\documentclass{beamer}

\mode<presentation>
{
  \usetheme{default}      % or try Darmstadt, Madrid, Warsaw, ...
  \usecolortheme{default} % or try albatross, beaver, crane, ...
  \usefonttheme{default}  % or try serif, structurebold, ...
  \setbeamertemplate{navigation symbols}{}
  \setbeamertemplate{caption}[numbered]
} 

\usepackage[utf8]{inputenc}
\usepackage{epstopdf}
\usepackage{graphicx}
\graphicspath{ {../figures/} }
\usepackage{xcolor}
\usepackage{hyperref}
\usepackage{soul} % for \st{...} aka strikethrough
\usepackage{amsmath} % for math ...
% \usepackage{gensymb} % degree and other symbols ...
\usepackage{tikz} % drawing stuff ..
\usepackage{enumitem}
\usepackage{siunitx} %% SI Units
\usetikzlibrary{tikzmark,shapes}

\tikzset{upper node/.style={fill=red!40},
  lower node/.style={ellipse,fill=red!20},
  ul line/.style={thick,red!20},
  upper/.code={\tikzset{upper node/.append style={#1}}},
  lower/.code={\tikzset{lower node/.append style={#1}}},
  line/.code={\tikzset{ul line/.append style={#1}}},
}

%% highlight with explanation beneath, see 
%% https://tex.stackexchange.com/questions/458694/highlighting-part-of-equation-with-text-underneath
\newcounter{HighLight}
\newcommand{\highlightwu}[3][]{%
  \stepcounter{HighLight}
  \tikzset{#1}
  \underset{\underset{\displaystyle\makebox[0pt]{\text{\tikzmarknode[lower node]{below-\theHighLight}{%
    #3}}}}{\phantom{!}}}{\tikzmarknode[upper node]{above-\theHighLight}{#2}}
    \tikz[overlay,remember picture]{\draw[ul line] (above-\theHighLight) --
      (below-\theHighLight);}
}
%% highlight term
\newcommand{\highlight}[2][]{%
  \stepcounter{HighLight}
  \tikzset{#1}
  {\tikzmarknode[upper node]{above-\theHighLight}{\textbf #2}}
}

%% define a box ...
\setbeamercolor{postit}{fg=black,bg=example text.fg!75!black!10!bg}

%% some itemize styles
\newcommand{\citem}{\item[\checkmark]}
\newcommand{\bitem}{\item[\textbullet]}
\newcommand{\ditem}{\item[\textendash]}
\newcommand{\aitem}{\item[\textasteriskcentered]}

\usepackage{natbib}
\bibliographystyle{plainnat}

\title[]{PhD Status}
\author{Xanthos}
\institute{DSO \& IGN}
\date{\today}

\begin{document}

\begin{frame}
  \titlepage
\end{frame}

\begin{frame}\frametitle{So far ...}\framesubtitle{}
  \textbf{What does the ``integrator'' do, it think ...}
  \medskip

  \begin{itemize}
    \bitem We are processing data and we are currently at epoch $t_i$. We 
    already have an estimate of the state at this epoch, $\bf{X}_{t_i}^-$ via 
    filtering.
    \bitem We use the observations at $t_i$ and compute a new state estimate
    (at $t_i$), $\bf{X}_{t_i}^+$
    \bitem We read in the next observation, which is at $t_{i+1}$. To process 
    this observation we need the state at $t_{i+1}$, hence, \textbf{we must 
    propagate the state from $t_i$ to $t_{i+1}$}. That is the job of the 
    ``integrator''.
  \end{itemize}
\end{frame}

\begin{frame}\frametitle{Currently ...}\framesubtitle{}
    \textbf{How does the ``integrator'' work}
    \medskip 

    The ``integrator'' will propage the state, by solving a 2\textsuperscript{nd} 
    order ODE system, comprised of the \emph{State Transition} and the 
    \emph{Variational Equations}.

    \smallskip
    That is an ``arithmetic'' solution, meaning that the ``integrator'' will 
    compute the equations in many epochs between the interval $t_i$ to 
    $t_{i+1}$

\end{frame}

\begin{frame}\frametitle{Currently ...}\framesubtitle{}
    \textbf{What goes in the ``integrator''}
    \medskip
    
    Pretty much everything apart from the filtering! E.g.
    \begin{itemize}
        \bitem Coordinate transformations (inertial/earth-fixed)
        \bitem Time scales / EOPs
        \bitem Planets position (DE)
        \bitem State transition
        \bitem Variational equations
        \bitem Force model (for now gravity)
        \bitem Partials/gradients (for now gravity)
        \bitem ODE solution
    \end{itemize}

    \medskip
    That is why it seems like a good test.
\end{frame}

\begin{frame}\frametitle{Currently ...}\framesubtitle{}
  \textbf{Idea for validation}
  \medskip

  \begin{itemize}
    \bitem Replace the state estimate $\bf{X}_{t_i}^+$ (we would normally get from 
  the filter), with the state recorded in the Sp3 file. That is, we assume 
  we processed the data at $t_i$ and came with this estimate, pretty accurate!

    \bitem Propagate the state to the next epoch in the sp3 file (normally $t_i + 60sec$

    \bitem Check the ``integrator'' result against the sp3 value for $t_i + 60sec$

    \bitem Set $t_{i+1}=t_i + 60$. Use the new state from sp3 at $t_i + 60sec$ as 
  initial value to propagate state to $t_{i+1}+$

    \bitem Loop ...
  \end{itemize}

\end{frame}

\begin{frame}\frametitle{Conversation ...}\framesubtitle{}
\centering
    \includegraphics[width=\textwidth]{diffs}
\end{frame}

\begin{frame}\frametitle{Conversation ...}\framesubtitle{}
\centering
    \includegraphics[width=\textwidth]{foo3}
\end{frame}

\bibliography{doris}

\end{document}
