\chapter{Celestial Reference System and \gls{eop}}
\label{ch:celestial-rf-and-eop}

The transformation used to relate the \gls{itrs} to the \gls{gcrs} at a 
given date $t$, can be written as (\cite{iers2010}):
\begin{equation}
    \bm{r}_{GCRS} = \bm{Q}(t) \cdot \bm{R}(t) \cdot \bm{W}(t) \cdot \bm{r}_{ITRS}
    \label{eq:iers1051}
\end{equation}

where:
\begin{itemize}
    \item $\bm{Q}(t)$ is the transformation matrix arising from the
    (celestial) \textbf{polar motion} in the celestial reference system,
    \item $\bm{R}(t)$ is the transformation matrix accounting for 
    \textbf{Earth rotation} around the axis ascociated with the pole, and
    \item $\bm{W}(t)$ accounts for \textbf{polar motion}
\end{itemize}

Note that the paremeter $t$ used in \ref{eq:iers1051} is given by:
\begin{equation}
    t = \left( TT - 2000 \text{ January } 1d 12h TT \right)
    \text{ in days } / 36525
\end{equation}
where $2000 \text{ January 1d 12h TT} = \text{ Julian Date } 2451545.0 \text{ TT}$.

\ul{Here, we follow the approach compliant with the \emph{IAU 2000/2006} 
resolutions}. Hence, the quantities to be used in the matrix $\bm{Q}(t)$ in 
\ref{eq:iers1051} must be based on the \emph{IAU 2006} precession and the 
\emph{IAU 2000A} or \emph{IAU 2000B} (depending on required precision).

\section{Terrestrial to Celestial Transformation}
\label{sec:ter2cel-trans}

Currently, IERS recommends two procedures for transforming between the 
Terrestrial and Celestial reference frames, called the ``equinox based'' 
and the ``\gls{cio} based'', differing in the origin of the \gls{cip} 
equator. For further details on the ``equinox based'' transformation, see 
e.g. \cite{iers2010} and \cite{esaa13}. In the following we will discuss the 
``\gls{cio} based'' transformation, since
\begin{displayquote}
    only the CIO based
procedure can be in agreement with IAU 2000 Resolution B1.8, which requires the use of the “non-
rotating origin” in both the GCRS and the ITRS as well as the position of the CIP in the GCRS
and in the ITRS.
\end{displayquote}, \cite{iers2010}.

Each of the three rotation matrices in \ref{eq:iers1051}, represents a series 
of elementary rotations, a product of the rotation matrices $R_x(\theta)$, 
$R_y(\theta)$ and $R_z(\theta)$, with positive angle about the $x-$, $y-$ and 
$z-$axis. The position of the \gls{cip} (in both \gls{itrs} and \gls{gcrs}) is 
provided by the \gls{cip} unit vector components $x$ and $y$, called 
``coordinates'' of the \gls{cip}.

\subsection{Polar Motion Matrix $W(t)$}
\label{ssec:polar-motion-matrix}
The rotation of the Earth is represented by the diurnal rotation around a
refernce axis, called the \gls{cip}. The \gls{cip} does not coincide with 
the axis of figure of the Earth, but slowly moves (in a terrestrial reference 
frame) (\cite{esaa13}). This motion of the terrestrial reference frame 
with respect to the \gls{cip} is known as \emph{polar motion}. Note that the 
\gls{cip} is not the instantenuous axis of rotation but the axis around which the 
diurnal rotation of earth is applied (in the celestial to terrestrial 
transformation). Polar motion is typically determined from \gls{vlbi} 
observation, as except from the principal periods of 365 days (annual wobble) 
and 428 days (Chandler wobble), it is also affected by unpredictable geophysical 
forces.

According to IAU 2006 Resolution B2, the system at date $t$ as realized 
from the \gls{itrs} by applying the transformation $\bm{W}(t)$ is the 
\gls{tirs}. It uses the \gls{cip} as its $z$-axis and the \gls{tio} as 
its $x$-axis (\cite{iers2010}). This matrix gives the position of the 
terrestrial reference frame with respect to the \gls{tio}.

The $\bm{W}$ matrix can be expressed as (\cite{iers2010}):
\begin{equation}
    \bm{W}(t) = \bm{R}_z(-s') \cdot \bm{R}_y(x_p) \cdot \bm{R}(y_p)
    \label{eq:iers1053}
\end{equation},
where $s'$ is the ``\gls{tio} locator'' and $x_p$, $y_p$ are the 
``polar coordinates'' of the \gls{cip} in the \gls{itrs}. The latter values, 
if not estimated, should be the ones published by the \gls{iers}, corrected for 
the effect of ocean tides and forced terms (aka ``libration''), with periods 
less than two days in space (\cite{iers2010}), so that:
\begin{equation}
    \begin{pmatrix} x_p & y_p \end{pmatrix} = 
    \begin{pmatrix} x & y \end{pmatrix}_{IERS} + 
    \begin{pmatrix} \Delta x & \Delta y \end{pmatrix}_{ocean\text{ }tides} + 
    \begin{pmatrix} \Delta x & \Delta y \end{pmatrix}_{libration} 
\end{equation}
Handling of ocean tides and forced terms is performed similar to the \gls{iers}-
published \texttt{INTERP.F}\footnote{available from IERS at \url{https://hpiers.obspm.fr/iers/models/interp.f}} routine. 
In principle, the same result should be obtained by computing the respective 
corrections from calling \texttt{RG\_ZONT2.F}\footnote{Available from the \gls{iers} \href{https://iers-conventions.obspm.fr/}{Conventions Centre} at \url{https://iers-conventions.obspm.fr/content/chapter8/software/RG_ZONT2.F}, provided by A. Brzezinski.} 
and \texttt{PMSDNUT2.F}\footnote{Available from the \gls{iers} \href{https://iers-conventions.obspm.fr/}{Conventions Centre} at \url{https://iers-conventions.obspm.fr/content/chapter5/software/PMSDNUT2.F}, provided by A. Brzezinski.}.

\begin{itemize}
    \item The subdaily variations are not part of the polar motion values 
    reported to and distributed by the \gls{iers} and are therefore to be 
    added after interpolation (\cite{iers2010}). To perform this correction, 
    two seperate effects are taken into account,
\end{itemize}

\section{Implementation}
\label{eop-implementation}

\subsubsection{\gls{eop} Information}
\Gls{eop} information for formulating the Celestial-to-Terrestrial transformation 
matrix, is extracted from the \gls{iers} \texttt{C04} files (\cite{Bizouard2019}).
These files contain tabulated \gls{eop} values at 0\textsuperscript{h} \gls{utc}. 

These files can be read into an \texttt{dso::EopLookUpTable} instance, using 
the function \texttt{dso::parse\_iers\_C04}.

According to \cite{Bradley2016850}:
\begin{displayquote}
    Prior to the interpolation of DUT1 and LOD, the tabulated values
    should be smoothed through regularization to enhance the
    interpolation accuracy. Regularization is the removal of
    zonal tidal variations with frequencies ranging from 5 days
    to 18.6 years.
\end{displayquote}


\begin{figure}
\centering
\begin{tikzpicture}
	\node (A) at (0,0) 
		[minimum height=2.5em, minimum width=10em, fill=blue!10,rounded corners, drop shadow] 
		{EOP C04 (published by \gls{iers})};
	
	\node (AtB) at (0,-3) 
		[minimum height=2.5em, minimum width=10em, fill=red!10,rounded corners, drop shadow]
		{\begin{minipage}{.5\textwidth}\begin{itemize}
		\item ``regularize''\footnote{(remove zonal Earth tide effect in $\Delta UT$ and LOD via \texttt{RG\_ZONT2})}
		\item \gls{utc} to \gls{tt}
		\end{itemize}\end{minipage}};
		
	\node (B) at (0,-6.0) 
		[minimum height=2.5em, minimum width=10em, fill=blue!10,rounded corners, drop shadow] 
		{\texttt{EopLookUpTable}};

	\node (BtI) at (0,-10) 
		[minimum height=2.5em, minimum width=10em, fill=red!10,rounded corners, drop shadow]
		{\begin{minipage}{.6\textwidth}\begin{itemize}
		\item Lagrangian inmterpolation (5\textsuperscript{th} order)
		\item compute effects of zonal Earth tides \& add to $\Delta UT$ and LOD (via \texttt{RG\_ZONT2})
		\item add effect of ocean tides (as in \texttt{INTERP.F}) to $x_p, y_p, \Delta UT, LOD$
		\item add libration effects (as in \texttt{INTERP.F}) to $x_p, y_p$
		\end{itemize}\end{minipage}};
	
	\node (I) at (-8,-8)
		[minimum height=2.5em, minimum width=10em, fill=blue!10,rounded corners, drop shadow]
		{Interpolate};

	\draw[-Latex,thick] (A.south) -- (AtB.north);
	\draw[-Latex,thick] (AtB.south) -- (B.north);
	\draw[-Latex,thick] (I.north) |- (B.west);
	\draw[-Latex,red,thick] (B.south) -| (BtI.north);
	\draw[-Latex,red,thick] (BtI.west) -| (I.south);
	%\draw[->] (A)--(B) node[midway]{Remove effects of zonal Earth tides on $\Delta$UT and LOD (\texttt{rg\_zont2})}; 
\end{tikzpicture}

\caption{Extracting \gls{eop} information from \gls{iers} \texttt{C04} data files.}
\label{fig:handling-eop}
\end{figure}
