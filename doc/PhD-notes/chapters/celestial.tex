\chapter{Celestial Reference System and \gls{eop}}
\label{ch:celestial-rf-and-eop}

The transformation used to relate the \gls{itrs} to the \gls{gcrs} at a 
given date $t$, can be written as (\cite{iers2010}):
\begin{equation}
    \bm{r}_{GCRS} = \bm{Q}(t) \cdot \bm{R}(t) \cdot \bm{W}(t) \cdot \bm{r}_{ITRS}
    \label{eq:iers1051}
\end{equation}

where:
\begin{itemize}
    \item $\bm{Q}(t)$ is the transformation matrix arising from the
    (celestial) \textbf{polar motion} in the celestial reference system,
    \item $\bm{R}(t)$ is the transformation matrix accounting for 
    \textbf{Earth rotation} around the axis ascociated with the pole, and
    \item $\bm{W}(t)$ accounts for \textbf{polar motion}
\end{itemize}

Note that the paremeter $t$ used in \ref{eq:iers1051} is given by:
\begin{equation}
    t = \left( TT - 2000 \text{ January } 1d 12h TT \right)
    \text{ in days } / 36525
\end{equation}
where $2000 \text{ January 1d 12h TT} = \text{ Julian Date } 2451545.0 \text{ TT}$.

\ul{Here, we follow the approach compliant with the \emph{IAU 2000/2006} 
resolutions}. Hence, the quantities to be used in the matrix $\bm{Q}(t)$ in 
\ref{eq:iers1051} must be based on the \emph{IAU 2006} precession and the 
\emph{IAU 2000A} or \emph{IAU 2000B} (depending on required precision).

\begin{figure}
\centering
\begin{tikzpicture}[
  every node/.style = {
    draw=black, 
    rounded corners, 
    fill=gray!30,
    minimum width=2cm,
    minimum height=0.5cm,
    align=center},
  every path/.style = {
    draw,
    -latex}
  ]

  \node (gcrs) {\acrfull{gcrs}};
  
  \node[fill=gray!5] (gcrs2cirs) [below =of gcrs] {
    apply matrix \(Q(t)\), using \gls{cip} coordinates 
    and the \gls{cio} locator:\\
    \(Q(t) = R_3 (E) R_2 (d) R_3 (-E) R_3 (-s)\)\\
    see \ref{ssec:gcrs-to-cirs-via-q}
  };

  \node (cirs) [below =of gcrs2cirs] {\acrfull{cirs}};

  \node[fill=gray!5] (cirs2tirs) [below =of cirs] {
    apply matrix \(R(t)\), using the \gls{era}:\\
    \(R(t) = R_3 (ERA) \)\\
    see \ref{ssec:cirs-to-tirs-via-r}
  };
  
  \node (tirs) [below =of cirs2tirs] {\acrfull{tirs}};

  \node[fill=gray!5] (tirs2itrs) [below =of tirs] {
    apply matrix \(W(t)\), that is:\\
    rotation to the \gls{itrs} origin: \(R_3 (s')\), and \\
    rotation for polar motion: \(R_2(-x_p) R_1(-y_p)\) \\
    see \ref{ssec:tirs-to-itrs-via-w}
  };
  
  \node (itrs) [below =of tirs2itrs] {\acrfull{itrs}};

  \draw (gcrs) -- (gcrs2cirs);
  \draw (gcrs2cirs) -- (cirs);
  \draw (cirs) -- (cirs2tirs);
  \draw (cirs2tirs) -- (tirs);
  \draw (tirs) -- (tirs2itrs);
  \draw (tirs2itrs) -- (itrs);

\end{tikzpicture}

\caption{Schematic representation of the ``\gls{cio}-based'' procedure to 
  transform between the \gls{gcrs} and \gls{itrs}.}
\label{fig:gcrs-to-itrs}
\end{figure}

\section{Terrestrial to Celestial Transformation}
\label{sec:ter2cel-trans}

Currently, \gls{iers} recommends two procedures for transforming between the 
Terrestrial and Celestial reference frames, called the ``equinox based'' 
and the ``\gls{cio} based'', differing in the origin of the \gls{cip} 
equator. For further details on the ``equinox based'' transformation, see 
e.g. \cite{iers2010} and \cite{esaa13}. In the following we will discuss the 
``\gls{cio} based'' transformation, since
\begin{displayquote}
    only the CIO based
procedure can be in agreement with IAU 2000 Resolution B1.8, which requires 
the use of the ``non-rotating origin'' in both the \gls{gcrs}and the \gls{itrs} 
as well as the position of the \gls{cip} in the \gls{gcrs} and in the \gls{itrs}.
\end{displayquote}, \cite{iers2010}.

Each of the three rotation matrices in \ref{eq:iers1051}, represents a series 
of elementary rotations, a product of the rotation matrices $R_x(\theta)$, 
$R_y(\theta)$ and $R_z(\theta)$, with positive angle about the $x-$, $y-$ and 
$z-$axis. The position of the \gls{cip} (in both \gls{itrs} and \gls{gcrs}) is 
provided by the \gls{cip} unit vector components $x$ and $y$, called 
``coordinates'' of the \gls{cip}.

\subsection{Polar Motion Matrix $W(t)$}
\label{ssec:polar-motion-matrix}
The rotation of the Earth is represented by the diurnal rotation around a
refernce axis, called the \gls{cip}. The \gls{cip} does not coincide with 
the axis of figure of the Earth, but slowly moves (in a terrestrial reference 
frame) (\cite{esaa13}). This motion of the terrestrial reference frame 
with respect to the \gls{cip} is known as \emph{polar motion}. Note that the 
\gls{cip} is not the instantenuous axis of rotation but the axis around which the 
diurnal rotation of earth is applied (in the celestial to terrestrial 
transformation). Polar motion is typically determined from \gls{vlbi} 
observation, as except from the principal periods of 365 days (annual wobble) 
and 428 days (Chandler wobble), it is also affected by unpredictable geophysical 
forces.

According to IAU 2006 Resolution B2, the system at date $t$ as realized 
from the \gls{itrs} by applying the transformation $\bm{W}(t)$ is the 
\gls{tirs}. It uses the \gls{cip} as its $z$-axis and the \gls{tio} as 
its $x$-axis (\cite{iers2010}). This matrix gives the position of the 
terrestrial reference frame with respect to the \gls{tio}.

The $\bm{W}$ matrix can be expressed as (\cite{iers2010}):
\begin{equation}
    \bm{W}(t) = \bm{R}_z(-s') \cdot \bm{R}_y(x_p) \cdot \bm{R}(y_p)
    \label{eq:iers1053}
\end{equation}
where $s'$ is the ``\gls{tio} locator'' and $x_p$, $y_p$ are the 
``polar coordinates'' of the \gls{cip} in the \gls{itrs}. The latter values, 
if not estimated, should be the ones published by the \gls{iers}, corrected for 
the effect of ocean tides and forced terms (aka ``libration''), with periods 
less than two days in space (\cite{iers2010}), so that:
\begin{equation}
    \begin{pmatrix} x_p & y_p \end{pmatrix} = 
    \begin{pmatrix} x & y \end{pmatrix}_{IERS} + 
    \begin{pmatrix} \Delta x & \Delta y \end{pmatrix}_{ocean\text{ }tides} + 
    \begin{pmatrix} \Delta x & \Delta y \end{pmatrix}_{libration} 
\end{equation}
Handling of ocean tides and forced terms is performed similar to the \gls{iers}-published 
\texttt{INTERP.F}\footnote{Available from IERS at \url{https://hpiers.obspm.fr/iers/models/interp.f} \label{fn:interp-f}} routine. 
In principle, the same result should be obtained by computing the respective 
corrections from calling \texttt{ORTHO\_EOPF}\footnote{Available from the \gls{iers} \href{https://iers-conventions.obspm.fr/}{Conventions Centre} at \url{https://iers-conventions.obspm.fr/content/chapter8/software/ORTHO_EOP.F}, provided by R. Eanes.\label{fn:ortho-eop-f}} 
to compute the diurnal and semidiurnal variations in the Earth orientation
and \texttt{PMSDNUT2.F}\footnote{Available from the \gls{iers} \href{https://iers-conventions.obspm.fr/}{Conventions Centre} at \url{https://iers-conventions.obspm.fr/content/chapter5/software/PMSDNUT2.F}, provided by A. Brzezinski. \label{fn:pmsdnut2-f}}
to compute the diurnal lunisolar effect on polar motion, see \ref{fig:handling-eop}.

The \gls{tio} locator $s'$, positioning the \gls{tio} on the equator of the \gls{cip}, 
is necessary to provide an exact realization of the ``instantaneous prime meridian'', 
designated by ``\gls{tio} meridian'' (\cite{iers2010}). $s'$ is obtained from 
polar motion observations by numerical integration, and so is in essence 
unpredictable. However, it is dominated by a secular drift of about 
\SI{47}{\micro\larcsecond \per \century}. The latter is used to actually compute 
$s'$ in \ref{eq:iers1053}\footnote{In accordance to the \gls{sofa} (\cite{SOFA20210125}) supplied \texttt{iauSp00} function} 
using the function:
\begin{equation}
  s' = \SI{-47}{\micro\larcsecond} \cdot t
  \label{eq:iers10513}
\end{equation}
obtained from C04 data (\cite{Lambert_2002}).

The $\bm{W}(t)$ matrix is computed using a variant of the \gls{sofa} (\cite{SOFA20210125}) 
supplied \texttt{iauPom00} function.

\subsection{Earth Rotation Matrix $R(t)$}
\label{ssec:earth-rotation-matrix}
The rotation of the Earth around the axis of the \gls{cip} (i.e. relating 
\gls{tirs} and \gls{cirs}), can be expressed as (\cite{iers2010}):
\begin{equation}
  \bm{R}(t) = \bm{R}_z (-ERA)
  \label{eq:iers1055}
\end{equation}
where $ERA$ is the \gls{era} between the \gls{cio} and the \gls{tio} 
at date $t$ on the equator of the \gls{cip}, which is the rigorous definition 
of the sidereal rotation of the Earth\footnote{The Earth rotation angle used in 
\ref{eq:iers1055}, is either Greenwich apparent sidereal time if we use the 
``equinox based'' approach, or $ERA$ if we follow ``CIO based'' procedure. Here, 
as already noted, we use the latter.}. Working with respect to the \gls{cio} 
(rather than the equinox) sweeps away sidereal time's complexities and opportunities 
for error. The Earth rotation angle, the \gls{cio} based counterpart of \gls{gst},
is simply a conventional linear transformation of \gls{ut1} (\cite{sofa_18141_eacb}):
\begin{equation}
  \label{eq:iers10515}
  \begin{split}
    ERA(T_u) = 2 \pi & ( \text{UT1 Julian day fraction } \\
                     & + 0.7790572732640 + 0.00273781191135448 \cdot T_u )
    \end{split}
\end{equation}
where $T_u = \left( \text{Julian UT1 date } - 2451545.0 \right)$ and 
$UT1=UTC+(UT1-UTC)$. 

Similarly to polar motion, additional components should 
be added to the values published by \gls{iers} for $\Delta UT$ to account for 
the effects of ocean tides and libration. These can be computed using the 
\gls{iers}-published \texttt{INTERP.F}\footnote{Available from IERS at \url{https://hpiers.obspm.fr/iers/models/interp.f} \label{fn:interp-f}} 
routine, or via a call to \texttt{ORTHO\_EOP}\footnote{Available from the \gls{iers} \href{https://iers-conventions.obspm.fr/}{Conventions Centre} at \url{https://iers-conventions.obspm.fr/content/chapter8/software/ORTHO_EOP.F}, provided by R. Eanes.} 
(to compute the diurnal and semidiurnal variations in the Earth orientation) 
followed by a call to \texttt{UTLIBR}\footnote{Available from the \gls{iers} \href{https://iers-conventions.obspm.fr/}{Conventions Centre} at \url{https://iers-conventions.obspm.fr/content/chapter5/software/UTLIBR.F}, provided by A. Brzezinski.\label{fn:utlibr-f}} 
to account for the subdiurnal librations in UT1. These corrections are applied 
after the interpolation, see \ref{fig:handling-eop}.

Note that according to \cite{Bradley2016850}:
\begin{displayquote}
    Prior to the interpolation of DUT1 and LOD, the tabulated values
    should be smoothed through regularization to enhance the
    interpolation accuracy. Regularization is the removal of
    zonal tidal variations with frequencies ranging from 5 days
    to 18.6 years.
\end{displayquote}
This ``regularization'' is implemented in the \gls{eop} interpolation proccess, 
see \ref{fig:handling-eop}.

\subsection{Celestial Motion Matrix $Q(t)$}
\label{ssec:celestial-motion-matrix}
The \gls{cio} based transformation matrix arising from the motion of the \gls{cip} 
in the \gls{gcrs} (i.e. relating \gls{cirs} and \gls{gcrs}), can be expressed as
\cite{iers2010}):
\begin{equation}
  \bm{Q}(t) = \bm{R}_z (-E)  \cdot 
              \bm{R}_y (-d) \cdot 
              \bm{R}_z (E) \cdot 
              \bm{R}_Z (s)
  \label{eq:iers1056}
\end{equation}
where $s$ is the ``\gls{cio} locator'' and $E$ and $d$ being such that the 
coordinates of the \gls{cip} in the \gls{gcrs} are:
\begin{equation}
  \begin{aligned}
    X & = \sin{d} \cos{E} \\
    Y & = \sin{d} \sin{E} \\
    Z & = \cos{d}
  \end{aligned}
\end{equation}
\ref{eq:iers1056} can be given in an equivalent form directly involving $X$ and 
$Y$ as (\cite{iers2010}):
\begin{equation}
  \bm{Q}(t) = \begin{pmatrix}
    1-\alpha X^2 & -\alpha XY & X \\
    -\alpha XY & 1-\alpha Y^2 & Y \\
    -X & -Y & 1-\alpha (X^2 + Y^2) \end{pmatrix}
    \cdot \bm{R}_Z (s)
    \label{eq:iers10510}
\end{equation}
with $\alpha = 1/(1+\cos{d})$ , which can also be written, with an accuracy of 
\SI{1}{\micro\larcsecond} as $\alpha = 1/2 + 1/8(X^2 + Y^2)$.

$X$ and $Y$ coordinates can be given by developments as function of time in the 
\si{\micro\larcsecond} level, based on the IAU 2006 precession and IAU 2000A
nutation (\cite{CapitaineAndWallace2006})\footnote{Implemented in the \gls{sofa} (\cite{SOFA20210125}) supplied \texttt{iauXy06} function.}
The IAU 2006/2000A developments are as follows (\cite{iers2010}):
\begin{equation}
  \label{eq:tn36516a}
  \begin{aligned}
  X &= \SI{-0.01661700}{\arcsecond} + \SI{2004.19189800}{\arcsecond} t - \SI{0.429782900}{\arcsecond} t^2 \\
  &- \SI{0.1986183400}{\arcsecond}t^3 + \SI{0.00000757800}{\arcsecond} t^4 + \SI{0.000005928500}{\arcsecond} t^5 \\
  &+ \sum_{i} \left[ (a_{s,0})_i \sin \theta + (a_{c,0})_i \cos \theta \right] \\ 
  &+ \sum_{i} \left[ (a_{s,1})_i t \sin \theta + (a_{c,1})_i t \cos \theta \right] \\ 
  &+ \sum_{i} \left[ (a_{s,2})_i t^2 \sin \theta + (a_{c,2})_i t^2 \cos \theta \right] \\ 
  &+ \cdots \\
  \end{aligned}
\end{equation}
and
\begin{equation}
  \label{eq:tn36516b}
  \begin{aligned}
  Y &= -\SI{0.00695100}{\arcsecond} - \SI{0.02589600}{\arcsecond} t - \SI{22.407274700}{\arcsecond} t^2 \\
  &+ \SI{0.0019005900}{\arcsecond} t^3 + \SI{0.00111252600}{\arcsecond} t^4 + \SI{0.000000135800}{\arcsecond} t^5 \\
  &+ \sum_{i} \left[ (b_{s,0})_i \sin \theta     + (b_{c,0})_i \cos \theta \right] \\ 
  &+ \sum_{i} \left[ (b_{s,1})_i t \sin \theta   + (b_{c,1})_i t \cos \theta \right] \\ 
  &+ \sum_{i} \left[ (b_{s,2})_i t^2 \sin \theta + (b_{c,2})_i t^2 \cos \theta \right] \\ 
  &+ \cdots \\
  \end{aligned}
\end{equation}

where $\theta$ is a function of the fundamental lunisolar and planetary arguments\footnote{Furter information and computation formulas for the fundamental arguments, can be found in \cite{iers2010}, e.g. Chapter 5.7}. Complete 
list of coefficients for \ref{eq:tn36516a} and \ref{eq:tn36516b} is provided 
by \gls{iers}. 

As \gls{vlbi} observations have shown that there are deficiencies in the 
IAU 2006/2000A precession-nutation model of the order of \SI{0.2}{\milli\larcsecond}, 
mainly due to the fact that the free core nutation (\gls{fcn}) is not part of 
the model, \gls{iers} publish observed estimates of the corrections to the 
IAU precession-nutation model. The observed differences with respect to the 
conventional celestial pole position defined by the models are monitored and 
reported by the \gls{iers}as ``celestial pole offsets''. Such time-dependent 
offsets from the direction of the pole of the \gls{gcrs} must be provided as 
corrections $\delta X$ and $\delta Y$ to the $X$ and $Y$ coordinates (\cite{iers2010}).
Using these offsets, the corrected celestial position of the \gls{cip} is 
given by (\cite{iers2010}):
\begin{equation}
  \begin{aligned}
    X = X_{\text{IAU 2006/2000}} + \delta X \\
    Y = Y_{\text{IAU 2006/2000}} + \delta Y
  \end{aligned}
\end{equation}
thus enabling to re-write \ref{eq:iers10510} as:
\begin{equation}
  \bm{\tilde{Q}}(t) = \begin{pmatrix}
    1 & 0 & \delta X \\
    0 & 1 & \delta Y \\
    -\delta X & -\delta Y & 1
    \end{pmatrix}
    \cdot \bm{Q}_{IAU}
    \label{eq:iers10527}
\end{equation}
where $\bm{Q}_{IAU}$ represents the $\bm{Q}(t)$ matrix based on the IAU 2006/2000 
precession-nutation model.

The ``\gls{cio} locator'' $s$, providing the position of the \gls{cio} in the 
\gls{gcrs} can also be computed using a developement described in \cite{Capitaineetal2003a}\footnote{\gls{sofa} (\cite{SOFA20210125}) supplies an implementation of the formula named \texttt{iauS06} function}.

\begin{figure}
     \centering
     \begin{subfigure}[b]{0.48\textwidth}
         \centering
         \includegraphics[width=\textwidth]{eop-00to22}
         \caption{\glspl{eop} extracted from \texttt{C04} file}
         \label{fig:erp-00to22}
     \end{subfigure}
     \hfill
     \begin{subfigure}[b]{0.48\textwidth}
         \centering
         \includegraphics[width=\textwidth]{eop-daily}
         \caption{\glspl{eop} interpolated values (see \ref{fig:handling-eop})}
         \label{fig:erp-00to22}
     \end{subfigure}
\caption{\gls{eop} information from \gls{iers} \texttt{C04} data files.}
\label{fig:erp-plots}
\end{figure}

\section{Implementation}
\label{eop-implementation}

\subsection{\gls{eop} Information}
\label{ssec:eop-information}

\Gls{eop} information for formulating the Celestial-to-Terrestrial transformation 
matrix, is extracted from the \gls{iers} \texttt{C04} files (\cite{Bizouard2019}).
These files contain tabulated \gls{eop} values at 0\textsuperscript{h} \gls{utc}. 

These files can be read into an \texttt{dso::EopLookUpTable} instance, using 
the function \texttt{dso::parse\_iers\_C04}. The function will parse the file 
for the requested date range, transform \gls{utc} dates to \gls{tt} and conviniently 
store them in an \texttt{dso::EopLookUpTable} instance. $\Delta UT$ and LOD values 
can be ``regularized'', aka have the zonal Earth tide effects removed, via the 
\texttt{dso::EopLookUpTable::regularize} function.

To interpolate \gls{eop} values from an \texttt{dso::EopLookUpTable} instance, 
users can call the \texttt{dso::EopLookUpTable::interpolate} function, given 
a datetime instance in \gls{tt}, also specifying the order of Lagrangian 
interpolation. Effects for ocean tides and libration (see \ref{ssec:polar-motion-matrix} 
and \ref{ssec:earth-rotation-matrix}) are handled in the function. The output 
is a complete set of \gls{eop}s, containing $x_p$, $y_p$, $\Delta UT$, $\delta X$ 
and $\delta Y$ at given $t$. A list of relevant function and data structures 
used to manipulate \gls{eop} is presented in \ref{table:eop-handling-fds}. The 
parsing and interpolation procedures are listed in \ref{fig:handling-eop}.

\begin{table}
\centering
\begin{tabular}{|p{3cm}|p{3cm}|p{2.5cm}|p{5cm}|}
\hline
\textbf{Function/Data Structure} & \textbf{Header File} & \textbf{Repository/ Library} & \textbf{Comment} \\
\hline
\texttt{EopLookUpTable} & \texttt{eop.hpp} & \texttt{doris} & 
Contains all relevant declerations, data structures and algorithms to handle 
\gls{eop} information. \\

\hline
\texttt{parse\_iers\_C04} & \texttt{eop.hpp} & \texttt{doris} & \\

\hline
\texttt{ortho\_eop} & \texttt{iers2010.hpp} & \texttt{iers2010} &
compute the diurnal and semidiurnal variations in \gls{eop} ($x$,$y$, $UT1$) from
 ocean tides; translated from \texttt{ORTHO\_EOP.F} \footref{fn:ortho-eop-f} \\

\hline
\texttt{pmsdnut2} & \texttt{iers2010.hpp} & \texttt{iers2010} &
evaluate the model of polar motion for a nonrigid Earth due to tidal gravitation; 
translated from \texttt{PMSDNUT2.F} \footref{fn:pmsdnut2-f} \\ 

\hline
\texttt{utlibr} & \texttt{iers2010.hpp} & \texttt{iers2010} & 
compute subdiurnal libration in the axial component of rotation, expressed by 
\gls{ut1} and LOD. This effect is due to the influence of tidal gravitation on the
 departures of the Earth's mass distribution from the rotational
 symmetry, expressed by the non-zonal components of geopotential. Translated 
 from \texttt{UTLIBR.F} \footref{fn:utlibr-f} \\

\hline
\texttt{interp\_pole} (and \texttt{interp} namespace) & \texttt{iers2010.hpp} & \texttt{iers2010} & 
account for ocean tidal and libration effects in \gls{eop}; translated from \texttt{INTERP.F} \footref{fn:interp-f}\\

\hline
\end{tabular}
\caption{List of functions and data structures relevant to handling \gls{eop} information.}
\label{table:eop-handling-fds}
\end{table}


\begin{figure}
\centering
\begin{tikzpicture}
	\node (A) at (0,0) 
		[minimum height=2.5em, minimum width=10em, fill=blue!10,rounded corners, drop shadow] 
		{EOP C04 (published by \gls{iers})};
	
	\node (AtB) at (0,-3) 
		[minimum height=2.5em, minimum width=10em, fill=red!10,rounded corners, drop shadow]
		{\begin{minipage}{.5\textwidth}\begin{itemize}
		\item ``regularize''\footnote{(remove zonal Earth tide effect in $\Delta UT$ and LOD via \texttt{RG\_ZONT2})}
		\item \gls{utc} to \gls{tt}
		\end{itemize}\end{minipage}};
		
	\node (B) at (0,-6.0) 
		[minimum height=2.5em, minimum width=10em, fill=blue!10,rounded corners, drop shadow] 
		{\texttt{EopLookUpTable}};

	\node (BtI) at (0,-10) 
		[minimum height=2.5em, minimum width=10em, fill=red!10,rounded corners, drop shadow]
		{\begin{minipage}{.6\textwidth}\begin{itemize}
		\item Lagrangian inmterpolation (5\textsuperscript{th} order)
		\item compute effects of zonal Earth tides \& add to $\Delta UT$ and LOD (via \texttt{RG\_ZONT2})
		\item add effect of ocean tides (as in \texttt{INTERP.F}) to $x_p, y_p, \Delta UT, LOD$
		\item add libration effects (as in \texttt{INTERP.F}) to $x_p, y_p$
		\end{itemize}\end{minipage}};
	
	\node (I) at (-8,-8)
		[minimum height=2.5em, minimum width=10em, fill=blue!10,rounded corners, drop shadow]
		{Interpolate};

	\draw[-Latex,thick] (A.south) -- (AtB.north);
	\draw[-Latex,thick] (AtB.south) -- (B.north);
	\draw[-Latex,thick] (I.north) |- (B.west);
	\draw[-Latex,red,thick] (B.south) -| (BtI.north);
	\draw[-Latex,red,thick] (BtI.west) -| (I.south);
	%\draw[->] (A)--(B) node[midway]{Remove effects of zonal Earth tides on $\Delta$UT and LOD (\texttt{rg\_zont2})}; 
\end{tikzpicture}

\caption{Extracting \gls{eop} information from \gls{iers} \texttt{C04} data files.}
\label{fig:handling-eop}
\end{figure}
