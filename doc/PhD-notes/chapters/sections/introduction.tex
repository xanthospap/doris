\section{Introduction}\label{sec:introduction}

Satellite Geodesy comprises the observational and computational techniques which allow
the solution of geodetic problems by the use of precise measurements to, from, or between
artificial, mostly near-Earth, satellites (\cite{Seeber2003}). Satellite geodesy began
shortly after the launch of the first artificial satellite, SPUTNIK-1, on October 4, 1957.
Observations of Explorer 1 and Sputnik 2 in 1958 allowed for an accurate determination of
Earth's flattening. Latter missions led to the accurate determination of the leading
spherical harmonic coefficients of the geopotential, the general shape of the geoid,
and linked the world's geodetic datums (e.g. \cite{KingHele1958} and \cite{Merson1958}).
Historical notes on the advancement of Satellite Geodesy can be found in \cite{Seeber2003}
and \cite{Barlier2001}.

Professor Veis was one of the framers of the early Satellite Geodesy Program at the
Smithsonian Astrophysical Observatory (SAO), which itself was a fundamental element of
the early \gls{nasa} Space Geodesy Activity. He was awarded with his PhD in 1958,
after defending his famous dissertation on the "Geodetic Applications of Observations
of the Moon, Artificial Satellites and Rockets" (\cite{Veis1960}). He actively contributed the early
concept and evolution of the Differential Orbit Improvement (DOI) Program, which became
the main analysis tool for satellite tracking, geopotential estimation, station
coordinate determination, and satellite drag research. He defined the fundamental
reference system used for many years, which now forms the basis of modern models
of earth rotation, precession, and nutation. He also initiated the SAO Star Catalogue
project, which provided a uniform all-sky catalogue for precision camera observations,
and was used for many years all over the world\footnote{"\textit{The passing of Professor George Veis}",
31/01/2022, \gls{nasa} Space Geodesy Project, \url{https://space-geodesy.nasa.gov/news/news.html}}.

Since the launch of the first artificial satellites, satellite geodesy has developed into a
self-contained field in geodetic teaching and research, with close relations and interactions
with adjacent fields. By virtue of the ever increasing accuracy achieved by satellite geodesy,
its results and methods are used by or are connected to a wide range of disciplines, spanning
basic sciences, geosciences and engineering.
Developments in space science and technology have spurred new interest
in the subject of the Earth's rotation,  and geophysical knowledge of the Earth's interior has
undergone a very considerable revision since 1960 leading to an improved understanding of the
geophysical excitation functions that perturb the Earth's rotation (\cite{Lambeck1980}).

The use of artificial satellites in geodesy, has
some prerequisites though; these are basically a comprehensive knowledge of the satellite
motion under the influence of all acting forces as well as the description of the satellite
trajectory and ground stations in suitable reference frames, both spatial and temporal.
Thus, in the core of satellite geodesy lies the problem of orbit determination. Satellite
orbit determination is the process by which knowledge of the satellite’s
motion relative to the center of mass of the Earth in a specified coordinate system can be
obtained (\cite{Tapley2004}). Precise satellite orbits are a prerequisite for determining
reliable geodetic parameters, including but not limited to station coordinates, \gls{eop},
or coefficients of the Earth's gravity field. They are also fundamental for satellite-based
observations of geodynamics and climate impacts, such as sea level changes in the order of
millimeters. Realization and maintenance of global reference frames is also relied upon
\gls{pod}.

\gls{pod} analysis is a very challenging task, requiring multi-scientific expertise,
coupled with efficient engineering. Given the accuracy requirements today (especially
for altimetry missions on-board which \gls{doris} receivers are installed), the
ever growing number of scientific satellites missions, and the widening of the application
range (e.g. climate change studies), it constitutes a field of growing dynamics.
Earth orbiting artificial differ from most natural objects in that, due to their size, mass,
and orbit characteristics, the nongravitational forces are of significant importance.
Further, most satellites orbit near to the surface and for objects close to a central
body, the gravitational forces depart from a central force in a significant way
(\cite{Tapley2004}).

To-date, \gls{pod} is dominated by three space geodetic techniques, namely \gls{slr}, \gls{gnss} and
\gls{doris}, which additionally, via the corresponding Technique Centers, provide the input
data time series of station positions and \gls{eop} for the realization of the \gls{itrf}
\footnote{ITRF2020, \url{https://itrf.ign.fr/en/solutions/ITRF2020}}. Along with \gls{vlbi}, these
techniques constitute the fundamental pillars of modern satellite ans space geodesy.
State-of-the-art \gls{pod} analysis, can deliver accuracies in the centimeter level
(\cite{Rudenko2023}).

Since its inception in the late 1980s, \gls{doris} has  played a crucial role in expanding
geodetic knowledge and enhancing our understanding of the Earth’s dynamics and is currently
considered a pillar of satellite geodesy. However, despite its prominence and significance,
it has failed to allure a dedicated scientific audience comparable in size to the other two
techniques. For the most recent realization of \gls{itrf}, the \gls{ilrs} contributed data
from 7 Analysis Centers (\cite{Pavlis2023}), \gls{igs} from 10 (\cite{Rebischung2021}), while
at the same time the \gls{ids}'s contribution was derived from only 4 Analysis Centers
(\cite{Moreaux2022}). This shortage of dedicated \gls{ids} analysis centers, also reflects the
scarcity in dedicated software tools to process \gls{doris} data, first and foremost for
orbit determination.

In the framework of the current Thesis, this scarcity of dedicated \gls{doris}
analysis tools for orbit determination is targeted, with the aim of creating
a high quality, scientific software solution. Designing for extensibility, reusability
and adaptability, and adhering to a free and open source policy, an as large as possible
impact and contribution is sought for in the scientific community. Coupled with recent
advancements in the technique (e.g. adoption of the RINEX format, dissemination of
non-preprocessed data including carrier phase observables, introduction of next generation
receivers and beacons), we hope to spur further, renewed interest in this fundamental
area of satellite geodesy.

To accomplish this task, the software is built in a modular fashion. The various components
are organized in different, independent, moderate-sized libraries, targeting well defined
problems. State-of-the-art models, algorithms and design patterns are used as building
blocks.

Earth observation via artificial satellites and its numerous
missions and applications is heavily relied upon orbit determination;
