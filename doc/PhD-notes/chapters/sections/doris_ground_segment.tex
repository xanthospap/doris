\section{DORIS Ground Segment}\label{sec:doris-ground-segment}

Estimation of precise coordinates of both \gls{doris}-equipped satellites and 
\gls{doris} beacons relies on accurate modeling of the signal path from the
ground antenna to the space antenna. It is thus important to have a clear view of 
the geometry of the respective antennae and reference points of signal transmission 
and reception. Note that these reference points depend on the frequency (or linear 
combination of frequencies) used to perform a measurement.

\subsection{Geometry of Ground Antennae}\label{ssec:antenae-geometry}
\gls{doris} observations are referred to the electronic reference points (RP) of the 
antenna, the points where the \gls{doris} observations are acquired. However, 
these points are ``virtual'' and for example may change while using another antenna 
type, hence they lack the accuracy required for geodetic studies.
Observations must therefore be referred to the conventional RP which is defined 
according to the geometry of the antenna and account for the distance between 
the electronic RP and the conventional RP of the antenna (\cite{Tourain2016}). 
The ability to get accurate \gls{doris} data relies 
for one part on the capability of providing accurate models to connect the 
electronic RP (or electronic \emph{phase center}) and the conventional RP, as well 
as, \glspl{pcv} as a function of the elevation angle to the tracking satellite.

Three antenna types are used in the \gls{doris} tracking network, namely 
\texttt{ALCATEL}, \texttt{STAREC-B} and \texttt{STAREC-C} (\cite{Saunier2016}). 
\autoref{table:antenna-gains} records antenna gains per type. 
ALCATEL is the oldest antenna type, installed throughout the \gls{doris} network 
at the time of its establishment. These antennas were later replaced by the STAREC-B 
model, a long-standing effort that ended in 2007 with the replacement of the last 
station in Toulouse (TLHA) (\cite{Stepanek2022b}).

\begin{table}[h!]
    \centering
    \begin{tabular}{|c | c | c | c | c|}
        \hline
        \textbf{Zenith Distance} & \multicolumn{2}{c}{\textbf{ALCATEL (dBi)}} & \multicolumn{2}{c|}{\textbf{STAREC (dBi)}} \\
                        & \SI{401.25}{\mega\hertz} & \SI{2036.25}{\mega\hertz} &  \SI{401.25}{\mega\hertz} & \SI{2036.25}{\mega\hertz}\\
        \hline
        \ang{0}&3.2&2.1&3.5&0 \\
        \ang{10}&3.5&2.6&3.6&0.4\\
        \ang{20}&4&2&3.7&0.5\\
        \ang{30}&4.4&4&3.8&1.5\\
        \ang{40}&4.6&4.4&3.7&3.2\\
        \ang{50}&4.2&4.6&3.2&3.9\\
        \ang{60}&2.7&2.7&2.5&4\\
        \ang{70}&0.6&-0.1&1&3.2\\
        \ang{80}&-2.7&-3.3&-1.3&0.2\\
        \ang{90}&-6&-7&-4.2&-5.6\\
        \hline
    \end{tabular}
    \caption{DORIS ground antennae gains, \cite{DORISGSM}.}
    \label{table:antenna-gains}
\end{table}

The type of antenna installed at a particular station, is identified by the 
4\textsuperscript{th} character of the beacon mnemonic: letter ``\texttt{A}'' for the 
Alcatel type; letter ``\texttt{B}'' or letter ``\texttt{C}'' for the Starec B 
or C type respectively. That is, in the \gls{doris} RINEX field ``STATION REFERENCE'', 
the last character of the second column (aka ``4-character station code''), 
defines the ground beacon antenna type; e.g.

\begin{adjustbox}{max width=\linewidth , fbox=0.5pt}
\begin{BVerbatim}[commandchars=\\\{\}]
    51                                                      # OF STATIONS       
D01  BEM\textcolor{red}{B} BELGRANO                      66018S002  3   0    STATION REFERENCE   
D02  ADH\textcolor{red}{C} TERRE ADELIE                  91501S005  3   0    STATION REFERENCE   
D03  SYQ\textcolor{red}{B} SYOWA                         66006S005  3   0    STATION REFERENCE   
D04  CRQ\textcolor{red}{C} CROZET                        91301S004  4   0    STATION REFERENCE   
D05  DIO\textcolor{red}{B} DIONYSOS                      12602S012  3   0    STATION REFERENCE
\end{BVerbatim}
\end{adjustbox}

\subsubsection{Phase Center Offsets}\label{sssec:doris-pco}
Depending on the antenna type, appropriate \glspl{pco} need to be applied to 
the observed quantities for the reduction of the observation vector to the 
Reference Point (from the respective ``virtual'' phase center) of the antenna. 
\autoref{table:antenna-pco} lists \gls{pco} values for the two fundamental frequencies 
of the \gls{doris} system per antenna type.

\begin{table}[h!]
    \centering
    \begin{tabular}{|c|c|c|}
        \hline
        \textbf{Antenna Type} & \textbf{ALCATEL} & \textbf{STAREC-B} \& \textbf{STAREC-C} \\
        \hline
        $\Delta h$ in \si{\mm} for \SI{2}{\GHz} & \SI{510}{\mm} & \SI{487}{\mm}\\
        $\Delta h$ in \si{\mm} for \SI{400}{\MHz} & \SI{335}{\mm} & \SI{0}{\mm}\\
        \hline
    \end{tabular}
    \caption{\gls{doris} ground antennae \glspl{pco}, \cite{DORISGSM}.}
    \label{table:antenna-pco}
\end{table}

When a linear combination of the observed quantities is used, a respective 
\gls{pco} needs to be computed and applied. E.g., for the case of the 
ionospheric-free linear combination, the respective \gls{pco} is:
\begin{equation}
    \vec{r}_{2GHz,iono-free} = \frac{\vec{r}_{400MHz,2GHz}}{\gamma - 1}
\end{equation}
where $\vec{r}_{2GHz,iono-free}$ is the vector from the \SI{2}{\GHz} phase
center to the iono-free phase center and $\vec{r}_{400MHz,2GHz}$ is
the vector from the \SI{400}{MHz} to the \SI{2}{\GHz} phase center. \autoref{table:antenna-pco} 
lists the \gls{pco} per antenna type.

The geometry of the \gls{doris} ALCATEL Antenna is depicted in \autoref{fig:alcatel-antenna}, 
with the reference points of interest.
\begin{figure}
  \centering
  \tikzset{cross/.style={cross out, draw=red, minimum size=2*(#1-\pgflinewidth), inner sep=0pt, outer sep=0pt},
%default radius will be 1pt. 
cross/.default={1pt}}
\begin{tikzpicture}
    \draw (2,-2) -- (2,-7); %% left vertical
    \coordinate (LT) at (2,-2);
    \draw (5,-2) -- (5,-7); %% right vertical
    \coordinate (RT) at (5,-2);
    \coordinate (L2PC) at (3.5, -4.0);
    \coordinate (L1PC) at (3.5, -3.0);
    \coordinate (BRP) at (3.5, -7.0);
    \draw (LT) to[out=60, in=120] (RT); %% dome
    \draw (LT) to (RT); %% top horizontal (under dome)
    \draw (1,-7) -- (6,-7); %% reference plain
    \draw (1, -7) -- (1, -7.5); %% left minor vertical
    \draw (6, -7) -- (6, -7.5); %% right minor vertical
    \draw [line width=0.25mm] (0,-7.5) -- (9,-7.5); %% ground
    \draw [-stealth] (3.5,-7.5) --node[anchor=west]{local vertical} (3.5, -8.5); %% local vertical to ground
    \draw [dotted] (3.5, 0) -- (3.5, -7.5); %% local vertical from dome
    
    \draw (L1PC) circle (2pt) node[anchor=south west]{phase center \SI{2}{\GHz}}; %% 2GHz pc
    \draw (L1PC) node[cross=3pt, rotate=0] {}; %% 2GHz cross
    \draw [dotted] (L1PC) -- (9.5, -3.0); %% horizontal from 2GHz PC
    \path (9.5, -7.0) -- node[](hl1){h\textsubscript{2GHz} = \SI{510}{\mm}} (9.5, -3.0);
    \draw [<-] (9.5, -7.0)--(hl1); \draw [->] (hl1)--(9.5, -3.0);
    
    \draw (L2PC) circle (2pt) node[anchor=south west]{phase center \SI{400}{\MHz}}; %% 400 MHz pc
    \draw (L2PC) node[cross=3pt, rotate=0] {}; %% 400MHz cross
    \draw [dotted] (L2PC) -- (9, -4.0); %% horizontal from 400MHz PC
    \path (8, -7.0) -- node[](hl2){h\textsubscript{400MHz} = \SI{335}{\mm}} (8, -4.0);
    \draw [<-] (8, -7.0)--(hl2); \draw[->] (hl2)--(8, -4.0);
    
    \draw (BRP) circle (2pt) node[anchor=south west]{DORIS reference point}; %% reference point
    \draw (BRP) node[cross=3pt, rotate=0] {}; %% RP cross
    \draw [dotted] (BRP) -- (10, -7.0); %% horizontal from RP
    \draw (10, -7) node[anchor=north west]{DORIS reference plane};
    
    \node[rectangle, align=left] (rptext) at (-1,-3.5) {DORIS reference point \\ \(\begin{bmatrix} x_s y_s z_s \end{bmatrix}^{T} \) beacon \\position in earth reference \\frame};
    \draw [-stealth] (rptext) to ($(BRP)-(3pt,-3pt)$);
    \node[above,font=\large\bfseries] at (current bounding box.north) {ALCATEL DORIS Ground Antenna};
\end{tikzpicture}

  \caption{Geometry of Alcatel DORIS Ground Antenna/Beacon}
  \label{fig:alcatel-antenna}
\end{figure}

STAREC antennae B and C are identical in terms of design and specification, the
difference being about the error budget in phase center position. For STAREC-C,
manufacturing process and error budget have been improved \cite{DORISGSM}. The geometry 
of these antennae is depicted in \autoref{fig:starec-antenna}.

\begin{figure}
  \centering
  \begin{tikzpicture}
    \coordinate (top) at (4,-1);
    \coordinate (tl) at (3.8, -1.3);
    \coordinate (tr) at (4.2, -1.3);
    \coordinate (blt) at (3.8, -3.0);
    \coordinate (brt) at (4.2, -3.0);
    \coordinate (blb) at (3.5, -3.3);
    \coordinate (brb) at (4.5, -3.3);
    \draw (top) to (tl);
    \draw (tl) to (blt);
    \draw (blt) to (blb);
    \draw (top) to (tr);
    \draw (tr) to (brt);
    \draw (brt) to (brb);
    \draw (blb) to (3.5, -9);
    \draw (brb) to (4.5, -9);
    \draw (3.5, -9) to (3.0, -9);
    \draw (3.0, -9) to (3.0, -9.3);
    \draw (4.5, -9) to (5.0, -9);
    \draw (5.0, -9) to (5.0, -9.3);
    \draw (3.0, -9.3) to (5.0, -9.3);

    \draw [dotted] (top) -- (4.0, -9.3); %% local vertical from top
    \draw [-stealth] (4.0, -9.3) -- node[anchor=west]{local vertical} (4.0, -10.3); %% local vertical arrow

    \coordinate (l1pc) at (4, -2.0);
    \draw (l1pc) circle (2pt) node[anchor=south west]{phase center \SI{2}{\GHz}}; %% 2GHz pc
    \draw (l1pc) node[cross=3pt, rotate=0] {}; %% 2GHz cross
    \draw (l1pc) to (10, -2.0);
    
    \coordinate (l2pc) at (4, -6.0);
    \draw [fill=black] (l2pc) circle (2pt) node[anchor=south west]{phase center \SI{400}{\MHz}}; %% 400MHz pc
    \draw (l2pc) node[cross=3pt, rotate=0] {}; %% 400MHz cross
    \draw ($(l2pc)-(1.0,0.0)$) -- node[below]{DORIS reference plane} (10, -6.0);
    \node (prt) at ($(l2pc)-(2.5,0.0)$) {Painted ring}; 
    \draw [-stealth] (prt) to ($(l2pc)-(0.2,0.0)$);

    \path ($(l1pc)+(5.5, 0.0)$) -- node[](h2g){h\textsubscript{2GHz} = \SI{487}{\mm}} ($(l2pc)+(5.5, 0.0)$);
    \draw [<-] ($(l1pc)+(5.5, 0.0)$) -- (h2g); 
    \draw [->] (h2g) -- ($(l2pc)+(5.5, 0.0)$);

    \node[rectangle, align=left] (rptext) at ($(l2pc)-(2.0,2.0)$) {DORIS reference point \\ \(\begin{bmatrix} x_s y_s z_s \end{bmatrix}^{T} \) beacon \\position in earth reference \\frame};
    \draw [-stealth] (rptext) to ($(l2pc)-(3.5pt,4.5pt)$);

    \node[above,font=\large\bfseries] at (current bounding box.north) {STAREC DORIS Ground Antenna};
\end{tikzpicture}

  \caption{Geometry of Alcatel STAREC Ground Antenna/Beacon}
  \label{fig:starec-antenna}
\end{figure}

\subsubsection{Phase Center Variations}\label{sssec:doris-pcv}
According to \cite{Tourain2016}, in order to check the consistency of the theoretical 
characteristics of the STAREC antennae, a measurement campaign was performed by 
the \gls{cnes} at the \gls{catr}. The \gls{catr} is a dedicated facility 
consisting of an anechoic chamber equipped with several specific devices 
allowing significant measurement for satellite characterization.

As a result of the campaign, a phase law was established by averaging the 
estimated phase law values obtained during the \gls{catr} characterization. 
The resulting couple phase center position to phase law correction is provided 
to the \gls{doris} users in \gls{antex} (\cite{ANTEXv14}) format, and 
made available by \gls{ids} (see \autoref{fig:doris-antennae-phase-law}).

A similar approach was eventually followed for the ALCATEL antenna type 
leading to an elevation-dependent \gls{pcv} model (see \autoref{fig:doris-antennae-phase-law}), 
conventionally called \texttt{PCV2.0}. \cite{Stepanek2022b} reports systematic differences in 
the estimated station heights of about \SI{15}{\milli\meter} when this model is 
adopted in the analysis, compared to older \gls{pcv} values. Note that both 
\gls{pcv} models (for ALCATEL and STAREC) do not consider azimuth dependence 
and have an elevation step of \ang{5}.

\begin{figure}
  \centering
  \begin{tikzpicture}
\pgfplotstableread{tikz/phase_law.dat}{\table}
\begin{groupplot}[
  group style={
        group name= phase-law-plots,
        group size=1 by 2,
        xlabels at=edge bottom,
        xticklabels at=edge bottom,
        vertical sep=8pt,
    },
  %title={\texttt{patch type=quadratic spline}},
  xmin=0, xmax=90,
  ymin=-5, ymax=25,
  xtick distance = 5,
  ytick distance = 5,
  grid = both,
  minor tick num = 1,
  major grid style = {lightgray},
  minor grid style = {lightgray!25},
  width = \textwidth,
  height = 0.4\textwidth,
  xlabel = {Zenith angle ($^{\circ}$)},
]
\nextgroupplot[]
\addlegendimage{empty legend};
%\addplot[blue, mark = *] table [x = {Zenith}, y = {Alcatel_1}] {\table};
\addplot[mark=o,mark size=0.6pt] table [x = {Zenith}, y = {Alcatel_1}] {\table};
\addlegendentry{\texttt{Alcatel Antenna}}[15 pt];
\coordinate (top) at (rel axis cs:0,1);% coordinate at top of the first plot

\nextgroupplot[]
\addlegendimage{empty legend};
\addplot[mark = *,mark size=0.6pt] table [x = {Zenith}, y = {Starec_1}] {\table};
\addlegendentry{\texttt{Starec Antenna}}[15 pt];
\coordinate (bot) at (rel axis cs:0,0);% coordinate at bottom of the last plot

\end{groupplot}

\path (top-|current bounding box.west)-- node[anchor=south,rotate=90] 
  {Phase Correction for 2GHz ($mm$)} (bot-|current bounding box.west);

\path (top-|current bounding box.west) --
  node[anchor=south]{\textbf{DORIS Antennae Phase Law}} 
  (top-|current bounding box.east);
%\node (title) at ($(group c1r1.center)!0.5!(group c2r1.center)+(0,2cm)$) 
%  {DORIS Antenae Phase Law};
\end{tikzpicture}

  \caption{DORIS Antennae Phase Law}
  \label{fig:doris-antennae-phase-law}
\end{figure}
