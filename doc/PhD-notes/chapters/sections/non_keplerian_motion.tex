\subsection{Perturbed Motion}
\label{ssec:perturbed-motion}

The Two-Body problem (\autoref{ssec:two-body-problem}) is the basis for most 
trajectory problems, due to (\cite{hintz}):
\begin{enumerate}
  \item the relative two-body problem can be solved analytically,
  \item it is often a good approximation to the real solution (central body 
    gravitational force is dominant compared to perturbing forces),
  \item provides a clear and illustrative picture of the situation, and
  \item it can be used as a reference trajectory for precise orbit 
    determination techniques
\end{enumerate}

In real world, a satellite orbiting the Earth departs from the Two-Body problem 
scenario; neither is the system isolated, acting only on mutual gravitational attraction, 
nor can the gravitational attraction of the Earth be attributed to a point-like mass. 
The force field acting on the satellite in this case, has to be augmented to account for 
forces other than the ones considered in the Two-Body problem. As a result, satellite 
orbits are \emph{perturbed} away from Keplerian orbits, making the actual motion 
follow a complex, ``disturbed'' trajectory. In \gls{pod}, effects other than a
spherical Earth central gravitational attraction are treated as perturbations of 
the hypothetical unperturbed motion of the satellite around the Earth. 

A perturbing force can either be \emph{conservative}\footnote{
A force field $\bm{F}$ is called a conservative force if it meets one of the 
following three demands:
  \begin{enumerate}
    \item the \emph{curl} of $\bm{F}$ is zero:\begin{equation}\vec{\nabla} \times \bm{F} = \bm{0}\end{equation}
    \item there is zero net work done by the force when moving a particle 
      through a loop trajectory:\begin{equation}W \equiv \oint _C \bm{F}d\bm{r}=0 \end{equation}
    \item the force can be written as the gradient of a potential:\begin{equation}\bm{F}=-\vec{\nabla} U\end{equation}
  \end{enumerate}
} (e.g. third body attraction) or \emph{non-conservative} (e.g. atmospheric drag). 
Systems driven by conservative forces have a constant total energy (kinetic and potential), 
whereas in the non-conservative case, energy may be gained or lost mainly through 
heat exchange (friction) or external sources (thrust). Conservative forces are 
derivable from a scalar function
\begin{equation}
  \bm{F} = -\nabla U(\bm{r})
\end{equation}
a fact that will be used later on, when dealing with perturbing forces.

The fundamental problem in perturbation analysis is orbit propagation. Unlike the 
Two-Body case, the most accurate way to analyze perturbations is numerically (\cite{Vallado}).
In general, solution techniques for the perturbation problem fall into two categories 
\begin{description}
  \item[Special perturbation techniques] where we numerically integrate the equations 
  of motion including all necessary perturbing accelerations, and
  \item[General perturbation techniques] where we consider an analytical approximation and 
  integration over some time interval
\end{description}

In this Thesis, we will be approximating the problem of perturbed orbit motion 
using special perturbation techniques. This approach is more recent (compared to 
the general perturbation approach), since it requires significant computational 
power. Today, with the limitations on computing power pretty much raised, a new 
era in analyzing perturbations has emerged. A drawback of this technique, is its 
specificity; new data means new integration, which can add lengthy computing times. 
Additionally, as most numerical methods, it suffers from errors that build up with 
truncation and round-off due to floating point arithmetic. These errors build up 
with the lengthening of integration intervals, and can cause a degradation of the 
solution accuracy. Even so however, they offer significantly better accuracy than 
the analytical approach. A nice overview of both approaches is given in \cite{Vallado}.

In the following, we are going to discuss the \emph{Force Model} acting on an 
Earth orbiting satellite, departing from the Two-Body problem.