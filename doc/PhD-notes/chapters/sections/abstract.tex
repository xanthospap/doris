Geodesy has greatly advanced since the introduction of artificial, Earth orbiting
satellites. A new era has emerged, where satellite-based data dominate the field,
providing results for a wide range of geodesy-related fields.
The use of artificial satellites in geosciences however requires
a comprehensive knowledge of the satellite motion under
the influence of all acting forces as well as the description of the satellite
trajectory in suitable reference frames. Thus exploitation of satellite-based 
observations is inherently coupled with the complex problem of orbit determination, 
a problem that lies in the core of satellite geodesy since its inception.

To-date, \gls{pod} is dominated by three space geodetic techniques, namely \gls{slr}, \gls{gnss} and
\gls{doris}, which additionally, via the corresponding Technique Centers, provide the input
data time series of station positions and \gls{eop} for the realization of the \gls{itrf}.

Despite its prominence and significance in the field of satellite geodesy,
\gls{doris} has failed to allure a dedicated scientific audience comparable in size to
the other techniques. The shortage of dedicated \gls{ids} analysis centers,
is indicative of the limited availability of dedicated software solutions designed to
handle \gls{doris} data, particularly for the purpose of orbit determination.

In the framework of the current Thesis, the scarcity of dedicated \gls{doris}
analysis tools for orbit determination is targeted, with the aim of creating
a high quality, scientific software solution. Specifically, the software 
package is \emph{non-proprietary} and \emph{open-source}, \emph{adaptable} and \emph{extensible} 
to meet the demands of scientific community, implements state-of-the-art \emph{algorithms}
and \emph{methodologies} and is designed using \emph{modern programming patterns} 
and paradigms.

The software is built in a modular fashion. The various components are organized 
in different, independent, moderate-sized libraries, targeting well defined problems.
Various different implementations are put to the test and robust algorithmic 
approaches are constructed based on criteria of accuracy and efficiency
(computing speed and resources).

Given the complexity of the problem and the inherent limitations of a thesis, both in
terms of time and resources, the objective of the current study is not to attain the
highest possible accuracy achievable by the \gls{ids} Analysis Centers. Instead, the
focus is on developing a brand-new toolset from scratch, which can serve as a
foundational component towards achieving that goal. The envisioned toolset, with
some additional fine refinements, has the potential to form the backbone of a
state-of-the-art, \gls{doris} \gls{pod} analysis pipeline.

Using the toolset developed, an orbit determination analysis scheme was designed and
tested using the \gls{jason}-3 satellite mission, using a high quality reference orbit
for validation. Position differences range within a few meters for one day, while
velocity discrepancies are in the order of a few millimeters per second. 

These results show that the software package developed can serve as a building 
block for a high quality \gls{doris} analysis software solution. Conclusions and 
recommendations for further enhancements and refinements are supplied to eventually 
reach the goal set.
