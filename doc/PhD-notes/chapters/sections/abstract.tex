Satellite Geodesy comprises the observational and computational techniques which
allow the solution of geodetic problems by the use of precise measurements to,
from, or between artificial, mostly near-Earth, satellites. The use of artificial 
satellites in geodesy, has some prerequisites though; these are basically a 
comprehensive knowledge of the satellite motion under the influence of all 
acting forces as well as the description of the satellite trajectory and
ground stations in suitable reference frames, both spatial and temporal. Thus, in
the core of satellite geodesy lies the problem of orbit determination.

\gls{pod} analysis is a very challenging task, requiring multi-scientific expertise,
coupled with efficient engineering. Given the accuracy requirements today,
the ever growing number of scientific satellites missions, and the widening of the
application range, it constitutes a dynamic, active and ever transforming field of 
growing interest.

To-date, \gls{pod} is dominated by three space geodetic techniques, namely 
\gls{slr}, \gls{gnss} and \gls{doris}, which additionally, via the corresponding 
Technique Centers, provide the input data time series of station positions and 
\gls{eop} for the realization of the \gls{itrf}. Along with \gls{vlbi}, these 
techniques constitute the fundamental pillars of modern satellite ans space geodesy.

Since its inception in the late 1980s, \gls{doris} has played a crucial role in 
expanding geodetic knowledge and enhancing our understanding of the Earth's 
dynamics and is currently considered of crucial importance for satellite geodesy. 
However, despite its importance, the scientific audience for the technique is 
disproportionately limited, as evidenced by the small number of dedicated Analysis 
Centers compared to other space geodetic techniques. This
lack of dedicated scientific software tools for a prominent space geodetic technique
is a significant challenge that needs to be addressed. The absence of specialized,
scientific software tools for a significant space geodetic method like \gls{doris} 
served as the primary objective and motivation for this thesis.

In the framework of the current Thesis, the scarcity of dedicated \gls{doris} 
analysis tools for orbit determination is targeted, with the aim of creating a 
high quality, scientific software solution, hoping to spur further interest and 
progress in this fundamental area of satellite geodesy. This software package
will hopefully serve as a fundamental building block for a dedicated analysis 
tool to process \gls{doris} observations at an \gls{ids} level.
