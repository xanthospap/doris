\section{Introduction}\label{sec:pod-introduction}

Orbit determination of artificial satellites began in essence in the 1960s. However, 
it was not until the 1980s that great advancements were made, facilitated by
a better modeling of the Earth's gravity field and its variations (e.g. tides) and
the large increase in computing capabilities. This enabled orbit determination 
accuracy to increase to the tens-of-centimeter level. This improvement was
motivated and further pushed by the ever-increasing demands of scientists in the 
oceanographic and geodetic communities, in search for centimeter-level accuracy 
in global ocean topography obtained from altimetric satellites (\cite{Tapley2004}).
To-date, \gls{pod} can yield results in the  few-centimeter level.

Orbit determination can be viewed as a special case of a parameter estimation problem, 
where the parameters characterizing the orbit of an Earth orbiting satellite have to be 
determined from observations. Observations are themselfes values of nonlinear functions 
(the so-called \emph{observed functions}, \cite{BeutlerVII}), thus for the solution 
of the problem, initial approximate parameter values are needed (a problem often 
called \emph{initial orbit determination}). Such approximate starting values can be 
obtained in a number of ways (see e.g. \cite{Vallado2001}), including published 
values of preliminary solutions, and will not be considered here. Hence, the \gls{pod}
process is in essence an \emph{orbit improvement} problem.

Apart from the orbital elements or equivalently the state vector, a number of
different parameters have to be considered in a \gls{pod} problem, including
\begin{itemize}
    \item dynamical parameters characterizing the force model, necessary to describe 
        the orbital motion of the satellites,
    \item coordinates of the observing sites in an \gls{ecef} system,
    \item motion of the observing sites (e.g. crustal deformation),
    \item Earth rotation and Earth orientation parameters,
    \item atmosphere parameters defining the tropospheric refraction correction,
    \item parameters defining the ionospheric refraction correction,
    \item technique-specific, hardware and (possibly site-specific) instrumentation
        bias parameters
\end{itemize}
All of these parameters have to be considered and dealt-with (either estimated or 
introduced) in a \gls{pod} analysis scheme. In this Thesis, a ``restricted'' problem 
is addressed, often labeled \emph{pure orbit determination}, where coordinates of 
reference observation sites are assumed to known (via their \gls{ecef} position vector).

It is not always possible to describe the entire time period covered by observations 
by one set of initial conditions and dynamic parameters (\cite{BeutlerVII}). Thus,
the period can be split into consecutive \emph{orbital arcs}; an orbital arc is 
thought of as contiguous, limited part of the satellite's trajectory, described by 
exactly one initial state vector and the dynamical parameters. Using short arcs, 
modeling deficiencies can be mitigated (absorbed by the initial state vectors),
but rapidly increases the number of arc-specific parameters, a fact that could lead 
to a considerably weakened solution (\cite{BeutlerVII}). To overcome this problem, 
an alternate method can be used, in which stochastic accelerations are introduced 
(added to the parameter list), with known mean values and variance-covariance matrices.
Not only observation noise but also system noise has to be introduced, and each 
\emph{deterministic} parameter is replaced by a stochastic process (see e.g. \cite{Jaggi2005b} 
and \cite{Jaggi2005a} for a least-squares approach).
