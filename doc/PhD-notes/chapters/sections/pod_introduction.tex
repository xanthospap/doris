\section{Introduction}\label{sec:pod-introduction}

Orbit determination of artificial satellites began in essence in the 1960s. However, 
it wasn't untill the 1980s that great advancements were made, facilitated by 
a better modelling of the Earth's gravity field and its variations (e.g. tides) and 
the large increase in computing capabilities. This enabled orbit determination 
accuracies to increase to the tens-of-centimeter level. This improvement was 
motivated and further pushed by the ever-increasing demands of scientists in the 
oceanographic and geodetic communities, in search for centimeter-level accuracy 
in global ocean topography obtained from altimetric satellites (\cite{Tapley2004}).
To-date, \gls{pod} can yield results in the  few-centimeter level.

Orbit determination can be viewed as a special case of a parameter estimation problem, 
where the parameters characterizing the orbit of an Earth orbiting satellite have to be 
determined from observations. Observations are themselfes values of nonlinear functions 
(the so-called \emph{observed functions}, \cite{BeutlerVII}), thus for the solution 
of the problem, initial approximate parameter values are needed (a problem often 
called \emph{initial orbit determination}). Such approximate starting values can be 
obtained in a number of ways (see e.g. \cite{Vallado2001}), including published 
values of preliminery solutions, and will not be considered here. Hence, the \gls{pod} 
process is in essence an \emph{orbit improvement} problem.

Apart from the orbital elements or equivelently the state vector, a number of 
different parameters have to be considered in a \gls{pod} problem, including
\begin{itemize}
    \item dynamical parameters characterizing the force model, necessary to describe 
        the orbital motion of the satellites,
    \item coordinates of the observing sites in an \gls{ecef} system,
    \item motion of the observing sites (e.g. crustal deformation),
    \item Earth rotation and Earth orientation parameters,
    \item atmosphere parameters defining the tropospheric refraction correction,
    \item parameters defining the ionospheric refraction correction,
    \item technique-specific, harware and (possibly site-specific) instrumentation 
        bias parameters
\end{itemize}
All of these parameters have to be considered and dealt-with (either estimated or 
introduced) in a \gls{pod} analysis scheme. 

The orbit parameters must uniquely specify one particular solution of the
equations of motion


The orbital arc or simply the arc as a contiguous, limited part of the satellite’s
trajectory plays an important role in satellite geodesy. it is not possible to describe the
entire time period covered by observations by one set of initial conditions
and the appropriate dynamical parameters. In such cases the orbital arc
may be broken up into arcs of lengths defined by the analyst. Each arc is
described by exactly one initial state vector and the dynamical parameters.

It is possible to define “restricted” problems. For permanent sites it is often
allowed to assume the coordinates and the motion of the sites as known
(e.g., for pure orbit determination).

One radical method of curing orbit modelling deficiencies of this kind is to
break up the original arc into shorter arcs. To a great extent the modelling
deficiencies are then absorbed by the initial state vectors of the shorter arcs.
This well known method usually is referred to as the short arc method. One
should be aware of the fact, however, that this simple method multiplies the
number of arc-specific parameters by the number of arcs generated – which
may considerably weaken the solutions.

Another method, very powerful and widely used technique to cope with this
problem is to replace the deterministic differential equation systems describ-
ing orbital motion by stochastic differential equation systems, which contain
on top of all deterministic forces so-called stochastic accelerations, which are
characterized by the (known) mean values (usually the zero vector) and the
associated (known) variance-covariance matrix. Stochastic modelling of this
kind is possible without major problems, provided the classical least-squares
approach is replaced by digital filter methods, in particular Kalman- and
Kalman-Bucy filters. In the framework of this more general theory every de-
terministic parameter estimated may be replaced by a stochastic process.
Not only the measurement noise, but also the system noise (represented by
the stochastic constituent in the differential equations in the case of orbit
parameters) has to be considered.

The precise orbit determination of LEO, such as Starlette, Stella, and AJISAI is more
demanding than the determination of the LAGEOS orbits, because of:
-a larger sensitivity to the Earth’s gravity field and to its temporal variations,
-a large sensitivity to atmospheric drag models and variations of air density in the
upper atmosphere, (diss_ks_front)

Gauss’s theory, Kalman filters, and so forth, ultimately needed more precise data.
Without improvements in observational instruments, many of the techniques and meth-
ods in this chapter would be of little use.

On the other hand, the stochastic approach (involving or con-
taining a random variable, a chance, or a probability) uses information in addition to
measurements to account for the fact that neither the mathematical models nor the mea-
surements are perfectly known but are corrupted by random processes. The stochastic
approach combines information from the dynamics, uncertainty with the force model,
and the measurement errors to obtain the best estimate possible. This combination is the
basis of estimation.

Even our best theories can’t exactly model the atmosphere, the Earth’s gravity field,
or the satellite’s attitude. Process noise, v, is a mathematical model of the errors in the
system dynamics. Bierman (1977:113) notes that in a general sense, process noise also represents effects of un-
modeled parameters, linearization, and even “effects such as leaky attitude controls or solar
wind.” It quantifies our ability to model accurately the actual dynamics
before and after a given epoch. The effect of process-noise is that the dynamics may introduce measurable error into
the solution. One difficulty is that we often assume the process-noise statistics are ran-
dom. Unfortunately, they’re often systematic errors such as we might expect from an
incomplete gravity model. These errors can be highly correlated with time and not well
approximated by white noise. Closely related is the power-density matrix or the second
moment of the process noise, Q, which is a covariance of acceleration errors induced by
the mathematical modeling of the system dynamics.

The procedure typically involves an iterative least-squares estimation
seeking best ts of selected model parameters and model-predicted satellite
position and velocity to the tracking data along an orbit arc. The predicted
state at each epoch is integrated by using the satellite equations of motion re-
quiring accurate force models.

