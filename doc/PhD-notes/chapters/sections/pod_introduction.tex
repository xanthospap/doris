\section{Introduction}\label{sec:pod-introduction}

Orbit determination of artificial satellites began in essence in the 1960s. However, 
it wasn't untill the 1980s that great advancements were made, facilitated by 
a better modelling of the Earth's gravity field and its variations (e.g. tides) and 
the large increase in computing capabilities. This enabled orbit determination 
accuracies to increase to the tens-of-centimeter level. This improvement was 
motivated and further pushed by the ever-increasing demands of scientists in the 
oceanographic and geodetic communities, in search for centimeter-level accuracy 
in global ocean topography obtained from altimetric satellites (\cite{Tapley2004}).
To-date, \gls{pod} can yield results in the  few-centimeter level.

Orbit determination can be viewed as a special case of a parameter estimation problem, 
where the parameters characterizing the orbit of an Earth orbiting satellite have to be 
determined from observations. Observations are themselfes values of nonlinear functions 
(the so-called \emph{observed functions}, \cite{BeutlerVII}), thus for the solution 
of the problem, initial approximate parameter values are needed (a problem often 
called \emph{initial orbit determination}). Such approximate starting values can be 
obtained in a number of ways (see e.g. \cite{Vallado2001}), including published 
values of preliminery solutions, and will not be considered here. Hence, the \gls{pod} 
process is in essence an \emph{orbit improvement} problem.

Apart from the orbital elements or equivelently the state vector, a number of 
different parameters have to be considered in a \gls{pod} problem, including
\begin{itemize}
    \item dynamical parameters characterizing the force model, necessary to describe 
        the orbital motion of the satellites,
    \item coordinates of the observing sites in an \gls{ecef} system,
    \item motion of the observing sites (e.g. crustal deformation),
    \item Earth rotation and Earth orientation parameters,
    \item atmosphere parameters defining the tropospheric refraction correction,
    \item parameters defining the ionospheric refraction correction,
    \item technique-specific, harware and (possibly site-specific) instrumentation 
        bias parameters
\end{itemize}
All of these parameters have to be considered and dealt-with (either estimated or 
introduced) in a \gls{pod} analysis scheme. In this thesis, a ``restricted'' problem 
is addressed, often labeled \emph{pure orbit determination}, where coordinates of 
reference observation sites are assumed to known (via their \gls{ecef} position vector).

It is not always possible to describe the entire time period covered by observations 
by one set of initial conditions and dynamic parametrs (\cite{BeutlerVII}). Thus, 
the period can be split into consecutive \emph{orbital arcs}; an orbital arc is 
thought of as contiguous, limited part of the satellite's trajectory, described by 
exactly one initial state vector and the dynamical parameters. Using short arcs, 
modelling deficiencies can be mitigated (absorbed by the initial state vectors), 
but rapidly increases the number of arc-specific parameters, a fact that could lead 
to a considerably weakened solution (\cite{BeutlerVII}). To overcome this problem, 
an alternate method can be used, in which stochastic accelerations are introduced 
(added to the parameter list), with known mean values and variance-covariance matrices.
Not only observation noise but also system noise has to be introduced, and each 
\emph{deterministic} parameter is replaced by a stochastic process (see e.g. \cite{Jaggi2005b} 
and \cite{Jaggi2005a} for a least-squares approach).

\iffalse
The precise orbit determination of LEO, such as Starlette, Stella, and AJISAI is more
demanding than the determination of the LAGEOS orbits, because of:
-a larger sensitivity to the Earth’s gravity field and to its temporal variations,
-a large sensitivity to atmospheric drag models and variations of air density in the
upper atmosphere, (diss_ks_front)

Gauss’s theory, Kalman filters, and so forth, ultimately needed more precise data.
Without improvements in observational instruments, many of the techniques and meth-
ods in this chapter would be of little use.

On the other hand, the stochastic approach (involving or con-
taining a random variable, a chance, or a probability) uses information in addition to
measurements to account for the fact that neither the mathematical models nor the mea-
surements are perfectly known but are corrupted by random processes. The stochastic
approach combines information from the dynamics, uncertainty with the force model,
and the measurement errors to obtain the best estimate possible. This combination is the
basis of estimation.

Even our best theories can’t exactly model the atmosphere, the Earth’s gravity field,
or the satellite’s attitude. Process noise, v, is a mathematical model of the errors in the
system dynamics. Bierman (1977:113) notes that in a general sense, process noise also represents effects of un-
modeled parameters, linearization, and even “effects such as leaky attitude controls or solar
wind.” It quantifies our ability to model accurately the actual dynamics
before and after a given epoch. The effect of process-noise is that the dynamics may introduce measurable error into
the solution. One difficulty is that we often assume the process-noise statistics are ran-
dom. Unfortunately, they’re often systematic errors such as we might expect from an
incomplete gravity model. These errors can be highly correlated with time and not well
approximated by white noise. Closely related is the power-density matrix or the second
moment of the process noise, Q, which is a covariance of acceleration errors induced by
the mathematical modeling of the system dynamics.

The procedure typically involves an iterative least-squares estimation
seeking best ts of selected model parameters and model-predicted satellite
position and velocity to the tracking data along an orbit arc. The predicted
state at each epoch is integrated by using the satellite equations of motion re-
quiring accurate force models.
\fi
