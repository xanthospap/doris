\section{Multistep Methods}\label{sec:multistep-methods}

% http://www.math.iit.edu/~fass/478578_Chapter_2.pdf

In the Runge-Kutta method (see \autoref{sec:runge-kutta}), each step in the 
integration process is completely independent and once used for computation is 
discarded (for this reason they are often called \emph{single-step} methods). 
To reduce the number of function calls and thus allow for efficiency, \emph{multi-step} 
methods have been introduced. These store values of previous steps, and reuse them 
in the next steps to be taken, so that to generate an approximation for the next 
step, the already computed $x$ and $\dot{x}$ values computed at the previous $k$ 
steps are combined.

In general, starting with the \gls{ode} $\bm{\dot{y}} = \bm{f}(t,\bm{y})$, and 
integrating both sides for the interval $t_i$ to $t_{i+j}$ the following expression 
is obtained
\begin{equation}\label{eq:mont442}
    \bm{y}(t_{i+1}) = \bm{y}(t_i) + \int_{t_i}^{t_{i+1}} \bm{f}(t,\bm{y}(t)) \,dt
\end{equation}
In multistep methods, the integrand is replaced by a polynomial $p(t)$, that interpolates 
a subset of the already available approximate values $\bm{\eta}_j$ of the solutions 
$\bm{y}(t_j)$, such that
\begin{equation}
    \bm{f}_j = \bm{f}(t_j, \bm{\eta}_j)
\end{equation}
Hence, if $\bm{\eta}_{i+1}$ is the approximate solution at the nest step to be taken, 
\begin{equation}
    \bm{\eta}_{i+1} = \bm{\eta}_i + \int_{t_i}^{t_{i+h}} p(t) \,dt
\end{equation}
and the increment function of a multistep method is therefore given by
\begin{equation}
    \bm{\Phi} = \frac{1}{h} \int_{t_i}^{t_{i+h}} p(t) \,dt
\end{equation}
Finally, the solution approximation at $t=t_{i+1}$ can be approximated by
\begin{equation}\label{eq:butcher241a}
    \bm{y}(t_{i+1}) = \bm{y}(t_i) + h \left( b_1 \bm{f}_i + b_2 \bm{f}_{i-1} + \dots + b_k \bm{f}_{i-k+1} \right)
\end{equation}
Methods of this form are known as \emph{Adams-Bashforth} methods. Coefficients for 
the representation \autoref{eq:butcher241a} for up to 4\textsuperscript{th} degree 
are available in e.g. \cite{Butcher2016}.