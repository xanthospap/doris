\section{Multistep Methods}\label{sec:multistep-methods}

% http://www.math.iit.edu/~fass/478578_Chapter_2.pdf

\subsection{Adams Method(s)}\label{ssec:adams-method}

In the Runge-Kutta method (see \autoref{sec:runge-kutta}), each step in the 
integration process is completely independent and once used for computation is 
discarded (for this reason they are often called \emph{single-step} methods). 
To reduce the number of function calls and thus allow for efficiency, \emph{multi-step} 
methods have been introduced. These store values of previous steps, and reuse them 
in the next steps to be taken, so that to generate an approximation for the next 
step, the already computed $x$ and $\dot{x}$ values computed at the previous $k$ 
steps are combined.

In general, starting with the \gls{ode} $\bm{\dot{y}} = \bm{f}(t,\bm{y})$, and 
integrating both sides for the interval $t_i$ to $t_{i+j}$ the following expression 
is obtained
\begin{equation}\label{eq:mont442}
    \bm{y}(t_{i+1}) = \bm{y}(t_i) + \int_{t_i}^{t_{i+1}} \bm{f}(t,\bm{y}(t)) \,dt
\end{equation}
In multistep methods, the integrand is replaced by a polynomial $p(t)$, that interpolates 
a subset of the already available approximate values $\bm{\eta}_j$ of the solutions 
$\bm{y}(t_j)$, such that
\begin{equation}\label{eq:mont443}
    \bm{f}_j = \bm{f}(t_j, \bm{\eta}_j)
\end{equation}
Hence, if $\bm{\eta}_{i+1}$ is the approximate solution at the next step to be taken, 
\begin{equation}
    \bm{\eta}_{i+1} = \bm{\eta}_i + \int_{t_i}^{t_{i+h}} p(t) \,dt
\end{equation}
and the increment function of a multistep method is therefore given by
\begin{equation}\label{eq:mont445}
    \bm{\Phi} = \frac{1}{h} \int_{t_i}^{t_{i+h}} p(t) \,dt
\end{equation}
Finally, the solution approximation at $t=t_{i+1}$ can be approximated by
\begin{equation}\label{eq:butcher241a}
    \bm{y}(t_{i+1}) = \bm{y}(t_i) + h \left( b_1 \bm{f}_i + b_2 \bm{f}_{i-1} + \dots + b_k \bm{f}_{i-k+1} \right)
\end{equation}
Methods of this form are known as \emph{Adams-Bashforth} methods. Coefficients for 
the representation \autoref{eq:butcher241a} for up to 4\textsuperscript{th} degree 
are available in e.g. \cite{Butcher2016}.

Multistep methods are typically derived by using an interpolating polynomial in
either of two ways. The first is to use an interpolating polynomial through 
$t_i, t_{i-1}, \dots , t_{i-k+1}$ for $\bm{f}(t, \bm{\eta})$ and then integrate 
the equation (as discussed above). The second method consists in using an 
interpolating polynomial (again through $t_i, t_{i-1}, \dots , t_{i-k+1}$) to 
approximate $\bm{y}(t)$ and then differentiate it, evaluate at $t_{i+1}$ for an 
implicit method and set it equal to the given slope ($\bm{f}(t, \bm{\eta})$) at 
that point to obtain the difference equation. This gives rise to a family of 
implicit methods called \emph{backward difference formulas}. In the latter case, 
the polynomial is given by the expression (\cite{Montenbruck2000})
\begin{equation}\label{eq:mont451}
    p^{i}_{m}(t) = p^{i}_{m}(t_i + \sigma h) = \sum_{j=0}^{m-1} \left(-1\right)^j 
        \begin{pmatrix}-\sigma \\ j \end{pmatrix} \nabla ^j \bm{f}_i
\end{equation}
where the binomial coefficient is used
\begin{equation}
    \begin{pmatrix}-\sigma \\ j \end{pmatrix} = 
        \begin{cases}
            \frac{(-\sigma)(-\sigma -1)\dots (-\sigma -j -1)}{j!} \text{ , if } j>0 \text{ and } \\
            1  \text{ , if } j=0
        \end{cases}
\end{equation}
The $\nabla ^i$ operator here denotes the backward difference operator, which is 
recursively defined by (see \autoref{tab:backward-differences-ab})
\begin{equation}
    \begin{aligned}
        \nabla ^0 \bm{f}_i &= \bm{f}_i \\
        \nabla ^1 \bm{f}_i &= \bm{f}_i - \bm{f}_{i-1} \\
        \nabla ^n \bm{f}_i &= \nabla ^{n-1}\bm{f}_i - \nabla ^{n-1} \bm{f}_{i-1}
    \end{aligned}
\end{equation}

Using \autoref{eq:mont445} the increment function of the $m$\textsuperscript{th}-order 
Adams-Bashforth methods, can be written as
\begin{equation}\label{eq:mont454}
    \bm{\Phi} = \frac{1}{h} \int_{t_i}^{t_{i+h}} p^{i}_{m}(t) \,dt 
        = \sum_{j=0}^{m-1} \gamma _j \nabla ^j \bm{f}_i
\end{equation}
with 
\begin{equation}\label{eq:mont455}
    \gamma _j = \left( -1 \right)^j \int_{0}^{1} \begin{pmatrix}-\sigma \\ j \end{pmatrix} \,d\sigma
\end{equation}
$\gamma _j$ coefficients can be found in e.g. \cite{Butcher2016} up to 
7\textsuperscript{th} order.

\begin{table}[]
    \centering
    \begin{tabular}{ccccccccccc}
        %\hline
        $\bm{f}_{i-4}$ & \dots &&&&&\\ 
        & $\searrow$ &&&&& \\ 
        $\bm{f}_{i-3}$ & $\rightarrow$ & $\nabla ^1 \bm{f}_{i-3}$ & \dots &&&&\\ 
        & $\searrow$ & & $\searrow$ &&& \\ 
        $\bm{f}_{i-2}$ & $\rightarrow$ & $\nabla ^1 \bm{f}_{i-2}$ & $\rightarrow$ & $\nabla ^2 \bm{f}_{i-2}$ & \dots &&& \\ 
        & $\searrow$ && $\searrow$ && $\searrow$ & \\ 
        $\bm{f}_{i-1}$ & $\rightarrow$ & $\nabla ^1 \bm{f}_{i-1}$ & $\rightarrow$ & $\nabla ^2 \bm{f}_{i-1}$ & $\rightarrow$  & $\nabla ^3 \bm{f}_{i-1}$ & \dots && \\ 
        & $\searrow$ && $\searrow$ && $\searrow$ &&  $\searrow$ \\ 
        $\bm{f}_{i}$   & $\rightarrow$ & $\nabla ^1 \bm{f}_{i}$   & $\rightarrow$ & $\nabla ^2 \bm{f}_{i}$   & $\rightarrow$  & $\nabla ^3 \bm{f}_{i}$   & $\rightarrow$  & $\nabla ^4 \bm{f}_{i}$ & \dots & \\ 
        %$\bm{f}_{i-}$ & $\rightarrow$ & $\nabla ^2 \bm{f}_{i-}$ & $\rightarrow$ & $\nabla ^2 \bm{f}_{i-}$ & $\rightarrow$  & $\nabla ^2 \bm{f}_{i-}$ & $\rightarrow$  & $\nabla ^2 \bm{f}_{i-}$ & $\rightarrow$  & $\nabla ^2 \bm{f}_{i-}$ & \dots \\
        %\hline
    \end{tabular}
    \caption{Schematic representation of backward differences for polynomial interpolation, \cite{Montenbruck2000}.}
    \label{tab:backward-differences-ab}
  \end{table}

The local truncation error of the Adams-Bashforth method decreases with the
order $m$ and may be estimated by comparing two methods of order $m$ and $m+1$ 
(\cite{Montenbruck2000})
\begin{equation}\label{eq:mont457}
    T_{m} = \norm{\bm{y}(t_i +h) - \eta _{m}} \approx \norm{\eta _{m+1} - \eta _{m}} 
    = h \norm{\gamma _m \nabla ^m \bm{f}_i}
\end{equation}
which can be approximated by
\begin{equation}
    T_{m} \approx h^{m+1} \norm{\gamma _m \bm{f}^{(m)}_{i}} = h^{m+1} \norm{\gamma _m \bm{y}^{(m+1)}_{i}}
\end{equation}
which shows that the order of the Adams-Bashforth method is equal to the number
$m$ of nodes $t_{i-m+1}, \dots , t_i$.

The method of backward differences has two major advantages:
\begin{enumerate}
    \item it allows a straightforward estimation of the local truncation error, and
    \item the order can be changed in-between integration steps
\end{enumerate}

Note that the polynomial $p$ (\autoref{eq:butcher241a}) is defined by $m$
function values, up to and including $\bm{f}_i$ at time $t_i$, but the integration 
\autoref{eq:mont445} is performed over the subsequent interval $t_i , \dots , t_{i+1}$.
Instead of \autoref{eq:butcher241a}, a slightly different polynomial can be chosen
\begin{equation}\label{eq:butcher241b}
    \bm{y}(t_{i+1}) = \bm{y}(t_i) + h \left( b_0 \bm{f}_{i+1} b_1 \bm{f}_i + b_2 \bm{f}_{i-1} + \dots + b_k \bm{f}_{i-k+1} \right)
\end{equation}
(that is, include a term for the $t_{i+1}$), which leads to the \emph{Adams-Moulton} 
method. For order $m$ in this case, the interpolating polynomial is (compare to 
\autoref{eq:mont451})
\begin{equation}\label{eq:mont461}
    p^{i+1}_{m}(t) = p^{i+1}_{m}(t_i + \sigma h) = \sum_{j=0}^{m-1} \left(-1\right)^j 
    \begin{pmatrix}-\sigma +1\\ j \end{pmatrix} \nabla ^j \bm{f}_{i+1}
\end{equation}
which yields the increment function
\begin{equation}\label{eq:mont462}
    \bm{\Phi} = \frac{1}{h} \int_{t_i}^{t_{i+h}} p^{i+1}_{m}(t) \,dt 
    = \sum_{j=0}^{m-1} \bar{\gamma}_j \nabla ^j \bm{f}_{i+1}
\end{equation}
with coefficients
\begin{equation}\label{eq:mont463}
    \bar{\gamma}_j = \left( -1 \right)^j \int_{0}^{1} \begin{pmatrix}-\sigma +1 \\ j \end{pmatrix} \,d\sigma
\end{equation}

The local truncation error in this case, is given by (\cite{Montenbruck2000})
\begin{equation}
    T_{m} \approx h^{m+1} \norm{\hat{\gamma}_m \bm{y}^{(m+1)}_i}
\end{equation}
and since in general $\hat{\gamma}_m < \gamma _m$, it is smaller that the respective 
$T_m$ value for an Adams-Bashforth method of equal size.

\subsection{\gls{pece} Method}\label{ssec:predictor-corrector-method}

Since the increment function \autoref{eq:mont462} depends on 
$\bm{f}_{i+1} \equiv \bm{f}(t_{i+1}, \bm{\eta}_{i+1})$, it is not possible to 
calculate an explicit solution at $t_{i+1}$ from \autoref{eq:butcher241b}, 
making this method an \emph{implicit} method. To surpass this difficulty, a combination 
of the Adams-Bashforth and Adams-Moulton methods can be used, in what is called a 
\gls{pece} scheme (\autoref{fig:pece-adams}). 
At a first step, an Adams-Bashforth of order $m$ is used to compute an approximate 
solution $\bm{\eta}^{(p)}_{t_{i+1}}$, using the already computed values 
$\bm{f}_i, \bm{f}_{i-1}, \dots , \bm{f}_{i-m+1}$. This \emph{predicted} value 
($\bm{\eta}^{(p)}_{t_{i+1}}$), can then be used to obtain a \emph{predicted} 
function value at $t_{i=1}$, $\bm{f}^{(p)}_{i+1}(t_{i+1},\bm{\eta}^{(p)}_{t_{i+1}})$ 
and proceed to the \emph{corrector} step, where an updated, improved solution 
$\bm{\eta}_{t_{i+1}}$ is computed, via an Adams-Moulton method of order $m$ or 
$m=1$. The ``final'' function value at $t_{i+1}$, 
$\bm{f}^{(p)}_{i+1}(t_{i+1},\bm{\eta}^{(p)}_{t_{i+1}})$ can be evaluated for the 
next integration step.

\begin{figure}
  \centering
  \begin{tikzpicture}[
    every node/.style = {
      draw=black, 
      rounded corners, 
      fill=gray!20,
      minimum width=2cm,
      minimum height=0.5cm,
      align=center},
      every path/.style = {draw, -latex}
    ]

    \node[fill=gray!5] (step1) [] {
        \emph{Predictor} compute initial estimate \\
        of the solution at $t_{i+1}$ using the\\
         (explicit) Adams-Bashforth formula (see \autoref{eq:butcher241a})\\
        $\bm{\eta}^{(p)}_{i+1} = \bm{\eta}_{i} + h \cdot \Phi _{AB}$
      };
    
    \node[fill=gray!5] (step2) [below =of step1] {
        Use the result obtained in the Predictor\\
        step to compute the function value at $t_{i+1}$ (see \autoref{eq:mont443}),\\
        $\bm{f}^{(p)}_{i+1} = \bm{f}(t_{i+1}, \bm{\eta}^{(p)}_{i+1})$
    };
    
    \node[fill=gray!8] (step3) [below =of step2] {
        \emph{Corrector} compute an improved approximate\\
        value $\bm{\eta}_{i+1}$ using the Adams-Moulton method (see \autoref{eq:butcher241b}),\\
        $\bm{\eta}_{i+1} = \bm{\eta}_{i} + h \cdot \Phi _{AM}$
    };

    \node[fill=gray!8] (step4) [below =of step3] {
        Evaluate the updated function value\\ 
        $\bm{f}_{i+1} = \bm{f}(t_{i+1}, \bm{\eta}_{i+1})$\\
        and use for nest integration step
    };

    \draw (step1) -- (step2);
    \draw (step2) -- (step3);
    \draw (step3) -- (step4);

\end{tikzpicture}

  \caption{Schematic representation of the \emph{Predictor-Corrector} algorithm using the Adams-Bashforth and Adams-Moulton methods.}
  \label{fig:pece-adams}
\end{figure}

Even though the \gls{pece} algorithms are complicated and difficult to implement, 
they offer increased stability, especially at large stepsizes (\cite{Montenbruck2000}).
Low-order methods are generally more stable even for large stepsizes.
