\section{Runge-Kutta Methods}\label{sec:runge-kutta}

In the \emph{Runge-Kutta} methods, a weighted average of the slopes ($\bm{f}$) of 
the solution computed at nearby points is used to determine the solution at 
$t = t_{n+1}$ from that at $t = t_n$. Typically, Runge-Kutta methods are further 
divided in \emph{explicit} and \emph{implicit} methods; the latter are more complicated 
but allow for higher order and improved stability (see \cite{Griffiths2010}).

The general $s$-stage Runge-Kutta methods, may be written in the form
\begin{equation}\label{eq:grif95}
    y_{n+1} = y_n + h \sum_{i=1}^{s} b_i k_i 
\end{equation}
where the $k_i$ terms can be computed from the function $f$
\begin{equation}\label{eq:grif96}
    k_i = f \left( t_n + c_i h , y_n + h \sum_{j=1}^{s} a_{i,j} k_j \right) \text{ with } i = 1,2,\dots s
\end{equation}
Typically, we impose the condition
\begin{equation}
    c_i = \sum_{j=1}^{s} a_{i,j}
\end{equation}
Thus, given a value of $s$, the method depends on $s^2 + s$ parameters $a_{i,j}$ and $b_j$.
These can be conveniently displayed in a tableau known as the \emph{Butcher array} 
(see e.g. \cite{Butcher2016}).

\noindent\begin{minipage}[t]{0.5\textwidth}%
    \centering
    \label{table:butcher-array-implicit}
    \begin{tabular}{c|cccc}
        $c_1$  & $a_{1,1}$ & $a_{1,2}$ & \dots & $a_{1,s}$ \\
        $c_2$  & $a_{2,1}$ & $a_{2,2}$ & \dots & $a_{2,s}$ \\
        \vdots & \vdots    & \vdots    &       & \vdots \\ 
        $c_s$  & $a_{s,1}$ & $a_{s,2}$ & \dots & $a_{s,s}$ \\
        \hline
               & $b_1$     & $b_2$     & \dots & $b_s$ \\
    \end{tabular}
    \captionsetup{width=0.8\linewidth}
    \captionof{table}{The Butcher array for a full (implicit) RK method}
\end{minipage}%
\begin{minipage}[t]{0.5\textwidth}%
    \centering
    \label{table:butcher-array-explicit}
    \begin{tabular}{c|ccccc}
        $0$    & $0$       & $0$       & \dots  & $0$         & $0$ \\
        $c_2$  & $a_{2,1}$ & $0$       & \dots  & $0$         & $0$ \\
        $c_3$  & $a_{3,1}$ & $a_{3,2}$ & \dots  & $0$         & $0$ \\
        \vdots & \vdots    & \vdots    & $\ddots$ & $0$         & $0$ \\ 
        $c_s$  & $a_{s,1}$ & $a_{s,2}$ & \dots  & $a_{s,s-1}$ & $0$ \\
        \hline
               & $b_1$     & $b_2$     & \dots & $b_{s-1}$    & $b_s$ \\
    \end{tabular}
    \captionsetup{width=0.8\linewidth}
    \captionof{table}{The Butcher array for an explicit RK method. Zeros are often omitted.}
\end{minipage}%

\autoref{eq:grif96} constitutes a nonlinear equation system of size $s$, that can 
be used to determine $k_i$; once found, they can be substituted into \autoref{eq:grif95} 
to determine $y_{n+1}$. Thus, a general Runge-Kutta method is implicit (\cite{Griffiths2010}).
The form of the Butcher array for an implicit Runge-Kutta method is shown in 
\autoref{table:butcher-array-implicit}. Although early studies were devoted entirely 
to explicit Runge-Kutta methods, interest has now moved to include implicit methods, 
which have become recognized as appropriate for the solution of stiff differential 
equations (\cite{Butcher2016}).

If the coefficients $a_{i,j}$ can be placed in a lower triangular matrix, i.e. 
$a_{i,j} = 0$ for all $j \ge i$, the $k_i$ terms can be computed directly (from 
\autoref{eq:grif96}) without the need to solve any nonlinear equations. These are 
the methods most often used, and are called explicit. In this case, the general form 
of the associated Butcher array is depicted in \autoref{table:butcher-array-explicit}.

The \emph{Local Truncation Error} of a Runge-Kutta method $T_{n+1}$, is defined 
to be the difference between the exact and the numerical solution at $t=t_{n+1}$
(\cite{Griffiths2010})
\begin{equation}
    T_{n+1} = x(t_{n+1}) - x_{n+1}
\end{equation}
with $y_n = y(t_n)$. If $T_{n+1} = \mathcal{O}(h^{p+1})(p>0)$ the method is 
said to be of \emph{order} $p$. In practice, two related Runge-Kutta methods are 
used, one of order $p$ and another of order $p+1$, to approximate the value of 
the local truncation error via $T_{n+1} = y^{p+1}_{n+1} - y^{p}_{n+1}$, where the 
$T_{n+1}$ estimate is for the lowest order method $p$. To perform the calculation 
in an efficient way, the values of $k$s for the lowest degree method, are chosen 
so that they are a subset of the higher degree method coefficients. Such methods 
of neighboring orders are often called \emph{embedded} Runge-Kutta methods.

While for $s<4$ there are always Runge-Kutta methods where $s=q$, this is not the 
case for methods of stages higher than four. The number of stages necessary for 
a given order is known up to order 8, but there are no precise results for higher 
orders (\cite{Griffiths2010}).

\subsection{Adaptive Step Size}\label{ssec:adaptive-step-size}

The step size $h$ is a crucial parameter in the integration methods; it dictates 
the number of steps required to integrate over a given interval, and the accuracy 
of the results obtained. Small step sizes (in general) improve accuracy but comes 
with an efficiency cost. To that end, \emph{step size control} can be used, to 
compute a value $h_n$ for step $n$, to obtain the same accuracy with fewer steps 
or better accuracy with the same number of steps. In \emph{adaptive} step size control,
the step size in each step is adapted to local conditions, so as to take short steps 
when the solution varies rapidly and longer steps when there is relatively little 
activity. Obviously, computing such variable step sizes, should be automatic and 
inexpensive. A thorough discussion on implementing sophisticated adaptive step size 
control, fit for computer programs, is given in the classic text by \cite{Shampine1975}.
