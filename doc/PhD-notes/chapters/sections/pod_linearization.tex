\section{Linearization}\label{sec:pod-linearization}

In a \gls{pod} problem, both the dynamics and the measurements involve significant 
nonlinear relationships. For both the trajectory and the observation models, a large 
number of partial derivatives have to be computed for a rigorous linearization. These, 
according to \cite{Montenbruck2000} can be divided into four different categories:
\begin{description}
  \item[The State Transition Matrix] $\bm{\Phi}(t,t_0)$, which maps deviations in the 
    state vector from an epoch $t_0$ to a later time $t$, and is given by
    \begin{equation}\label{eq:mont71}
      \bm{\Phi} (t,t_0) = \left(
          \frac{\partial \bm{y}(t)}{\partial \bm{y}(t_0)} 
        \right) _{(6 \times 6)}
    \end{equation}

  \item[The Sensitivity Matrix] $\bm{S}(t)$, which describes the dependence of the 
    orbit on the dynamical parameters $p_i$ with $i=1,2,\dots ,n_p$ and is formed 
    by the respective partial derivatives of the state with respect to the force 
    model, 
    \begin{equation}\label{eq:mont72}
      \bm{S} (t) = \left(
          \frac{\partial \bm{y}(t)}{\partial \bm{p}} 
        \right) _{(6 \times n_p)}
        = \begin{pmatrix} 
          \frac{\partial x(t)}{\partial p_1} & \frac{\partial x(t)}{\partial p_2} & \dots & \frac{\partial x(t)}{\partial p_n} \\
          \frac{\partial y(t)}{\partial p_1} & \frac{\partial y(t)}{\partial p_2} & \dots & \frac{\partial y(t)}{\partial p_n} \\
          \frac{\partial z(t)}{\partial p_1} & \frac{\partial z(t)}{\partial p_2} & \dots & \frac{\partial z(t)}{\partial p_n} \\
          \frac{\partial \dot{x}(t)}{\partial p_1} & \frac{\partial \dot{x}(t)}{\partial p_2} & \dots & \frac{\partial \dot{x}(t)}{\partial p_n} \\
          \frac{\partial \dot{y}(t)}{\partial p_1} & \frac{\partial \dot{y}(t)}{\partial p_2} & \dots & \frac{\partial \dot{y}(t)}{\partial p_n} \\
          \frac{\partial \dot{z}(t)}{\partial p_1} & \frac{\partial \dot{z}(t)}{\partial p_2} & \dots & \frac{\partial \dot{z}(t)}{\partial p_n}
      \end{pmatrix}
    \end{equation}

  \item[Partials of the measurements with respect to the state vector], which given 
    an observation $z$  at some instant $t$, is given by
    \begin{equation}\label{eq:mont73}
      \left( \frac{\partial z}{\partial \bm{y}(t)} \right) _{(1 \times 6)}
    \end{equation}, and

  \item[Partials with respect to measurement model parameters], which given the observation 
    model parameters $q_i$ for $i=1,2, \dots ,n_q$, given by 
    \begin{equation}\label{eq:mont74}
       \left( \frac{\partial z}{\partial \bm{q}} \right) _{(1 \times n_q)}
    \end{equation}

\end{description}

Note that from \autoref{eq:mont71}, the dependence of an individual observation $z$ 
on the initial state $\bm{y}(t_0)$ is
\begin{equation}
  \frac{\partial z}{\partial \bm{y}(t_0)} = 
    \frac{\partial z}{\partial \bm{y}(t)}\frac{\partial \bm{y}(t)}{\partial \bm{y}(t_0)} 
    = \frac{\partial z}{\partial \bm{y}(t)} \bm{\Phi} (t,t_0)
\end{equation}
and from \autoref{eq:mont72}
\begin{equation}
  \frac{\partial z}{\partial \bm{p}} = 
    \frac{\partial z}{\partial \bm{y}(t)}\frac{\partial \bm{y}(t)}{\partial \bm{p}}
    = \frac{\partial z}{\partial \bm{y}(t)} \bm{S} (t)
\end{equation}

Analytical computation of the partial derivatives is tedious, cumbersome and an 
error prone procedure. However, it constitutes an essential part of orbit determination 
and have a noticable impact on the achieved performance and convergence speed 
(\cite{Montenbruck2000}).

The orbit determination problem, can in the general case be described by the 
1\textsuperscript{st} order \gls{ode} system for the dynamics accompanied with 
the observation functions
\begin{align}
  \bm{\dot{Y}} &= \bm{F}(t, \bm{Y}) \label{eq:tapley421} \\
  \bm{Z}_i     &=  G(t_i, \bm{Y}_i) + \bm{\epsilon}_i \label{eq:tapley422}
\end{align}
where $\bm{y}$ is the state vector and $\bm{z}_i$ is a $p-$dimensional set of 
observations, for $i=1,2,\dots ,l$. If a sufficiently precise reference (initial) 
trajectory $\bm{y}^{*}$ is available, then the actual, ``true'' trajector $\bm{y}$ 
can be expanded in a Taylor series about this reference trajectory at each point 
in time. Truncating higher order terms, the deviation in state from the 
reference trajectory can be described by a set of linear differential equations 
with time-dependent coefficients. The same procedure can be used in \autoref{eq:tapley422} 
to obtain a linear relation between the observation deviation and the state deviation.
In this way, the original, non-linear orbit determination problem is transformed 
to a linear proble, in which the deviation from some reference solution must be 
determined.
If $\bm{y}$ denotes the state deviation vector and $\bm{z}$ the observation deviation 
vector, then
\begin{equation}
  \begin{aligned}
    \bm{\delta y}(t) &= \bm{y}(t) - \bm{y}^{*}(t) \\
    \bm{\delta z}(t) &= \bm{z}(t) - \bm{z}^{*}(t)
  \end{aligned}
\end{equation}
so that
\begin{equation}\label{eq:tapley424}
  \dot{\bm{\delta y}}(t) = \dot{\bm{y}}(t) - \dot{\bm{y}}^{*}(t)
\end{equation}
Expanding \autoref{eq:tapley421} and \autoref{eq:tapley422} in a Taylor's series 
about the reference trajectory, leads to (\cite{Tapley2004})
\begin{equation}\label{eq:tapley425a}
  \begin{aligned}
  \bm{\dot{y}}(t) &= \bm{f}(t, \bm{y}) \\
                  &= \bm{f}(t, \bm{y}^{*}) + 
                  \frac{\partial \bm{f}(t, \bm{y})}{\partial \bm{y}^{*}(t)} 
                  \left( \bm{y}(t) - \bm{y}^{*}(t) \right) 
                  + \mathcal{O}\left( \bm{y}(t) - \bm{y}^{*}(t) \right)
  \end{aligned}
\end{equation}
and
\begin{equation}\label{eq:tapley425b}
  \begin{aligned}
  \bm{z}_i &= \bm{g}(t_i, \bm{y}_i) + \epsilon _i \\ 
           &= \bm{g}(t, \bm{y}^{*}_{i}, t_i) 
           + \frac{\partial \bm{g}}{\partial \bm{y}^{*}}\at{i} 
           \left( \bm{y}(t_i) - \bm{y}^{*}(t_i) \right) 
           + \mathcal{O}\left( \bm{y}(t_i) - \bm{y}^{*}(t_i) \right) 
           + \epsilon _i
  \end{aligned}
\end{equation}
where the partials are evaluated on the reference solution, $\bm{y}^{*}(t)$ can be 
obtained by integrating \autoref{eq:tapley421} and $\mathcal{O}$ indicate terms 
containing products of the difference $\bm{y}(t) - \bm{y}^{*}(t)$ higher than the 
first term. Given that these terms are sufficiently small to be neglecting and using 
$ \bm{\dot{y}}^{*} = \bm{f}(t, \bm{y}^{*})$, \autoref{eq:tapley425a} and 
\autoref{eq:tapley425b} can be written as
\begin{align}
  \dot{\bm{\delta y}}(t) &= A(t) \delta \bm{y}(t) \label{eq:tapley426a}\\
  \bm{\delta z}_i &= H_i \delta \bm{y}_i + \epsilon _i \label{eq:tapley426b}
\end{align}
where 
\begin{equation}\label{eq:tapley426c}
  A(t) = \frac{\partial f (t)}{\partial \bm{y}(t)}\at{\bm{y}^{*}} 
  \text{ and }
  H_i = \frac{\partial g}{\partial \bm{y}}\at{\bm{y}^{*}_i}
\end{equation}
Hence the original non-linear problem describe by the equations \autoref{eq:tapley421} 
and \autoref{eq:tapley422} is replaced by the linear \autoref{eq:tapley426a} and 
\autoref{eq:tapley426b}.
