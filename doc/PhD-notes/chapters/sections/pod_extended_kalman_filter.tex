\section{The Extended Kalman Filter}\label{sec:pod-extended-kalman-filter}

To address the problem of growing the errors due to higher order terms that are 
ignored in the sequential Kalman filter (see \autoref{ssec:pod-kalman-filter-shortcomings}), 
an extended form of the algorithm can be used, labeled the \emph{Extended Kalman Filter}.
The advantage of this approach is that convergence (to the best estimate) is 
accelerated because of the reduced linearization errors. The major disadvantage 
of the extended sequential algorithm is that the differential equations for the 
reference trajectory must be reinitialized after each observation is processed.
The concept of this filter involves the notion that the true state is sufficiently 
close to the estimated state (\cite{Markley2019}).
A flowchart of the algorithm is depicted in \autoref{fig:efk-pod}.

The primary difference between the classic formulation and the extended algorithm
is that the reference trajectory for the extended Kalman filter is updated 
after each observation to reflect the best estimate of the true trajectory. E.g., 
after processing the $k$\textsuperscript{th} observation, the computed best estimate 
is used to provide a new initial condition for the reference orbit,
\begin{equation}\label{eq:tapley4728}
    \bm{y}^{*}_{k, new} = \hat{\bm{y}}_k = \bm{y}^{*}_k + \hat{\delta \bm{y}}_k
\end{equation}

Note that using $\hat{\bm{y}}_k$ as the reference orbit, implies that $\hat{\delta \bm{y}}_k=\bm{0}$ 
and thus $\bar{\delta \bm{y}}_{k+1}=\bm{0}$. The integration for the reference 
trajectory and the state transition matrix is reinitialized at each observation 
epoch, and the equations are integrated forward from $t_k$ to $t_{k+1}$. After 
the time update step, the best estimate can be computed as 
\begin{equation}
    \hat{\delta \bm{y}}_{k+1} = K_{k+1} \bm{y}_{k+1}
\end{equation}
with $K_{k+1}$ and $\bm{y}_{k+1}$ computed based on the updated reference orbit.
The process of incorporating the estimate at each observation point into the 
reference trajectory for propagating to the next observation epoch leads to
the reference trajectory being the prediction of the estimate of the nonlinear state 
(\cite{Tapley2004}), e.g. $\bm{y}^{*}_{t} = \hat{\bm{y}}(t)$.

When implementing the extended Kalman filter for applications demanding high accuracy, 
care must be taken when updating the reference orbit at the beginning of the
processing. Often, the update is omitted for the first few observations, especially 
if these contain significant noise. After a few observations have been processed, 
the estimates of $\hat{\delta \bm{y}}$ will stabilize, and the trajectory update step 
can be added to the process (\cite{Tapley2004}).

\begin{figure}
    \centering
    \begin{tikzpicture}[align=center, scale=1.0, every node/.style={scale=1.0}]

    \node[rectangle,
        draw,
        text centered,
        rounded corners,
        minimum height=2em,
        align=center,
        thick,
        fill=red!5] 
    (init) at (0,0) {
        Initialize at $t_0$, $i = 0$\\
        Set $t_{i-1} \coloneqq t_0$,\\
        $\bm{Y}^{*}_{t-1} \coloneqq \bm{Y}^{*}_{0}$
        $\hat{\bm{y}}_{i-1} \coloneqq \bar{\bm{y}}_0$
        $\bm{P}_{t-1} \coloneqq \bm{P}_{0}$
    };

    \node[rectangle,
        draw,
        text centered,
        rounded corners,
        thick,
        minimum height=2em,
        align=center,
        fill=gray!5] 
    (readobs) at (0,-2) {
        Read next observation $\bm{Z}_i$ at $t_i$, with $\bm{R}_{i}$
    };

    \node[rectangle,
        draw,
        text centered,
        rounded corners,
        minimum height=2em,
        align=center,
        thick,
        fill=gray!5] 
    (integrate) at (0,-5) {
        Integrate reference trajectory and state transition matrix from 
        $t_{i-1}$ to $t_i$\\
        $
        \begin{aligned}
            \dot{\bm{Y}}^{*} &= F(t, \bm{Y}^{*}(t)) \\
            \dot{\Phi (t,t_0)} &= A(t) \Phi(t,t_0)
        \end{aligned}
        $\\
        with initial conditions\\
        $\bm{Y}^{*}(t_{i-1})$ and $\Phi (t_{i-1},t_{i-1}) = \bm{I}$
    };
    
    \node[rectangle,
        draw,
        text centered,
        rounded corners,
        minimum height=2em,
        align=center,
        thick,
        fill=gray!5] 
    (tupd) at (0,-8) {
        Time Update step\\
        $\bar{\bm{y}}_i = \Phi _{t_i,t_{i-1}}  \hat{\bm{y}}_{i-1}$\\
        $\bar{\bm{P}}_i = \Phi _{t_i,t_{i-1}} \bm{P}_{i-1} \Phi ^{T}_{t_i,t_{i-1}}$
    };

    \node[rectangle,
        draw,
        text centered,
        rounded corners,
        minimum height=2em,
        align=center,
        thick,
        fill=gray!5] 
    (obs) at (0,-10.5) {
        Form observation equations and linearize\\
        $\bm{z}_i = \bm{Z}_i - G(\bm{Y}^{*}_i, t_i)$,\\
        $\tilde{H}_i = \frac{\partial G(t_i, \bm{Y}(t))}{\partial \bm{Y}}\at{\bm{y}^{*}}$\\
        Compute gain matrix $\bm{K}_i$
    };

    \node[rectangle,
        draw,
        text centered,
        rounded corners,
        minimum height=2em,
        align=center,
        thick,
        fill=gray!5] 
    (mupd) at (0,-13.5) {
        Measurement Update\\
        $\hat{y}_i = \bar{y}_i + \bm{K}_i \left( \bm{y}_i - \tilde{H} \bar{y}_i \right)$,\\ \\
        $\bm{P}_i = \left( \bm{I} - \bm{K}_i \tilde{H}_i \right) \bar{\bm{P}}_i$,\\
        $\bm{Y}^{*}_i \coloneqq \bm{Y}^{*}_i + \hat{y}_i$
    };

    \node[rectangle,
        draw,
        text centered,
        rounded corners,
        minimum height=2em,
        align=center,
        thick,
        fill=gray!5] 
    (roupd) at (0,-16.5) {
        Reference orbit update\\
        $t_i \coloneqq t_{i-1}$\\
        $\bm{Y}^{*}(t_i) \coloneqq \bm{Y}^{*}(t_{i-1})$\\
    };

    \node[rectangle,
        draw,
        text centered,
        rounded corners,
        minimum height=2em,
        thick,
        fill=red!5] 
    (more) at (0,-18.5) {
        End of observations ?
    };
    
    \node[diamond,
        draw,
        minimum height=0.5em,
        thick,
        fill=red!10] 
    (end) at (5,-18.5) {
        End
    };

    \draw [-stealth, thick, red] (init) -- (readobs);
    \draw [-stealth, thick] (readobs) -- (integrate);
    \draw [-stealth, thick] (integrate) -- (tupd);
    \draw [-stealth, thick] (tupd) -- (obs);
    \draw [-stealth, thick] (obs) -- (mupd);
    \draw [-stealth, thick] (mupd) -- (roupd);
    \draw [-stealth, thick] (roupd) -- (more);
    \draw [thick] (more.west) -| ($(readobs.west)-(3.5,0)$);
    \node[] (no) at ($(more.west)-(3.1,-.3)$) {No};
    \draw [-stealth, thick] ($(readobs.west)-(3.5,0)$) -- (readobs);
    \node[red] (yes) at ($(more.east)-(-1.4,-.3)$) {Yes};
    \draw [-stealth, thick, red] (more) -- (end);
    %\draw [-stealth, thick] ($(upd.west)-(1.5,0)$) |- (integrate.west) node[midway, above]{$i = i + 1$}; 
    %\draw [-stealth, thick, red] (upd) -- (output);

\end{tikzpicture}
    \caption{Flowchart of the Extended Kalman filter algorithm for orbit determination.}
    \label{fig:efk-pod}
\end{figure}
