\subsection{Kinematics of the Earth}\label{ssec:earth-attitude}

Earth attitude is complicated by earth kinematics, including precession and nutation 
of the axes of rotation, the motion of the pole of rotation within the Earth (polar 
motion), and the variability of the rate of rotation of the Earth (resulting in variations 
in the length of day). While the first two phenomena (precession and nutation) can be 
predicted quite accuratelly, the latter two have to be observed. To connect observations 
performed on the Earth's surface in a local coordinate system to a celestial system, 
these effects have to be considered.

Precession, which constists of the \emph{precession of the equator} and 
\emph{precession of the ecliptic}, is the motion of the equator with respect to 
the ecliptic. Nutation is the oscillations in the motion of the Earth's pole due 
to torques from external gravitational forces, limited to motions with periods longer 
than two days (\cite{Gurfil18}). Polar Motion is the motion of the Earth's pole of 
rotation with respect to the Earth's solid body, that is the angular excursion of 
the \gls{cip} from the \gls{itrs} $z$-axis.

Earth kinematics, including all the aforementioned phenomena, plays a crucial role in 
relating the \gls{icrf} and \gls{itrf}, a task needed in \gls{pod} since the equations 
of motions of an Earth orbiting satellite need to be formulated in an inertial frame.
Description of these variations and the related adopted models can be found in detail 
in e.g. \cite{Gurfil18} and \cite{Urban2013}. In this thesis, we will be using the 
nomenclature and resolutions adopted in the framework of the \gls{iau} 2000/2006 resolutions 
(see \cite{iers2010} and references therein).