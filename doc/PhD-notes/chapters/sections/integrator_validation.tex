\subsection{Integrator Validation}\label{sec:integrator-validation}

Checking and validating integrator results is not a straight-forward process, since 
the computation of the complex force field acting on the satellite is required 
(see \autoref{ssec:perturbed-motion}). Some of these forces further depend on 
satellite-specific characteristics (e.g. atmospheric drag, solar radiation 
pressure). Additionally, validation of trajectory extrapolation requires a ``reference'' 
orbit (i.e. reference state results), which may have been computed using a non 
identical set of models, reference frames and algorithms.

To test the integrator designed and implemented for this Thesis, \gls{ids}-published 
(\cite{Willis2016a}) \texttt{sp3} files were acquired and used as ``reference'' orbits. 
These orbits are the accumulated results for the satellite state, tabulated at 
equidistant epochs, of the individual analysis centers contributing to the service. 
Hence, they represent state-of-the-art \gls{pod} results using the \gls{doris} 
satellite system.

The test devised to check the integrator proceed as follows:
\begin{enumerate}
  \item Read a state vector off from the sp3 file for an epoch $t$
  \item Extrapolate the state to a later time $t_i$ (prior to the end of sp3 records)
  \item Read the state off from the sp3 file for the epoch $t_i$ and compare 
    with the results obtained from the integrator. Go to (1) and repeat the 
    process.
\end{enumerate}
The force model used within the integrator, does not contain contributions of the 
solar radiation pressure and atmospheric drag. Ocean tidal loading effects on the 
geopotential, are limited to the 11 major tides. All other contributions, are 
computed as described in \autoref{ssec:perturbed-motion}. Handling of the 
transformation between the \gls{itrf} and \gls{gcrf} is described in 
\autoref{ssec:the-celestial-reference-frame}.

The validation test is performed for a time interval of approximately one week, 
starting on 26/08/2022. The satellite mission picked for the test is Jason-3 
(\cite{Bannoura2011}), since this is the mission used later on, to test the 
integrated \gls{pod} process.

Normally, in a \gls{pod} process using the \gls{doris} system, there is no need to 
extrapolate an orbit further than some seconds, after it has been adjusted by the 
inclusion of observations (see \autoref{ch:pod}). Two distinct measurements are made 
every \SI{10}{\sec} and the ground network is dense enough so that at least one 
ground stations is nearly always visible by any \gls{doris}-equipped satellite. 
For the validation process, extrapolation is tested up to time intervals of 
\SI{15}{\minute} to gain a thorough view on the robustness of the implemented 
algorithm (\autoref{fig:sp3-vs-integration-1min}).

\autoref{fig:sp3-vs-integration-1min} depicts the differences in the state vector 
$\bm{y} = \begin{pmatrix}\bm{r} & \bm{v}\end{pmatrix}^T$ between the reference values 
obtained by the \texttt{sp3} file and the ones computed using the integrator with 
an extrapolation interval of \SI{1}{\minute}. Differences in the position range 
%between \qtyrange{-2}{2}{\milli\metre} while for the velocity components the 
between \numrange{-2}{2} \si{\milli\metre} while for the velocity components the 
%differences are within \qtyrange{-1e-5}{1e-5}{\metre\per\second}. A diurnal 
differences are within \numrange{-1e-5}{1e-5} \si{\metre\per\second}. A diurnal 
signal seems to be present in the latter case, which could be due to unmodeled 
effects of \begin{itemize}
  \item (remaining) ocean tidal constituents, 
  \item solid earth pole tides and ocean pole tides,
  \item solar radiation pressure acting on the satellite,
  \item atmospheric drag
\end{itemize}
\begin{figure}
  \centering
  \includegraphics[height=.4\textheight,keepaspectratio]{intrp.ja3.1min}
  \caption{Integration results of satellite state compared to respective 
    \gls{ids}-distributed sp3 file records. Extrapolation is performed for an 
    interval of \SI{1}{\minute}.}
  \label{fig:sp3-vs-integration-1min}
\end{figure}

\autoref{fig:sp3-vs-integration-3min} depicts the differences in the state vector 
when the integration interval is expanded to \SI{3}{\minute}. Differences in 
%position range between \qtyrange{-5}{5}{\milli\metre} while for the velocity 
position range between \numrange{-5}{5} \si{\milli\metre} while for the velocity 
%components the differences are within \qtyrange{-2e-5}{2e-5}{\meter\per\second}. As in the 
components the differences are within \numrange{-2e-5}{2e-5} \si{\meter\per\second}. As in the 
case of \SI{1}{\minute} extrapolation (see \autoref{fig:sp3-vs-integration-1min}), 
velocity differences are dominated by a harmonic, diurnal signal and the same can 
be told for the position discrepancies.
\begin{figure}
  \centering
  \includegraphics[height=.4\textheight,keepaspectratio]{intrp.ja3.3min}
  \caption{Integration results of satellite state compared to respective 
    \gls{ids}-distributed sp3 file records. Extrapolation is performed for an 
    interval of \SI{3}{\minute}.}
  \label{fig:sp3-vs-integration-3min}
\end{figure}

Differences when extrapolating for an interval of \SI{15}{\minute} are depicted 
in \autoref{fig:sp3-vs-integration-15min}. In this case, differences in position 
%range between \qtyrange{-5}{5}{\centi\metre} while for the velocity components 
range between \numrange{-5}{5} \si{\centi\metre} while for the velocity components 
%the range is \qtyrange{-.1}{.1}{\milli\meter\per\second}
the range is \numrange{-.1}{.1} \si{\milli\meter\per\second}
\begin{figure}
  \centering
  \includegraphics[height=.4\textheight,keepaspectratio]{intrp.ja3.15min}
  \caption{Integration results of satellite state compared to respective 
    \gls{ids}-distributed sp3 file records. Extrapolation is performed for an 
    interval of \SI{15}{\minute}.}
  \label{fig:sp3-vs-integration-15min}
\end{figure}

In general, due to the very small magnitude in the discrepancies computed for the 
extrapolated vs the ``reference'' state (see \autoref{fig:sp3-vs-integration-1min}, 
\autoref{fig:sp3-vs-integration-3min} and \autoref{fig:sp3-vs-integration-15min}), 
the design of the algorithm and its implementation seem to be robust, while the 
discrepancies are attributed to either mismodelled, or unmodelled effects.
