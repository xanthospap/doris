\section{Introduction}\label{sec:integration-introduction}

The equations of motion governing the orbital path followed by an Earth orbiting 
satellite, constitute a system of \glspl{ode}. To ``propagate'' the orbit, we need 
to solve this system, using a reference trajectory at a given instant $t=t_0$ as 
\emph{initial conditions}. The high accuracy that is nowadays required for \gls{pod} 
applications, can only be achieved via numerical methods (\cite{Montenbruck2000}).
Although there are many types of numerical integrators developed for solving \gls{ode} 
problems, only a few of them are suitable for \gls{pod} due to their limited 
numerical precision and stability (\cite{Yufeng2020}).

A variety of methods have been successfully applied to the problem of orbit 
propagation, each with its own drawbacks, limitations and advantages (see e.g.
\cite{Somodi2011} and \cite{Atallah2020}.)
Thus, in general, it is not possible to simply select one method as best 
suited for the prediction of satellite motion.

In this Thesis, two different techniques of numerical integration are investigated, 
namely:
\begin{description}
    \item[Runge—Kutta] methods, that have the advantage of being well established 
      and easy to implement, and
    \item[multistep] methods, that a high efficiency and accuracy at a cost of 
      increased implementation complexity
\end{description}
Both of these techniques, are further subdivided into more specialized algorithms, 
depending on problem constraints, choice of parameters and algorithmic approach. 
Typically, application needs indicate the appropriate methodology.

A comparisson of different integration techniques considering a simple, Keplerian
orbit is discussed in \cite{Montenbruck1992}, where the highly flexible, variable
order and step size multistep methods are shown to be ideal candidates for use 
in general satellite orbit prediction and determination systems. A more realistic 
scenario is studied in \cite{Somodi2011}, where again the efficiency and robustness 
of the \emph{Adams} methods is shown. 
\cite{Papanikolaou2016}, compare integration techniques (including direct 
2\textsuperscript{nd}-order integrators such as \emph{Gauss-Jackson} methods) 
in the frame of the \gls{grace} and \gls{goce} satellite missions; the efficiency 
of the Adams \gls{pece} algorithm compared to \emph{Runge-Kutta} methods is evident, 
even though only constant-step implementations are considered. A comprehensive 
study of the precision and efficiency of different \emph{Runge-Kutta} methods 
(among others) is presented in \cite{Atallah2020}.

It should be noted that the equations of motions considered, constitute a system 
of \glspl{ode} of seconds degree. Even though methods for direct integration of 
such equations exist, they will not be considered here, since such methods assume 
that forces acting on a satellite do not depend on its velocity (an assumption not 
fulfilled in the \gls{pod} case considered here, see e.g. \autoref{sssec:atmospheric-drag}).
For a more detailed and general discussion on numerical methods for \glspl{ode}, 
see \cite{Hairer2009I} and \cite{Hairer2010II}.

A first order system of \glspl{ode} can always be obtained from a respective second 
order, by combining the position $\bm{r}$ and velocity $\bm{\dot{r}}$ vectors into 
the 6-dimensional \emph{state vector} $\bm{y}$
\begin{equation}
    \bm{y} = \begin{pmatrix}\bm{r}\\ \bm{\dot{r}} \end{pmatrix}
\end{equation}
with 
\begin{equation}
    \bm{\dot{y}} = \bm{f}(t,\bm{y}) = 
      \begin{pmatrix}\bm{\dot{r}} \\ 
      \bm{a}(t, \bm{r}, \bm{\dot{r}}) \end{pmatrix}
\end{equation}
so that the original system
\begin{equation}
    \bm{\ddot{r}} = \bm{a}(t, \bm{r}, \bm{\dot{r}})
\end{equation}
can be written in the general form
\begin{equation}\label{eq:genode0}
    \bm{\dot{y}} = \bm{f}(t,\bm{y}), \text{ with } \bm{y}, \bm{\dot{y}}, \bm{f} \in \Re
\end{equation}

\subsection{Goals of Current Chapter}\label{ssec:goals-of-integrator}

Orbit integration plays a major role in \gls{pod} as it can be used to ``connect'' 
the trajectory between observation epochs. Nowadays, the increasingly enhanced 
ground and space-borne geodetic observation techniques have put forward even higher 
requirements on the precision of numerical orbit integration. Efficiency 
constraints and computational resources have to be thoroughly considered in such 
an integration process, as the computational burden it incurs is significant.

In this chapter a quick overview of the most widely used \gls{ode} 
solution techniques, as applied to the orbit determination problem, is presented. 
Focus is placed on \emph{multistep} methods, and more specifically the \emph{Adams} 
\gls{pece} family of algorithms, as they represent state-of-the-art integration 
schemes.

Unfortunately, there is a lack of relevant software or software availability; well 
established, scientific packages performing \gls{pod} are mostly either not freely 
available or not open-source. Hence, the decision was made to design and implement 
an efficient, robust integrator from scratch within the framework of this Thesis. 
The \emph{Adams} \gls{pece} method was adopted as the algorithm of choice, due to  
the advantages it offers both in terms of precision and efficiency. The implemented 
scheme follows a \emph{variable step}, \emph{variable order} approach, based on 
the theoretical groundwork laid by \cite{Shampine1975}.

The implementation details of the software built to address orbit integration, are discussed 
in \autoref{sec:integrator-implementation}. Elaborate mathematical formulations are 
employed to allow for as much efficiency as possible; variable, but not rapidly 
changing step sizes and orders of the integrator are seeked for in every individual 
step, to accommodate robustness and precision and limit computational burden.

The integrator constitutes a big part of the software package built for this Thesis. 
As such, it pertains to the overall philosophy of free and open-source software, 
adopting a modern, generic design and accommodating reusability and adaptability 
(see \autoref{ssec:design-considerations}).
