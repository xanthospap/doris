\section{Introduction}\label{sec:integration-introduction}

The equations of motion governing the orbital path followed by an Earth orbiting 
satellite, constitute a system of \glspl{ode}. To ``propagate'' the orbit, we need 
to solve this system, using a reference trajectory at a given instant $t=t_0$ as 
\emph{initial conditions}. The high accuracy that is nowadays required for \gls{pod} 
applications, can only be achieved via numerical methods (\cite{Montenbruck2000}).

A variety of methods have been successfully applied to the problem of orbit 
propagation, each with its own drawbacks, limitations and advantages. Thus, in 
general, it is not possible to simply select one method as best suited for the 
prediction of satellite motion.

In this Thesis, we investigate two kind of integration techniques, namely:
\begin{description}
    \item[Runge—Kutta] methods, that have the advantage of being well established 
      and easy to implement, and
    \item[multistep] methods, that a high efficiency and accuracy at a cost of 
      increased implementation complexity
\end{description}
Both of these techniques, are further subdivided into more specialized algorithms, 
depending on problem constraints, choice of parameters and algorithmic approach. 
Typically, application needs indicate the appropriate methodology.

It should be noted that the equations of motions considered, constitute a system 
of \glspl{ode} of seconds degree. Even though methods for direct integration of 
such equations exist, they will not be considered here, since such methods assume 
that forces acting on a satellite do not depend on its velocity (an assumption not 
fulfilled in the \gls{pod} case, see e.g. \autoref{sssec:atmospheric-drag}).
For a more detailed and general discussion on numerical methods for \glspl{ode}, 
see \cite{Hairer2009I} and \cite{Hairer2010II}.

A first order system of \glspl{ode} can always be obtained from a respective second 
order, by combining the position $\bm{r}$ and velocity $\bm{\dot{r}}$ vectors into 
the 6-dimensional \emph{state vector} $\bm{y}$
\begin{equation}
    \bm{y} = \begin{pmatrix}\bm{r}\\ \bm{\dot{r}} \end{pmatrix}
\end{equation}
with 
\begin{equation}
    \bm{\dot{y}} = \bm{f}(t,\bm{y}) = \begin{pmatrix}\bm{\dot{r}} \\ \bm{a}(t, \bm{r}, \bm{\dot{r}}) \end{pmatrix}
\end{equation}
so that the original system
\begin{equation}
    \bm{\ddot{r}} = \bm{a}(t, \bm{r}, \bm{\dot{r}})
\end{equation}
can be written in the general form
\begin{equation}
    \bm{\dot{y}} = \bm{f}(t,\bm{y}), \text{ with } \bm{y}, \bm{\dot{y}}, \bm{f} \in \Re
\end{equation}
