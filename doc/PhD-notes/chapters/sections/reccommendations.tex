\section{Recommendations}\label{sec:recommendations}

As discussed before, the software tools presented in this Thesis constitute a 
fundamental infrastructure for a state-of-the-art \gls{doris} analysis software 
package. Incorporation of a few refinements (outlined in \autoref{sec:conclusions}) 
can result to analysis results meeting the highest standards. The basis of further 
development, should evolve around the following considerations:
\begin{itemize}
  \item[] Adherence to a free and open source policy; this policy results in attraction 
    of new users, which will pose new scientific questions, explore limitations and 
    drawbacks, and eventually drive the development forward to meet new, wider and 
    higher demands. This cycle will in turn promote scientific knowledge and push 
    for technological advance of the technique itself.

  \item[] Adherence to a modular design paradigm, favoring genericity and efficiency. 
    Given that the software aims to be used by the scientific community, which by large is 
    familiar with concepts such as programming, these two attributes should be preferred 
    over e.g. user-friendliness.

  \item[] Upgrade the filtering process to a more robust estimator including the 
    introduction of process noise (stochastic properties).

  \item[] Incorporate more space geodetic techniques. As said, most scientific 
    software packages of high caliber, are not limited to one technique. This happens 
    because once the groundwork is laid, the introduction of a new space geodetic 
    technique is a matter of efficiently handling the corresponding observation 
    equation model. Complex earth dynamics, reference and time frame transformations, 
    filtering and adjustment, integration and linear algebra are all already in place. 
    Additionally, the introduction other techniques, allows for validation and 
    can further provide crucial scientific results and insight.

  \item[] Promote the package to meet \gls{ids} standards and contribute to the 
    community. This will require confirming to high precision demands and 
    in parallel further drive updateting and refining, since it will necessitate adoption of 
    the latest developments in geodesy.
\end{itemize}
