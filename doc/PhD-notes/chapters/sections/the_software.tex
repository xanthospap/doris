\section{The Software}\label{sec:the-software}

In the framework of the Thesis, a software package was designed and implemented, to 
perform orbit determination using the \gls{doris} satellite system. The software 
was built from scratch, with minimum (external) dependencies. The purpose of the 
package is to act as the fundamental building block of a state-of-the-art scientific 
software toolset, utilizing \gls{doris} observations for a wide range of geodetic 
studies, including but not limited to \gls{pod}, reference frame maintenance, 
earth dynamics and atmospheric studies. A first, imminent step towards this direction, 
would be for the software to reach \gls{ids} standards, a goal not too far-fetched, 
since the bulk of the work has already been done.

\begin{figure}
  \centering
    %every node/.style = {
  %  draw=black, 
  %  rounded corners, 
  %  fill=gray!20,
  %  minimum width=2cm,
  %  minimum height=0.5cm,
  %  align=center},
  %  every path/.style = {draw, -latex}
  %]

\tikzset{bin/.style=   {black, draw=blue, fill=red!20,   rounded corners=3pt, align=center, minimum height=1.2cm, minimum width=2.5cm}}
\tikzset{level1/.style={black, draw=blue, fill=blue!25,  rounded corners=3pt, align=center, minimum height=1.2cm, minimum width=2.5cm}}
\tikzset{level2/.style={black, draw=blue, fill=blue!20,  rounded corners=3pt, align=center, minimum height=1.2cm, minimum width=2.5cm}}
\tikzset{level3/.style={black, draw=blue, fill=blue!15,  rounded corners=3pt, align=center, minimum height=1.2cm, minimum width=2.5cm}}
\tikzset{level4/.style={black, draw=blue, fill=blue!10, rounded corners=3pt, align=center, minimum height=1.2cm, minimum width=2.5cm}}
\tikzset{external/.style={black, draw=blue, fill=brown!10, rounded corners=3pt, align=center, minimum height=1.2cm, minimum width=2.5cm}}
\tikzset{data/.style=  {black, draw=blue, fill=green!20, rounded corners=3pt, align=center, minimum height=1.2cm, minimum width=3.0cm}}
  
\begin{tikzpicture}
  \draw [red, very thick] (-8,-1.5) rectangle (8,.4);
  \node [] at (-5,-1.7) {\texttt{Executables}};
  \node [bin] (bin) at (0,-.6) {\texttt{bin}};
  
  \draw [blue, very thick] (-8,1.2) rectangle (8,9.7);
  \node [] at (-5,.9) {\texttt{Libraries \& Dependencies}};
  \node [level1] (doris) at (2,2)  {\texttt{doris}};
  \node [external] (yaml) at (-2,2) {\texttt{yaml-cpp}};
  
  \node [external] (cspice) at (6,4.5)  {\texttt{cspice}};
  \node [level2] (sinex) at (2,4.5)  {\texttt{sinex}};
  \node [level2] (iers10) at (-2,4.5) {\texttt{iers2010}};
  \node [level2] (sp3) at (-6,4.5) {\texttt{sp3}};
  
  \node [level3] (geodesy) at (0,6.8)  {\texttt{geodesy}};

  \node [external] (eigen) at (2,9)  {\texttt{eigen}};
  \node [level4] (datetime) at (-2,9) {\texttt{datetime}};
  
  \draw [green, very thick] (-8,-2) rectangle (8,-6);
  \node [data] (din1) at (6,-3)    {\texttt{eop (C04)}};
  \node [data] (din2) at (2,-3)    {\texttt{satellite}\\\texttt{macromodel}};
  \node [data] (din3) at (-2,-3)   {\texttt{space}\\ \texttt{weather}};
  \node [data] (din4) at (-6,-3)   {\texttt{gravity}\\ \texttt{model}};
  \node [data] (din5) at (0,-5)    {\texttt{Configuration File}};
  \node []     (din6) at (-5,-6.5) {\texttt{User Input and Models}};

  \draw[thick,-Latex] (yaml.south) to[bend right] (bin.west);
  \draw[thick,-Latex] (doris.south) to[bend left] (bin.east);
  %\draw[->] (cspice.south) to[bend left] (doris.east);
  \draw[thick,-] (-6,3.2) to (6,3.2);
  \draw[thick,-Latex] (sinex.south) to (doris.north);
  \draw[thick,-Latex] (cspice.south) to (6,3.2);
  \draw[thick,-Latex] (sinex.south) to (2,3.2);
  \draw[thick,-Latex] (iers10.south) to (-2,3.2);
  \draw[thick,-Latex] (sp3.south) to (-6,3.2);
  
  \draw[thick,-] (-6,5.8) to (2,5.8);
  \draw[thick,-Latex] (geodesy.south) to (0,5.8);
  \draw[thick,Latex-] (sinex.north) to (2,5.8);
  \draw[thick,Latex-] (iers10.north) to (-2,5.8);
  \draw[thick,Latex-] (sp3.north) to (-6,5.8);
  
  \draw[thick,-Latex] (datetime.south) to[bend right] (geodesy.west);
  \draw[thick,-Latex] (eigen.south) to[bend left] (geodesy.east);

  \draw[thick,-Latex] (0,-2) -- (bin.south);
  \draw[-Latex] (din1.south) to[bend left] (din5.east);
  \draw[-Latex] (din4.south) to[bend right] (din5.west);
  \draw[-Latex] (din3.south) to[] ($(din5.north) + (-1,0)$);
  \draw[-Latex] (din2.south) to[] ($(din5.north) + (+1,0)$);
  %\path[->] (din1) to[bend left] node[midway,inner sep=2pt,draw=blue, fill=green!20, rounded corners=3pt, minimum height=.5cm, minimum width=.95cm] {$\cdots$} (din5);

\end{tikzpicture}

  \caption{Overview of the software structure, dependencies and hierarchy.}
  \label{fig:software-hierarchy}
\end{figure}

Writing scientific software to meet such high accuracy and efficiency standards 
is a challenging task. This is evident by the limited number of packages delivering 
such robust products. A deep understanding of the underlying scientific notions is 
a prerequisite (including e.g. celestial mechanics, spatial/Temporal reference frames, 
geodesy and estimation theory) coupled with software engineering skills to match the 
high volume and high calibre of work needed (about 100000 lines of source code were 
written for this Thesis, \autoref{table:software-components}).

\begin{table}[h!]
  \centering
  \begin{tabular}{p{0.20\linewidth} | p{0.2\linewidth} | p{0.20\linewidth} | p{0.25\linewidth}}
    \toprule
    \textbf{Library} & \textbf{Language} & \textbf{Lines of Code}\footnote{Counted using \texttt{tokei} (\url{https://github.com/XAMPPRocky/tokei}) excluding blank and comment lines.} & \textbf{Comment}\\
    \midrule
    \multirow{2}{*}{\texttt{doris}}  & \texttt{C++} & 18286 & \multirow{2}{4cm}{Including source code for executables}\\
      & \texttt{Python} & 2068 & \\
    \hline
    \multirow{3}{*}{\texttt{iers2010}}  & \texttt{C++} & 50000 & \multirow{3}{4cm}{Python \& Fortran mainly used for unit testing} \\
      & \texttt{C} & 1801 \\
      & \texttt{Python \& Fortran} &  17120 \\
    \hline
    \texttt{sp3}    & \texttt{C++} & 1287 & \\
    \hline
    \texttt{sinex}  & \texttt{C++} & 1392 & \\
    \hline
    \multirow{2}{*}{\texttt{datetime}}  & \texttt{C++} & 4432 & \multirow{2}{*}{Mostly header files}\\
      & \texttt{C}      & 136 \\
    \hline
    \multirow{2}{*}{\texttt{geodesy}}  & \texttt{C++} & 3755 & \multirow{2}{*}{}\\
      & \texttt{Python} & 120 \\
    \hline
    Total & & 100397\\
    \bottomrule
  \end{tabular}
  \caption{Software components of the package designed and implemented for the Thesis.}
  \label{table:software-components}
\end{table}

\subsection{Policy And Software Philosophy}\label{sec:the-software-policy}
The software is hosted and developed on the public domain, adopting a \emph{free} 
and \emph{open}-source policy. Any interested party is free to use and/or adapt 
the source code, fitting individual needs. It is a strong belief that adhering to 
such an open-development, open-access paradigm, can prove to be highly beneficial 
both for the scientific community and for the development team. Already individual 
modules/libraries built for the Thesis have been ``forked'' from different users.

\subsection{Architecture}\label{sec:the-software-architecture}
The general architectural approach followed in designing the software is that of 
\emph{modular software development}, emphasizing on building simple, compact, 
clear, modular, and extensible code that can be easily maintained and repurposed 
by developers other than its creators. This approach favors composability, an attribute 
greatly valued in scientific programming since it accommodates ease of adoption and 
expansion. Since the aim of the package is to act as a fundamental building block 
for an analysis tool matching the highest standards, this design pattern is crucial 
to its development and further success.

A mixture of the well established \emph{object-oriented design} (OOP) and the 
newer approach of \emph{data-oriented design} (\cite{Fabian2018}) (DOP) is used 
throughout the codebase. In coarse terms, the former favors code clarity and 
encapsulation while the latter promotes efficiency. Depending on the problem at 
hand, the constraints imposed and the abstraction level used, the technique better 
matching the challenge was chosen.

The language of choice for code development is \texttt{C++}, while minor parts 
are written in \texttt{C} and \texttt{Python} (see \autoref{table:software-components}). 
\texttt{C++} is a language supporting OOP, renowned for its speed and efficiency. 
Via its versatility (especially in handling memory resources) and closeness to 
the hardware, this programming language is fit for problems posing constraints on 
computational resources and speed (such as a \gls{pod} analysis procedure). The source 
code employs \emph{generic programming} techniques and makes heavy usage of the 
emerging technique of \emph{template metaprogramming} (\cite{Esterie2014}, 
\cite{Gawlik2018}), to get the highest possible performance gains.

The source code is split into different modules/libraries, where each collection 
serves a well established, individual goal (see \autoref{table:software-components2}). 
This division favors the overall modularity of the package.

\begin{table}[h!]
  \centering
  \begin{tabular}{p{0.20\linewidth} | p{0.60\linewidth} | p{0.15\linewidth}}
    \textbf{Library} & \textbf{Purpose} & \textbf{Repository Name}\footnote{The repository can be found online by adding the name at the static path \url{https://github.com/xanthospap/}} \\
    \toprule
    
    \texttt{libdatetime} & A library to handle datetime instances. Includes data structures, 
      operators and transformations between different scales, systems and representations. 
      Facilitates the parsing, formatting and writing of instances in different formats.
      & \texttt{datetime} \\
    
    \texttt{libgeodesy} & A library implementing fundamental geodetic concepts and 
      computations. Includes handling of reference systems, reference ellipsoids, 
      and associated geometric operations, as well as transformations between 
      systems and frames. & \texttt{geodesy} \\

    \texttt{libsp3} & Faciliatate the parsing and data extraction and manipulation 
      (e.g. indexing and iterating) of \texttt{sp3} data files (\cite{Hilla2010}). 
      & \texttt{sp3} \\
    
    \texttt{libiers2010} & Implementation of the \gls{iers}-2010 standards, based on  
      \cite{iers2010}. Additional material is added (where needed) to handle the 
      convention's updates. To-date model updates are also included, as well as 
      source code to handle data resources in various formats, often encountered 
      when implementing the standards.
      & \texttt{iers2010} \\
    
    \texttt{libsinex} & Facilitate the parsing and data extraction and manipulation 
      of \texttt{SINEX} files (\cite{Sinex202}) including additional blocks 
      used by the \gls{ids}, see \cite{Moreaux2023}. Utilities include reference 
      frame realization, reductions between site and reference points (via eccentricity 
      vectors) and inspection of time interval validity.
      & \texttt{sinex} \\
    
    \texttt{libdoris} & A library focused on the \gls{doris} satellite system; includes 
      relevant data handling, implementation of the observation equation model 
      (\autoref{ch:doris}) and its application in \gls{pod}. Facilitates data 
      retrieval from a wide range of different formats, for various steps of a
      \gls{doris} data analysis scheme.
      & \texttt{libdoris} \\
      \bottomrule
  \end{tabular}
  \caption{Individual libraries (modules) of the package designed and implemented for the Thesis.}
  \label{table:software-components2}
\end{table}

\subsection{External Dependencies}\label{sec:the-software-dependencies}
An effort was made while developing the source code, to strive for minimal external 
dependencies. As a consequence, only three external components are used to build 
the whole package, namely,
\begin{description}
  \item[The \texttt{SPICE} Toolkit] (\cite{Acton2018}), release January 3, 2022. 
  This is \gls{nasa}'s 
  \emph{Observation Geometry System for Space Science Missions}, developed by the 
  Navigation and Ancillary Information Facility (NAIF). This is a multi-purpose 
  toolkit, with a wide range of functionalities, available via a list of dedicated 
  APIs. For the software implemented for this Thesis, \texttt{SPICE} is only used to 
  retrieve planetary and lunar ephemeris records. More information on the 
  toolkit are provided in \url{https://naif.jpl.nasa.gov/naif/index.html}.

  \item[The \texttt{eigen} Library] (\cite{eigenweb}), release 3.4. This is an 
  open-source \texttt{C++} library for linear algebra, matrix and vector 
  operations, geometrical transformations, numerical solvers and related algorithms.
  The main use of \texttt{eigen} within the implemented software, is to perform 
  matrix/vector operations. More information on the library can be found at 
  \url{http://eigen.tuxfamily.org/}.

  \item[The \texttt{yaml-cpp} Library], developed by Jesse Beder. This is an open-source 
  library, hosted at \url{https://github.com/jbeder/yaml-cpp}, that enables efficient 
  parsing of \texttt{YAML} (\url{https://yaml.org/}) files. This library is only 
  used by the executables (i.e. not needed for any of the libraries) to parse the 
  configuration file(s).
\end{description}
