\section{The Software}\label{sec:the-software}

In the framework of the Thesis, a software package was designed and implemented, to 
perform orbit determination using the \gls{doris} satellite system. The software 
was built from scratch, with minimum (external) dependencies. The purpose of the 
package is to act as the fundamental building block of a state-of-the-art scientific 
software toolset, utlizing \gls{doris} observations for a wide range of geodetic 
studies, including but not limited to \gls{pod}, reference frame maintenance, 
earth dynamics and atmospheric studies. A first, imminent step towards this direction, 
would be for the software to reach \gls{ids} standards, a goal not too far-fetched, 
since the bulk of the work has already been done.

Writing scientific software to meet such high accuracy and efficiency standards 
is a chalenging task. This is evident by the limited number of packages delivering 
such robust products. A deep understanding of the underlying scientific notions is 
a prerequisit (including e.g. celestial mechanics, spatial/Temporal reference frames, 
geodesy and estimation theory) coupled with software engineering skills to match the 
high volume and high calibre of work needed (about 100000 lines of source code were 
written for this Thesis, \autoref{table:software-components}).

\begin{figure}
  \centering
    %every node/.style = {
  %  draw=black, 
  %  rounded corners, 
  %  fill=gray!20,
  %  minimum width=2cm,
  %  minimum height=0.5cm,
  %  align=center},
  %  every path/.style = {draw, -latex}
  %]

\tikzset{bin/.style=   {black, draw=blue, fill=red!20,   rounded corners=3pt, align=center, minimum height=1.2cm, minimum width=2.5cm}}
\tikzset{level1/.style={black, draw=blue, fill=blue!25,  rounded corners=3pt, align=center, minimum height=1.2cm, minimum width=2.5cm}}
\tikzset{level2/.style={black, draw=blue, fill=blue!20,  rounded corners=3pt, align=center, minimum height=1.2cm, minimum width=2.5cm}}
\tikzset{level3/.style={black, draw=blue, fill=blue!15,  rounded corners=3pt, align=center, minimum height=1.2cm, minimum width=2.5cm}}
\tikzset{level4/.style={black, draw=blue, fill=blue!10, rounded corners=3pt, align=center, minimum height=1.2cm, minimum width=2.5cm}}
\tikzset{external/.style={black, draw=blue, fill=brown!10, rounded corners=3pt, align=center, minimum height=1.2cm, minimum width=2.5cm}}
\tikzset{data/.style=  {black, draw=blue, fill=green!20, rounded corners=3pt, align=center, minimum height=1.2cm, minimum width=3.0cm}}
  
\begin{tikzpicture}
  \draw [red, very thick] (-8,-1.5) rectangle (8,.4);
  \node [] at (-5,-1.7) {\texttt{Executables}};
  \node [bin] (bin) at (0,-.6) {\texttt{bin}};
  
  \draw [blue, very thick] (-8,1.2) rectangle (8,9.7);
  \node [] at (-5,.9) {\texttt{Libraries \& Dependencies}};
  \node [level1] (doris) at (2,2)  {\texttt{doris}};
  \node [external] (yaml) at (-2,2) {\texttt{yaml-cpp}};
  
  \node [external] (cspice) at (6,4.5)  {\texttt{cspice}};
  \node [level2] (sinex) at (2,4.5)  {\texttt{sinex}};
  \node [level2] (iers10) at (-2,4.5) {\texttt{iers2010}};
  \node [level2] (sp3) at (-6,4.5) {\texttt{sp3}};
  
  \node [level3] (geodesy) at (0,6.8)  {\texttt{geodesy}};

  \node [external] (eigen) at (2,9)  {\texttt{eigen}};
  \node [level4] (datetime) at (-2,9) {\texttt{datetime}};
  
  \draw [green, very thick] (-8,-2) rectangle (8,-6);
  \node [data] (din1) at (6,-3)    {\texttt{eop (C04)}};
  \node [data] (din2) at (2,-3)    {\texttt{satellite}\\\texttt{macromodel}};
  \node [data] (din3) at (-2,-3)   {\texttt{space}\\ \texttt{weather}};
  \node [data] (din4) at (-6,-3)   {\texttt{gravity}\\ \texttt{model}};
  \node [data] (din5) at (0,-5)    {\texttt{Configuration File}};
  \node []     (din6) at (-5,-6.5) {\texttt{User Input and Models}};

  \draw[thick,-Latex] (yaml.south) to[bend right] (bin.west);
  \draw[thick,-Latex] (doris.south) to[bend left] (bin.east);
  %\draw[->] (cspice.south) to[bend left] (doris.east);
  \draw[thick,-] (-6,3.2) to (6,3.2);
  \draw[thick,-Latex] (sinex.south) to (doris.north);
  \draw[thick,-Latex] (cspice.south) to (6,3.2);
  \draw[thick,-Latex] (sinex.south) to (2,3.2);
  \draw[thick,-Latex] (iers10.south) to (-2,3.2);
  \draw[thick,-Latex] (sp3.south) to (-6,3.2);
  
  \draw[thick,-] (-6,5.8) to (2,5.8);
  \draw[thick,-Latex] (geodesy.south) to (0,5.8);
  \draw[thick,Latex-] (sinex.north) to (2,5.8);
  \draw[thick,Latex-] (iers10.north) to (-2,5.8);
  \draw[thick,Latex-] (sp3.north) to (-6,5.8);
  
  \draw[thick,-Latex] (datetime.south) to[bend right] (geodesy.west);
  \draw[thick,-Latex] (eigen.south) to[bend left] (geodesy.east);

  \draw[thick,-Latex] (0,-2) -- (bin.south);
  \draw[-Latex] (din1.south) to[bend left] (din5.east);
  \draw[-Latex] (din4.south) to[bend right] (din5.west);
  \draw[-Latex] (din3.south) to[] ($(din5.north) + (-1,0)$);
  \draw[-Latex] (din2.south) to[] ($(din5.north) + (+1,0)$);
  %\path[->] (din1) to[bend left] node[midway,inner sep=2pt,draw=blue, fill=green!20, rounded corners=3pt, minimum height=.5cm, minimum width=.95cm] {$\cdots$} (din5);

\end{tikzpicture}

  \caption{Overview of the software structure, dependencies and hierarchy.}
  \label{fig:software-hierarchy}
\end{figure}

The software is hosted and developed on the public domain, adopting a \emph{free} 
and \emph{open}-source policy. Any interested party is free to use and/or adapt 
the source code, fitting indivdual needs. It is a strong belief that adhering to 
such an open-developement, open-access paradigm, can prove to be highly beneficial 
both for the scientific community and for the development team. Already individual 
modules/libraries built for the Thesis have been ``forked'' from different users.

\begin{table}[h!]
    \centering
    \begin{tabular}{p{0.20\linewidth} | p{0.2\linewidth} | p{0.20\linewidth} | p{0.25\linewidth}}
        \hline
        \textbf{Library} & \textbf{Language} & \textbf{Lines of Code} & \textbf{Comment}\\
        \hline
        \multirow{2}{*}{\texttt{doris}}  & \texttt{C++} & 18286 & \multirow{2}{4cm}{Including source code for executables}\\
          & \texttt{Python} & 2068 & \\
        \hline
        \multirow{3}{*}{\texttt{iers2010}}  & \texttt{C++} & 50000 & \multirow{3}{4cm}{Python \& Fortran mainly used for unit testing} \\
          & \texttt{C} & 1801 \\
          & \texttt{Python \& Fortran} &  17120 \\
        \hline
        \texttt{sp3}    & \texttt{C++} & 1287 & \\
        \hline
        \texttt{sinex}  & \texttt{C++} & 1392 & \\
        \hline
        \multirow{2}{*}{\texttt{datetime}}  & \texttt{C++} & 4432 & \multirow{2}{*}{Mostly header files}\\
          & \texttt{C}      & 136 \\
        \hline
        \multirow{2}{*}{\texttt{geodesy}}  & \texttt{C++} & 3755 & \multirow{2}{*}{}\\
          & \texttt{Python} & 120 \\
        \hline
        Total & & 100397\\
        \hline
    \end{tabular}
    \caption{Software components of the package designed and implemented for the Thesis.}
    \label{table:software-components}
\end{table}

The software is built using modular
