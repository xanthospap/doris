\subsection{Implementing Time Scales}\label{ssec:time-scales-implementation}

Designing and implementing software for handling dates, time scales and related transformations 
is a rather challenging task. The complexity is evident, considering:
\begin{itemize}
    \item The long list of different time-scales involved in Satellite Geodesy and 
        Astronomy (see \ref{ssec:time-scales})
    \item The different representation conventions (e.g. \gls{jd}, \gls{mjd}, year/month/day and hour/minute/seconds)
    \item The accuracy required when transforming between different time scales 
    \item The heavy usage of dates and time in a \gls{pod} process (i.e. efficiency)
\end{itemize}

All of the above must be taken into account in software design, making the ``datetime'' 
problem a field of continuous investigation. Despite the fact that there are a few datetime libraries, 
(some languages have even relevant implementations in their standard libraries), none 
complies with the accuracy and complexity involved in Satellite Geodesy. The exception 
is the official \gls{iau} implementation of Fundamental Astronomy, i.e. the \gls{sofa} 
library (\cite{sofa2021}). However, \gls{sofa} is implemented in the FORTRAN (and C) 
programming languages and has to reserve a level of ``backwards compatibility'', thus 
its design paradigm is rather outdated (e.g. no Object Oriented Design). Additionally, 
the package provides core functionality, hence one should implement various utility functions 
(e.g. parsers) to interact with this functions.

For all the above, a decision was taken to design and implement a new software library, 
from scratch, to address the ``datetime'' problem. The design follows recent developments 
and paradigms in Software Engineering, such as Template Metaprogramming 
(\cite{Vandevoorde2017}). The library is open and free for any interested user.

A ``datetime'' or \emph{epoch} within the library is represented as an \gls{mjd}, 
where the integral and fractional part of day are stored separately, to preserve 
precision. In the special case of \gls{utc} dates, the epoch can be stored in the 
Year Month Date plus Hours Minutes Seconds (YMD/HMS) format. Depending on the user/application 
needs, the fractional part of the day can be stored as either a floating point numeric 
value, or the accumulated number of \si{\second}, \si{\milli\second}, \si{\micro\second} or 
\si{\nano\second}, since the beginning of the day. Both implementations are supported by 
individual ``classes''; the latter is achieved via heavy template usage. 

While storing the fractional part of day as a floating point number is intuitive and 
straightforward, it suffers from roundoff errors. This is avoided in the case where 
the fraction of day is stored as an integer numeric value (i.e. accumulated second 
submultiple); however, in this case, precision is limited by the chosen submultiple 
and non-integer division will also introduce truncation. It is up to the user to 
choose the suitable representation to meet application demands.

The components of the library have been extensively tested using the \gls{sofa} results 
as reference, as well as the relevant (limited) standard library functions. The 
software is developed as a stand-alone library, complying with the latest C++ standards 
(C++17 \& C++20) and include $\approx$5000 lines of source code (including test suits).
