\subsection{The Two-Body Problem}
\label{ssec:two-body-problem}

The main features of the motion of artificial satellites can be described by a 
reasonably simple approximation, due to the fact that the force induced by the 
Earth's gravity field outrules all other forces (acting on the satellite) by 
several orders of magnitude. The problem of determining the motion of two 
bodies, due solely to their own mutual gravitational attraction, is usually 
reffered to as the \emph{Two-Body Problem} or \emph{Kepler's Problem}. In the 
case of Earth orbiting satellites, we will be approximating Earth's gravity as 
if it were a spherically symmetric force field, induced by a central mass.

In the \emph{Two-Body Problem}, we are considering a satellite of mass $m$, 
where $m \ll M_{\Earth}$. Assuming a spherical Earth, the acceleration of the 
satellite $\bm{\ddot{r}}$ is given by Newton's law of gravitational attraction:
\begin{equation}
  \label{eq:mont32}
  \bm{\ddot{r}} = - \frac{G M_{\Earth}}{r^3} \bm{r}
\end{equation}

A full treatment of the \emph{Two-Body Problem} is beyond the scope of this 
thesis and well documented in e.g. \cite{curtisb} and \cite{chobotov}. It can 
be shown that the path of the satellite, relative to the Earth, is a conic section 
(ellipse) whose shape is determined by the \emph{eccentricity}. Using the laws 
of conservation of angular momentum and energy, the period of the elliptic orbit 
can also be deduced.

\iffalse
\subsection{Two-Body Problem}
\label{ssec:keplerian-orbits}
Starting from Newton's law of gravitation for two masses $M$ and $m$ at 
distance $r$, assuming an inertial reference frame, the gravitational 
attraction is:
\begin{equation}
  \label{eq:chobotov31}
  \bm{F} = \frac{G M m_1}{r^3} \bm{r}
\end{equation}
where $G$ is the gravitational constant. The force acts along the line joining 
the centers of the two masses. The force exerted on $M$ by $m$, is:
\begin{equation}\label{eq:curtis29}
  \bm{F}_{Mm} = \frac{G M m}{r^3} \bm{r}
\end{equation}
and analogously, the force exerted on $m$ by $M$, is
\begin{equation}\label{eq:curtis210}
  \bm{F}_{mM} = -\frac{G M m}{r^3} \bm{r}
\end{equation}
By Newton's third law (action-reactin principle) the two forces are equal in 
magnitude with oppposite directions. Additionaly, Newton's second law of motion
dictates that:
\begin{equation}\label{eq:curtis2105}
  \bm{F}_{Mm} = M \ddot{\bm{r}}_M \quad \bm{F}_{mM} = m \ddot{\bm{r}}_m
\end{equation}
hence:
\begin{subequations}
  \begin{align}
    M \ddot{\bm{r}}_M &= \frac{G M m}{r^3} \bm{r} \label{eq:curtis211} \\
    m \ddot{\bm{r}}_m &= \frac{G M m}{r^3} \bm{r} \label{eq:curtis212}
  \end{align}
\end{subequations}
Note that by addition of \ref{eq:curtis211} and \ref{eq:curtis212}, we get that:
\begin{equation}
  M \ddot{\bm{r}}_M + m \ddot{\bm{r}}_m = \bm{0}
\end{equation}
i.e., the acceleration of the barycenter is zero (see \ref{eq:curtis24}), a 
condition holding for any system free of external forces.

The \emph{barycenter} of the (two-body) system is defined by the formula,
\begin{equation}\label{eq:curtis22}
  \bm{r}_{cm} = \frac{M \bm{r}_M + m \bm{r}_m }{M + m}
\end{equation}
with absolute\footnote{\emph{Absolute} means that the quantities are measured relative to an inertial reference frame.} 
velocity and acceleration:
\begin{subequations}
  \begin{align}
    \dot{\bm{r}}_{cm} &= \frac{M \dot{\bm{r}}_M + m \dot{\bm{r}}_m }{M + m} \label{eq:curtis23} \\
    \ddot{\bm{r}}_{cm} &= \frac{M \ddot{\bm{r}}_M + m \ddot{\bm{r}}_m }{M + m} \label{eq:curtis24}
  \end{align}
\end{subequations}

\begin{figure}
\centering
\begin{tikzpicture}
  \tikzstyle{every node}=[font=\scriptsize]
  \pgfmathsetmacro\a{4.0}
  \pgfmathsetmacro\e{0.7}
  \pgfmathsetmacro\xstart{0}
  \pgfmathsetmacro\ystart{0}
  % geometry of ellipse
  \pgfmathsetmacro\asqr{\a *\a}
  \pgfmathsetmacro\esqr{\e *\e}
  \pgfmathsetmacro\b{{\a * sqrt(1.0 - \e * \e)}}
  \pgfmathsetmacro\bsqr{\b *\b}
  \pgfmathsetmacro\c{{sqrt(\asqr - \bsqr)}}
  % coordinates of center
  \pgfmathsetmacro\xc{{\xstart - \a *cos(0)}}
  \pgfmathsetmacro\yc{{\ystart - \a *sin(0)}}
  % focal point (aka earth)
  \pgfmathsetmacro\xe{\xc + \c }
  \pgfmathsetmacro\ye{\yc }
  % left focal point
  \pgfmathsetmacro\xf{\xc - \c }
  \pgfmathsetmacro\yf{\yc }
  % true anomaly
  \pgfmathsetmacro\v{120}
  % point on ellipse, satellite
  \pgfmathsetmacro\vr{{\v}}
  \pgfmathsetmacro\p{{\a * (1e0 - \e * \e )}}
  \pgfmathsetmacro\r{{\p / (1+\e*cos(\v))}}
  \pgfmathsetmacro\xsat{{\r * cos(\v) - \xc}}
  \pgfmathsetmacro\ysat{{\r * sin(\v ) - \yc}}
  % semilatus rectum
  \pgfmathsetmacro\vsr{{90}}
  \pgfmathsetmacro\xsr{\xe}
  \pgfmathsetmacro\xsrq{{\xsr - \xc}}
  \pgfmathsetmacro\ysr{{\b *sqrt(1-(\xsrq *\xsrq)/\asqr) -\yc}}

  % mark perigee
  \node[] (perigee) 
    at ( \xstart , \ystart )
    {};
  \draw[fill=black] (perigee) circle[radius=1pt];
  \node[anchor=north west,align=center] () at (perigee){P\\Perigee};
  
  % mark empty focal point
  \node[] (empty-focal) 
    at ( \xf , \yf )
    {};
  \draw[fill=black] (empty-focal) circle[radius=1pt];
  \node[anchor=north,align=center] () at (empty-focal){F'\\Empty Focus};
  
  % mark earth
  \node[] (earth)
    at ( \xe , \ye )
    {};
  \draw[fill=black] (earth) circle[radius=1pt];
  \node[anchor=north,align=center] () at (earth){F\\Focus};
  
  % center
  \node[] (center) at (\xc , \yc ) {};
  \draw[fill=black] (center) circle[radius=1pt];
  \node[anchor=north, align=center] () at (center){C\\Geometric Center};

  % mark apogee
  \node[] (apogee) 
    at ({\xstart - \a *cos(0)+ \a *cos(180)}, {\ystart - \a *sin(0)+ \a *sin(180)})
    {};
  \draw[fill=black] (apogee) circle[radius=1pt];
  \node[anchor=north east, align=center] () at (apogee){A\\Apogee};
  
  % ellipse arc
  \draw[] ( \xstart , \ystart ) arc(0:360:\a cm and \b cm);
  
  % x-axis
  \draw[] ($(apogee) - (.2,0)$)  -- ($(perigee)+(.2,0)$);

  % true anomaly (coordinate system centered at F/Earth)
  \pgfmathsetmacro\xsat{{\xe - \r * cos(180 -\v)}}
  \pgfmathsetmacro\ysat{{\ye + \r * sin(180 -\v )}}
  \coordinate (Shift) at (\xe,\ye,0);
  \tdplotsetrotatedcoordsorigin{(Shift)}
  \draw[tdplot_rotated_coords,black,-stealth] (0,0) -- ++(\v:\r) node[midway,below left,align=center] {Radius\\$r$};
  \node[anchor=south west] () at (\xsat, \ysat) {Satellite};
  \draw[tdplot_rotated_coords,fill=black] (\v:\r) circle[radius=1pt];
  \tdplotdrawarc[tdplot_rotated_coords,-stealth]
        {(Shift)}{0.1*\b}{0}{\v}{anchor=south,align=center}{True Anomaly\\$\theta$};

  % semilatus rectum
  \draw[gray,thin] ($(perigee)+(.2,0)$) -- ($(perigee)+(.2,0)+(1.0,0)$);
  \draw[gray,thin] ($(\xe,\ye)+(0,\b/1.8)$) -- ($(\xe,\ye)+(0,\b-\b/10)$);
  \draw[gray,thin] ($(\xsr,\ysr)$) -- ($(perigee)+(.2,0)+(1.0,0)+(0,\ysr)$);
  \path ($(perigee)+(.2,0)+(1.0,0)+(0,\ysr)$) 
    -- node[align=center](semirec){$p=\alpha (1-e^2)$\\Semi-latus Rectum}
    ($(perigee)+(.2,0)+(1.0,0)$);
  \draw[gray,stealth-] ($(\xstart,\ysr)+(.2,0)+(1.0,0)$) -- (semirec);
  \draw[gray,-stealth] (semirec) -- ($(perigee)+(.2,0)+(1.0,0)$);

  % satellite velocity
  \pgfmathsetmacro\gfa{60} % angle gamma
  \pgfmathsetmacro\bfa{90-\gfa} % angle beta
  \pgfmathsetmacro\vel{{\r / 2}}
  \pgfmathsetmacro\velx{{cos(\gfa) * \vel}}
  \pgfmathsetmacro\vely{{sin(\gfa) * \vel}}
  \coordinate (Shift) at (\xsat,\ysat);
  \tdplotsetrotatedcoordsorigin{(Shift)}
  \draw[tdplot_rotated_coords,gray,thin] (0,0) -- ++(\v:\vely);
  \draw[tdplot_rotated_coords,black,-stealth] (0,0) -- ++(\v + \bfa : \vel) node[midway,above]{$\nu$};
  \draw[tdplot_rotated_coords,black,thin,-stealth] (0,0) -- ++(\v+90:\velx) node[midway,below] {$\nu _n$};
  \tdplotsetrotatedcoordsorigin{($(Shift)+(\v+90:\velx)$)}
  \draw[tdplot_rotated_coords,black,thin,-stealth] (0,0) -- ++(\v:\vely) node[midway,below,xshift=-0.6ex] {$\nu _r$};
  \tdplotdrawarc[tdplot_rotated_coords,-stealth]
        {(Shift)}{0.1*\b}{\v+90}{\v+90-\gfa}{anchor=center,yshift=1ex}{$\gamma$};
  \tdplotdrawarc[tdplot_rotated_coords,-stealth]
        {(Shift)}{0.15*\b}{\v}{\v+\bfa}{anchor=center,yshift=1.0ex}{$\beta$};


  % lower part dimensioning
  % -----------------------
  \pgfmathsetmacro\ygoff{{-\b / 4}}

  % semi-major axis
  \pgfmathsetmacro\ydaoff{{-\b -\b / 4}}
  \draw[gray,thin] ($(apogee) +  (0, \ygoff)$) 
    -- ($(apogee) +  (0, -\b -\b / 3)$);
  \draw[gray,thin] ($(perigee) + (0, \ygoff)$) 
    -- ($(perigee) + (0, -\b -\b / 3)$);
  \draw[gray,thin] ($(center) +  (0, -\b -\b / 8)$) 
    -- ($(center) + (0, -\b -\b / 3)$);
  \path ($(apogee) +  (0, \ydaoff)$) 
    --node[](alphaleft){$\alpha$} ($(center) + (0, \ydaoff)$);
  \draw[gray,thin,stealth-] ($(apogee) +  (0, \ydaoff)$) -- (alphaleft);
  \draw[gray,thin,-stealth] (alphaleft) -- ($(center) + (0, -\b \ygoff)$);
  \path ($(center) +  (0, \ydaoff)$) 
    -- node[align = center](alpharight){$\alpha$\\Semi-Major axis}
      ($(perigee) + (0, \ydaoff)$);
  \draw[gray,thin,stealth-] ($(center) +  (0, \ydaoff)$) -- (alpharight);
  \draw[gray,thin,-stealth] (alpharight) -- ($(perigee) + (0, \ydaoff)$);
  
  % apogee radius
  \pgfmathsetmacro\ydaroff{{-\b / 1.5}}
  \draw[gray,thin] ($(\xe,\ye) +  (0, \ygoff)$) 
    -- ($(\xe,\ye) +  (0, -\b / 1.2)$);
  \path ($(apogee)+(0,\ydaroff)$)
    -- node[align = center](apogeeradius){$r_\alpha =\alpha (1+e)$\\Apogee Radius}
     ($(\xe,\ye)+(0,\ydaroff)$);
  \draw[gray,stealth-] ($(apogee)+(0,\ydaroff)$) -- (apogeeradius);
  \draw[gray,-stealth] (apogeeradius) -- ($(\xe,\ye)+(0,\ydaroff)$);
  % perigee radius
  \draw[gray, thin, -stealth] ($(\xe,\ye) + (- \a * \e / 5, \ydaroff + \a /8)$) -- ($(\xe, \ye) + (0,\ydaroff + \a /8)$);
  \draw[gray, thin, -stealth] ($(perigee) + (\a * \e / 5, \ydaroff + \a /8)$) -- ($(perigee) + (0, \ydaroff + \a /8)$);
  %\path ($(\xe, \ye) + (0,\ydaroff + \a /8)$)
  %  -- node[align = center](perigeeradius){$r_p =\alpha (1-e)$\\Perigee Radius}
  %  ($(perigee) + (0, \ydaroff + \a /8)$);
  \node[anchor = west, align=center] () at ($(perigee) + (\a * \e / 5, \ydaroff + \a /8)$)
    {$r_p =\alpha (1-e)$\\Perigee Radius};

  % to foci
  \pgfmathsetmacro\ydaroff{{-\b / 3}}
  \draw[gray,thin] ($(\xf,\yf) +  (0, \ygoff)$)
    -- ($(\xf,\yf) +  (0, -\b / 2)$);
  \path ($(\xf,\yf) +  (0, \ydaroff)$) --node[](aeleft){$\alpha e$}($(\xc,\yc) +  (0, \ydaroff)$);
  \draw[gray,thin,stealth-] ($(\xf,\yf)+(0,\ydaroff)$) -- (aeleft);
  \draw[gray,thin,-stealth] (aeleft) -- ($(\xc,\yc) +  (0, \ydaroff)$);
  \draw[gray,thin] ($(\xc,\yc) +  (0, \ygoff)$)
    -- ($(\xc,\yc) +  (0, -\b / 2)$);
  \path ($(\xc,\yc) +  (0, \ydaroff)$) --node[](aeright){$\alpha e$}($(\xe,\ye) +  (0, \ydaroff)$);
  \draw[gray,thin,stealth-] ($(\xc,\yc)+(0,\ydaroff)$) -- (aeright);
  \draw[gray,thin,-stealth] (aeright) -- ($(\xe,\ye) +  (0, \ydaroff)$);
  
\end{tikzpicture}

\caption{Ellipse geometry (\cite{chobotov})}
\label{fig:ellipse-geometry}
\end{figure}
\fi
