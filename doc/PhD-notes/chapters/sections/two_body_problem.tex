\subsection{The Two-Body Problem}
\label{ssec:two-body-problem}

The main features of the motion of artificial satellites can be described by a
reasonably simple approximation, due to the fact that the force induced by the 
Earth's gravity field outrules all other forces by several orders of magnitude. 
Assume that the masses are spherically symmetrical, thus acting like point masses, 
and isolated, so the only force acting is gravitational attraction along the line 
joining the centers. The problem of predicting the orbit of the masses, given the 
described setup, is usually referred to as the \emph{Two-Body Problem} (or 
\emph{Kepler Problem} since the attracting force is gravity) and lies in the core 
of celestial mechanics. The solution can be expressed as a \emph{Kepler orbit}, using 
six \emph{orbital elements} (see \autoref{ssec:orbital-elements}).

In the case of an Earth orbiting satellite, we can safely assume that the mass of 
the satellite $m$ is insignificant compared to Earth's mass, $m \ll M_{\Earth}$, 
(thus considering what is often called a \emph{central-force problem}). The 
acceleration of the satellite $\bm{\ddot{r}}$ is given by Newton's law of gravitational 
attraction
\begin{equation}
  \label{eq:mont32}
  \bm{\ddot{r}} = - \frac{G M_{\Earth}}{r^3} \bm{r}
\end{equation}

A full treatment of the Two-Body Problem is beyond the scope of this thesis and 
well documented in e.g. \cite{curtisb} and \cite{chobotov}. It can be shown that 
the path of the satellite, relative to the Earth, is a conic section 
(ellipse) whose shape is determined by the \emph{eccentricity}. Using the laws 
of conservation of angular momentum and energy, the period of the elliptic orbit 
can also be deduced.

The Two-Body Problem results in what is called the \emph{Keplerian orbit}.
In this ideal case, a set of six parameters is required to uniquely identify a specific 
orbit. This parameter set in not unique (there are different ways to mathematically 
describe the same orbit), but the most oftenly used set is the \emph{Keplerian elements} 
(see \autoref{ssec:orbital-elements}).