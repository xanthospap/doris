\subsubsection{\gls{eop} Information \& Interpolation}\label{ssec:eop-interpolation}

\Gls{eop} information for formulating the Celestial-to-Terrestrial transformation 
matrix, is extracted from the \gls{iers} \texttt{C04} files (\cite{Bizouard2019}).
These files contain tabulated \gls{eop} values at 0\textsuperscript{h} \gls{utc}. 

These files can be read into an \texttt{dso::EopLookUpTable} instance, using 
the function \texttt{dso::parse\_iers\_C04}. The function will parse the file 
for the requested date range, transform \gls{utc} dates to \gls{tt} and conviniently 
store them in an \texttt{dso::EopLookUpTable} instance. $\Delta UT$ and LOD values 
can be ``regularized'', aka have the zonal Earth tide effects removed, via the 
\texttt{dso::EopLookUpTable::regularize} function.

To interpolate \gls{eop} values from an \texttt{dso::EopLookUpTable} instance, 
users can call the \texttt{dso::EopLookUpTable::interpolate} function, given 
a datetime instance in \gls{tt}, also specifying the order of Lagrangian 
interpolation. Effects for ocean tides and libration (see \ref{ssec:polar-motion-matrix} 
and \ref{ssec:earth-rotation-matrix}) are handled in the function. The output 
is a complete set of \gls{eop}s, containing $x_p$, $y_p$, $\Delta UT$, $\delta X$ 
and $\delta Y$ at given $t$. A list of relevant function and data structures 
used to manipulate \gls{eop} is presented in \ref{table:eop-handling-fds}. The 
parsing and interpolation procedures are listed in \ref{fig:handling-eop}.

\begin{table}
\centering
\begin{tabular}{|p{3cm}|p{3cm}|p{2.5cm}|p{5cm}|}
    \hline
    \textbf{Function/Data Structure} & \textbf{Header File} & \textbf{Repository/ Library} & \textbf{Comment} \\
    \hline
    \texttt{EopLookUpTable} & \texttt{eop.hpp} & \texttt{doris} & 
    Contains all relevant declerations, data structures and algorithms to handle 
    \gls{eop} information. \\

    \hline
    \texttt{parse\_iers\_C04} & \texttt{eop.hpp} & \texttt{doris} & \\

    \hline
    \texttt{ortho\_eop} & \texttt{iers2010.hpp} & \texttt{iers2010} &
    compute the diurnal and semidiurnal variations in \gls{eop} ($x$,$y$, $UT1$) from
    ocean tides; translated from \texttt{ORTHO\_EOP.F} \footref{fn:ortho-eop-f} \\

    \hline
    \texttt{pmsdnut2} & \texttt{iers2010.hpp} & \texttt{iers2010} &
    evaluate the model of polar motion for a nonrigid Earth due to tidal gravitation; 
    translated from \texttt{PMSDNUT2.F} \footref{fn:pmsdnut2-f} \\ 

    \hline
    \texttt{utlibr} & \texttt{iers2010.hpp} & \texttt{iers2010} & 
    compute subdiurnal libration in the axial component of rotation, expressed by 
    \gls{ut1} and LOD. This effect is due to the influence of tidal gravitation on the
    departures of the Earth's mass distribution from the rotational
    symmetry, expressed by the non-zonal components of geopotential. Translated 
    from \texttt{UTLIBR.F} \footref{fn:utlibr-f} \\

    \hline
    \texttt{interp\_pole} (and \texttt{interp} namespace) & \texttt{iers2010.hpp} & \texttt{iers2010} & 
    account for ocean tidal and libration effects in \gls{eop}; translated from \texttt{INTERP.F} \footref{fn:interp-f}\\

    \hline
\end{tabular}
\caption{List of functions and data structures relevant to handling \gls{eop} information.}
\label{table:eop-handling-fds}
\end{table}


\begin{figure}
\centering
\input{tikz/handling_eop.tex}
\caption{Extracting \gls{eop} information from \gls{iers} \texttt{C04} data files.}
\label{fig:handling-eop}
\end{figure}