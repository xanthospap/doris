\section{The Kalman Filter}\label{sec:pod-kalman-filter}

In this section, the sequential estimation algorithm referred to as the 
\emph{Kalman filter} is discussed, with emphasis on its application on the orbit 
determination problem. A more thorough discussion on the sequential estimation 
techniques can be found at \cite{Gelb1974}, while \cite{Montenbruck2000} and 
\cite{Tapley2004} describe the filter's application and variations thereof for 
\gls{pod}. One important advantage of the Kalman filter is that the matrix to be 
inverted will be of the same dimension as the observation vector, which means 
that given that the observations can be processed one at a time, only scalar 
divisions will be required to obtain the estimate of $\delta \bm{y}(t_k)$. For the 
rest of this section, the notation $\bm{y}_k$ will be used to denote the value 
$\delta \bm{y}(t_k)$ to reduce complexity and follow relevant literature in the 
field.

Given that an estimate $\hat{\bm{y}}_j$ and the associated covariance matrix $\bm{P}_j$ 
are available at a given epoch $t_j$, the state and its variance-covariance matrix can 
be propagated according to
\begin{equation}\label{eq:tapley471}
    \begin{aligned}
        \bar{\bm{y}}_k &= \phi (t_k, t_j) \hat{\bm{y}}_j \\
        \bar{\bm{P}}_k &= \phi (t_k, t_j) \bm{P}_j \phi ^{T}(t_k, t_j)
    \end{aligned}
\end{equation}
This step of the algorithm, is often called the \emph{time update} step. 
Assuming that an observation is available at $t_k$
\begin{equation}\label{eq:tapley472}
    \bm{z}_k = \tilde{H}_k \bm{y}_k + \bm{\epsilon}_k
\end{equation}
with $E [\bm{\epsilon}_k] = \bm{0}$ and  
$E [\bm{\epsilon}_k \bm{\epsilon}^{T}_k] = \bm{R}_k \delta _{kj}$, it can be 
shown (see \cite{Tapley2004}) that:
\begin{align}
    K_k &= \bar{\bm{P}}_k \tilde{H}^{T}_{k} 
        \left( \tilde{H}_{k} \bar{\bm{P}}_k \tilde{H}^{T}_{k} 
            + \bm{R}_k \right)^{-1} \label{eq:tapley4711} \\
    \bm{P}_k &= \left( \bm{I} - \bm{K}_k \tilde{H}_{k} \right) \bar{\bm{P}}_k 
        \label{eq:tapley4712} \\
    \hat{\bm{y}}_k &= \bar{\bm{y}}_k + \bm{K}_k 
        \left( \bm{z}_k - \tilde{H}_{k} \bar{\bm{y}}_k \right) \label{eq:tapley4716}
\end{align}
\autoref{eq:tapley4711}, \autoref{eq:tapley4712} and \autoref{eq:tapley4716} are 
collectively labeled the \emph{measurement update} step. The matrix $\bm{K}$ is 
called the \emph{(Kalman) gain matrix}.
The above equations can be used in a recursive fashion to compute the estimate of 
$\hat{\bm{y}}_k$, incorporating the observation $\bm{z}_k$. A flowchart of the 
process is depicted in \autoref{fig:kalman-pod}.

\begin{figure}
    \centering
    \begin{tikzpicture}[align=center, scale=1.0, every node/.style={scale=1.0}]

    \node[rectangle,
        draw,
        text centered,
        rounded corners,
        minimum height=2em,
        align=center,
        thick,
        fill=red!5] 
    (init) at (0,0) {
        Initialize at $t_0$, $i = 0$\\
        Set $t_{i-1} \coloneqq t_0$,\\
        $\bm{Y}^{*}_{t-1} \coloneqq \bm{Y}^{*}_{0}$
        $\hat{\bm{y}}_{i-1} \coloneqq \bar{\bm{y}}_0$
        $\bm{P}_{t-1} \coloneqq \bm{P}_{0}$
    };

    \node[rectangle,
        draw,
        text centered,
        rounded corners,
        thick,
        minimum height=2em,
        align=center,
        fill=gray!5] 
    (readobs) at (0,-2) {
        Read next observation $\bm{Z}_i$ at $t_i$, with $\bm{R}_{i}$
    };

    \node[rectangle,
        draw,
        text centered,
        rounded corners,
        minimum height=2em,
        align=center,
        thick,
        fill=gray!5] 
    (integrate) at (0,-5) {
        Integrate reference trajectory and state transition matrix from 
        $t_{i-1}$ to $t_i$\\
        $
        \begin{aligned}
            \dot{\bm{Y}}^{*} &= F(t, \bm{Y}^{*}(t)) \\
            \dot{\Phi (t,t_0)} &= A(t) \Phi(t,t_0)
        \end{aligned}
        $\\
        with initial conditions\\
        $\bm{Y}^{*}(t_{i-1})$ and $\Phi (t_{i-1},t_{i-1}) = \bm{I}$
    };
    
    \node[rectangle,
        draw,
        text centered,
        rounded corners,
        minimum height=2em,
        align=center,
        thick,
        fill=gray!5] 
    (tupd) at (0,-8) {
        Time Update step\\
        $\bar{\bm{y}}_i = \Phi _{t_i,t_{i-1}}  \hat{\bm{y}}_{i-1}$\\
        $\bar{\bm{P}}_i = \Phi _{t_i,t_{i-1}} \bm{P}_{i-1} \Phi ^{T}_{t_i,t_{i-1}}$
    };

    \node[rectangle,
        draw,
        text centered,
        rounded corners,
        minimum height=2em,
        align=center,
        thick,
        fill=gray!5] 
    (obs) at (0,-10.5) {
        Form observation equations and linearize\\
        $\bm{z}_i = \bm{Z}_i - G(\bm{Y}^{*}_i, t_i)$,\\
        $\tilde{H}_i = \frac{\partial G(t_i, \bm{Y}(t))}{\partial \bm{Y}}\at{\bm{y}^{*}}$\\
        Compute gain matrix $\bm{K}_i$
    };

    \node[rectangle,
        draw,
        text centered,
        rounded corners,
        minimum height=2em,
        align=center,
        thick,
        fill=gray!5] 
    (mupd) at (0,-13.5) {
        Measurement Update\\
        $\hat{y}_i = \bar{y}_i + \bm{K}_i \left( \bm{y}_i - \tilde{H} \bar{y}_i \right)$,\\ \\
        $\bm{P}_i = \left( \bm{I} - \bm{K}_i \tilde{H}_i \right) \bar{\bm{P}}_i$,\\
        $\bm{Y}^{*}_i \coloneqq \bm{Y}^{*}_i + \hat{y}_i$
    };

    \node[rectangle,
        draw,
        text centered,
        rounded corners,
        minimum height=2em,
        align=center,
        thick,
        fill=gray!5] 
    (roupd) at (0,-16.5) {
        Reference orbit update\\
        $t_i \coloneqq t_{i-1}$\\
        $\bm{Y}^{*}(t_i) \coloneqq \bm{Y}^{*}(t_{i-1})$\\
    };

    \node[rectangle,
        draw,
        text centered,
        rounded corners,
        minimum height=2em,
        thick,
        fill=red!5] 
    (more) at (0,-18.5) {
        End of observations ?
    };
    
    \node[diamond,
        draw,
        minimum height=0.5em,
        thick,
        fill=red!10] 
    (end) at (5,-18.5) {
        End
    };

    \draw [-stealth, thick, red] (init) -- (readobs);
    \draw [-stealth, thick] (readobs) -- (integrate);
    \draw [-stealth, thick] (integrate) -- (tupd);
    \draw [-stealth, thick] (tupd) -- (obs);
    \draw [-stealth, thick] (obs) -- (mupd);
    \draw [-stealth, thick] (mupd) -- (roupd);
    \draw [-stealth, thick] (roupd) -- (more);
    \draw [thick] (more.west) -| ($(readobs.west)-(3.5,0)$);
    \node[] (no) at ($(more.west)-(3.1,-.3)$) {No};
    \draw [-stealth, thick] ($(readobs.west)-(3.5,0)$) -- (readobs);
    \node[red] (yes) at ($(more.east)-(-1.4,-.3)$) {Yes};
    \draw [-stealth, thick, red] (more) -- (end);
    %\draw [-stealth, thick] ($(upd.west)-(1.5,0)$) |- (integrate.west) node[midway, above]{$i = i + 1$}; 
    %\draw [-stealth, thick, red] (upd) -- (output);

\end{tikzpicture}
    \caption{Flowchart of the Kalman filter algorithm for orbit determination.}
    \label{fig:kalman-pod}
\end{figure}

Note that the differential equations for the state transition matrix are 
reinitialized at each observation epoch. If observations are introduced as 
scalars, and more than one measurements are available at each epoch, $\Phi (t_i,t_i)$ 
would be set to the unity matrix $\Phi (t_i,t_i) = \bm{I}$ after processing the first 
observation in the epoch, and $\bm{P}$ and $\hat{\bm{y}}$ would only be time updated 
at the next observation epoch $i+1$.

\subsection{Filter Shortcomings}\label{ssec:pod-kalman-filter-shortcomings}
One disadvantage of the sequential algorithm presented here, lies in the fact
that if the true state and the reference state are not close together then the 
linearization assumption leading to \autoref{eq:tapley426} may not be valid and 
the estimation process may diverge (\cite{Tapley2004}). To overcome this problem, 
the \emph{Extended Kalman Filter} algorithm (\autoref{sec:pod-extended-kalman-filter}) 
was used in this Thesis.

Yet another disadvantage is that the state estimation error covariance matrix may 
approach zero as the number as the number of observations becomes large. As 
$\bm{P}_k \to \bm{0}$, gain also approaches zero $\bm{K} \to \bm{0}$, and the 
estimation procedure will become insensitive to further observations. Consequently, 
the estimate will diverge due to either errors introduced in the linearization 
procedure, computational errors, or errors due to an incomplete mathematical model. 
To overcome this problem, process noise often is added to the state propagation 
equations (\cite{Tapley2004}).

In addition to these two problems, the Kalman filter may diverge because of
numerical difficulties associated with the covariance measurement update, given
by \autoref{eq:tapley4712} (\cite{Tapley2004}). This can be the case if the 
covariance matrix loses its properties of symmetry and become non-positive definite 
due to roundoff error (this pitfal is especially possible when a large a priori 
covariance is reduced by the incorporation of very accurate observation data). 
An alternative equation to \autoref{eq:tapley4712} is to use the more numerically 
stable formula introduced by \emph{Bucy} and \emph{Joseph}, which reads
\begin{equation}\label{eq:tapley4719}
    \bm{P}_k = \left( \bm{I} - \bm{K}_k \tilde{\bm{H}} \right) \bar{\bm{P}}_k
    \left( \bm{I} - \bm{K}_k \tilde{\bm{H}} \right)^{T} 
    + \bm{K}_k \bm{R}_k \bm{K}^{T}_k
\end{equation}
More information and a detailed discussion on the stability of relevant approaches 
can be found in \cite{Bierman1977}.
