\section{Introduction}\label{sec:astro-introduction}

Orbital motion for Earth orbiting satellites, is dominated by the Earth's attraction on 
the space vehicle. In the absence of all other forces, and assuming point-like 
gravitational attraction the trajectory would follow a \emph{Keplerian orbit}. While 
this approximation is the starting point for orbital mechanics, it does not suffice 
for modern requirements of \gls{pod}.

The complex force model acting on satellites must be efficiently modeled to derive 
robust results required for geodetic studies. The growing number of satellite missions, 
the technological advance coupled with an ever increasing quantity and quality of 
data sets available, and the quest for understanding complicated 
dynamics and underlying processes (e.g. density variations in high atmosphere) 
affecting and describing satellite motion, have resulted in sophisticated, high accuracy 
models.

Describing (in mathematical terms) and computing trajectories assumes the 
introduction of spatial and temporal reference systems, ones that are defined and 
realized in a precision level appropriate to accommodate modern day \gls{pod} 
analysis. Transforming back and forth between such frames to fit computation and 
modeling requirements is frequent within a \gls{pod} analysis chain, hence 
such transformation should be efficient and precise.

In an effective \gls{pod} analysis, such models have to be carefully considered 
and implemented, a task that raises both analytical and engineering challenges. 

\subsection{Goals of the Chapter}\label{sec:astro-goals}

In this chapter the fundamental concepts of celestial mechanics are presented, in an approach oriented 
towards \gls{pod} for \gls{leo} satellites. Starting from fundamental Keplerian motion, 
orbital perturbations are introduced, as well as the underlying perturbing forces. 
State-of-the-art models for computing induced accelerations are discussed and their 
design, implementation and validation are given in detail. Through a thorough 
examination of the perturbing forces, especially those relevant to geodesy (e.g. 
Earth's gravity field, Earth tides, etc) recent developments in the field are 
reviewed as well as their weaknesses, limitations and strengths.

The goal of this discussion, is to derive efficient and robust algorithms, for computing 
perturbing accelerations, using the most recent models, altering, adapting and re-formulating 
on the way according to application needs and problem constraints.

A presentation of the spatial and temporal reference systems and their realizations follows, 
based on the most recent, international standards. Since orbital motion has to be 
expressed in an inertial reference frame, but a number of modeling techniques (e.g. 
spherical harmonics expansions) usually assume a reference system co-rotating with 
the Earth, transforming between systems is customary in \gls{pod}. The latest 
\gls{iau} standards for such transformations are presented, with the aim of deriving 
algorithmic implementations serving efficiency and robustness.
