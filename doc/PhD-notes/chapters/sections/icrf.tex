\subsection{\gls{icrf}}\label{ssec:icrf}

A reference system is a theoretical concept of coordinates, and includes the time
and the standards necessary to specify the bases for giving positions and motions
in the system (\cite{Gurfil2018}). There are celestial and terrestrial reference systems.

The \gls{icrs} is based on the theory of relativity, observations of distant 
extragalactic radio sources, and a fixed origin, thus it is essentially ``fixed'' 
in space (since there is no apparent motion of distant sources). Two distinct 
systems are defined, the \gls{bcrs}, centered at the barycenter of the solar 
system and the \gls{gcrs}, centered at the geocenter. The \gls{gcrs} is defined 
such that its spatial coordinates are not kinematically rotating with respect 
to the \gls{bcrs} (\cite{Gurfil2018}). The axes of the \gls{gcrs} are considered 
non-rotating in the Newtonian absolute sense, but the geocenter is accelerated 
within the solar system, thus this system is in reallity a ``quasi-inertial'' 
system.

The \gls{icrs} is materialized by a celestial reference frame called the \gls{icrf}, 
consisting of the precise coordinates of extragalactic objects, mostly quasars. The 
necessity of keeping the reference directions fixed and the continuing improvement 
in the source coordinates requires regular maintenance of the frame.

The \gls{iers} Earth Orientation Parameters provide the permanent tie of the \gls{icrf}
to the \gls{itrf}. They describe the orientation of the \gls{cip}
in the terrestrial system and in the celestial system (polar coordinates $x$,
$y$; celestial pole offsets $\delta \psi$, $\delta \epsilon$) and the orientation 
of the Earth around this axis (UT1-UTC), as a function of time (\cite{iers2010}). 
This tie is available daily with an accuracy of $\pm\SI{0.1}{\milli\larcsecond}$ in the 
\gls{iers} publications.
