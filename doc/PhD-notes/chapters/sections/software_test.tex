\section{Software Tests And Orbit Determination}\label{sec:jason3-pod}

Using the software designed and implemented in the framework of this Thesis, a fundamental 
orbit determination processing pipeline was setup, to analyze \gls{jason}-3 
\gls{doris} measurements. The different, individual parts put together to achieve 
this task have already been discussed in the previous chapters, including both 
theoretical implications and implementation details. \autoref{table:processing-options} 
lists the models and parameters used for the test suite.

\gls{jason}-3 was chosen 
to act as the test bed since it is an on-going, modern satellite mission, extensively 
documented, and its attitude determination can be performed via quaternions 
(see \autoref{ssec:jason3-attitude}). A drawback however, is its \gls{uso} susceptibility 
to \gls{saa}-induced effects (see \autoref{sec:saa}).

\begin{landscape}
\begin{table}[h!]
    %\tiny
    \centering
    \resizebox{.8\hsize}{!}{%
    \begin{tabularx}{\hsize}{p{3cm}XX}
        \toprule
        \textbf{Description} & \textbf{Model Used} & \textbf{Model Parameters} \\
        \hline
        Gravity Model & CNES/GRGS RL04 \gls{tvg} (\cite{Lemoine2019}) & up to $120^{th}$ degree and order \\
        Third Body    & Sun and Moon & Ephemeris DE421 (\cite{Folkner2009}) \\
        Solid Earth \newline Tides & IERS2010 (\cite{iers2010}) & step-1 and step-2 corrections \\
        Ocean Tide    & FES2014 (\cite{Lyard2021}) & Only major components included \\
        Atmospheric Drag & According to \autoref{sssec:atmospheric-drag}, using the NRLMSISE-00 (\cite{nrlmsise00}) model & Flux data from \href{https://celestrak.org/}{CelesTrack} (\cite{Vallado2013}) \\
        Solar Radiation \newline Pressure & According to \autoref{sssec:solar-radiation-pressure} using the macromodel provided by \cite{Cerri2022} & Occultation due to Earth modelled using the conical model (\cite{Zhang2019}) \\

        Site Positions \newline (Terrestrial RF) & DPOD2020 (\cite{Moreaux2023}) realized by the \gls{ids} Combination Center & version \texttt{dpod2020\_01}; beacon positions and meta-data extracted from SINEX \\
        Celestial/Inertial RF & \gls{gcrf} (\cite{iers2010} and \autoref{ssec:itrs-to-gcrs}) & \gls{iau} 2000/2006 resolutions, using the \gls{cio}-based transformation with full series for the polar motion (\gls{iau} 2006 precession and \gls{iau} 2000A nutation) \\
        Earth Orientation \newline Parameters & \gls{iers} 2020 C04 products (see \footnote{\url{https://datacenter.iers.org/data/2/message_471.txt}}) & interpolated as described in \autoref{ssec:eop-interpolation} \\
        Orbit Integration & Adams-Bashforth-Moulton, variable step variable order integrator (\autoref{ch:orbit-integration}) & Integration performed in \gls{tai}; equations of motions solved for including the system of variational equations (\autoref{sec:pod-variational-equations}) \\
        Filtering & Extended Kalman Filter algorithm (\autoref{sec:pod-extended-kalman-filter}), no process noise & Parameter set includes: 
        \begin{itemize}[topsep=0pt,partopsep=0pt,itemsep=1pt,parsep=1pt]
          \item[] satellite state, 
          \item[] atmospheric drag coefficient (1/\si{revolution}),
          \item[] solar radiation pressure coefficient (1/\si{revolution}),
          \item[] beacon frequency offest (per pass) as bias,
          \item[] per-beacon hydrostatic zenith path delay (per pass)
        \end{itemize} \\
        Troposphere Modelling & GPT3/VMF3 (\cite{Landskron2018}), see \autoref{ssec-tropospheric-correction} & wet zenith path delay estimated per beacon, per pass\\
       \bottomrule
    \end{tabularx}}
    \caption{Processing options used for \gls{jason}-3 orbit determination.}
    \label{table:processing-options}
\end{table}
\end{landscape}

The large number of data and product files to be processed, along with the complexity of 
the models, corresponding input parameters and user options needed to properly determine 
the analysis procedure, poses a challenging task. An efficient but on the same time 
user-friendly way of setting such input parameters must be devised. The solution adopted 
for the software package described here, is to archive such parameters in \emph{configuration files} 
using the \texttt{YAML} (\url{https://yaml.org/}) format and ``feed'' them to the 
program to drive the analysis scheme. \texttt{YAML} is a data serialization
language with the important advantage of being human-friendly. An example of such 
a file is depicted in \autoref{tab:config-example}.

\begin{adjustbox}{center, max width=\linewidth , fbox=0.5pt, captionbelow={Configuration file example.}, float=table}
\begin{BVerbatim}
---
data:
  doris-rinex: data/Jason-3/ja3rx22248.001
  sp3: data/Jason-3/ssaja320.b22238.e22258.DG_.sp3.001

reference-frame:
  station-coordinates: data/dpod2020_01.snx

eop-info:
  eop-file: data/eopc0420.1962-now

gravity:
  model: data/gfc/EIGEN-GRGS.RL04.MEAN-FIELD.gfc2
  degree: 120
  order: 120

troposphere:
  vmf3:
    grid: data/2022248.v3gr_d

force-model:
  atmospheric-drag:
    density-model: nrlmsise00
    atmo-data-csv: data/SW-All.csv
    Cd-apriori: 2e0
  srp:
    Cr-apriori: 1.5e0
...
\end{BVerbatim}
\label{tab:config-example}
\end{adjustbox}

\begin{figure}
    \centering
    \includegraphics[height=.5\textheight,keepaspectratio]{state}
    \caption{Satellite state (in \gls{ecef} RF) estimated for one day of \gls{jason}-3 orbit}
    \label{fig:state}
\end{figure}

To validate and check the results obtained for the satellite state estimates, 
\gls{jason}-3 sp3 files were retrieved from \gls{cddis} (\cite{Noll2010}), recording 
state (in cartesian components) for the day of interest, with a sampling rate of 
\SI{1}{\second}. These solutions are computed by \gls{cnes}'s new-generation ground segment 
called SSALTO\footnote{https://www.aviso.altimetry.fr/en/newsletter07/ssalto-a-new-ground-segment-for-a-new-generation-of-altimetry-satellites.html} 
and are produced using \gls{doris} and \gls{gps} observations. Documentation 
for the data analysis procedure and the models used therein, is available at 
\url{ftp://ftp.ids-doris.org/pub/ids/data/POD_configuration_POEF.pdf}. Note that 
according to the \gls{pod} specifications used by SSALTO, a large number of models 
and options are different from the ones used in the test-suite program listed in 
\autoref{table:processing-options}.

Despite the fact that inclusion of \gls{gps} data may introduce inconsistencies 
when compared to a \gls{doris}-only solution, the high sampling rate of these 
product files makes extrapolation of reference trajectories trivial and only introduces 
small errors (see \autoref{sec:integrator-validation}). The procedure for computing 
reference solutions (state estimates) included the following steps:
\begin{enumerate}
  \item \label{en:st1} get (next) epoch $t$ for which the test-suite has produced a state estimate
  \item find the record in the reference sp3 file closest to $t$, $t_0$
  \item extrapolate the state from $t_0$ to $t$; go to \autoref{en:st1} and repeat 
    until a reference state is computed for all $t$
\end{enumerate}

Results of the comparison are depicted in \autoref{fig:statediffs}.
\begin{figure}
    \centering
    \includegraphics[height=.5\textheight,keepaspectratio]{statediffs}
    \caption{Discrepancies of estimated state against the orbits estimated at \gls{cnes}/AALTO for one day of \gls{jason}-3 orbit (in \gls{ecef} RF). Red colored regions depict passing through shadow.}
    \label{fig:statediffs}
\end{figure}

\begin{table}[h!]
    \centering
    \begin{tabularx}{\textwidth}{cccccc}
        \toprule
        \textbf{Component} & \textbf{Mean} & \textbf{Std. deviation} & \textbf{Max Value} & \textbf{Min Value} & \textbf{Units}\\
        \hline
        $x$   &  +0.17 & $\pm$ 0.64 & 2.09 & -1.59 & \si{\metre}\\
        $y$   &  +0.00 & $\pm$ 0.57 & 1.33 & -2.01 & \si{\metre}\\
        $z$   &  +0.86 & $\pm$ 0.72 & 3.52 & -0.17 & \si{\metre}\\
        $v_x$ &  -0.03 & $\pm$ 0.58 & 2.19 & -1.48 & \si{\milli\metre\per\second}\\
        $v_y$ &  +0.01 & $\pm$ 0.44 & 1.11 & -1.09 & \si{\milli\metre\per\second}\\
        $v_z$ &  +0.03 & $\pm$ 0.65 & 2.29 & -2.22 & \si{\milli\metre\per\second}\\
       \bottomrule
    \end{tabularx}
    \caption{Details of discrepancies between estimated state against the orbits estimated at \gls{cnes}/AALTO for one day of \gls{jason}-3 orbit, depicted in \autoref{fig:statediffs}.}
    \label{table:statediffs}
\end{table}

\begin{figure}
    \centering
    \includegraphics[height=.25\textheight,keepaspectratio]{t4-resVsTimeRaw}
    \caption{\gls{doris} residuals computed from one day of \gls{jason}-3 orbit determination.}
    \label{fig:ResVsTimeRaw}
\end{figure}

\begin{figure}
    \centering
    \includegraphics[height=.5\textheight,keepaspectratio]{t4-resVsTime}
    \caption{\gls{doris} residuals computed from one day of \gls{jason}-3 orbit determination.}
    \label{fig:ResVsTime}
\end{figure}

\begin{figure}
    \centering
    \includegraphics[height=.5\textheight,keepaspectratio]{t4-resVsEle}
    \caption{\gls{doris} residuals w.r.t. elevation angle computed from one day of \gls{jason}-3 orbit determination.}
    \label{fig:ResVsEle}
\end{figure}

\begin{figure}
    \centering
    \includegraphics[height=.5\textheight,keepaspectratio]{t4-obsbeacons}
    \caption{Number of measurements performed by the on-board \gls{jason}-3 \gls{doris} receiver per ground beacon.}
    \label{fig:obsPerBeacon}
\end{figure}

\begin{figure}
    \centering
    \includegraphics[height=.5\textheight,keepaspectratio]{t4-dynamicparams}
    \caption{Estimated dynamic orbit parameters (drag and solar radiation pressure corfficients, $C_d$ and $C_r$), for one day of \gls{jason}-3 orbit determination.}
    \label{fig:EstimDynamicParams}
\end{figure}
