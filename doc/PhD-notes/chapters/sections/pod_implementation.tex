\section{Implementation}\label{sec:pod-implementation}

\subsection{Solution of Variational Equations}\label{ssec:vareq-implementation}
The computation of the differential equation system comprisig the variational 
equations and its subsequent solution is a demading task in \gls{pod}, posing 
challenges both in efficiency and in precision. A large number of computations 
must be performed, mainly including evaluation of partial derivatives. Analytic 
formulas for the latter are rather complicated making their implementation error 
prone.

Unfortunately, testing and validation of software designed to tackle this problem 
is cumbersome, and based on trial-and-error. To test the implemetation, a large 
number of tests was performed, starting from a simple, two-body formulation and 
gradually increasing complexity, checking each step with respect to the previously 
estimated solution. The gradual increase of complexity was expected to be paired with 
a increase in solution accuracy.

The solution of the variational equations is based on numeric integration, via an 
Adams-Bashforth-Moulton \gls{pece} algorithm, with varying step size and order. The 
fundametals and implementation of this method is already discussed in \autoref{sec:multistep-methods} 
and \autoref{sec:integrator-implementation}.

\subsection{Exteded Kalman Filter Implementation}\label{ssec:ekf-implementation}
For the purposes of the current Thesis, a software package was designed and 
implemented to perform orbit determination using the Extended Kalman Filter 
algorithm (see \autoref{sec:pod-extended-kalman-filter}). This algorithm was 
chosen due to a number of factors, including
\begin{itemize}
  \item robust and efficient estimation algorithm,
  \item the filter's ability to utilize state models for dynamic processing,
  \item compensation for dynamic model inaccuracy (\emph{process noise}),
  \item estimation of varying state and easily adaptable to (near) real-time scenarios; 
    \cite{Zhou2020} use an extended Kalman filterig algorithm to determine in real time 
    the orbit of the HY2A \gls{leo}, using \gls{doris} and spaceborne \gls{gps} 
    observations
  \item adaptability and fine tuning of statistical properties of process noise, 
    and measurement error to design an ``optimal'' filter,
  \item widely used across various engineering fields and under growing progress
\end{itemize}

To address the issue raised in \autoref{ssec:pod-kalman-filter-shortcomings}, 
concerning the possible divergence of the estimates due to numerical instabilities, 
the \emph{Bucy} and \emph{Joseph} formula (\autoref{eq:tapley4719}) is adopted.

Kalman filtering offers great verstility in handling stochastic and statistical 
properties of parameters, state and system dynamics. Special care was taken in 
order to preserve this versatility and transfer it to the user, through the control
of relevant options by means of (user) input. Fundamental stochastic properties of 
the analysis are set via a user-friendly configuration file, including but not 
limited to a-priori sigmas (standard deviation) for all parameters considered and 
observation statistics. Numerous tests have been performed with combinations of 
different values to derive sensible defaults, an option alsoprovided to the user.

The implementation processes one observation at a time, so as to take full advantage 
of the scalar computations of various formulas included in the previous chapters. 
This makes the algorithm more efficient and less demading on memory resources.

Filter design is as generic as possible, so as to allow (except from varying user 
input descussed above) reusage in different parts of the software package, to perform 
different tasks. E.g. a variant of the same algorithm performs linear regression 
to estimate relative frequency offsets biases (see \autoref{ch:doris}).

It is worth noting that most state-of-the-art software packages for \gls{pod} using 
DORIS observations, use the method of least squares for parameter estimation. This 
is true e.g. for the Bernese GNSS Software (\cite{Dach2015}), GINS (\cite{Gins2013}) and 
GEODYN (\cite{Geodyn2015}), all of which are packages used by Aalysis Centers 
actively contributing to the \gls{ids}. This choice has to do with the fact that 
at the time these packages were first developed, Least Squares was the prevailing 
method for parameter estimation, while Kalman filtering had miimum intrusion into the 
geodetic community. Nevertheless, using alternate but equally robust techniques 
can provide insight and drive further research and progress both within the DORIS 
community (via the combination of Analysis Centers individual results) and in the 
scientific world in general.
