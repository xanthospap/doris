\section{DORIS RINEX}\label{sec:doris-rinex}
With the adoption of the \gls{doris} DGXX receivers, first installed onboard the 
Jason-2 satellite, signal tracking on seven different channels simultaneously became 
possible, with synchronous dual frequency phase and pseudo-range measurements 
(\cite{Mercier2010}). This development made it possible for \gls{doris} data to 
be described in a manner similar to \gls{gnss} data, and hence an extension of the 
RINEX 3.0 format (\cite{RINEX305}) was defined and adopted for \gls{doris} observations to be 
recorded and published in. One major advantage of these new measurement data is 
that they are available with a very short latency (older data needed to be pre-processed 
before published). The new data exchange format is expected to further aid analysis 
development, since it allows each analysis center to be independent from \gls{cnes}
data preprocessing, as users have access to synchronous phase and 
pseudorange measurements (\cite{Cerri2011}).

\subsection{General Format Description}
The \gls{doris} RINEX format consists of one ASCII file containing both space based 
and meteorological data collected at \gls{doris} stations and relayed by satellites.
It bears close resemblance to the GNSS RINEX Version 3 (\cite{RINEX305});
data files consist of a header section and a data section. The first contains 
global information for the entire file, while the latter contains the actual 
observations and a date tag, keeping strict chronological order.
Observation types recorded in the DORIS RINEX files are the given in 
\autoref{table:doris-rinex-observation-types}.

\gls{doris} is basically running on its own proper time which is constantly linked 
to \gls{tai}. Time tags are given in instrument time, and clock offset values are 
provided between instrument time and \gls{tai}.

\begin{table}[h!]
  \centering
  \begin{tabular}{|c c c|}
  \hline
  Descriptor & Observation Type & Units \\
  \hline
  \texttt{L} & carrier phase observation & cycles \\
  \texttt{C} & pseudo-range observation & \si{\metre}\\
  \texttt{W} & power level received at each frequency & \si{\dBm} \\
  \texttt{F} & relative frequency offset of the receiver’s oscillator $\frac{f-f_0}{f_0}$ & $10^{-11}$ \\
  \texttt{P} & ground pressure at the station & 100 \si{\pascal} (\si{\milli\bar}) \\
  \texttt{T} & ground temperature at the station & \si{\degreeCelsius} \\
  \texttt{H} & ground humidity at the station & \si{\percent} \\
  \hline
  \end{tabular}
  \caption{\gls{doris} RINEX observation types.}
  \label{table:doris-rinex-observation-types}
\end{table}
A detailed description of the data files, can be found in \cite{DORISRNX3}.

\iffalse
\subsection{Receiver Clock Offsets}
DORIS RINEX contain a ``receiver clock offset'' \(\tau_{r_{offset}}\) (as an 
optional field) at the header record of every epoch. E.g. the line
\begin{verbatim}
    ...
    > 2020 01 01 00 00 35.099949800  0  1       -3.248177132 0
    ...
\end{verbatim}
records a receiver clock offset of \SI{35.099949800}{\second} followed by the 
receiver clock offset flag, which in this case is \num{0}. In such record 
lines, the date (given as YYYY MM DD HH MM SS.) is given in the on-board time 
scale (aka proper time of the receiver) and the conversion to the \gls{tai} 
time scale is obtained by adding the receiver clock offset.

\textit{Depending on the version of the RINEX file, this offset can have been either 
computed by the DORIS-DIODE navigator, or through a post-fit processing using PANDOR. 
In the first case, the file ends with ``.001'' whereas in the second with ``.010''. 
More information on the computation of (\(\tau_{r_{offset}}\) are provided in 
\cite{lemoine-2016}.}

\subsection{Types of DORIS measurements}\label{ssec:types-of-doris-measurements}

\subsubsection{Relative Frequency Offset}\label{sssec:relative-frequency-offset}
\gls{doris} RINEX files (usually) contain a measurement type labeled ``F'' (should be 
recorded in the field \verb|SYS/#/OBS TYPES| at the RINEX header). This measurement 
type is provided for every epoch and is a measure of the relative frequency 
offset of the receiver's oscillator (aka \(\frac{f-f_0}{f_0}\) \num{10e-11}).
Example (rinex header):

\begin{adjustbox}{max width=\linewidth , fbox=0.5pt}
\begin{BVerbatim}
        0.9768        0.0001        0.0011                  CENTER OF MASS: XYZ
D   10  L1  L2  C1  C2  W1  W2   F   P   T   H              SYS / # / OBS TYPES
  2018    01    01    00    00   28.8526816     DOR         TIME OF FIRST OBS
...
                                                            END OF HEADER
> 2018 01 01 00 00 32.589951370  0  3       -3.737269708 0
D01  -2104936.480    -1241282.301   131837622.17412 131837685.14912      -124.650 7
         -114.150 7      5643.911        1025.000 1         6.000 1        85.000 1
D02  -1063469.796    -1862572.573   136575183.93813 136575216.89713      -118.350 7
         -109.250 7      5643.911        1014.000 0       -18.800 0        73.000 0
D03   -826480.354    -2642364.157   148404267.53014 148404098.13114      -115.200 7
         -104.700 7      5643.911         934.000 0       -21.500 0        72.000 0
...                                                            
\end{BVerbatim}
\end{adjustbox}
\fi
