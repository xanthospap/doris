\section{Implementation of the \gls{doris} Observation Equation}\label{sec:doris-observation-equation-implementation}
The observation equation formed to process the \gls{doris} data, is based on 
\autoref{eq:lem17}. Consequently, two parts are computed, $v_{measured}$ which 
represents the ``observed'' or ``measured'' relative velocity between the receiver 
and the transmitter, and $v_{theo}$ which is the ``computed'' or ``theoretical'' 
counterpart. In this way, during the processing phase, all quantities that do not 
need adjustment can be placed on the ``measured'' side of the equation 
(\cite{Lemoine2016}). A short discussion follows, describing the implementation 
of the \gls{doris} observation equation in the software designed for this Thesis.

\subsection{Coordinate and Proper Time}\label{ssec:coordinate-proper-time}
\gls{tai} is used as coordinate time; to transform RINEX observation time (given in 
proper time $\tau$) to coordinate time $t$, the \emph{receiver clock offset} values 
are used, extracted from the RINEX file (one value per observation block).

Hence, if an observation block is tagged at proper time $\tau _i$ at the RINEX 
file, and the receiver clock offset for this block is $\Delta \tau _i$ (again from 
RINEX), then the coordinate time of the event in \gls{tai} is computed as
\begin{equation}
  t^{TAI}_i = \tau _i + \Delta \tau _i
\end{equation}
According to \cite{Lemoine2016} however, there is no need to make a time conversion 
for the time interval of the Doppler count, i.e. the term $\Delta \tau _r$ in 
\autoref{eq:lem17}, due to the time-tagging method used in \gls{doris} RINEX.

\subsection{Receiver Emitter Geometric Distance}
The geometric distances between the emitter and the receiver, $\rho _1$ and $\rho _2$, 
when computed, are corrected for the aberration effect, i.e. the slight displacement of 
the emitter due to Earth's rotation between signal emission and signal reception at
the receiver. The algorithm for this correction is described in \autoref{sssec:doris-aberration}.
Note that checks performed have validated that more than one iterations are practically 
redundant.

\subsection{Relativistic Correction}\label{ssec:relativistic-correction}
\autoref{eq:lem17} contains a relativistic correction term, $\Delta u_{REL}$. This 
correction is split into two parts, $\Delta u_{REL_c}$ the part containing the 
clock correction and $\Delta u_{REL_r}$, containing the effect of the travel 
path (see \autoref{eq:lem14}). In the implementation followed for this Thesis, 
only the $\Delta u_{REL_c}$ part is considered (\autoref{eq:lem14a}).

For the receiver part (with subscripts $r$), the respective quantities in 
\autoref{eq:lem14a} are given by
\begin{equation}
  \begin{aligned}
    V^2_r &= \norm{\bm{v}_{ecef}}^2 \\
    U_r   &= \frac{\mu _{\Earth}}{\norm{\bm{r}_{ecef}}} \cdot \left( 1 - 
      \left(\frac{\alpha}{\norm{\bm{r}_{ecef}}}\right) ^2 \cdot J_2 \cdot
        \frac{3 \cdot \sin ^2{\phi} -1}{2} \right)
  \end{aligned}
  \label{eq:potential-receiver}
\end{equation}
where $\bm{r}_{ecef}$ and $\bm{v}_{ecef}$ are the position and velocity of the 
satellite at the given instant, in the terrestrial reference frame, 
aka \gls{itrf}. Note that the Earth's oblateness cannot be ignored here (see 
discussion in \autoref{sssec:doris-geopotential}).

For the emitter, potential computation is further simplified. Since $V_e = 0$, 
the potential is computed as 
\begin{equation}
  U_e = \frac{\mu _{\Earth}}{\norm{\bm{r}_{ecef}}}
  \label{eq:potential-emitter}
\end{equation}
where $\bm{r}_{ecef}$ is the position vector of the beacon in \gls{itrf}.

In \autoref{eq:lem17}, relativistic corrections must be ``differentiated'' between 
two consecutive epochs (used for the Doppler count)
\begin{equation}\label{eq:drel-diff}
  \begin{aligned}
    \Delta v_{REL} &= \frac{1}{c} 
      \left( 
        \left[ U_r - U_e + \frac{V^2_r - V^2_e}{2} \right]\at{t=t_i} 
        - \left[ U_r - U_e + \frac{V^2_r - V^2_e}{2} \right]\at{t=t_{i-1}} 
      \right) \\
      &= \frac{1}{c} \cdot \left( 
        U_r\at{t_i} - U_r\at{t_{i-1}} +  \frac{V_r\at{t_i}-V_r\at{t_{i-1}}}{2} 
        \right) \si{\m\per\s}
  \end{aligned}
\end{equation}

\subsection{Receiver Proper Frequency $f_{r_T}$}\label{ssec:receiver-true-proper-frequency}
In the implementation, ``smoothed'' values of RINEX-provided $\Delta f_r / f_{r_N}$ estimates 
are used to compute the receiver's proper frequency, given by \autoref{eq:frt-rinex}.
In a first RINEX pass, $F_{t_i}$ are used to estimate a linear model spanning the 
whole RINEX time span. These smoothed values are then used to compute relative 
frequency offsets at the observation epochs (see discussion in \autoref{sssec:true-proprtfrequency-of-the-receiver}). 

\subsection{Ionospheric Correction}\label{ssec:iono-correction}
For each observation in the RINEX file, the ionospheric correction is computed 
and applied to the \SI{2}{\GHz} measurement (as described in \autoref{ssec:iono-correction}), 
thus transforming it to an ``iono-free'' measurement.

When applied to \autoref{eq:lem17}, ``differenciation'' of the ionospheric 
delays computed from \ref{eq:iono-delay-cycles} must be performed, affecting two 
observations (the same ones used to derive the Doppler count). Hence, the term 
$\Delta v_{IONO}$ appearing in \autoref{eq:lem17} is
\begin{equation}
  \Delta v_{IONO} [\si{\m \per \s}] = 
    %\tikz[baseline]{
    %  \node[fill=blue!20,anchor=base](en1){$\frac{c}{f_{e_N}}$};
    %}
    \frac{c}{f_{e_N}}
    \cdot 
    \frac{\delta_{ION}\at[\big]{t=t_{i-1}} 
    - \delta_{ION}\at[\big]{t=t_i}}{\Delta \tau}
  \label{eq:dion-diff}
\end{equation}

\subsubsection{2GHz and Iono-Free Phase Center}\label{sssec:2ghz-ionofree-pco}
When using the transformed, ``iono-free'' phase measurement, a geometric correction 
has to be applied to get to the respective beacon (and satellite antenna) phase 
center. This offset is computed as
\begin{equation}\label{eq:ionf-pco}
  \bm{r}_{iono-free} = \bm{r}_{\SI{2}{\GHz}} + \frac{\bm{r}_{\SI{2}{\GHz}} 
    - \bm{r}_{\SI{400}{\MHz}}}{\gamma - 1}
\end{equation}
where $\bm{r}_{\SI{2}{\GHz}}$ and $\bm{r}_{\SI{400}{\MHz}}$ are the 
eccentricities for the \SI{2}{\GHz} and the \SI{400}{\MHz} carriers respectively 
from the beacon antenna phase center (given at \cite{DORISGSM}, Sec. 5.2.1).

Note that in \autoref{eq:ionf-pco}, the eccentricity vector $\bm{r}_{iono-free}$ 
is in a topocentric reference frame. Hence, to compute the \gls{ecef} coordinates of the 
iono-free phase center, given the (cartesian) \gls{ecef} coordinates of the beacon's 
\gls{arp} $\bm{r}_{arp}$ (see \ref{ssec:beacon_coordinates})
\begin{equation}
  \bm{r}^{ecef}_{iono-free} = \bm{r}_{arp} + \bm{R}^T \cdot \bm{r}_{iono-free}
  \label{eq:arp-to-if-pc}
\end{equation}
where $\bm{R}$ is the cartesian-to-topocentric rotation matrix, computed 
at $\bm{r}_{arp}$.

In accordance to the beacons, a similar geometric reduction must be applied 
at the satellite's end, to correct for the discrepancy between the \SI{2}{\GHz} 
and the ``iono-free'' phase center
\begin{equation}
  \bm{r}^{satf}_{iono-free} = \bm{r}^{satf}_{\SI{2}{\GHz}} + 
    \frac{\bm{r}^{satf}_{\SI{2}{\GHz}} - 
    \bm{r}^{satf}_{\SI{400}{\MHz}}}{\gamma - 1}
\end{equation}
where the superscript $satf$ denotes the \emph{satellite-fixed} body/reference 
frame. On-board satellite antenna phase center offset values can be found in 
\cite{DorisSatModels}.

\subsection{Tropospheric Correction}\label{ssec-tropospheric-correction}
The GPT3/VMF3 (\cite{Landskron2018}) model is used to handle tropospheric refraction. 
The hydrostatic zenith delay $zd_{hydrostatic}$, is computing via the ``refined'' 
\emph{Saastamoinen} model (\cite{Davisetal85} and \cite{Saastamoinen72}). 
The corresponding value for the wet delay, $zd_{wet}$ is estimated during the 
analysis, \emph{per beacon and per pass}, using an initial value provided by 
\cite{Askneetal87}.

Using the mapping function and the zenith delay, the tropospheric 
delay for an observation at $t=t_i$ is given by
\begin{equation}\label{eq:tropo-delay}
  \delta _{TRO} [\si{\m}] = zd_{hydrostatic} \cdot mf_{hydrostatic} + zd_{wet}\at{t=t_i} \cdot mf_{wet}
\end{equation}

The tropospheric correction term in \autoref{eq:lem17}, is actually the ``time-differenced'' 
tropospheric delay between two measurements (the same ones used to derive the 
Doppler count), given in \si{\m \per \s}. That is:
\begin{equation}\label{eq:dtrop-diff}
  \begin{aligned}
    \Delta v_{TROPO} [\si{\m \per \s}] 
      &= \left( \delta _{TRO} \at{t=t_{i-1}} - \delta _{TRO} \at{t=t_{i}} \right) / \Delta \tau\\
      &= \left( \left[ zd_{h} \cdot mf_{h} + zd_{w}\at{t=t_{i-1}} \cdot mf_{w} \right]\at[\big]{t=t_i} - 
        \left[ zd_{h} \cdot mf_{h} + zd_{w}\at{t=t_{i-1}} \cdot mf_{w} \right]\at[\big]{t=t_{i-1}} \right) \ \Delta \tau
  \end{aligned}
\end{equation}

Note that in the above equation the same value $zd_{w}\at{t=t_{i-1}}$ 
for the wet part of the zenith delay is used, that is the best estimate prior to 
incorporating the (new) measurement at $t=t_i$.

The parameter $zd_{w}$ is estimated using no constraints and a simple white 
noise model (no process noise). Since it is an estimated parameter, 
the (partial) derivative of the observation equation w.r.t to this parameter, 
is required
\begin{equation}\label{eq:partials-zwd}
  \frac{\partial v_{theo}}{\partial zd_{w}} = \frac{mf_{w}\at[\big]{t=t_i} 
    - mf_{w}\at[\big]{t=t_{i-1}}}{\Delta \tau}
\end{equation}

\iffalse
\subsubsection{Mapping Functions}
The GPT3/VMF3 (\cite{Landskron2018}) model is used to handle tropospheric refraction. 
For the hydrostatic zenith delay $zd_{hydrostatic}$, the ``refined'' 
\emph{Saastamoinen} model (\cite{Davisetal85} and \cite{Saastamoinen72}). 
The corresponding value for the wet delay, $zd_{wet}$ is estimated during the 
analysis, \emph{per beacon and per pass}, using an initial value provided by 
\cite{Askneetal87}.

Given the elevation angle $el$ and the beacon's ellipsoidal coordinates 
$\bm{r}=\begin{pmatrix} \lambda & \phi & h\end{pmatrix}$, we use the \texttt{gpt3} 
$\ang{5} \times \ang{5}$ grid to interpolate the $ah$ and $aw$ coefficients; 
we then compute the mapping function (with height correction) $mf_{wet}$ and 
$mf_{hydrostatic}$.

\subsubsection{Zenith Delays}
For the hydrostatic zenith delay $zd_{hydrostatic}$ we use the ``refined'' 
\emph{Saastamoinen} model (\cite{Davisetal85} and \cite{Saastamoinen72}). 
The corresponding value for the wet delay, $zd_{wet}$ is estimated during the 
analysis, \emph{per beacon and per pass}, using an initial value provided by 
\cite{Askneetal87}.

\subsubsection{Tropospheric Correction}
Using the mapping function and the zenith delay, we derive the tropospheric 
delay for an observation at $t=t_i$, given by:
\begin{equation}
  \delta _{TRO} [\si{\m}] = zd_{hydrostatic} \cdot mf_{hydrostatic} + zd_{wet}\at{t=t_i} \cdot mf_{wet}
  \label{eq:tropo-delay}
\end{equation}

The tropospheric correction term in \ref{eq:lem13}, is actually the ``time-differenced'' 
tropospheric delay between two measurements (the same ones used to derive the 
Doppler count), given in \si{\m \per \s}. That is:
\begin{equation}
  \begin{aligned}
    \Delta v_{TROPO} [\si{\m \per \s}] 
      &= \left( \delta _{TRO} \at{t=t_{i-1}} - \delta _{TRO} \at{t=t_{i}} \right) / \Delta \tau\\
      &= \left( \left[ zd_{h} \cdot mf_{h} + zd_{w}\at{t=t_{i-1}} \cdot mf_{w} \right]\at[\big]{t=t_i} - 
        \left[ zd_{h} \cdot mf_{h} + zd_{w}\at{t=t_{i-1}} \cdot mf_{w} \right]\at[\big]{t=t_{i-1}} \right) \ \Delta \tau
    \label{eq:dtrop-diff}
  \end{aligned}
\end{equation}

Note that in the above equation we use the same value $zd_{w}\at{t=t_{i-1}}$ 
for the wet part of the zenith delay, that is the best estimate prior to 
incorporating the (new) measurement at $t=t_i$.

The parameter $zd_{w}$ is estimated using no constraints and a simple white 
noise model (no process noise). Since it is an estimated parameter, we need 
to compute the (partial) derivative of the observation equation w.r.t to it, 
that is:
\begin{equation}
  \frac{\partial v_{theo}}{\partial zd_{w}} = \frac{mf_{w}\at[\big]{t=t_i} 
    - mf_{w}\at[\big]{t=t_{i-1}}}{\Delta \tau}
  \label{eq:partials-zwd}
\end{equation}
\fi
