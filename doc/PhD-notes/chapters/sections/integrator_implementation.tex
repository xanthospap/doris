\section{Integrator implementation}\label{sec:integrator-implementation}

According to \cite{Shampine1975}, use of Adams methods are most advantageous for 
problems in which:
\begin{itemize}
    \item function evaluations are expensive,
    \item mnoderate to high accuracy is requested,
    \item many output points are required
\end{itemize}

\cite{Shampine1975}:
The overhead in these codes is rather high, partly because of the methods
and partly because of the features they provide. For example, DE detects
and deals with discontinuities; it detects and deals with moderate stiffness;
it detects and tells the user of severe stiffness; it detects requests for very
high accuracy and switches on propagated roundoff controls; it detects
requests for more accuracy than is possible on the machine being used and
tells the user what is possible; it is, for practical purposes, independent
of the number and location of output points; it monitors the work being
expended; it allows the user to change direction without restarting; and
it is extremely efficient in terms of function evaluations. 

\subsection{The Predictor}\label{ssec:integrator-predictor}
The $k$th order Adams-Bashforth predictor at $x_n$ is defined by the expression 
\begin{equation}\label{eq:sg51}
  y_n + \int_{x_n}^{x} P_{k,n}(t) \,dt
\end{equation}
where $P_{k,n}(x)$ satisfies the interpolation conditions
\begin{equation}\label{eq:sg51b}
  P_{k,n}(x_{n+1-j}) = \bm{f}_{n+1-j} \text{ for } j=1,\dots ,k
\end{equation}
\autoref{eq:sg51} is used to predict both the solution and its derivative at 
$x_{n+1}$
\begin{equation}
  \begin{aligned}
    p_{n+1}    &= y_n + \int_{x_n}^{x_{n+1}}  P_{k,n} (t) \,dt \label{eq:sg52a}\\
    {p'}_{n+1} &= P_{n,k}(x_{n+1}) \label{eq:sg52b}
  \end{aligned}
\end{equation}

Obviously we must determine an easy way to integrate the interpolating
polynomial. We start with the polynomial in divided difference form
\begin{equation}\label{eq:sg53}
  P_{k,n}(x) = f[x_n] 
    + (x-x_n)f[x_n,x_{n-1}
    + (x-x_n)(x-x_{n-1})f[x_n,x_{n-1},x_{n-2}]
    + \dots 
    + (x-x_n)(x-x_{n-1})\dots (x-x_{n-k+2})f[x_n,x_{n-1},x_{n-2},\dots ,x_{n-k-1}]
\end{equation}
where
\begin{equation}
  \begin{aligned}
    f[x_0] & & \equiv f(x_0) \text{ and}\\
    f[x_0,x_1,\dots ,x_n] &= \frac{f[x0,\dots ,x_{n-1}]-f[x1,\dots ,x_n]}{x_0 - x_n}
  \end{aligned}
\end{equation}
and we assume that $n \ge 1$.
Because we intend to vary the step size, we will take a code variables the take the 
step sizes $h_i = x_i - x_{i-1}$ as fundamental variables along with their sums 
$\psi (n+1) = h_{n+1} + h_{n} + \dots + h_{n+2-i} = x_{n+1} - x_{n+1-i}$. The 
divided differences will be modified so that they reduce to backward differences 
when the step sizes are constant. Using these modified divided differences $\phi _i (n)$ 
leads to economical, simple formulas. For integrations like the one in \autoref{eq:sg52a} 
we again find it convenient to introduce a normalized variable $s$ by $x=x_n +sh_{n+1}$ 
so that $s$ runs from 0 to 1 as $x$ runs from $x_n$ to $x_{n+1}$. We will also need 
two more fundamental quantities $\alpha _i (n+1)$ and $\beta _i (n+1)$
\begin{equation}\label{eq:sg54}
  \begin{aligned}
    h_i  &= x_i - x_{i-1} \\
    s    &= \frac{x - x_n}{h_{n+1}} \\
    \psi _i (n+1)   &= h_{n+1} + h_{n} + \dots + h_{n+2-i}, \text{ } i \ge 1 \\
    \alpha _i (n+1) &= \frac{h_{n+1}}{\psi _i (n+1)}, \text{ } i \ge 1 \\
    \beta _1 (n+1)  &= 1 \\
    \beta _i (n+1)  &= \frac{\psi _1 (n+1) \psi _2 (n+1) \dots \psi _{i-1}(n+1)}{\psi _1 (n) \psi _2 (n) \dots \psi _{i-1}(n)}, \text{ } i > 1 \\
    \phi _1(n) &= f[x_n] = f_n \\
    \phi _i(n) &= \psi _1 (n) \psi _2 (n) \dots \psi _{i-1}(n) f[x_n,x_{n-1}],\dots ,x_{n-i+1}], \text{ } i > 1
  \end{aligned}
\end{equation}

It is worth noting that if the step size is a constant $h$, then 
\begin{equation}
  \begin{aligned}
    \psi _i(n+1)   &= ih \\
    \alpha _i(n+1) &= \frac{1}{i} \\
    \beta _i(n+1)  &= 1 \\
    \phi _i(n)     &= \nabla ^{i-1} f_{n}
  \end{aligned}
\end{equation}
$\phi _i (n)$ values can be computed efficiently using divided differences. For 
the $\psi _i$ and $\beta _i$ values, the following formulas are used for recursion 
\begin{equation}
  \begin{aligned}
    \psi _i (n+1)  &= \psi _{i-1} (n) + h_{n+1} \\
    \beta _i (n+1) &= \beta _{i-1}(n+1) \frac{\psi _{i-1} (n+1)}{\psi _{i-1}(n)} \\
  \end{aligned}
\end{equation}
starting with the initial values
\begin{equation}
  \begin{aligned}
    \psi _1 (n+1) &= h_{n+1} \\
    \beta _1(n+1) &= 1
  \end{aligned}
\end{equation}

The interpolating polynomial, analytical work can be done (see \cite{Shampine1975}) 
\begin{equation}\label{eq:sg57}
  P_{k,n} (x) = \sum_{i=1}^{k} c_{i,n} (s) \phi ^{*}_{i} (n)
\end{equation}
where
\begin{equation}
  \phi ^{*}_{i} (n) = \beta _i (n+1) \phi _i (n)
\end{equation}
and
\begin{equation}\label{eq:sg56}
  c_{i,n}(s) = \begin{cases}
    1, i=1 \\
    \frac{s h_{n+1}}{\psi _1 (n+1)} = s, i=2 \\
    \left(\frac{sh_{n+1}}{\psi _1(n+1)}\right)
    \left(\frac{sh_{n+1} + \psi _1(n)}{\psi _2(n+1)}\right)
    \dots
    \left(\frac{sh_{n+1} + \psi _{i-2}(n)}{\psi _{i-1}(n+1)}\right), i \ge 3
  \end{cases}
\end{equation}
For variable step sizes, the above becoms (using \autoref{eq:sg54} and \autoref{eq:sg56})
\begin{equation}\label{}
  c_{i,n}(s) = \begin{cases}
    1, i=1 \\
    \alpha _1 (n+1) s = s, i=2 \\
    \left( \alpha _{i-1} (n+1) s + \frac{\psi _{i-2}(n)}{\psi _{i-1}(n+1)} \right) c_{i-1,n}(s) , i \ge 3
  \end{cases}
\end{equation}

Hence, to approximate the derivative of the solution by \autoref{eq:sg52b}, we 
take $s=1$ in \autoref{eq:sg57}, using \autoref{eq:sg54} and \autoref{eq:sg56}
\begin{equation}
  P_{k,n} (x_{n+1}) = p'_{n+1} = \sum^{k}_{i=1} \phi^{*}_{i} (n)
\end{equation}
To approximate the solution at $x_{n+1}$, we can substitute \autoref{eq:sg57} 
into \autoref{eq:sg52a} and integrate to obtain
\begin{equation}\label{eq:sg58}
  p_{n+1} = y_n + h_{n+1} \sum_{i=1}^{k} \phi ^{*}_{i} (n) \int_{0}^{1} c_{i,n} \,ds
\end{equation}
When the step size is constant, \autoref{eq:sg58} reduces to
\begin{equation}
  p_{n+1} = y_n +h \sum_{i=1}^{k} \gamma _{i-1} \nabla ^{i-1} f_n
\end{equation}
If we define the quantity $g_{i,q}=(q-1)!c_{i,n^{(-q)}}$, and 
\begin{equation}\label{eq:sg510}
  g_{i,q} = \begin{cases}
    \frac{1}{q}, \text{ } i=1,\\
    \frac{1}{q(q+1)}, \text{ } i=2,\\
    g_{i-1,q} - \alpha_{i-1}(n+1)g_{i-1,q+1}, \text{ } i \ge 3
  \end{cases}
\end{equation}
then
\begin{equation}\label{eq:sg511}
  p_{n+1} = y_n + h_{n+1} 
    \sum_{i=1}^{k} \phi ^{*}_{i} (n) g_{i,1}
\end{equation}


\subsection{The Corrector}\label{ssec:integrator-corrector}
The code uses a corrector one order higher that the predictor and accepts as an 
approximation to the solution and its derivative at $x_{n=1}$
\begin{equation}
    \begin{aligned}
        y_{n+1} &= y_n + \int_{x_n}^{x_{n+1}} P^{*}_{k+1,n}(t) \,dt \\
        f_{n+1} &= f(x_{n=1}, y_{n+1})
    \end{aligned}
\end{equation}
where
\begin{equation}
    \begin{aligned}
        P^{*}_{k+1,n}(x_{n+1-j}) &= f_{n+1-j}, \text{ } j=1,\dots ,k \\
        P^{*}_{k+1,n}(x_{n+1}) &= f^{p}_{n+1} = f(x_{n+1}, p_{n+1})
    \end{aligned}
\end{equation}
The corrector polynomial interpolates to the same data as the predictor plus 
the additional value $f^{p}_{n+1}$. For the error estimation we also want to consider 
the case of the corrector of order $k$, $P^{*}_{k,n}(x)$. 
Since $P^{*}_{k+1,n}(x)$ interpolates the same data as $P^{*}_{k,n}(x)$ plus one 
extra point, the basic intent of the divided difference form is to represent it as a small 
correction to $P_{k,n}(x)$
\begin{equation}\label{eq:sg512}
    P^{*}_{k+1,n}(x) = P_{k,n}(x) + (x-x_n)(x-x_{n-1})\dots (x-x_{n-k+1})
        f^{p}[x_{n+1}, \dots x_{n-k+1}]
\end{equation}
The superscript $p$ on this divided difference is to remind us that $P^{*}_{k+1,n}(x)$ 
interpolates to $f^{p}_{n+1}$. 
Using alternate notation, this is 
\begin{equation}
    P^{*}_{k+1,n}(x) = P_{k,n}(x) + c_{k+1,n}(s) \phi ^{p}_{k+1} (n+1)
\end{equation}
and after integration we obtain
\begin{equation}\label{eq:sg513}
    \begin{aligned}
        y_{n+1} &= p_{n+1} + h_{n+1} g_{k+1,1} \phi ^{p}_{k+1} (n+1) \\
        f_{n+1} &= f(x_{n+1}, y_{n+1})
    \end{aligned}
\end{equation}
This is a very convenient way to compute $y_{n+1}$ since we can compute the 
coefficient $g_{k+1,1}$ along with the $g_{i,1}$ needed in the prediction process.
if we use a corrector of order $k$ instead of $k+1$ the only change in the formula 
for $y_{n+1}$ is to change $g_{k+1,1}$ to $g_{k,1}$. We shall write $y_{n+1}(k)$ 
to distinguish this value:
\begin{equation}
    y_{n+1}(k) = p_{n+1} + h_{n+1} g_{k,1} \phi ^{p}_{k+1} (n+1)
\end{equation}
These two formulas generalize to variable step size the formulas for constant step size:
\begin{equation}\label{eq:sg513}
    \begin{aligned}
        y_{n+1} &= p_{n+1} + h \gamma _k \nabla ^{k} f^{p}_{n+1} \\
        y_{n+1}(k) &= p_{n+1} + h \gamma _{k-1} \nabla ^{k} f^{p}_{n+1}
    \end{aligned}
\end{equation}
All that remains is to see how to compute the $\phi ^{p}_{i} (n+1)$ and 
$\phi _{i} (n+1)$ to complete the step.
Omitting details we can came up with
\begin{equation}\label{eq:sg514}
    \phi ^{p}_{i+1} (n+1) = \phi ^{p}_{i} (n+1) -  \phi ^{*}_{i} (n)
\end{equation}
and
\begin{equation}\label{eq:sg515}
    \phi _{i+1} (n+1) = \phi _{i} (n+1) -  \phi ^{*}_{i} (n)
\end{equation}
where we note that  $\phi ^{p}_{1} (n+1) = f^{p}_{n+1}$ and 
$\phi _{1} (n+1) = f_{n+1}$ by definition.
It would be ideal if we could form $\phi ^{*}_{i} (n)$ and overwrite $\phi _{i} (n)$. 
To do this we can replace \autoref{eq:sg514} with 
\begin{equation}\label{eq:sg516}
    \phi ^{p}_{i} (n+1) = \phi ^{p}_{i+1} (n+1) + \phi ^{*}_{i} (n)
\end{equation}
recursively in the order $i=k,k-1,\dots ,1$. To start the recursion we can use
\begin{equation}
    \phi ^{p}_{k+1} (n+1) = f^{p}_{n+1} - {p'}_{n+1} = 
    f^{p}_{n+1} - \sum_{i=1}^{k}\phi ^{*}_{i} (n)
\end{equation}
From \autoref{eq:sg514} and \autoref{eq:sg515} we can see that
\begin{equation}\label{eq:sg517}
    \phi _{i} (n+1) = \phi ^{p}_{i} (n+1) + \left( f_{n+1} - \phi ^{p}_{1} (n+1) \right)
\end{equation}
We can find that
\begin{equation}\label{eq:sg518}
    \phi ^{e}_{i} (n+1) = \phi ^{e}_{i+1} (n+1) + \phi ^{*}_{i} (n)
\end{equation}
we can use \autoref{eq:sg518} to generate $\phi ^{e}_{i} (n+1)$ in the sequence 
$i=k, k-1, \dots ,1$ and overwrite them on the $\phi ^{*}_{i} (n)$. 
We can also prove that
\begin{equation}
    \begin{aligned}
        \phi ^{p}_{i} (n+1) &= \phi ^{e}_{i} (n+1) + 
            \left( f^{p}_{n+1} - \phi ^{e}_{1} (n+1) \right) \label{eq:sg519} \\
        \phi _{i} (n+1) &= \phi ^{e}_{i} (n+1) + 
            \left( f_{n+1} - \phi ^{e}_{1} (n+1) \right) \label{eq:sg520}
    \end{aligned}
\end{equation}
Using \autoref{eq:sg519} we generate those $\phi ^{p}_{i} (n+1)$ required for 
the error estimation. If 
the step is successful, we then use \autoref{eq:sg520} to compute the 
$\phi _{i} (n+1)$ and overwrite them on the $\phi ^{e}_{i} (n+1)$.
Let us now summarize all the computation required to advance one step.
The code that is given uses a PECE Adams method which advances from $x_n$ to 
$x_{n+1}$ by
\begin{description}
    \item[Initialize] compute $g_{i,1}$ for $i=1,2, \dots ,k+1$
    \item[P Predicting] 
        \begin{equation}
            \begin{aligned}
                \phi ^{*}_{i} (n) &= \beta _i(n+1) \phi _i(n) \text{ } i=1,2, \dots ,k \\
                p_{n+1} &= y_n + h_{n+1} \sum_{i=1}^{k} g_{i,1} \phi ^{*}_{i} (n) \\
                \phi ^{e}_{k+1} (n+1) &= 0 \\
                \phi ^{e}_{i} (n+1) &= \phi ^{e}_{i+1} (n+1) + \phi ^{*}_{i} (n) \text{ } k,k-1,\dots ,1
            \end{aligned}
        \end{equation}
    \item[E Evaluating] $f^{p}_{n+1} = f(x_{n+1}, p_{n+1})$
    \item[C Correcting] $y_{n+1} = p_{n+1} + h_{n+1} g_{k+1,1} 
        \left( f^{p}_{n+1} - \phi ^{e}_{i} (n+1) \right)$
    \item[E Evaluating]
        \begin{equation}
            \begin{aligned}
                f_{n+1} &= f(x_{n+1}, y_{n+1}) \\
                \phi _{k+1} (n+1) &= f_{n+1} - \phi ^{e}_{1} (n+1) \\
                \phi _{i} (n+1) &= \phi ^{e}_{i} (n+1) + \phi _{k+1} (n+1) \text{ } i=k,k-1,\dots ,1
            \end{aligned}
        \end{equation}
\end{description}
Notice that advancing the differences is very cheap though there may be a 
lot of them. Much of the overhead results from the computation of the $g_{i,1}$.

We stated at the beginning of this chapter that our code would take
advantage of steps taken with a constant size. Let us go into this matter
now and examine the programming of the computation.
Let the variable $n_s$ be the number of successive steps taken with constant 
step size $h$, including the current one. When $\psi _i(n+1)$ is computed it is 
written over the stored value of $\psi _i(n)$.
To compute the $\psi _{i}(n+1)$ we need only start with the value 
$\psi _{n_s}(n+1) = n_s h$ and obtain the remaining elements by the iteration 
\begin{equation}
    \psi _{i}(n+1) = \psi _{i-1}(n) + h \text{ } i= n_s+1 , \dots , k
\end{equation}
and for the $\alpha$ and $\beta$ arrays to advance from $n$ to $n+1$
\begin{equation}
    \begin{aligned}
        \alpha _{n_s} (n+1) &= \frac{1}{n_s} \\
        \alpha _{i} (n+1)   &= \frac{h}{\psi _i(n+1)} \text{ } i= n_s+1, \dots ,k \\
        \beta _{n_s} (n+1)  &= 1 \\
        \beta _{i} (n+1)    &= \beta _{i-1} (n+1) \frac{\psi _{i-1}(n+1)}{\psi _{i-1}(n)} 
            \text{ } i= n_s +1 , \dots ,k
    \end{aligned}
\end{equation}
Obviously the closer we are to working with a constant step size, the less
work need be done in computing these quantities.