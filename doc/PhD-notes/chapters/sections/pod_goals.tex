\subsection{Goals of Current Chapter}\label{ssec:pod-goals}
In this section the fundamentals of orbit determination are discussed, focusing on 
two of the most crucial problems:
\begin{itemize}
  \item the derivation, computation and solution of the so called \emph{variational equations}, and 
  \item efficient, robust and precise parameter estimation via a variation of the 
    Kalman filter, labeled \emph{Extended Kalman Filter}
\end{itemize}

Variational equations are a set of differential equations that describe how small changes 
(i.e. perturbations) in initial conditions propagate over time and thus play a important 
role in predicting the satellite's trajectory. For the derivation of these equations, 
a linearization of the (non-linear) equations of motion is needed. Solution of the 
variational equations is coupled with the computation of the \emph{state transition matrix}, 
which relates the perturbed state at one epoch to a perturbed state at a later time and 
can be used to propagate the covariance matrix. Propagating the latter forward in time, 
a prediction can be made of the evolution of uncertainty in the estimate of the satellite's 
state. Formulas and numerical recipes for the linearization, as well as the differential 
equations of the state transition matrix and variational equations are presented. 
The discussion focuses on a hands-on approach, limiting analytical derivations which can be 
found in relevant literature. Equations are presented in matrix form, to enable an 
as much as possible easy translation to source code, and special care is taken to 
single out and exploit features, equations and particularities that can be used to derive 
a more efficient and/or robust algorithmic design.

Subsequently, the problem of parameter estimation is considered. There are two major 
methodologies that come into play in \gls{pod} problems, the method of \emph{Least Squares} 
and the extended family of filters called \emph{Kalman filters} and variations. 
For the current thesis, the latter methodology was used and more specifically the 
\emph{Extended Kalman filter} to derive a robust estimator. Computational aspects of the 
methodology and the implementation are discussed, as well as advantages and shortcomings.

Since a discussion on the input data for the \gls{pod} problem considered in this Thesis 
is not yet touched upon, the computation of \emph{observation equations} (needed in the 
estimation process) is not thoroughly presented here; this issue will be revisited once the 
basics of the DORIS system are presented (??see chapter 5??).
