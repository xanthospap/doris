\section{Implementation}\label{sec:doris-implementation}
In the framework of the current Thesis, a software package was designed to analyze 
\gls{doris} measurements for \gls{pod}. The practical implementation of the observation 
equation model is already discussed in \autoref{sec:doris-observation-equation-implementation}. 
In this section, general issues of the algorithmic design are presented.

The first issue that arises when attempting to process \gls{doris} data is the 
parsing of input files. The decision was made for the software to adhere to 
the new RINEX format (\autoref{sec:doris-rinex}), as suggested by the \gls{ids}.
For the software created, the decision was made to stick to the new 
RINEX format (\autoref{sec:doris-rinex}), as suggested by the \gls{ids}. This decision 
provides for a wider range of analysis options (compared to the older data format), 
regarding both the data types available and the subsequent processing options and 
scheme (e.g. it is possible to parse and process pseudorange measurements). The 
drawback though, is the complexity of the RINEX format.

A dedicated module of the software package is thus designed to handle parsing of the 
RINEX files, in a generic way, so that users can extract the observables and meta-data 
needed for the processing scheme they decide on. Note that recent studies have shown that 
the phase measurements themselves can be used (instead of Doppler counts) to effectively 
process data (see e.g. \cite{Mercier2010}, \cite{Dettmering2014} and \cite{Zhou2020}), 
although this method is not yet widely used.

Ground beacon geometry, including \gls{pco} and \gls{pcv} (see \autoref{sec:doris-ground-segment}) 
information, is hard-coded in the software library. The implementation is based on a 
\emph{type-safe}, \emph{meta-programming} paradigm (\cite{Vandevoorde2017}), so that 
users can request eccentricities and corrections for any \gls{doris} frequency, including 
their linear combinations. This design pattern allows for efficiency, safety and 
versatility.

Site eccentricities are read from the respective \emph{log} files, per beacon. 
\gls{ids} maintains updates log-files for each of the ground beacons included in 
the \gls{doris} network (see \url{https://ids-doris.org/doris-system/tracking-network/site-logs.html}). 
To that end, a \emph{Python} module has been created that can handle the acquisition, 
validation and parsing of relevant information off from the site-specific log file. 
Eccentricities can thus be extracted and sourced into the main processing module. 
This design was preferred (e.g. to harcoding eccentricity information), as it accommodates 
an easier adoption of site changes (e.g. instrumentation, of installation).

Once all of the above components are in place, the software can implement the 
observation model, as described in \autoref{sec:doris-observation-equation-implementation}.
The processing algorithm applies observation-specific corrections to the extracted 
measurements, and keeps track (in chronological order) of the previous observations 
encountered. Thus, it can compute Doppler counts at every new epoch. All available 
observations can be taken into account, assuming they comply to a number of user-defined 
criteria:
\begin{description}
  \item[Observation flags]: every observation in a RINEX file is followed by 
    a list of flags, denoting the instrumentation status (see \cite{DORISRNX3}). Users 
    can define selection criteria based on these flags.
  \item[Minimum Elevation]: measurements performed on low 
    elevation angles include an increased error budget. Users can define an 
    elevation cut-off angle, under which observations are removed for analysis.
  \item[Time Offset]: to compute the Doppler count between two consecutive observations, 
    a time distance criterion is set. If the measurements are performed within a  
    time period larger that this interval, then the Doppler count is rest.
  \item[Running Statistics]: while processing observations, the software computes 
    ``running'' statistic values (e.g. average and standard deviation). Users can 
    set boundary values for accepting observations based on these statistics (e.g. 
    a 3-$\sigma$ statistical test).
\end{description}

The whole analysis can be performed in two ``passes'' of the RINEX file: the first is 
performed to model the receiver's proper frequency $f_{r_T}$ (see 
\autoref{ssec:receiver-true-proper-frequency}). 
In a second pass, the ``core'' part of the analysis is performed, including 
parameter estimation. Details on the latter are given in \autoref{ch:pod-using-doris}.
