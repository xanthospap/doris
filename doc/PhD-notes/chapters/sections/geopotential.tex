\subsubsection{Geopotential}
\label{sssec:geopotential}

In the two-body problem (\ref{ssec:two-body-problem}) we are assuming a radially 
symmetric gravity force. For precise orbit determination however, we need to 
drop this assumption and take into account that the Earth is not a perfect sphere, 
but rather resembles an oblate spheroid, with different equatorial and polar 
diameter. To derive a more realistic model, it is convenient to use an equivalent 
representation involving the gradient of the corresponding gravity potential $U$ 
(\cite{Montenbruck2000}):
\begin{equation}
  \label{eq:geopotential}
  \ddot{\bm{r}} = \nabla U \textrm{ where } U = G M_{\Earth} \frac{1}{r}
\end{equation}
so that 
$\bm{F} = \begin{pmatrix} \frac{\partial U}{\partial x} & \frac{\partial U}{\partial y} & \frac{\partial U}{\partial z} \end{pmatrix}$
This formulation enables the replacement of the three components of the vector 
$\bm{F}$ by a single function $U$, thus simplifying notation and further 
developments. The geopotential is widely used in geodesy (see e.g. \cite{moritz}).

Given an arbitrary mass distribution, we can sum up the individual elementary 
mass contributions $dm = \rho (\bm{s}) d^3 s$, and express the potential as:
\begin{equation}
  U = G \int \frac{\rho (\bm{s})}{\norm{\bm{r}-\bm{s}}} \, d^3 s
\end{equation}
where $\bm{r}$ is the vector from the mass center to the attracted body (i.e. satellite) 
and $\bm{s}$ is the geocentric vector the the elementary mass $dm$.

For any point $\bm{r}$ outside the mass, $r > s$, we can expand the inverse of the 
distance using a series of \emph{Legendre polynomials}, as
\begin{equation}
  \frac{1}{\norm{\bm{r}-\bm{s}}} = \frac{1}{r} \sum_{n=0}^{\infty} 
    \left( \frac{s}{r} \right) ^n  P_n ( \cos \gamma )
\end{equation}
where $\cos \gamma = \frac{\bm{r} \cdot \bm{s}}{rs}$ (the angle between $\bm{r}$
and $\bm{s}$) and $P_n(u)$ is the Legendre polynomial of degree $n$. Making use of 
the addition theorem of Legendre polynomials (\cite{Montenbruck2000}):
\begin{equation}
  \label{eq:mont38}
  P_n (\cos \gamma ) = \sum_{m=0}^{n} \left( 2 - \delta _{0m} \right) 
    P_{nm} (\sin \phi ) P_{nm} (\sin{\phi ^\prime}) 
    \cos{m \left( \lambda - \lambda ^\prime \right) }
\end{equation}
with $P_{nm}$ the \emph{associated Legendre polynomials} of degree $n$ and order 
$m$ and $\begin{pmatrix} \phi & \lambda \end{pmatrix}$ and 
$\begin{pmatrix} \phi ^\prime & \lambda ^\prime \end{pmatrix}$ being the longitude 
and geocentric latitudes of points $\bm{r}$ and $\bm{s}$ respectively.

We can now write the Earth's gravity potential as:
\begin{equation}
  U = \frac{G M_{\Earth}}{r} \sum_{n=0}^{\infty} \sum_{m=0}^{n} 
      \left( \frac{R_{\Earth}}{r} \right) ^n P_{nm} (\sin \phi )
      \left( C_{nm} \cos {m\lambda} + S_{nm} \sin{m\lambda} \right)
\end{equation}
with coefficients\footnote{The coefficients $C_{nm}$ and $S_{nm}$ are often called 
\emph{Stokes' coefficients}, e.g. \cite{Barthelmes2018}.}:
\begin{equation}
  \begin{aligned}
    C_{nm} = \frac{2 - \delta _{0m}}{M_{\Earth}} \frac{(n-m)!}{(n+m)!} 
      \int \left({\frac{s}{R_{\Earth}}}\right)^n P_{nm} (\sin \phi ^\prime) \cos {m\lambda ^\prime} \rho (\bm{s}) \, d^3 s \\
    S_{nm} = \frac{2 - \delta _{0m}}{M_{\Earth}} \frac{(n-m)!}{(n+m)!} 
      \int \left({\frac{s}{R_{\Earth}}}\right)^n P_{nm} (\sin \phi ^\prime) \sin {m\lambda ^\prime} \rho (\bm{s}) \, d^3 s
  \end{aligned}
\end{equation}
which describe the dependence on the Earth's internal mass distribution. In geodetic 
applications, we most oftenly the \emph{normalized} geopotential coefficients 
$\bar{C}_{nm}$ and ${\bar{S}}_{nm}$, defined as
\begin{equation}
  \begin{Bmatrix} \bar{C}_{nm} & \bar{S}_{nm} \end{Bmatrix} = 
  \sqrt{\frac{(n+m)!}{\left(2-\delta _{0m}\right) \left(2n+1\right) \left(n-m\right)!}}
  \begin{Bmatrix} C_{nm} & S_{nm} \end{Bmatrix}
\end{equation}
which are much more unifom in magnitude.

We can now write the acceleration due to the Earth's gravity potential as 
(\cite{Montenbruck2000})
\begin{equation}
  \bm{\ddot{r}} = \nabla \frac{GM_{\Earth}}{r} \sum_{n=0}^{\infty} \sum_{m=0}^{n} 
    \left(\frac{R_{\Earth}}{r}\right) ^n \bar{P}_{nm} (\sin \phi )
    \left( \bar{C}_{nm} \cos{m\lambda} + \bar{S}_{nm} \sin{m\lambda} \right)
\end{equation}
where we use the \emph{normalized} associated Legendre functions
\begin{equation}
  \bar{P}_{nm} = \sqrt{\frac{\left(2-\delta _{0m}\right) \left(2n+1\right) \left(n-m\right)!}{(n+m)!}} P_{nm}
\end{equation}

\paragraph{Gravity Models}
\label{par:gravity-models}

Earth gravity models, contain the potential coefficients $\bar{C}_{nm}$ and $\bar{S}_{nm}$ 
up to a given degree $n$ and order $m$ (with $m \le n$) and the parameters $GM_{\Earth}$ 
and $R_{\Earth}$ via which one can compute the geopotential or the induced acceleration 
on a satellite given its position vector $\bm{r}$. Gravity models are derived from:
\begin{description}
  \item \emph{Satellite data}, making use of the fact that the Earth's gravity field 
    can produce perturbations ``seen'' in satellite orbits via \gls{pod}. It is worth 
    noting that since the Gravity Recovery and Climate Experiment (GRACE) was launched 
    in March 2002, satellite gravimetry has brought a new era of studying global mass 
    variation and redistribution through measuring the time-variable gravity field with 
    unprecedented accuracy (see e.g. \cite{Chen2022} and \cite{Jaggi2023}). GRACE is a 
    twin satellites mission, accurately tracking variations of inter-satellite range between 
    the two satellites via a K-band ranging system.
  \item \emph{Terrestrial observations} (surface gravimetry), providing precise local 
    and regional (short-wavelength) information on the gravity field. Due to their 
    inhomogenous distribution though, deriving a global gravity model is quite challenging.
  \item \emph{Altimeter data}, which provide detailed information about the form of the 
    geoid, which may in turn be used to derive geopotential coefficients.
  \item \emph{Combinations} of the above methods/data.
\end{description}

Within the framework of this Thesis, we will make use of two Earth gravity field 
models, namely \emph{EIGEN6-C} \cite{Eigen6} and \emph{CNES/GRGS RL04 \cite{lemoine-rl04}.

\paragraph{The \gls{icgem}}
\label{par:icgem}
The \href{http://icgem.gfz-potsdam.de/home}{\gls{icgem}} (\cite{icgempub}) hosts a large 
number of gravity field models, published in what is called the 
\emph{The ICGEM-format}(\cite{ICGEMFormat}). \gls{icgem} is one of five services coordinated 
by the \gls{igfs} of the \gls{iag}. Among other services, \gls{icgem} collects and 
archives all existing global gravity field models and provides a web interface for 
getting access.



%% -------------------------------------------------------------------- %%
Models for the geopotential usually describe the static part of the field. For 
precision applications, time varying effects should also be taken into account. 
According to \cite{iers2010}, these are:
\begin{itemize}
  \item secular variations of coefficients,
  \item solid Earth tides,
  \item ocean tides,
  \item solid Earth pole tide and
  \item ocean pole tide
\end{itemize}

The expression for the potential (\ref{eq:geopotential}) may easily be 
generalized to an arbitrary mass distribution by summing up the contributions 
created by individual mass elements
\(dm = \rho(\vec{s}) d^3 \vec{s}\) according to (\cite{Montenbruck2000})
\begin{equation}
    U = G \int{\frac{\rho(\vec{s}) d^3 \vec{s}}{\lvert \vec{r} - \vec{s} \rvert}}
\end{equation}


\subsection{Earth's Geopotential Models}
The geopotential field $U$ at the point $(r, \phi , \lambda )$ is expanded in 
spherical harmonics up to degree $N$, as
\begin{equation}
  \label{eq:iers201061}
  U(r, \phi , \lambda ) = \frac{G M_{\Earth}}{r} \sum_{n=0}^N 
    {\left(\frac{\alpha _e}{r}\right)}^n 
     \sum_{m=0}^n \left[ \bar{C}_{nm} \cos {m \lambda} + \bar{S}_{nm} \sin{m \lambda} \right] 
     \bar{P}_{nm}( \sin \phi )
\end{equation}
with $\bar{S}_{n0} = 0$ and $\bar{C}_{nm}$, $\bar{S}_{nm}$ the normalized
\footnote{Since the geopotential coefficients $C_{nm}$ and $S_{nm}$ span ten or 
more orders of magnitude (\cite{Montenbruck2000}), making the computation 
susceptible to overflow and round-off errors, we usually use the corresponding 
normalized values. The latter are much more uniform in magnitude, and their 
size is approximately given by the empirical \emph{Kaula rule} (see 
\cite{Montenbruck2000} and \cite{Kaula2000}):
\begin{equation} \bar{C}_{nm} , \bar{S}_{nm} \approx \frac{10^{-5}}{n^2} \end{equation}}
geopotential coefficients the normalized. $\bar{P}_{nm}$ are the normalized 
associated Legendre functions, related to the unnormalized ones by:
\begin{subequations}
  \begin{align}
    \bar{P}_{nm} &= N_{nm} P_{nm} \label{eq:iers201062a} \\
    N_{nm} &= \sqrt{\frac{(n-m)!(2n+1)(2-\delta _{0m})}{(n+m)!}} 
      \quad \delta _{0m} = 
        \begin{cases}
          1 \quad \text{ if } m = 0 \\
          0 \quad \text{ if } m \neq 0
        \end{cases}
        \label{eq:iers201062b}
  \end{align}
\end{subequations}
Correspondingly, the normalized and unnormalized geopotential coefficients are 
related by:
\begin{subequations}
  \begin{align}
    C_{nm} &= N_{nm} \bar{C}_{nm} \\
    S_{nm} &= N_{nm} \bar{S}_{nm}
  \end{align}
\end{subequations}

For all \emph{zonal} terms, we have:
\begin{equation}
  S_{n,0} = 0 \quad C_{n,0} = -J_n
\end{equation}
where we use the ``traditional'' notation $J_n$ for \emph{zonal coefficients}. 
All $S_{n0}$ coefficients vanish (due to their definition).

Coefficients for $m<n$ are known as ``\emph{tesseral}'', while when $m=n$, the 
respective coefficients are called ``\emph{sectorial}''.


\subsubsection{Static Geopotential Models}
For computations, the Earth's gravity potential at point $(r, \phi , \lambda )$
can be approximated using a spherical harmonics expansion 
(e.g. \cite{Montenbruck2000})
\begin{equation}
    U = \frac{G M_{\Earth}}{r} \sum_{n=0}^\infty \sum_{m=0}^n 
    \frac{R_{\Earth}^n}{r^n} P_{nm} \left( \sin \phi \right) 
    \left( C_{nm} \cos{m\lambda} + S_{nm} \sin{m\lambda} \right)
\end{equation}

with the coefficients \(C_{nm}\) and \(S_{nm}\) describing the dependence on the 
Earth's internal mass distribution. These coefficients can be extracted from a 
suitable geopotential model.

Note that Geopotential coefficients with \(m=0\) are called  \emph{zonal} coefficients, 
since they describe the part of the potential that does not depend on the longitude. 
All \(S_{n0}\) vanish due to their definition, and the notation \(J_n = -C_{n0}\) 
is commonly used for the remaining zonal terms. The other geopotential coefficients
are known as \emph{tesseral} and \emph{sectorial} coefficients for \(m<n\) and 
\(m = n\), respectively.

\fbox{\begin{minipage}{.9\textwidth}
To compute the acceleration \(\ddot{\vec{r}}\), which is equal to the 
gradient of \(U\), we use the recursion formulas described in \cite{Montenbruck2000}.
Note that these expressions, use the \underline{non-normalized} harmonic 
coefficients. 
\end{minipage}}

%\subsubsection{Gravity Models}
The \href{http://icgem.gfz-potsdam.de/home}{International Centre for Global Earth Models (ICGEM)}  
(\cite{icgempub}) web service hosts a large number of gravity field models, 
published in what is called the \emph{The ICGEM-format}(\cite{ICGEMFormat}). 

Such static model files can be parsed and the respective harmonic coeefficients 
(\(C_{nm}\) and \(S_{nm}\)) be used to compute satellite acceleration induced by 
the geopotential.

Practicaly, the degree and order used depends on the accuracy of the 
application and can be truncated. According to \cite{iers2010}, truncation levels 
providing a 3-dimensional orbit accuracy of better than \SI{0.5}{\mm} as a 
function of orbit radius, are listed in \ref{table:egm2008-truncation-levels} 
for EGM2008 (\cite{pavlisegm08}).

\begin{table}
\centering
\begin{tabular}{c c c}
 \hline
 Orbit radius [km] & Example Satellite & Truncation level \\
 \hline
  7331  & Starlette & 90 \\
  12270 & Lageos    & 20 \\
  26600 & GPS       & 12 \\
 \hline
\end{tabular}
\caption{Suggested truncation levels for use of EGM2008 at different orbits according to \cite{iers2010}.}
\label{table:egm2008-truncation-levels}
\end{table}
