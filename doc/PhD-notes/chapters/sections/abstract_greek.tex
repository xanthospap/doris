\selectlanguage{greek}
Η επιστήμη της Γεωδαισίας έχει κάνει μεγάλα άλματα προόδου μετά την εισαγωγή των τεχνητών
δορυφόρων που βρίσκονται σε τροχιά γύρω από τη Γη. Σε αυτή τη ''νέα εποχή'', τα δεδομένα που
προέρχονται από δορυφορικές αποστολές παίζουν σημαίνοντα ρόλο στο ευρύτερο γνωστικό πεδίο,
προσφέροντας σημαντικά αποτελέσματα σε ένα ευρύ φάσμα γεωεπιστημών. Η χρήση όμως τέτοιων
δεδομένων, είναι συχνά συνυφασμένη με την περιγραφή της δορυφορικής κίνησης και των δυνάμεων που
την επηρεάζουν, σε κατάλληλα, υψηλής ακρίβειας χωρικά και χρονικά συστήματα αναφοράς. Συνεπώς, η
εκτίμηση των δορυφορικών τροχιών αποτελεί ένα θεμελιώδες πρόβλημα στην σύγχρονη γεωδαισία.


Η εκτίμηση υψηλής ακρίβειας δορυφορικών τροχιών, γίνεται κυρίως με χρήση τριών συστημάτων και
συγκεκριμένα του \textlatin{SLR}, του \textlatin{GNSS} και του \textlatin{DORIS}. Οι τεχνικές αυτές,
συμβάλουν καταλυτικά και στην υλοποίηση των σύγχρονων παγκόσμιων συστημάτων αναφοράς (π.χ.
\textlatin{ITRF}).


Παρά τη σημαντική συμβολή και την εξέχουσα θέση που καταλαμβάνει τις τελευταίες δεκαετίες, το
σύστημα \textlatin{DORIS} δεν έχει έως τώρα προσελκύσει το ανάλογο επιστημονικό κοινό και συνεπώς
μια κατάλληλη υπολογιστική πλατφόρμα λογισμικού για την εκμετάλλευση των δυνατοτήτων που μπορεί
να προσφέρει. Ενδεικτικός είναι ο μικρός αριθμός εξειδικευμένων κέντρων ανάλυσης, ειδικά σε σχέση με
τις λοιπές τεχνικές προσδιορισμού τροχιών.


Με αφορμή την έλλειψη αυτή, στο πλαίσιο της παρούσας Διδακτορικής Διατριβής, επιχειρείται ο
σχεδιασμός και η ανάπτυξη μια δέσμης λογισμικών εργαλείων επιστημονικής ποιότητας, για την
επεξεργασία δεδομένων \textlatin{DORIS} και δη για τη χρήση δεδομένων καταγραφής του συστήματος
για την εκτίμηση δορυφορικών τροχιών. Τα υπολογιστκά εργαλεία που αναπτύχθηκαν είναι ελεύθερα
διαθέσιμα στην επιστημονική κοινότητα, ακολουθώντας την πολιτική του ''ανοιχτού κώδικα'',
προσαρμόσιμα και επεκτάσιμα για να ανταποκρίνονται σε ερευνητικές απαιτήσεις υψηλής ακρίβειας,
περιλαμβάνοντας τους πλέον σύγχρονους αλγορίθμους και μεθοδολογίες. Ο σχεδιασμός και η υλοποίηση
ακολουθεί και υιοθετεί σύγχρονα προγραμματιστικά πρότυπα.


Η διάρθρωση της εργαλειοθήκης που αναπτύχθηκε, αποτελείται από ανεξάρτητα ''πακέτα'' (ή
βιβλιοθήκες), καθένα εκ των οποίων έχει σαφώς καθορισμένη στόχευση. Με τον τρόπο αυτό, τα
ξεχωριστά κομμάτια συγκρατούνται σε εύκολα διαχειρίσιμα μεγέθη, καθιστώντας το πακέτο ιδανικό για
χρήση από την επιστημονική κοινότητα σε ένα ευρύ φάσμα εφαρμογών.


Τα τρέχοντα διεθνή, υψηλής ακρίβειας πρότυπα μοντελοποίησης έχουν ακολουθηθεί αυστηρά, σε μία
σειρά προβλημάτων που αφορούν ενδεικτικά: τις δυνάμεις που επιδρούν σε τεχνητούς δορυφόρους γύρω
από τη γη, την ίδια κίνηση της γης, τα χωρικά και χρονικά συστήματα αναφοράς, την αριθμητική επίλυση
διαφορικών εξισώσεων. Όπου χρειάστηκε, δοκιμάστηκαν και αξιολογήθηκαν διαφορετικές μέθοδοι και
εν συνεχεία προσαρμόστηκαν και επεκτάθηκαν με βάση κριτήρια ακριβείας και υπολογιστικού φόρτου.
Αξίζει να σημειωθεί ότι το λογισμικό εστιάζει σε μία μοντέρνα, καινοφανή προσέγγιση επεξεργασίας
δεδομένων \textlatin{DORIS} υποστηρίζοντας τις τελευταίες εξελίξεις τις τεχνικής όπως π.χ. η υιοθέτηση
του μορφότυπου ανταλλαγής δεδομένων \textlatin{DORIS RINEX}, η διαχείριση δεδομένων φέροντος
κύματος (\textlatin{carrier phase}) καθώς και η δυνατότητα επεξεργασίας δεδομένων σε σχεδόν
πραγματικό χρόνο.


Δεδομένης της πολυπλοκότητας του προβλήματος και των περιορισμών που τίθενται σε μία Διδακτορική
Διατριβή, ο σκοπός της παρούσας δεν είναι η ανάπτυξη ενός λογισμικού που θα φτάσει τα επίπεδα
ποιότητας που αυτή τη στιγμή πετυχαίνουν κέντρα ανάλυσης εγνωσμένης αξίας, με δεκαετίες εμπειρίας
και μεγάλες ερευνητικές ομάδες υποστήριξης (\textlatin{IDS}). Ο στόχος που τέθηκε είναι η δημιουργία
μιας καινούριας, σύγχρονης εργαλειοθήκης που θα μπορεί να παίξει το ρόλο του θεμελιώδους δομικού
στοιχείου για την επίτευξη τέτοιων ακριβειών στο άμεσο μέλλον. Με μικρές, σαφώς ορισμένες
βελτιώσεις, εκτιμάται ότι το πακέτο που σχεδιάστηκε για την παρούσα Διατριβή μπορεί να υπηρετήσει το
ρόλο αυτό.


Για τον έλεγχο της ποιότητας του πακέτου που σχεδιάστηκε, αναπτύχθηκε ένα πρόγραμμα για τον
προσδιορισμό της τροχιάς της δορυφορικής αποστολής \textlatin{JASON}-3. Οι εκτιμήσεις θέσης και
ταχύτητας του δορυφόρου που προέκυψαν, για μία ημέρα ενδεικτικά, συγκρίθηκαν με αντίστοιχα
αποτελέσματα της υψηλότερης δυνατής ακρίβειας από την \textlatin{CNES/SSALTO}. Οι διαφορές που
προέκυψαν είναι της τάξης των λίγων μέτρων και των λίγων χιλιοστών ανά μέτρο για την θέση και την
ταχύτητα αντίστοιχα.


Τα αποτελέσματα υποδεικνύουν ότι η εργαλειοθήκη που σχεδιάστηκε και αναπτύχθηκε μπορεί να
αποτελέσει το κύριο δομικό στοιχείο ενός προγράμματος επεξεργασίας δεδομένων \textlatin{doris},
ποιότητας εφάμιλλης των κέντρων ανάλυσης της τεχνικής.
\iffalse
Η επιστήμη της Γεωδαισίας έχει κάνει μεγάλα άλματα προόδου μετά την εισαγωγή των τεχνητών δορυφόρων
που βρίσκονται σε τροχιά γύρω από τη Γη. Σε αυτή τη ``νέα εποχή'', τα δεδομένα που προέρχονται
από δορυφορικές αποστολές παίζουν σημαίνοντα ρόλο στο ευρύτερο γνωστικό πεδίο, προσφέροντας
σημαντικά αποτελέσματα σε ένα ευρύ φάσμα γεωεπιστημών. Η χρήση όμως τέτοιων δεδομένων, είναι συχνά
συνυφασμένη με την περιγραφή της δορυφορικής κίνησης και των δυνάμεων που την επηρεάζουν,
σε κατάλληλα, υψηλής ακρίβειας χωρικά και χρονικά συστήματα αναφοράς. Συνεπώς, η εκτίμηση
των δορυφορικών τροχιών αποτελεί ένα θεμελιώδες πρόβλημα στην σύγχρονη γεωδαισία.

Η εκτίμηση υψηλής ακρίβειας δορυφορικών τροχιών, γίνεται κυρίως με χρήση τριών συστημάτων
και συγκεκριμένα του \textlatin{SLR}, του \textlatin{GNSS} και του \textlatin{DORIS}. Οι τεχνικές αυτές, συμβάλουν
καταλυτικά και στην υλοποίηση των σύγχρονων παγκόσμιων συστημάτων αναφοράς (π.χ. \textlatin{ITRF}).

Παρά τη σημαντική συμβολή και την εξέχουσα θέση που καταλαμβάνει τις τελευταίες δεκαετίες,
το σύστημα \textlatin{DORIS} δεν έχει έως τώρα προσελκύσει το ανάλογο επιστημονικό κοινό και
συνεπώς μια κατάλληλη υπολογιστική πλατφόρμα λογισμικού για την εκμετάλλευση των δυνατοτήτων
που μπορεί να προσφέρει.
Ενδεικτικός είναι ο μικρός αριθμός εξειδικευμένων κέντρων ανάλυσης, ειδικά σε σχέση με τις
λοιπές τεχνικές προσδιορισμού τροχιών.

Με αφορμή την έλλειψη αυτή, στο πλαίσιο της παρούσας Διδακτορικής Διατριβής, επιχειρείται ο σχεδιασμός και
η ανάπτυξη μια δέσμης λογισμικών εργαλείων επιστημονικής ποιότητας, για την επεξεργασία
δεδομένων \textlatin{DORIS} και δη για τη χρήση δεδομένων καταγραφής του συστήματος για την
εκτίμηση δορυφορικών τροχιών. Τα υπολογιστκά εργαλεία που αναπτύχθηκαν είναι ελεύθερα διαθέσιμα στην επιστημονική
κοινότητα, ακολουθώντας την πολιτική του ``ανοιχτού κώδικα'', προσαρμόσιμα και επεκτάσιμα για να
ανταποκρίνονται σε ερευνητικές απαιτήσεις υψηλής ακρίβειας, περιλαμβάνοντας τους πλέον σύγχρονους αλγορίθμους
και μεθοδολογίες. Ο σχεδιασμός και η υλοποίηση ακολουθεί και υιοθετεί σύγχρονα
προγραματιστικά πρότυπα.

Η διάρθρωση της εργαλειοθήκης που αναπτύχθηκε, αποτελείται από ανεξάρτητα ``πακέτα'' (ή βιβλιοθήκες),
καθένα εκ των οποίων έχει σαφώς καθορισμένη στόχευση. Με τον τρόπο αυτό, τα ξεχωριστά κομμάτια
κρατούνται σε εύκολα διαχειρίσιμα μεγέθη, καθιστώντας το πακέτο ιδανικό για χρηση από την
επιστημονική κοινότητα σε ένα ευρύ φάσμα εφαρμογών.

Τα τρέχοντα διεθνή, υψηλής ακρίβειας πρότυπα μοντελοποίησης έχουν ακολουθηθεί αυστηρά, σε μία σειρά
προβλημάτων που αφορούν ενδεικτικά: τις δυνάμεις που επιδρούν σε τεχνητούς δορυφόρους γύρω από τη γη,
την ίδια κίνηση της γης, τα χωρικά και χρονικά συστήματα αναφοράς, την αριθμητική επίλυση
διαφορικών εξισώσεων. Όπου χρειάστηκε, δοκιμάστηκαν και αξιολογήθηκαν διαφορετικές μέθοδοι και
εν συνεχεία προσαρμόστηκαν και επεκτάθηκαν με βάση κριτήρια ακριβείας και υπολογιστικού φόρτου.

Αξίζει να σημειωθεί ότι το λογισμικό εστιάζει σε μία μοντέρνα, καινοφανή προσέγγιση επεξεργασίας 
δεδομένων \textlatin{DORIS} υποστηρίζοντας τις τελευταίες εξελίξεις τις τεχνικής όπως π.χ. η υιοθέτηση του 
μορφότυπου ανταλλαγής δεδομένων \textlatin{DORIS RINEX}, η διαχείριση δεδομένων φέροντος κύματος 
(\textlatin{carrier phase}) καθώς και η δυνατότητα επεξεργασίας δεδομένων σε σχεδόν πραγματικό χρόνο.

Δεδομένης της πολυπλοκότητας του προβλήματος και των περιορισμών που τίθενται σε μία Διδακτορική Διατριβή,
ο σκοπός της παρούσας δεν είναι η ανάπτυξη ενός λογισμικού που θα φτάσει τα επίπεδα ποιότητας
που αυτή τη στιγμή πετυχαίνουν κέντρα ανάλυσης εγνωσμένης αξίας, με δεκαετίες εμπειρίας και
μεγάλες ερευνητικές ομάδες υποστήριξης (\textlatin{IDS}). Ο στόχος που τέθηκε είναι η δημιουργία μιας
καινούριας, σύγχρονης εργαλειοθήκης που θα μπορεί να παίξει το ρόλο του θεμελιώδους δομικού στοιχείου για
την επίτευξη τέτοιων ακριβειών στο άμεσο μέλλον. Με μικρές, σαφώς ορισμένες βελτιώσεις, εκτιμάται ότι
το πακέτο που σχεδιάστηκε για την παρούσα Διατριβή μπορεί να υπηρετήσει το ρόλο αυτό.

Για τον έλεγχο της ποιότητας του πακέτου που σχεδιάστηκε, αναπτύχθηκε ένα πρόγραμμα για τον προσδιορισμό
της τροχιάς της δορυφορικής αποστολής \textlatin{JASON}-3. Οι εκτιμήσεις θέσης και ταχύτητας του δορυφόρου
που προέκυψαν, για μία ημέρα ενδεικτικά, συγκρίθηκαν με αντίστοιχα αποτελέσματα της υψηλότερης
δυνατής ακρίβειας από την \textlatin{CNES/SSALTO}. Οι διαφορές που προέκυψαν είναι της τάξης των λίγων μέτρων
και των λίγων χιλιοστών ανά μέτρο για την θέση και την ταχύτητα αντίστοιχα.

Τα αποτελέσματα υποδεικνύουν ότι η εργαλειοθήκη που σχεδιάστηκε και αναπτύχθηκε μπορεί να
αποτελέσει το κύριο δομικό στοιχείο ενός προγράμματος επεξεργασίας δεδομένων \textlatin{doris},
ποιότητας εφάμιλλης των κέντρων ανάλυσης της τεχνικής.
\fi
\selectlanguage{english}
