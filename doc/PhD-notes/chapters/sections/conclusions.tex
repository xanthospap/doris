\section{Conclusions}\label{sec:conclusions}

\iffalse
\gls{doris} is one of the fundamental techniques of Satellite Geodesy, steadily contributing 
since the early 1990s significant results, products and insight in a wide range of geosciences 
and related fields. It has played a crucial role in expanding the geodetic knowledgebase 
and our understanding of the Earth's dynamics. Along with \gls{gnss}, \gls{slr} and 
\gls{vlbi}, it constitutes a fundamental pillar of reference frame maintenance. Yet, 
its scientific audience is dsiproportionally limited, a fact evident by the small number 
of dedicated Analysis Centers compared to the other space geodetic techniques. This 
lack of dedicated, scientific software tools for a prominent space geodetic technique 
such as \gls{doris} was the target and driving force for this Thesis.
\fi
\gls{doris} is a fundamental technique in Satellite Geodesy that has been steadily 
contributing significant results, products, and insights in a wide range of 
geosciences and related fields since the early 1990s. It has played a crucial 
role in expanding the geodetic knowledgebase and enhancing our understanding 
of the Earth's dynamics. Together with \gls{gnss}, \gls{slr}, and \gls{vlbi}, \gls{doris} 
forms a fundamental pillar of reference frame maintenance. However, despite its 
importance, the scientific audience for \gls{doris} is disproportionately 
limited, as evidenced by the small number of dedicated Analysis Centers compared 
to other space geodetic techniques. This lack of dedicated scientific software 
tools for a prominent space geodetic technique is a significant challenge that 
needs to be addressed. The absence of specialized, scientific software tools for 
a significant space geodetic method like \gls{doris} served as the primary 
objective and motivation for this thesis.

\iffalse
Within the framework of this Thesis, a software package was designed and implemented from scratch, 
with the aim of acting as a fundamental building block for a state-of-the-art, \gls{ids} 
level, dedicated analysis tool for processing \gls{doris} observations. The design 
principle adopted was twofold, based on efficiency on the one hand and on the same time 
allowing for genericity and modularity. This architectural approach will hopefully 
attract new users, since it favors reusability, maintenance and repurposing. The software 
package is free and open-source, a policy adopted to further aid the growth of the 
user community and in turn boost interest in the technique and thus accommodate further 
developments.
\fi
This thesis presents the development and implementation of a software package 
that serves as a fundamental building block for a dedicated analysis tool to 
process \gls{doris} observations at an \gls{ids} level. The software package 
was designed with two main principles in mind: efficiency and genericity/modularity. 
This architectural approach is expected to attract new users by promoting 
reusability, maintenance, and repurposing. The software package is free and open-source, 
aimed at fostering growth in the user community and stimulating interest in the 
\gls{doris} technique for further development.

The software built amounts to approximately 100000 lines of source code, divided into 
different modules/libraries (\autoref{sec:the-software}). Using this design, it is 
trivial to create a \gls{doris} analysis program to perform orbit determination 
(\autoref{sec:jason3-pod}), since all required functionality is delivered via the 
different modules. Adding yet another satellite in the processing chain, would only 
require the development of a dedicated attitude model (as e.g. done for \gls{jason}-3, \autoref{ssec:jason3-attitude}) 
and accommodating for the new state parameters in the filter.

\iffalse
Results obtained for the case of orbit determination using the \gls{jason}-3 satellite, 
show average discrepancies, with respect to the \gls{cnes}/SSALTO (multitechnique) solution, 
that amount to a few \si{\meter} in position and a few \si{\milli\meter\per\sec} for 
velocity estimates. Evidently, this cannot be qualified as a \gls{pod} result, but limitations 
of the current analysis pipeline are well identified and easy to transcend. These are
\fi
The results obtained from orbit determination using the \gls{jason}-3 satellite 
reveal average discrepancies, with position discrepancies amounting to a few 
meters and velocity estimates differing by a few millimeters per second compared 
to the CNES/SSALTO (multitechnique) solution. While these outcomes fall short of 
a \gls{pod} result, limitations of the current analysis pipeline are well 
recognized and easily surmountable:
\begin{itemize}
  \item Limitations in the force model, including:
    \begin{itemize}
      \item Currently only a few major ocean tidal constituents are considered. More 
        can be added since the capacity is already there.
      \item Account for earth pole tides and ocean pole tides. These 
        tidal effects are less pronounced than the earth and ocean loading effects and 
        simpler to model.
      \item Apply more complex Earth radiation modeling (e.g. include albedo and empirical models)
      \item Add dealiasing products (some gravity models were determined using this model)
      \item Account for atmospheric loading effects
    \end{itemize}
  %\item Adjust tropospheric for more robust tropospheric delay estimation
  \item Handle displacement of reference points (effects of tides on crustal deformation)
  \item Implement a more robust treatment of \emph{proper time} (currently estimates extracted from RINEX)
  \item Discriminate between \emph{zero tide} and \emph{tide free} models and adopt a unified approach
\end{itemize}

Yet the most important step towards a more precise estimation process, would be the adoption 
of a more robust filtering design. Currently the \emph{Extended Kalman Filter} algorithm is used 
as estimator (\autoref{sec:pod-extended-kalman-filter}). The more efficient and 
rigorous formulation of the Square Root Information Filtering (SRIF) algorithm could 
replace the current implementation. More crucially though, the introduction of stochastic 
properties via \emph{process noise}. Currently, no process noise is introduced in the filter, 
meaning that the force model is considered ``errorless'', and the estimation is not 
allowed to deviate and account for mismodeled or even unmodeled effects. Process noise 
needs to be included in the model also due to modeling approximations and model integration errors.
In a least squares sense, such effects are treated using stochastic accelerations.

\iffalse
It should be noted that the above constitute refinements to the already existing 
software package. The bulk of the work, to reach a state-of-the-art, scientific 
software tool has already been performed and via this Thesis is made available to all 
interested parties. By far the biggest part of such a tool is already in place, 
extensively documented and thoroughly tested. Considering the fact that the few 
software packages that do exist for accurate processing of \gls{doris} observations 
are either non-free and/or not open-sourced, the tools presented here constitute a 
novel alternative.
\fi

It is noteworthy that the aforementioned improvements pertain to the enhancement 
of an existing software package. The majority of the work, aimed at achieving a 
state-of-the-art, scientific software tool, has already been accomplished and 
is now available to all interested parties through this thesis. The majority 
of this tool has been implemented, extensively documented, and rigorously tested. 
Given that the limited number of software packages available for accurate processing 
of \gls{doris} observations are either not free and/or not open-source, the 
tools presented in this thesis represent a new and innovative alternative.

\iffalse
\gls{pod} analysis is a very challenging task, requiring multi-scientific expertise, 
coupled with efficient engineering. Given the accuracy requirements today (especially 
for altimetry missions on-board which \gls{doris} receivers are installed), the 
ever growing number of scientific satellites missions, and the widening of the application 
range (e.g. climate change studies), it constitutes a field of growing dynamics.
\fi
