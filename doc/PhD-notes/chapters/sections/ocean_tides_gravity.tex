\subsubsection{Ocean Tide}\label{sssec:ocean-tide-perturbations}

As in the case of Earth tides (see \autoref{sssec:earth-tide-perturbations}), the 
dynamical effects of ocean tides can be modeled as periodic variations in the 
Stoke's coefficients $\Delta \bar{C}_{nm}$ and $\Delta \bar{S}_{nm}$ (for 
degree $n$ and order $m$). To compute these values, the prograde and retrograde 
geopotential harmonic amplitudes $\mathcal{C}^{\pm}$ and $\mathcal{S}^{\pm}$ for 
$\mathcal{C}$ and $\mathcal{S}$ respectively are needed, for each tidal constituent 
$f$ involved. Adopting the development presented in \cite{iers2010}, we can compute 
the variations from
\begin{equation}
  \label{eq:iers10615}
  \left( \Delta \bar{C}_{nm} - \iim \Delta \bar{S}_{nm} \right) (t) =
    \sum_{f} \sum_{+}^{-} \left( 
      \mathcal{C}_{f,nm}^{\pm} \mp \mathcal{S}_{f,nm}^{\pm} \right)
    e^{\pm \iim \theta _f (t)}
\end{equation}
where
\begin{description}
  \item $f$ is a given tidal constituent,
  \item $\theta _f (t)$ is the argument of the constituent $f$ at epoch $t$, and
  \item $\mathcal{C}_{f,nm}^{\pm}$ and $\mathcal{S}_{f,nm}^{\pm}$ are the geopotential 
    harmonic amplitudes
\end{description}

Note that not all of the available ocean loading models are published in the form of 
$\mathcal{C}_{f,nm}^{\pm}$ and $\mathcal{S}_{f,nm}^{\pm}$ coefficients
\footnote{Typically, most are developed and distributed as gridded maps of tide 
height amplitudes. These models provide in-phase and quadrature amplitudes of tide 
heights for selected, main tidal frequencies (or main tidal waves), on a variable 
grid spacing over the oceans (\cite{iers2010}).}. However, using spherical harmonic 
decomposition and with the use of an Earth loading model, the maps of ocean tide 
height amplitudes have been converted to spherical harmonic coefficients for use 
in \autoref{eq:iers10615}. The procedure is outlined in e.g. \cite{iers2010}.

Typically, ocean tide models provide maps for only the largest tides or ``main waves''. 
Interpolation from the main waves to the smaller, ``secondary waves'' can be performed, 
using an assumption of linear variation of tidal admittance between closely spaced 
tidal frequencies. Through this method, the spectrum of tidal geopotential perturbations 
can be completed to the requested degree. For a secondary wave $f$, the needed 
coefficients can be computed as
\begin{equation}\label{eq:iers10616}
  \begin{aligned}
    \mathcal{C}_{f,nm}^{\pm} = \frac{\dot{\theta}_f - \dot{\theta}_1}{\dot{\theta}_2 - 
        \dot{\theta}_1} \frac{H_f}{H_2} \mathcal{C}_{2,nm}^{\pm} 
        + \frac{\dot{\theta}_2 - \dot{\theta}_f}{\dot{\theta}_2 - \dot{\theta}_1}
        \frac{H_f}{H_1} \mathcal{C}_{1,nm}^{\pm} \\
    \mathcal{S}_{f,nm}^{\pm} = \frac{\dot{\theta}_f - \dot{\theta}_1}{\dot{\theta}_2 - \dot{\theta}_1} 
      \frac{H_f}{H_2} \mathcal{S}_{2,nm}^{\pm} 
      + \frac{\dot{\theta}_2 - \dot{\theta}_f}{\dot{\theta}_2 - \dot{\theta}_1} 
      \frac{H_f}{H_1} \mathcal{S}_{1,nm}^{\pm}
  \end{aligned}
\end{equation}
where the subscript $1$ and $2$ denote the two nearby main lines, or \emph{pivot waves} 
and $H$ is the astronomic amplitude of the considered wave.

As in the case of solid Earth tides (\autoref{sssec:earth-tide-perturbations}), 
ocean loading can induce deformation on the Earth's crust and thus displacements 
on instrumentation sites. These effects are treated in ??.

\paragraph{Ocean Tide Models}\label{par:ocean-tide-models}

Ocean tide models are defined using an underlying tide height model, and further 
include the  maximum degree and order of the expansion and identification of the 
main, pivot waves. Since the mid-1990s, a series of \texttt{FES} (finite element 
solution) global ocean tidal atlases has been produced and released with the primary 
objective to provide altimetry missions with tidal de-aliasing correction at the 
best possible accuracy. In this Thesis, we will make use of the latest model in 
this series, labeled \texttt{FES2014} (\cite{Lyard2021}), since it is the one suggested 
from the \gls{ids} for the ITRF2020 processing campaign. \autoref{table:tidal-constituents-fes14b} 
lists the tidal constituents contained within the aforementioned data file.

\begin{figure}
  \begin{adjustbox}{max width=\linewidth , fbox=0.5pt}
  \begin{BVerbatim}
  Coefficients to compute variations in normalized Stokes coefficients (unit = 10^-11)
  Ocean tide model: FES2014b up to (180,180)
  Doodson Darw  l   m    DelC+     DelS+       DelC-     DelS-
  055.565 om1   1   0  -0.84987   0.00000    -0.84987   0.00000
  055.565 om1   2   0   2.55417  -0.00000     2.55417  -0.00000
  055.565 om1   3   0   0.02827  -0.00000     0.02827  -0.00000
  055.565 om1   4   0  -0.25307   0.00000    -0.25307   0.00000
  055.565 om1   5   0   0.34383  -0.00000     0.34383  -0.00000
  ...
  055.575 om2   1   0   0.00830   0.00000     0.00830   0.00000   
  055.575 om2   2   0  -0.02493   0.00000    -0.02493   0.00000   
  055.575 om2   3   0  -0.00028   0.00000    -0.00028   0.00000   
  055.575 om2   4   0   0.00247   0.00000     0.00247   0.00000   
  055.575 om2   5   0  -0.00336   0.00000    -0.00336   0.00000   
  \end{BVerbatim}
  \end{adjustbox}
  \caption{Part of \texttt{FES2014b} geopotential harmonic amplitudes
  $\mathcal{C}_{f,nm}^{\pm}$ and $\mathcal{S}_{f,nm}^{\pm}$ for tidal constituents
  055.565 and 055.575. File retrieved from the \texttt{COST-G} benchmark test 
  repository, \url{ftp://ftp.tugraz.at/outgoing/ITSG/COST-G/}.}
  \label{fig:part-of-fes2014b}
\end{figure}

\begin{table}[]
  \centering
  \begin{tabular}{ccc}
      \hline
      \textbf{Doodson Number} & \textbf{Darwin Symbol} & \textbf{Description} \\
      %& \multicolumn{4}{c}{\si{\metre\per\square\second}} \\
      \hline
      055.565 & $\Omega _1$ & Lunar Saros \\
      055.575 & $\Omega _2$ & \\
      056.554 & $S_a$ & Solar annual \\
      057.555 & $S_{sa}$ & Solar semiannual \\
      065.455 & $M_m$ & Lunar monthly \\
      075.555 & $M_f$ & Lunisolar fortnightly \\
      085.455 & $M_{tm}$ & \\
      093.555 & $M_{sqm}$ & \\
      135.655 & $Q_1$ & Larger lunar elliptic diurnal \\
      145.555 & $O_1$ & Principal lunar declinational \\
      163.555 & $P_1$ & Principal solar declination \\
      164.555 & & \\
      165.555 & $K_1$ & Lunisolar diurnal \\
      175.455 & $J_1$ & Smaller lunar elliptic diurnal \\
      227.655 & $\epsilon _2$ & \\
      235.755 & $2N_2$ & Lunar elliptical semidiurnal second-order \\
      237.555 & $\mu _2$ & Variational \\
      245.655 & $N_2$ & Larger lunar elliptic semidiurnal \\
      247.455 & $\nu _2$ & Larger lunar evectional \\
      255.555 & $M_2$ & Principal lunar semidiurnal \\
      263.655 & $\lambda _2$ & Smaller lunar evectional \\
      265.455 & $L_2$ & Smaller lunar elliptic semidiurnal \\
      272.556 & $T_2$ & \\
      273.555 & $S_2$ & Principal solar semidiurnal \\
      274.554 & $R_2$ & \\
      275.555 & $K_2$ & Lunisolar semidiurnal \\
      355.555 & $M_3$ & Lunar terdiurnal \\
      435.755 & & \\
      445.655 & $M_{N4}$ & Shallow water quarter diurnal \\
      455.555 & $M_4$ & Shallow water overtides of principal lunar \\
      473.555 & $M_{s4}$ & Shallow water quarter diurnal \\
      491.555 & $S_4$ & Shallow water overtides of principal solar \\
      655.555 & $M_6$ & Shallow water overtides of principal lunar \\
      855.555 & $M_8$ & Shallow water eighth diurnal\\
      \hline
  \end{tabular}
  \caption{List of ``main'' tidal constituents contained listed in \texttt{FES2014b}; published 
    file via \texttt{COST-G}. Description is extracted from \cite{Beauducel2023}.}
  \label{table:tidal-constituents-fes14b}
\end{table}

\paragraph{Implementation}\label{par:ocean-tide-acceleration-implementation}
Design and implementation of the procedure outlined above for the computation of 
variations in the Stoke's coefficients $\Delta \bar{C}_{nm}$ and $\Delta \bar{S}_{nm}$, 
poses a series of challenges. First off, a decision must be made to allow for the 
distinction between different tidal waves. This allows for efficiency, adaptability, 
extensibility and ease of use. To that end, a generic class was designed to represent 
different waves, based on the Doodson number; this approach offers several advantages, 
as well as the ability to compute arguments of the tide constituents (e.g. $\theta$ 
in \autoref{eq:iers10615}), via the Doodson multipliers, which are in turn obtained 
through the computation of the fundamental arguments of nutation theory or 
\emph{Delaunay variables} $l$, $l_0$, $F$, $D$ and $\Omega$, augmented by the \gls{gmst}. 
These values are computed using the latest formulations (\cite{iers2010}).

Next, efficient mapping/loading of the coefficients $\mathcal{C}_{f,nm}^{\pm}$ and 
$\mathcal{S}_{f,nm}^{\pm}$ must be developed. Note that we have four such coefficients 
per tidal wave, and for every combination of degree and order $(n,m)$; as can be seen 
in \autoref{fig:part-of-fes2014b}, the maximum degree and order in this case is 180;
that amounts to a huge number of floating point numerics that need to be stored 
and loaded to/from memory. The software developed, uses efficient data structures and 
contiguous memory chunks to speed-up computations and only allocate the memory needed.

Computation of $\Delta \bar{C}_{nm}$ and $\Delta \bar{S}_{nm}$ values follows 
\autoref{eq:iers10615}. When needed, interpolation through tidal admittance (see 
\autoref{eq:iers10616}) is also handled in the software, per user/application needs. 