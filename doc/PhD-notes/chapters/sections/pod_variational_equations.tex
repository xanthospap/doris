\section{Variational Equations}\label{sec:pod-variational-equations}

For application with high accuracy demands, the state transition matrix should include 
terms at least the major perturbations. That is, the initial value problem should be 
expanded to include differential equations to account for perturbed motion. An 
analytical solution of this problem is close to impossible, thus this extended  
formulation should be solved for numerically. The added differential equations (to 
the state transition matrix system) are labelled \emph{variational equations}. 
Aside from the increased accuracy that may be obtained by accounting for perturbations, the
concept of the variational equations offers the advantage that it is not limited to the
computation of the state transition matrix, but may also be extended to the treatment
of partial derivatives with respect to force model parameters (\cite{Montenbruck2000}).

\subsection{Differential Equations for the State Transition Matrix}\label{ssec:pod-state-transition-ode}
Denoting the state vector as $\bm{y}(t) = \begin{pmatrix}\bm{r}(t) & \bm{v}(t) \end{pmatrix}^{T}$, 
the differential equation, which describes the change of the state transition matrix 
with time in a first-order \gls{ode} 
\begin{equation}\label{eq:mont738}
    \frac{d \bm{y}(t)}{d t} = \bm{f}(t,\bm{y}(t)) = 
        \begin{pmatrix}\bm{v}(t)\\\bm{a}(t,\bm{r},\bm{v})\end{pmatrix}
\end{equation}
An differentiating with respect to $\bm{y}(t_0)$ gives
\begin{equation}
    \frac{\partial}{\partial \bm{y}(t_0)} \left( \frac{d \bm{y}(t)}{d t} \right) 
        = \frac{\partial \bm{f}(t,\bm{y}(t))}{\partial \bm{y}(t_0)} 
        = \frac{\partial \bm{f}(t,\bm{y}(t))}{\partial \bm{y}(t)}\frac{\partial \bm{y}(t)}{\partial \bm{y}(t_0)}
\end{equation}
Since the state transition matrix $\Phi (t,t_0)$ is given by
\begin{equation}\label{eq:mont740}
    \Phi (t,t_0) = \frac{\partial \bm{y}(t)}{\partial \bm{y}(t_0)}
\end{equation}
its derivative can be computed from
\begin{equation}\label{eq:mont7412}
    \begin{aligned}
        \frac{d}{dt} \Phi (t,t_0) & = \frac{\partial \bm{f}(t,\bm{y}(t))}{\partial \bm{y}(t)} 
        \Phi (t,t_0) \\
        &= \begin{pmatrix}
            \frac{\partial \bm{v}(t,\bm{r}, \bm{v})}{\partial \bm{r}(t)} &
            \frac{\partial \bm{v}(t,\bm{r}, \bm{v})}{\partial \bm{v}(t)} \\
            \frac{\partial \bm{a}(t,\bm{r}, \bm{v})}{\partial \bm{r}(t)} &
            \frac{\partial \bm{a}(t,\bm{r}, \bm{v})}{\partial \bm{v}(t)}
        \end{pmatrix} \Phi (t,t_0) \\
        &= \begin{pmatrix}
            \bm{0}_{3 \times 3} & \bm{I}_{3 \times 3} \\
            \frac{\partial \bm{a}(t,\bm{r}, \bm{v})}{\partial \bm{r}(t)} &
            \frac{\partial \bm{a}(t,\bm{r}, \bm{v})}{\partial \bm{v}(t)}
        \end{pmatrix} \Phi (t,t_0)
    \end{aligned}
\end{equation}
Paired with the initial condition $\Phi (t_0,t_0) = \bm{I}_{6 \times 6}$, 
\autoref{eq:mont7412} can be solved for as an initial value problem, using 
numerical integration.

\subsection{Differential Equations for the Sensitivity Matrix}\label{ssec:pod-sensitivity-ode}
To form the variational equations, the partial derivatives of the state with respect 
to the $n_p$ dynamical, or force model parameters $p_i$ are needed. Taking the 
time derivatives
\begin{equation}
    \frac{d}{dt}\frac{\partial \bm{y}(t)}{\partial \bm{p}} 
        = \frac{\partial \bm{f}(t,\bm{y}(t), \bm{p})}{\partial \bm{y}(t)} 
            \frac{\partial \bm{y}(t)}{\partial \bm{p}}
            + \frac{\partial \bm{f}(t,\bm{y}(t), \bm{p})}{\partial \bm{p}}
\end{equation}
and hence using the sensitivity matrix defined in \autoref{sec:pod-linearization},
\begin{equation}
    \frac{d}{dt}\bm{S}(t) = 
    \begin{pmatrix}
        \bm{0}_{3 \times 3} & \bm{I}_{3 \times 3} \\
        \frac{\partial \bm{a}(t,\bm{r}, \bm{v}, \bm{p})}{\partial \bm{r}(t)} &
        \frac{\partial \bm{a}(t,\bm{r}, \bm{v}, \bm{p})}{\partial \bm{v}(t)}
    \end{pmatrix}_{6 \times 6} \bm{S}(t)
    +
    \begin{pmatrix}
        \bm{0}_{3 \times n_p} \\
        \frac{\partial \bm{a}(t,\bm{r}, \bm{v}, \bm{p})}{\partial \bm{p}}
    \end{pmatrix}_{6 \times n_p}
\end{equation}
The initial condition for the above system is $\bm{S}(t_0) = \bm{0}$, since the 
state vector at $t_0$ does not depend on the force model parameters.

\subsection{Solving the Variational Equations}\label{ssec:pod-varequations-solution}
Combining the differential equation systems formed above for the state transition matrix 
(\autoref{ssec:pod-state-transition-ode}) and the sensitivity matrix (\autoref{ssec:pod-sensitivity-ode}),
the full system of variational equations is formed, which reads
\begin{equation}\label{eq:mont745}
    \frac{d}{dt} \begin{pmatrix} \Phi & \bm{S} \end{pmatrix} = 
    = \begin{pmatrix}
        \bm{0}_{3 \times 3} & \bm{I}_{3 \times 3} \\
        \frac{\partial \bm{a}}{\partial \bm{r}} &
        \frac{\partial \bm{a}}{\partial \bm{v}}
    \end{pmatrix}_{6 \times 6} \begin{pmatrix} \Phi & \bm{S} \end{pmatrix}
    + 
    \begin{pmatrix}
        \bm{0}_{3 \times 6} & \bm{0}_{3 \times n_p} \\
        \bm{0}_{3 \times 6} & \frac{\partial \bm{a}}{\partial \bm{p}}
    \end{pmatrix}_{6 \times n_p}
\end{equation}

Given also the initial conditions (desribed in \autoref{ssec:pod-state-transition-ode} and 
\autoref{ssec:pod-sensitivity-ode})
\begin{equation}
    \begin{pmatrix}\Phi (t_0,t_0) _{6 \times 6} & \bm{S}(t_0) _{}\end{pmatrix}
    \begin{pmatrix} \bm{I}_{6 \times 6} & \bm{0}_{6 \times n_p} \end{pmatrix}
\end{equation}
an initial value problem of $1$\textsuperscript{st} degree is formed, which can 
be solved for by methods of numerical integration (see \autoref{ch:orbit-integration}).
A slightly different approach is presented in \autoref{sec:alternate-variational-equations}, 
starting from the state-space representation.

It is important to note that the variational equations have to be integrated 
simultaneously with the state vector. Otherwise the position and velocity of the 
satellite, which are required to evaluate the acceleration partials (right-hand 
side of the variational equations), would be unknown. 
