\subsubsection{Solar Radiation Pressure}\label{sssec:solar-radiation-pressure}

Quantum mechanics says that each photon of frequency \(\nu\) and wavelength
\(\lambda = c / \nu\) (where \(c\) is the speed of light in vacuum) carries the energy
\begin{equation}
    E = h \nu
\end{equation}
and the linear momentum
\begin{equation}
    \bm{p} = \frac{h \nu}{c} \bm{e}
\end{equation}
where \(h=6.62\cdot10^{-34}\) Js is Planck's constant and \(\bm{e}\) is the unit 
vector of the propagation of the photon.

The momentum transferred per time unit onto a unit surface in a radiation
field is called radiation pressure (hence a vectorial quantity).

In a general radiation field it may be quite an elaborate task to calculate
the radiation pressure acting on a surface element. The underlying principle
is simple, however (the subsequent elaboration follows \cite{BeutlerVII}): 
We have to calculate the change \(\Delta \bm{p} = \bm{p}_a -\bm{p}_b\) of 
linear momentum \(\bm{p}\) for each of the photons hitting the surface element 
\(S\) (the index \(b\) characterizes the momentum before, \(a\) that after 
hitting the surface). The momentum transferred to the surface element must be 
\(\bm{p_s} \equiv - \Delta \bm{p} = \bm{p}_b -\bm{p}_a \).
After having computed the momentum transferred by each of the photons
we simply have to compute the vectorial sum of the momenta \(\bm{p_s}\).

According to the law of conservation of linear momentum \(-\Delta \bm{p}\) is 
the momentum gained by the surface element. If the photon is absorbed by the surface, 
\(\bm{p}_a = 0\) and the surface gains the momentum \(\bm{p}_{abs} = \bm{p}_b\).
If the photon is reflected by the surface, the photon is reflected according to Snellius' law
of reflection. The surface gains linear momentum normal to its surface with
absolute value \(2 \lvert \vec{p}_b \rvert cos\alpha = \lvert \bm{p}_{ref} \rvert\).

The primary radiation source to be considered in satellite geodesy is the
Sun. The radiation pressure due to the direct solar radiation is also referred
to as \emph{direct radiation pressure}. Other sources of radiation are the Earth, which
reflects and/or re-emits the radiation received by the Sun, or to a much lesser
extent, the Moon, reflecting the solar radiation.

The radiation field due to the direct solar radiation may be considered as
parallel to the direction \(Sun \to Satellite\). Consequently, the momentum trans-
ferred per second to a unit surface normal to this direction may be expressed
by the solar constant \(S = \SI{1368}{\watt\per\square\m}\), giving the energy 
flowing through this surface per time unit at the distance of 
\SI{1}{\astronomicalunit}.

If 100\% of the radiation is absorbed,  the linear momentum
\begin{equation}
    \bm{p} = \frac{S}{c} \bm{e}
\end{equation}
is gained in one second by a surface element of \SI{1}{\square\metre} normal to 
the direction \(Sun \to Satellite\) element at \SI{1}{\astronomicalunit}. The 
absolute value of the vector is
\begin{equation}
  \label{eq:beutler3141}
    P_{\Sun} \equiv \norm{\bm{p}} = \frac{S}{c} = \SI{4.56316e-6}{\N\per\square\m}
\end{equation}
Obviously, the surface of the cross section of the satellite normal to the 
direction \(Sun \to Satellite\) and the mass \(m\) of the satellite play the same roles as 
the corresponding quantities when dealing with drag (\ref{ssec:acceleration-due-to-drag}).

Assuming rotational symmetry of the satellite w.r.t. the axis \(Sun \to Satellite\), 
the acceleration due to the direct solar radiation may be written as (\cite{BeutlerVII})
\footnote{Note that \cite{Vallado} uses an alternate formulation for \ref{eq:beutler3142}, 
namely: 
\( \bm{\alpha}_{srp} =  - \frac{p_{srp} C_R A_{\Sun}}{m} \frac{\bm{r}_{sat\Sun}}{\lvert \bm{r}_{sat\Sun} \rvert} \), 
where \(p_{srp} \approx \SI{4.57e-6}{\N\per\square\m}\) is the solar pressure force 
per unit area, or the change in momentum, \(C_R\) is the \emph{reflectivity} and 
\(A_{\Sun}\) the exposed area to the Sun. Determining \(C_R\) is extremely
difficult. It changes over time and is virtually impossible to predict. This is especially
true for complex satellites made of various materials, that enter and exit eclipse regions,
and have a constantly changing orientation. For this reason, it's almost always a solution
parameter, determined in differential correction (just like the drag coefficient).}:
\begin{equation}
    \label{eq:beutler3142}
    \bm{\alpha}_{rad} = \frac{\tilde{C}}{2} 
    \frac{{AU}^2}{{\norm{\bm{r} - \bm{r}_{\Sun}}}^2}
    \frac{S}{c} \frac{A}{m}
    \frac{\bm{r} - \bm{r}_{\Sun}}{\norm{\bm{r} - \bm{r}_{\Sun}}}
\end{equation}
where:
\begin{itemize}
\item[\(\tilde{C}\)] depends on the reflective properties of the satellite surface,
\item[\(\bm{r}\)] is the geocentric position vector of the satellite,
\item[\(\bm{r_{\Sun}}\)] is the geocentric position vector of the Sun,
\item[\(AU\)] is the Astronomical Unit,
\item[\(m\)] is the mass of the satellite, and
\item[\(A\)] is the cross section of the satellite normal to the direction \(Sun \to Satellite\)
\end{itemize}

Using \ref{eq:beutler3141}, \ref{eq:beutler3142} can be simplified to:
\begin{equation}
  \label{eq:mont375}
  \bm{\alpha}_{rad} = \norm{\bm{p}} \tilde{C} \frac{A}{m} \frac{\bm{r} - \bm{r}_{\Sun}}{\norm{\bm{r} - \bm{r}_{\Sun}} ^3} {AU}^2
\end{equation}

Rotational symmetry w.r.t. the axis \(Sun \to Satellite\) may not always
be assumed. Nevertheless, formula \ref{eq:beutler3142} is in practice an excellent first-
order approximation to take radiation pressure into account. If the radiation
is mainly absorbed (as it should be the case, e.g., for solar panels), formula 
\ref{eq:beutler3142} is correct for satellites of arbitrary shape (\cite{BeutlerVII}).
Note that in contrast to atmospheric drag, direct radiation pressure practically does 
not depend on the height of the satellite above the Earth's surface. This is one of 
the reasons why radiation pressure is the dominating non-gravitational perturbation 
above heights of about \SI{600}{\km} (\cite{BeutlerVII}).

Radiation pressure obviously is ``turned off'' if the radiation is blocked by an 
``obstacle'', i.e the Moon or Earth, between the Sun and the satellite. 

\subsection{Solar Radiation Pressure Partials}
\label{ssec:solar-radiation-pressure-partials}
Starting from \ref{eq:mont375}, the acceleration due to solar radiation pressure 
aries with the satellite position in the same way as the gravitational attraction
of the Sun, with derivatives given by (\cite{Montenbruck2000}):
\begin{equation}
  \label{eq:mont777}
  \frac{\partial \ddot{\bm{r}}}{\partial \bm{r}} = 
  P_{\Sun} C \frac{A}{m} {AU}^2 \left( 
    \frac{1}{\norm{\bm{r} - \bm{r}_{\Sun}}^3} \bm{I}_{3\times 3}
    -3 \left( \bm{r} - \bm{r}_{\Sun} \right) 
      \frac{\left( \bm{r} - \bm{r}_{\Sun} \right) ^T}{\norm{\bm{r} - \bm{r}_{\Sun}}^5}
      \right)
\end{equation}

Here $\bm{r}$ and $\bm{r}_{\Sun}$ are the geocentric coordinates of the 
satellite and the Sun, respectively.

Due to the fact that the radiation pressure coefficient $C$ is usually estimated 
during orbit determination, the derivative of the acceleration w.r.t $C$ must 
be calculated. The respective partials are (\cite{Montenbruck2000}):
\begin{equation}
  \label{eq:mont778}
  \frac{\partial \ddot{\bm{r}}}{\partial C} = \frac{1}{C} \ddot{\bm{r}} = 
  -P_{\Sun} \frac{A}{m} 
  \frac{\bm{r} - \bm{r}_{\Sun}}{\norm{\bm{r} - \bm{r}_{\Sun}}^3} {AU}^2
\end{equation}
required to compute the influence of variations in the radiation pressure
coefficient on the satellite trajectory.

\section{Satellite Eclispes and Earth Shadow}
Satellites routinely enter eclipse periods during their orbits, and
this can cause a great fluctuation in the available solar radiation for the solar 
arrays.

Examining \ref{fig:eclipse-geometry}, we are interested in two main regions: the 
\emph{umbra} is totally eclipsed by the primary body (Earth); the \emph{penumbra} 
is only partially obscured by the primary.
\begin{equation}
    \sin({\alpha}_{umb}) = \frac{r_S - r_E}{R} \text{ and }
    \sin({\alpha}_{pen}) = \frac{r_S + r_E}{R}
\end{equation}

If we consider the the primary (or occulting) body to be the Earth and the 
secondary body to be the Sun, then:
\begin{equation}
    \sin({\alpha}_{umb}) = \frac{r_{\Sun} - r_{\Earth}}{R_{\Earth\Sun}} 
    \Rightarrow {\alpha}_{umb} \approx \SI{0.264}{\degree}
    \text{ and }
    \sin({\alpha}_{pen}) = \frac{r_{\Sun} + r_{\Earth}}{R_{\Earth\Sun}}
    \Rightarrow {\alpha}_{pen} \approx \SI{0.269}{\degree}
\end{equation}

\begin{figure}
\centering
\input{tikz/eclipse-geometry.tikz}
\caption{Eclipse Geometry; \emph{scales are greatly exaggerated}}
\label{fig:eclipse-geometry}
\end{figure}

The distance from the occulting planet to the vertex (\(V\)) can be found via 
using the similar triangles (see \ref{fig:eclipse-geometry}):
\begin{equation}
    \label{eq:val501}
    \frac{\overline{EV}}{r_{E}} = \frac{\overline{SV}}{r_S} \Rightarrow \overline{EV} = \overline{SV} \frac{r_E}{r_S}
\end{equation}
We know the (ralative) position of the two (primary and secondary) planets, hence:
\(\overline{SV} = R + \overline{EV}\); substituting into \ref{eq:val501} yields:
\begin{align*}
    \label{eq:val51}
    \overline{EV} &= R \frac{r_E}{r_S} + \overline{EV} \frac{r_E}{r_S} \\
    \overline{EV} (1-\frac{r_E}{r_S}) &= R \frac{r_E}{r_S} \\
    \overline{EV} &= R \frac{r_E}{r_S - r_E}
\end{align*}

Using values for the Earth and the Sun, we find \(\overline{PV}\) to 
be \SI{1.384e6}{\km}. This is almost four times the distance from the 
Earth to the Moon (\SI{3.8268e5}{\km}) (\cite{Vallado}). Hence, it comes as little surprise 
that the Earth's shadow sometimes completely obscures the Moon, causing a 
lunar eclipse. Using values for the Moon and the Sun, 
\ref{eq:val51} yields a value of about \SI{3.735e5}{\km} (\cite{Vallado}) 
which is not too far from the Earth-Moon distance (approximately 
\SI{3.844e5}{\km}). When the Moon obscures part of the Earth's surface with its 
shadow, it causes a \emph{solar eclipse}.
