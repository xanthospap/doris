\section{Variational Equations Differential Equations}\label{sec:alternate-variational-equations}

A slightly different but enlighting approach is presented here for the formation 
and solution of the variational equation system of differential equations. The 
developments presented here can be found in \cite{Tapley2004}. The $\Phi$ matrix 
here will contain the combined state transition matrix and the sensitivity 
matrix $\bm{S}(t)$, see \autoref{ssec:pod-sensitivity-ode}.

The differential equation for the variational equations can be written in the 
form
\begin{align}
  \dot{\Phi}(t, t_0) &= A(t) \Phi (t,t_0) \text{, with} \label{eq:tapleyg111}\\
  \Phi (t_0, t_0) &= \bm{I} \label{eq:tapleyg112}
\end{align}
The state transition matrix can be split in the following way
\begin{equation}\label{eq:tapleyg21}
   \Phi (t,t_0) \equiv \frac{\partial \bm{y}(t)}{\partial \bm{y}_0} \equiv 
    \begin{pmatrix}
      \phi _1 (t,t_0) \\ \phi _2 (t,t_0) \\ \phi _3 (t,t_0) \end{pmatrix}
    = \begin{pmatrix}
      \frac{\partial \bm{r}(t)}{\partial \bm{y}_0} \\
      \frac{\partial \bm{v}(t)}{\partial \bm{y}_0} \\
      \frac{\partial \bm{p}(t)}{\partial \bm{y}_0} \end{pmatrix}
\end{equation}
where 
\begin{equation}
  \phi _3 (t,t_0) = \begin{pmatrix} \bm{0}_{n_p \times 6} & \bm{I}_{n_p \times n_p} \end{pmatrix}
\end{equation}
By differentiating \autoref{eq:tapleyg21}, \autoref{eq:tapleyg111} in terms of a 
second order differential equation,
\begin{equation}\label{eq:tapleyg22}
  \dot{\Phi}(t,t_0) = \frac{\partial \dot{\bm{y}}(t)}{\partial \bm{y}_0} = 
    \begin{pmatrix}
      \dot{\phi} _1 (t,t_0) \\ \dot{\phi} _2 (t,t_0) \\ \dot{\phi} _3 (t,t_0) \end{pmatrix}
    = \begin{pmatrix}
      \frac{\partial \dot{\bm{r}(}t)}{\partial \bm{y}_0} \\
      \frac{\partial \dot{\bm{v}}(t)}{\partial \bm{y}_0} \\
      \bm{0}_{n_p \times 6} \end{pmatrix}
    = \begin{pmatrix}
      \frac{\partial \dot{\bm{r}(}t)}{\partial \bm{y}(t)} \\
      \frac{\partial \dot{\bm{v}}(t)}{\partial \bm{y}(t)} \\
      \bm{0}_{n_p \times 6} \end{pmatrix} \frac{\partial \bm{y}(t)}{\partial \bm{y}_0}
\end{equation}
Note that $\dot{\phi} _2 = \ddot{\phi} _1$.

The second order differential equation could be solved to obtain $\Phi (t, t_0)$
\begin{equation}\label{eq:tapleyg24}
  \begin{aligned}
    \ddot{\phi _1} (t,t_0) &= \frac{\partial \ddot{r}(t)}{\partial \bm{y}_0} = 
    \frac{\partial \ddot{\bm{r}}(t)}{\partial \bm{y}(t)}
      \frac{\partial \bm{\ddot{y}}(t)}{\partial \bm{y}_0} \\
    &=\begin{pmatrix}
      \frac{\partial \ddot{\bm{r}}(t)}{\partial \bm{r}(t)} & 
      \frac{\partial \ddot{\bm{r}}(t)}{\partial \bm{v}(t)} & 
      \frac{\partial \ddot{\bm{r}}(t)}{\partial \bm{p}} \end{pmatrix} 
    \begin{pmatrix}
      \frac{\partial       \bm{r} (t)}{\partial \bm{y}_0} \\
      \frac{\partial  \dot{\bm{r}}(t)}{\partial \bm{y}_0} \\
      \frac{\partial       \bm{p}    }{\partial \bm{y}_0} \end{pmatrix} 
  \end{aligned}
\end{equation}
which reduces to
\begin{equation}\label{eq:tapleyg25}
  \ddot{\phi _1} (t,t_0) = 
    \frac{\partial \ddot{\bm{r}}(t)}{\partial \bm{r}(t)} \phi _1 (t,t_0) 
    + \frac{\partial \ddot{\bm{r}}(t)}{\partial \bm{v}(t)} \dot{\phi} _1 (t,t_0) 
    + \frac{\partial \ddot{\bm{r}}(t)}{\partial \bm{p}} \phi _3 (t,t_0)
\end{equation}
with initial conditions
\begin{align}
  \phi _1 (t_0, t_0) &= \begin{pmatrix} \bm{I} & \bm{0} \end{pmatrix} \\
  \dot{\phi} _1 (t_0, t_0) &=  \phi _2 (t_0, t_0) = 
    \begin{pmatrix} \bm{0} & \bm{I} & \bm{0} \end{pmatrix}
\end{align}
This second order \gls{ode} system can be transformed to a $n \times n$ first-order system 
(if the solution of first order \gls{ode} is preffered)
\begin{equation}
  \begin{aligned}
    \dot{\phi} _1 (t,t_0) &= \phi _2 (t,t_0)\\
    \dot{\phi} _2 (t,t_0) &= \frac{\partial \ddot{\bm{r}}(t)}{\partial \bm{r}(t)} \phi _1 (t,t_0) 
      + \frac{\partial \ddot{\bm{r}}(t)}{\partial \bm{v}(t)} \phi _2 (t,t_0) 
      + \frac{\partial \ddot{\bm{r}}(t)}{\partial \bm{p}} \phi _3 (t,t_0) \\
    \dot{\phi} _3 (t,t_0) &= \bm{0}
  \end{aligned}
\end{equation}
or $\dot{\Phi}_(t,t_0) = A(t) \Phi (t,t_0)$, or
\begin{equation}
  \begin{pmatrix}
     \dot{\phi} _1 (t,t_0) \\ \dot{\phi} _2 (t,t_0) \\ \dot{\phi} _3 (t,t_0) \end{pmatrix}_{n \times n}
  = \begin{pmatrix}
      \bm{0}_{3 \times 3} & \bm{I}_{3 \times 3} & \bm{0}_{3 \times m} \\
      \left(\frac{\partial \ddot{\bm{r}}(t)}{\partial \bm{r}(t)}\right)^{*}_{3 \times 3} & 
      \left(\frac{\partial \ddot{\bm{r}}(t)}{\partial \bm{v}(t)}\right)^{*}_{3 \times 3} &
      \left(\frac{\partial \ddot{\bm{r}}(t)}{\partial \bm{p}}   \right)^{*}_{3 \times m} \\
      \bm{0}_{m \times 3} & \bm{0}_{m \times 3} & \bm{0}_{m \times m}
    \end{pmatrix}_{n \times n}
    \begin{pmatrix}
      \phi _1 (t,t_0) \\
      \phi _2 (t,t_0) \\
      \phi _3 (t,t_0)
    \end{pmatrix}_{n \times n}
\end{equation}
