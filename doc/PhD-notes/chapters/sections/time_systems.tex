\subsection{A Short Introduction to Time Scales}\label{ssec:time-scales}

This section only touches upon the fundamentals of time scales used in Satellite Geodesy 
(see \autoref{table:time-scales})  
and relevant transformation algorithms, as applied in the software designed for the 
purposes of this Thesis. An extensive discussion on the subject can be found in i.e. 
\cite{Urban2013} and \cite{sofa2021ts}.

Historically, solar time was the basis of time keeping, based on the diurnal rotation
of the Earth. However, with the  discovery of the variability of the 
rate of rotation of the Earth and later on the development of atomic clocks, 
ans the SI second different time systems and scales have been introduced to match the 
ever growing precision demands.

\begin{table}[h!]
  \centering
  \begin{tabular}{p{7cm}p{5cm}p{1cm}}
      %\hline
      \textbf{Time Scale} & \textbf{Usage} & \textbf{Type} \\
      \hline
      \gls{tai} & the official timekeeping standard & Atomic \\
      %\arrayrulecolor{gray}\hline
      \gls{utc} & the basis of civil time & Atomic/Solar hybrid\\
      %\arrayrulecolor{gray}\hline
      \gls{ut1} & based on Earth rotation & Solar \\
      %\arrayrulecolor{gray}\hline
      \gls{tt}  & used for solar system ephemeris look-up & Dynamical\\
      %\arrayrulecolor{gray}\hline
      \gls{tcg} & used for calculations centered on the Earth in space  & Dynamical\\
      %\arrayrulecolor{gray}\hline
      \gls{tcb} & used for calculations beyond Earth orbit  & Dynamical\\
      %\arrayrulecolor{gray}\hline
      \gls{tdb} & a scaled form of \gls{tcb} that keeps in step with \gls{tt} on the average  & Dynamical\\
      \hline
  \end{tabular}
  \caption{Fundamental time scales used in Geodesy and Astronomy.}
  \label{table:time-scales}
\end{table}

\paragraph{\gls{tai}}\label{par:tai}
The unit of \gls{tai} is the SI second, defined as ``the duration of 9,192,631,770 periods 
of the radiation corresponding to the transition between the two hyperfine levels of the ground state 
of the caesium 133 atom''. \gls{tai} is a laboratory time scale, independent of 
astronomical phenomena apart from having been synchronized to solar time when first 
introduced. It is realized via a weighted average of a number of high-precision atomic 
clocks held around the world. It is close to proper time for an observer on the
geoid, and is an appropriate choice for terrestrial applications (\cite{sofa2021ts}).

\paragraph{\gls{utc}}\label{par:utc}
\gls{utc} is a compromise between the demands of precise timekeeping and the desire to maintain
the current relationship between civil time and daylight. Until 1972, rate changes were 
introduced to keep \gls{utc} roughly in step with \gls{ut1}; since then, adjustments have been 
made by occasionally inserting a whole second, called a \emph{leap second}, a procedure 
that can be thought of as stopping the \gls{utc} clock for a second to let the Earth catch 
up (\cite{sofa2021ts}). Leap seconds are introduced as necessary to keep \gls{ut1}-\gls{utc} in the
range $\pm\SI{0.9}{\second}$. The difference between \gls{ut1} and \gls{utc} is usually 
designated as $\Delta AT$.

\paragraph{\gls{ut1}}\label{par:ut1}
\gls{ut1} is the modern equivalent of mean solar time. In a physical sense, it is an 
angle rather than actual time, and is defined through its relationship with Earth 
rotation angle. Because of the variability of Earth's rotation, \gls{ut1} second is not 
precisely matched to the SI second. The difference between \gls{ut1} and \gls{tt} 
is normally designated as $\Delta T$, and can be written out as
\begin{equation}
    \Delta T = TT - UT1 = \SI{32.184}{\second} + \Delta AT - \Delta UT1
\end{equation}

\paragraph{The Dynamical Time Scales \gls{tcg}, \gls{tcb} \& \gls{tdb}}
\label{par:dynamical-time-scales}
The coordinate time scales \gls{tcg}, \gls{tcb} and \gls{tdb} are the independent
variable in General Relativity based theories which describe the motions of bodies in 
the vicinity of the Earth (\gls{tcg}) and in the solar system (\gls{tcb}, \gls{tdb}).
\gls{tt} and \gls{tdb} are close to each other (less than \SI{2}{\milli\second}) and 
run at the same rate as \gls{tai} (exactly in the case of \gls{tt}). \gls{tcg}
and \gls{tcb}, used in theoretical work, run at different rates and so have long 
term drifts relative to \gls{tai} (\cite{sofa2021ts}). \gls{tcg} and \gls{tcb} are 
the time coordinates of two \gls{iau} spacetime metrics called, respectively, the
geocentric and barycentric celestial reference systems (\gls{gcrs} and \gls{bcrs}).
\gls{tcg}, is appropriate for theoretical studies of geocentric ephemerides.
Its relationship with \gls{tt} is this conventional linear transformation:
\begin{equation}
    TCG = TT + L_G \times (JD_{TT} - TT_{0})
\end{equation}
where $TT_0 = 2443144.5003725$ (i.e. TT at 1977 January 1.0 TAI) and 
$LG = 6.969290134 \times 10-10$. The rate change $LG$ means that \gls{tcg} gains about
\SI{2.2}{\second} per century with respect to \gls{tt} or \gls{tai}; this represents 
the combined effect on the terrestrial clock of the gravitational potential from the 
Earth and the observatory's diurnal speed (\cite{sofa2021ts}).

\paragraph{\gls{tt}}\label{par:tt}
\gls{tt}, is the theoretical time scale for clocks at sea-level (on the geoid): for 
practical purposes it is tied to \gls{tai} through
\begin{equation}
    TT = TAI + \SI{32.184}{\second}
\end{equation}

Note that \gls{utc} has to be expressed as hours, minutes and seconds (or at least in seconds in a
given day) if leap seconds are to be taken into account in the correct manner. In particular, it
is inappropriate to express UTC as a Julian Date, because there will be an ambiguity during a
leap second—so that for example 1994 June 30 23h 59m 60s.0 and 1994 July 1 00h 00m 00s.0 would
both come out as MJD 49534.00000—and because subtracting two such JDs would not yield the
correct interval in cases that contain leap seconds.
