\section{Fundamentals of DORIS System}\label{sec:doris-introduction}

In the late 1980s, \gls{cnes}, in conjunction with \gls{ign} and \gls{grgs} developed 
a new geodetic tracking system called \gls{doris} for precise orbit determination 
of \gls{leo} satellites for oceanographic missions. Since then, the number of 
applications has increased, including but not limited to geiphysics, Earth rotation 
and atmospheric studies, leading in 2003 to the creation of \gls{ids} (\cite{Willis2016}), making it a part 
of the \gls{ggos} within the \gls{iag} (\cite{Willis2006}). DORIS origiated in the
preperation for the US-French TOPEX/Poseidon mission (\cite{Fu1994}), and constitutes a 
Doppler up-lik system, optimaized for orbit determination (both in real-time and 
post-processed). Radio signals are generated from a ground-tracking network, and 
Doppler measurements are performed on-board the satellite. Since its initialization, 
the system's technology has greatly improved, and its applications have gradually 
and logically expanded from orbit determination to gravity-field determination, 
terrestrial reference frame maintenance and geodynamics.

\begin{figure}
  \centering
  \includegraphics[height=.4\textheight,keepaspectratio]{Constellation2021}
  \caption{DORIS-ecquiped satellites; image courtesy of \gls{ids}.}
  \label{fig:doris-constellations}
\end{figure}

\subsection{The DORIS Tracking Netwok}\label{ssec:doris-tracking-network}
The tracking network is a key factor in the success of the \gls{doris} system. 
This network was deployed and is actively maitained by \gls{ign}; it is global, dense 
and homogeneous, and thus unique among the different techniques that contribute to 
\gls{itrf}. Since its establishment, very few changes of sites and/or instrumentation 
have been performed, thus enforcing network stability. Furthermore, it is (spatially) 
dense and well distributed geographically (especially between the Northern and  
Southern hemispheres). This spatial balance along with several collocations with 
other space-geodetic techniques, make the \gls{doris} system an integral part of 
reference frame maintenance.

\begin{figure}
  \centering
  \includegraphics[height=.4\textheight,keepaspectratio]{permanent_network_Nov2020}
  \caption{The \gls{doris} Network (as of Nov. 2020); image courtesy of \gls{ids}.}
  \label{fig:doris-network}
\end{figure}

The strengths and advantages of the \gls{doris} tracking network, can be summarized 
by (\cite{Soudarin2019})
\begin{description}
  \item[Centralized control and management] of the network deployment and evolution, 
    including site instrumentation
  \item[Long operation time]; time-series of current stations span a 21 year period in average, 
    with a median of 26.4 years
  \item[Homogeneous spatial distribution]; half of the stations are located on 
    islands or coastal areas and the network is well balanced between the 
    Northern and Southern hemispheres
  \item[Large number of co-locations]; 48 out of 59 stations are co-located with 
    other techniques (\gls{gnss}: 47, \gls{slr}: 10, \gls{vlbi}: 7), plus 28 of 59 are 
    co-located with tide gauges
\end{description}

\begin{figure}
  \centering
  \includegraphics[height=.4\textheight,keepaspectratio]{colocation_IERS_TG_Nov2020}
  \caption{\gls{doris} stations co-located with other space-geodetic techniques and tide-gauges (as of Nov. 2020); image courtesy of \gls{ids}.}
  \label{fig:doris-constellations}
\end{figure}

\iffalse
\fi
