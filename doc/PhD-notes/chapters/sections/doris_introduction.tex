\section{Introduction}\label{sec:doris-introduction}
Individual components of \gls{pod} analysis have been discussed in the preceding 
sections, including the force model, orbit integration, and parameter estimation. 
This chapter focuses on the one crucial component that is still missing: actual data.

The most widely used satellite observation techniques in geodesy, 
are \gls{slr}, \gls{gnss} and \gls{doris}. In this Thesis the latter technique 
is considered. 
Since its inception in the late 1980s, DORIS has been constantly evolving, and 
it is now of critical importance for geodesy, with applications spanning a wide 
range of related elds, including reference frame maintenance. There is currently 
an effort underway to deploy 4\textsuperscript{th} generation ground beacons, 
securing and strengthening the technique's performance and thus its future.

To date, orbital accuracies achieved using \gls{doris} data, can reach the few-centimeter 
level (see e.g. \cite{Rudenko2023} and \cite{Kong2017}), and since most altimetry satellite missions 
are equipped with onboard \gls{doris} receivers, the technique plays a crucial role in 
the study of sea level changes and hence, indirectly, in the monitoring of the Earth's 
climate.

Currently \gls{doris} is not as popular as other satellite geodetic techniques (e.g. 
\gls{gnss}), for a variety of reasons, including the technique's complexity and 
its limited (if any) commercial usage. This is evident by the number of Analysis 
Centers contributing to the \gls{ids} (according to \cite{Moreaux2022}, four Analysis Centers 
were involved in \gls{ids}'s contribution to ITRF2020). 
The current Thesis work aims to lay the groundwork for a new, state-of-the-art 
software package that will allow \gls{doris} data analysis in accordance with 
\gls{ids} quality standards.

\subsection{Goals of Current Chapter}
In this chapter emphasis is given on the \gls{doris} satellite system. A short 
introduction of the system's origins and technological advancements is given first, and a short 
description of its operation principle follows. Two pillars of the system are 
introduced: 
\begin{itemize}
  \item the \gls{ids}, a service dedicated to facilitating access to \gls{doris} 
data and products to the scientific community, while at the same time deriving products 
of the utmost quality to all interested parties, and 
  \item the \gls{doris} network, i.e. the transmitting ground beacons, scattered around the 
globe in a homogeneous spatial distribution, which along with the instrumentation stability 
is a key factor in the technique's precision
\end{itemize}

Subsequently, the geometry of the ground stations is discussed. In applications  
with high accuracy demands, the signal path between emitter and receiver  
needs to be referred to the appropriate, yet virtual, exact point of transmission 
(and reception respectively). Hence, beacon geometry, reference points and related 
offsets and variations (\gls{pco} and \gls{pcv}) are presented and discussed, as 
well as the reductions involved in data analysis.

The theoretical investigation of the measurements obtained via the \gls{doris} 
technique, namely the relative velocity between the transmitter and the receiver, 
follows (via Doppler counts).
It is crucial to gain a clear view on both the 
observation model and the measurement conduction by the receiver electronics, to 
be able to identify discrepancies, error sources and ambiguities that enter the model, 
measurements and computations. An observation model is developed and extensively 
discussed in order to match the acquired measurements as precisely as possible. 
Basic formulas involved, along with theoretical implications are also presented.

Last but not least, a thorough discussion on the implementation of the observation 
equations, obtained via the \gls{doris} system, follows. Starting from the theoretical 
background (discussed previously), implementation details and practical aspects of the 
computations involved are presented, following a hands-on approach. Given that analysis 
of \gls{doris} observations is not as popular and not as ``standardized'' as other 
techniques (e.g. \gls{gnss}), especially since the new, extended \gls{doris} RINEX 
format was introduced, this section is key in designing a robust processing pipeline.
