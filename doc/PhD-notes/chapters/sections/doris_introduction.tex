\section{Introduction}\label{sec:doris-introduction}
In the sections above, individual componets of \gls{pod} analysis have been 
discussed, including the force model, orbit integration and parameter estimation. 
This chapter is devoted to the one fundamental component still missing: the actual data.

The most widely used statellite observation techniques in satellite geodesy, 
are \gls{slr}, \gls{gnss} and \gls{doris}. In this Thesis is the latter technique 
is considered. \gls{doris} has been constanty evolving since it originated in the 
late 1980s, and is currently of fundamental importance for geodesy, with applications 
spanning a wide range of related fields, including reference frame maintenance. 
Currently, there is an ongoing effort to deploy the 4\textsuperscript{th} generation 
ground beacons, securing and further strengthening the techique's performance and 
thereby future.

To date, orbital accuracies achieved using \gls{doris} data, can reach the few-centimeter 
level (see e.g. \cite{Rudenko2023} and \cite{Kong2017}), and since most altimetry satellite 
are equiped with onboard \gls{doris} receivers, the technique plays a crucial role in 
the study of sea level chages and hence in general the monitoring of the Earth's 
climate.

\subsection{Goals of Current Chapter}
In this chapter emphasis is given on the \gls{doris} satellite system. A short 
introduction of the system's origins and evolvement is given first, and a short 
description of its operation principle follows. Two pillars of the system are 
then introduced: 
\begin{itemize}
  \item the \gls{ids}, a service dedicated to facilitating access of \gls{doris} 
data and products to the scietific community and at the same time deriving products 
of the utmost quality to all interested parties, and 
  \item the \gls{doris} network, i.e. the transmitting ground beacons, scatter around the 
globe in a homogenous spatial distribution, which along with the instrumentation stability 
are a key factor in the technique's precision
\end{itemize}

Subsequently, the geometry of the ground stations is discussed, since it wiil be needed 
to reference the signal path to the appropriate, yet virtual point of transmittion. 
Related offsets and variations (\gls{pco} and \gls{pcv}) are also discussed.

Focus is then turned to the theoretical study of the measurements obtained via the 
\gls{doris} technique, i.e. the relative velocity between the transmitter and the 
receiver (via Doppler counts). A brief derivation is given of the basic formulas 
involved, along with a short discussion on theoretical implications and models.

Last but not least, a thorough discussion on the implementation of the observation 
equations, obtained via the \gls{doris} system, follows. Starting from the theoretical 
background (discussed previously), implementation details and practical aspects of the 
computations involved are presented, following a hands-on approach. Given that analysis 
of \gls{doris} observations is not as popular and not as ``standarized'' as other 
techniques (e.g. \gls{gnss}), especially since the new, extended \gls{doris} RINEX 
format was introduced, this section is key in designing a robust processing chain.

\section{Fundamentals of \gls{doris} System}\label{sec:doris-system-fundamentals}
In the late 1980s, \gls{cnes}, in conjunction with \gls{ign} and \gls{grgs} developed 
a new geodetic tracking system called \gls{doris} for precise orbit determination 
of \gls{leo} satellites for oceanographic missions. Since then, \gls{doris} has 
made huge leaps forward, and proved to be an invaluable tool to the scientific community, 
greatly expanding its application range and significance, 
%contributing to applications in the fields of geophysics, Earth rotation and 
%atmospheric studies, 
leading in 2003 to the creation of \gls{ids} (\cite{Willis2016}), 
part of the \gls{ggos} within the \gls{iag} (\cite{Willis2006}). 

\gls{doris} (\cite{Barlier2005}) originated in the preperation for the US-French 
TOPEX/Poseidon mission (\cite{Fu1994}), 
and constitutes a Doppler up-lik system, optimized for orbit determination (both 
in real-time and post-processed). Radio signals are generated from a ground-tracking 
network, and Doppler measurements are performed on-board the satellite. Since its 
initialization, the system's technology has greatly improved, and its applications 
have gradually and logically expanded from orbit determination to gravity-field 
determination, terrestrial reference frame maintenance and geodynamics. Since TOPEX/Poseidon, an 
ever-increasing number of \gls{leo} satellite missions are ecquiped with on-board 
\gls{doris} receivers. \autoref{fig:doris-constellations} depicts past, current and 
future \gls{doris}-ecquiped satellite missions.

\begin{figure}[h!]
  \centering
  \includegraphics[height=.4\textheight,keepaspectratio]{Constellation2021}
  \caption{DORIS-ecquiped satellites; image courtesy of \gls{ids}.}
  \label{fig:doris-constellations}
\end{figure}

\iffalse
\begin{description}
  \item[Atmospheric Studies] including troposphere () and ionosphere ()
  \item[Reference Frames] \cite{Willis2007}
  \item[Space Weather and Solar Activity] \cite{Willis2005}
  \item[Tectonics] \cite{}
\end{description}
\fi

\subsection{The \gls{doris} Tracking Network}\label{ssec:doris-tracking-network}
The tracking network is a key factor in the success of the \gls{doris} system. 
This network was established and is actively maitained by \gls{ign} and it currently 
(Feb. 2023) constists of $\approx$ 60 sites (\autoref{fig:doris-network}). It is global, dense 
and homogeneous, and thus unique among the different techniques that contribute to 
\gls{itrf}. Since its establishment, very few changes of sites and/or instrumentation 
have been performed, thus enforcing network stability. Furthermore, it is (spatially) 
dense and well distributed geographically (especially between the Northern and  
Southern hemispheres). This spatial balance along with several collocations with 
other space-geodetic techniques (\autoref{fig:doris-network-ties}), make the 
\gls{doris} system an integral part of reference frame maintenance (\cite{Moreaux2022}).

\begin{figure}[h]
  \centering
  \includegraphics[height=.4\textheight,keepaspectratio]{permanent_network_Nov2020}
  \caption{The \gls{doris} Network (as of Nov. 2020); image courtesy of \gls{ids}.}
  \label{fig:doris-network}
\end{figure}

The strengths and advantages of the \gls{doris} tracking network, can be summarized 
by (\cite{Soudarin2019})
\begin{description}
  \item[Centralized control and management] of the network deployment and evolution, 
    including site instrumentation.
  \item[Long operation time]; time-series of current stations span a 21 year period in average, 
    with a median of 26.4 years.
  \item[Homogeneous spatial distribution]; half of the stations are located on 
    islands or coastal areas and the network is well balanced between the 
    Northern and Southern hemispheres (\autoref{fig:doris-network}).
  \item[Large number of co-locations]; 48 stations are co-located with 
    other techniques (\gls{gnss}: 47, \gls{slr}: 10, \gls{vlbi}: 7), plus 28 are 
    co-located with tide gauges (\autoref{fig:doris-network-ties})
\end{description}

\begin{figure}[h]
  \centering
  \includegraphics[height=.4\textheight,keepaspectratio]{colocation_IERS_TG_Nov2020}
  \caption{\gls{doris} stations co-located with other space-geodetic techniques 
    and tide-gauges (as of Nov. 2020); image courtesy of \gls{ids}.}
  \label{fig:doris-network-ties}
\end{figure}

\subsection{The \gls{ids}}\label{ssec:ids}
\gls{ids} was established in 2003, with the mission to (\cite{Soudarin2019})
\begin{itemize}
  \item provide support to research activities in geodesy and geophysics, based 
    on \gls{doris} data and derived products, and
  \item give access to data, products and documents related to the \gls{doris} system
\end{itemize}

The service is based on international cooperation on a volunteer basis and just like 
its geodetic counterparts (i.e. \gls{igs}, \gls{ilrs} and \gls{ivs} for \gls{gnss}, 
\gls{slr} and \gls{vlbi} respectively) plays a crucial role in the development of 
the technique and most importantly drives and facilitates its usage from the scientific 
community. Major products published by \gls{ids} include times series of the \gls{doris} 
tracking stations, along with their positions and velocities  and time series of 
geocenter motion and Earth orientation parameters. \gls{ids} also coordinates the 
technique's contribution to the \gls{itrf} (see e.g. \cite{Moreaux2022}).

\gls{ids}'s organization chart, icludes a Governing Board and a Central Bureau.
The Analysis Centers process the \gls{doris} data available at the \gls{ids} Data 
Centers and generate products with the assistace of the Analysis Center Coordinator.
The \gls{ids} Combination Center provides regular combination of products of 
the Analysis Centers ad is also in charge of the realization of the so-called 
DPOD (\gls{doris} extension to the current \gls{itrf} for Precise Orbit Determination) 
which contains update mean positions and velocities of all the \gls{doris} stations. 
Last but not least, the \gls{ids} Combination Center produces the technique's 
contribution the \gls{itrf}.

\iffalse
\fi
