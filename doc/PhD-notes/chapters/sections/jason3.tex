\section{The Jason-3 Satellite Mission}\label{sec:jason-3}

\gls{jason}-3 is an international cooperative mission between \gls{noaa} 
\gls{cnes}, \gls{eumetsat}, and \gls{nasa}. The objective of the mission is to 
provide a continuation of the TOPEX/Poseidon, \gls{jason}-1 and \gls{jason}-2 
missions and their collection of high accuracy radar altimetry 
measurements for global ocean circulation and sea surface studies, without any 
data gaps (\cite{Couderc2015}). \gls{jason}-3 follows a near-circular and frozen orbit 
that follows an exact repeating ground track every 127 revolutions in a little 
less that ten days (see \autoref{table:jason-3-characteristics}).

\begin{table}[h]
    \centering
    \begin{tabular}{c|c}
        \textbf{Parameter} & \textbf{Value} \\
        \toprule
        Semi-major axis ($\alpha$) & \SI{7714431}{\meter} \\
        Eccentricity ($e$) & \num{9.5e-5} \\
        Inclination ($i$) & \ang{66.038309} \\
        Argument of periapsis & \ang{90} \\
        Reference equatorial altitude & \SI{1336}{\kilo\meter} \\
        Cycle duration & \SI{9.91564}{\day} \\
        Number of revolution in a cycle & 127 orbits \\
        \bottomrule
    \end{tabular}
    \caption{Characteristics of \gls{jason}-3 orbit, \cite{Couderc2015}}
    \label{table:jason-3-characteristics}
\end{table}

\paragraph{Mass and Center of Gravity:}
Initial values of mass and center of gravity coordinates in satellite reference 
frame are given in \autoref{table:jason-3-mass}. Actual values of the satellite 
mass and center of gravity coordinates are obtained by adding the offsets
published by the \gls{ids} at the dedicated \emph{mass \& Center of gravity history} 
file\footnote{Available at \url{ftp://ftp.ids-doris.org/pub/ids/satellites/ja3mass.txt}}.
Direction of the body-fixed reference frame axis are shown in \autoref{fig:jason1} 
for \gls{jason}-1; \gls{jason}-3 follows a very similar design (same PROTEUS platform). 

\begin{table}[h!]
    \centering
    \begin{tabular}{cccc}
        \toprule
        \textbf{Mass} & \textbf{X} & \textbf{Y} & \textbf{Z} \\
        \si{\kilo\gram} & \multicolumn{3}{c}{\si{\meter}} \\
        \hline
        509.6 & 1.0023 & 0.0000 & -0.0021 \\
       \bottomrule
    \end{tabular}
    \caption{Mass and Center of Gravity coordinates in the SV-fixed reference 
        frame for \gls{jason}-3, \cite{Cerri2022}}
    \label{table:jason-3-mass}
\end{table}

\begin{figure}
    \centering
    \includegraphics[height=.3\textheight,keepaspectratio]{jason1}
    \caption{Image of \gls{jason}-1 satellite including the body frame, \cite{Cerri2022}.}
    \label{fig:jason1}
\end{figure}

\paragraph{\gls{doris} Receiver Phase Center:}
The position of the \SI{2}{\giga\hertz} and \SI{400}{\mega\hertz} receiver phase 
centers in the spacecraft-fixed refernce frame, are given in \autoref{table:jason-3-receiver-arp}.
\begin{table}[h!]
    \centering
    \begin{tabular}{cccc}
        \toprule
        \textbf{Frequency} & \textbf{X} & \textbf{Y} & \textbf{Z} \\
         & \multicolumn{3}{c}{\si{\meter}} \\
        \hline
        \SI{2}{\giga\hertz} &   2.4128 & -0.1325 & 0.9235 \\
        \SI{400}{\mega\hertz} & 2.4128 & -0.1325 & 0.7555 \\
       \bottomrule
    \end{tabular}
    \caption{DORIS receiver phase center coordinates in the SV-fixed reference 
        frame for \gls{jason}-3, \cite{Cerri2022}}
    \label{table:jason-3-receiver-arp}
\end{table}

\paragraph{\gls{jason}-3 MacroModel:}
Macromodel parameters to be used for computing 
the solar radiation effect (see \autoref{sssec:solar-radiation-pressure}) are 
published by the \gls{ids} (\cite{Cerri2022}). Details of the macromodel are 
shown in \autoref{table:jason-3-nacromodel}.
\begin{table}[h!]
    \centering
    \begin{tabular}{cccccccccc}
        \toprule
        \textbf{Surface \si{\meter\tothe{2}}} & 
            \multicolumn{3}{c}{Normal in satellite RF} &
            \multicolumn{3}{c}{Optical Properties} &
            \multicolumn{3}{c}{Infrared Properties} \\
         & 
         \textbf{X} & \textbf{Y} & \textbf{Z} &
         \textbf{spec} & \textbf{diff} & \textbf{abs} &
         \textbf{spec} & \textbf{diff} & \textbf{abs} \\
        \hline
        0.783& -1&   0&  0& 0.3410& 0.6460& 0.0130& 0.0000& 0.9870& 0.0130 \\
        0.783&  1 &  0&  0& 0.1490& 0.8510& 0.0000& 0.0000& 1.0000& 0.0000 \\
        2.040&  0 & -1&  0& 0.5730& 0.3840& 0.0430& 0.1040& 0.5690& 0.3280 \\
        2.040&  0 &  1&  0& 0.5390& 0.4240& 0.0370& 0.0890& 0.6270& 0.2830 \\
        3.105&  0 &  0& -1& 0.2460& 0.7520& 0.0020& 0.0050& 0.9770& 0.0170 \\
        3.105&  0 &  0&  1& 0.2130& 0.4530& 0.3340& 0.0370& 0.2870& 0.6760 \\
        \multicolumn{10}{c}{Solar Array} \\
        9.800&  1& 0& 0& 0.0600& 0.4070& 0.5330& 0.0970& 0.0980& 0.8030 \\
        9.800& -1& 0& 0& 0.0040& 0.2980& 0.6970& 0.0350& 0.0350& 0.9310 \\
       \bottomrule
    \end{tabular}
    \caption{Satellite Macromodel for \gls{jason}-3, \cite{Cerri2022}.}
    \label{table:jason-3-nacromodel}
\end{table}

\subsection{\gls{jason}-3 Attitude}\label{ssec:jason3-attitude}
The spacecraft's attitude is needed for various steps in the \gls{pod} processing 
pipeline (including e.g. the computation of atmospheric drag (see 
\autoref{sssec:atmospheric-drag}) and solar radiation pressure (see 
\autoref{sssec:solar-radiation-pressure})). Satellite attitude refers to the 
orientation of the satellite body system in a reference frame, specifically, the 
coordinate transformation matrix between the latter frame and an Earth-fixed 
or inertial system. The \gls{ids} strongly recommends the use of \emph{measure attitude} 
if available (\cite{IdsRecommendationItrf2020}). 

To describe the orientation of a satellite in space, a \emph{body-fixed} reference 
frame $\mathcal{R}_{BF}$ needs to be defined (see \autoref{fig:jason3-view}). 
This frame is permanently tied to the mechanical 
structure of the satellite and enables the specification of antenna and coordinates 
relative to the CoM. The body-fixed frame is likewise required to describe the 
position and alignment of individual surface elements (\cite{Montenbruck2015}).
For the \gls{jason}-3 satellite, the origin of the body reference system is the center of the launcher interface ring. 
The $X-$axis points along the satellite's main direction towards the radiometer 
antenna and the axis $Z-$axis is oriented towards the Earth (nadir pointing). 
$Y-$axis is parallel to the rotation axis of the solar panels, positively 
towards the right solar panel (\cite{Zeitlhofler2019}).

The transformation between the body-fixed reference frame and the \gls{gcrf} is 
given by the transformation:
\begin{equation}
  \bm{r}_{BF} = \mathcal{R} \bm{r}_{GCRS}
\end{equation}
The transformation denoted here by $\mathcal{R}$ can be represented in different 
forms (e.g. \emph{Euler Axis/Angle}), one of which is the 
\emph{quaternion representation}. This way of transforming between the body-fixed 
and the \gls{gcrf} frames is adopted and implemented in the software designed for 
the current Thesis. The quaternion representation expresses the attitude matrix as 
a homogenous quadratic function of the elements of the quaternion, requiring no 
trigonometric or other transcendental function evaluations. Quaternions are more 
efficient for specifying rotations thatn the attitude matrix itself, having only 
four components (instead of nine) and obeying only one constraint, namely that 
their norm is equal to unity (\cite{Markley2019}). Discussions on the the use of 
quaternions for attitude determination and undelrying algebraic concepts can be 
found in \cite{Yang2012} and \cite{Markley2019}.

In the case of \gls{jason}-3, measured attitude is published by \gls{cnes} in 
dedicated data files (\cite{Ferrage2020}) by means of quaternions, covering a 
period of 28 hours (2 overlapping hours between two consecutive files). Data records 
are retrieved from these files to relate the body-fixed reference frame to the 
\gls{gcrf}. Since quaternions are provided in distinct epochs, an interpolation 
lagorithm must be used to estimate the quaternions at the epoch of request. In this 
Thesis, the \emph{spherical linear interpolation} method (often called \emph{Slerp} in 
computer graphics) introduced by \cite{Shoemake1985} is used. According to \cite{Dam2000}, 
Slerp is the optimal interpolation curve between two rotations.

Respective solar panel files are also distributed by \gls{cnes}, containing the 
angular positions of the solar arrays.

\begin{figure}
    \centering
    \begin{subfigure}[b]{0.45\textwidth}
      \centering
      \includegraphics[angle=180,width=\textwidth,keepaspectratio]{jason3-scetch}
      \caption{Body-fixed frame}
      \label{fig:jason3-scetch}
    \end{subfigure}
    \begin{subfigure}[b]{0.45\textwidth}
      \centering
      \includegraphics[width=\textwidth,keepaspectratio]{jason3-photo}
      \caption{Surfaces for $+Z$ and $-Y$}
      \label{fig:jason3-photo}
    \end{subfigure}
    \caption{\gls{jason}-3 images depicting the satellite and the embeded, 
      body-fixed reference frame (source \cite{Couderc2015})}
    \label{fig:jason3-view}
\end{figure}
