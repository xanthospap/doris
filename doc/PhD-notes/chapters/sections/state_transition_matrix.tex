\subsection{State Transition Matrix}\label{ssec:pod-state-transition-matrix}

\autoref{eq:tapley426a} represents a system of linear differential equations with 
time-dependent coefficients (notice that the matrix $A$ in \autoref{eq:tapley426c} 
is derived from a particular solution of $\dot{\bm{y}}=\bm{f}(t,\bm{y})$, generated 
with the initial conditions $\bm{y}(t_0)=\bm{y}^{*}_{0}$). The general solution for 
this system can be expressed as 
\begin{equation}\label{eq:tapley427}
    \delta \bm{y}(t) = \Phi (t,t_k) \delta \bm{y}_k
\end{equation}
with $\delta \bm{y}_k \equiv \delta \bm{y}(t_k)$
Differentiating \autoref{eq:tapley427} and noting that $\delta \bm{y}_k$ is constant, 
gives
\begin{equation}\label{eq:tapley429}
    \delta \dot{\bm{y}}(t) = \dot{\Phi} (t,t_k) \delta \bm{y}_k
\end{equation}
and using \autoref{eq:tapley426a} and \autoref{eq:tapley427}, the \gls{ode} system 
\begin{equation}\label{eq:tapley4210}
    \begin{aligned}
        \dot{\Phi} (t,t_k) &= A(t) \Phi (t,t_k) \\
        \Phi (t_k,t_k) &= \bm{I}
    \end{aligned}
\end{equation}
The great advantage of this formulation, is that it allows the solution $\delta \bm{y}(t)$ 
to be expressed in terms of the unknown initial state $\delta \bm{y}_k$. Hence, the 
state transition matrix enables relating observations made at different times.
    
The \gls{ode} system for the state transition matrix \autoref{eq:tapley429} 
is prefered against a direct solution of $\delta \bm{y}(t)$ from the system
\autoref{eq:tapley426a} for computational reasons, \cite{Tapley2004}.

\cite{Montenbruck2000} give the formulae for forming the state transition matrix 
using orbital elements and the associated partials with respect to the state vector.
\cite{Battin1999} uses the fact that $\Phi$ is a \emph{sympletic} matrix, to obtain 
an analyticaly its inverse.
