\subsection{Orbital Elements}
\label{ssec:orbital-elements}

A total of six independent parameters are needed to to describe the motion of 
an orbiting body. Two parameters, eccentricity $e$ and angular momentum $h$ (or 
alternatively the semi-major axis, $\alpha$) define the form of the orbit. To 
locate a point on the orbit requires a third parameter, the true 
anomaly $\theta$. Describing the orientation of an orbit in three dimensions 
requires three additional parameters, inclination $i$, argument of perigee $\omega$ 
and right ascension of the ascending node, $\Omega$. These parameters are called 
the \emph{orbital elements} (see \ref{fig:orbital-elements-3d} and 
\ref{fig:orbital-elements-Ooi}).

\begin{figure}
  \centering
  %% see https://tex.stackexchange.com/questions/386030/how-to-draw-orbital-elements
\tdplotsetmaincoords{70}{110}
\begin{tikzpicture}[tdplot_main_coords,scale=5]
  \pgfmathsetmacro{\r}{.8}
  \pgfmathsetmacro{\O}{45} % right ascension of ascending node [deg]
  \pgfmathsetmacro{\i}{30} % inclination [deg]
  \pgfmathsetmacro{\f}{35} % true anomaly [deg]

  \coordinate (O) at (0,0,0);
  
  % rotate and plot orbital plane (so that the ecliptic goes above it)
  \tdplotsetrotatedcoords{-\O}{\i}{0};
  \tdplotdrawarc [tdplot_rotated_coords,fill opacity=0.8,fill=red!20]
    {(O)}{\r}{0}{360}{}{};

  % rotate back and plot the ecliptic RF
  \tdplotsetrotatedcoords{\O}{-\i}{0};
  \node at (0,-\r,0) [left,text width=4em] {Ecliptic Plane}; % name of plane
  \tdplotdrawarc [dashed,fill=gray!10,fill opacity=0.9]
    {(O)}{\r}{0}{360}{}{}; % draw plane and fill it
  \draw [] (O) -- (\r,0,0); % x-axis, solid part
  \draw [-stealth,dashed] (\r,0,0) -- (2*\r + \r /2,0,0) 
    node[anchor=north east] {$\bm{x} (\aries)$}; % x-axis, dashed part
  \draw [-stealth] (O) -- (0,\r,0) node[anchor=north west] {$\bm{y}$};
  \draw [-stealth] (O) -- (0,0,\r) node[anchor=south] {$\bm{z}$};
  
  % rotate and plot the line of nodes
  \tdplotsetrotatedcoords{\O}{0}{0};
  \draw [tdplot_rotated_coords] (-1,0,0) -- (1,0,0) 
    node [below right] {Line of Nodes};
  \tdplotdrawarc[thick,-stealth]{(O)}{.33*\r}
    {0}{\O}{anchor=north}{$\bm{\Omega}$} % Omega angle

  % rotate back and plot the orbital RF
  \tdplotsetrotatedcoords{-\O}{\i}{0};
  % re-plot the part of the orbital plane above the ecliptic
  \tdplotdrawarc [tdplot_rotated_coords,fill opacity=0.8,fill=red!20]
    {(O)}{\r}{90}{270}{}{};
  \begin{scope}[tdplot_rotated_coords]
    \draw[red,-stealth] (O) -- (0,0,\r) node [above] {$\hat{\bm{h}}$};
    \tdplotdrawarc[thick,-stealth,red]{(O)}{.33*\r}{90}{180}
      {anchor=west}{$\bm{\omega}$};
    \coordinate (Sat) at (180+\f:\r);
    \draw [-stealth] (O) -- (Sat);
    \filldraw [black] (Sat) circle (0.5pt) node[anchor=south west]{Sat};
    \tdplotdrawarc[thick,-stealth,red]{(O)}{.33*\r}{180}{180+\f}
      {anchor=south west}{$\bm{\theta}$};
    \draw [thick] (O) -- (-\r,0,0) node[anchor=south west]{Periapsis/Perigee} ;
  \end{scope}
 
  % inclination ....
  \pgfmathsetmacro\ANx{\r * cos(\O)}   
  \pgfmathsetmacro\ANy{\r * sin(\O)}   
  \coordinate (Shift) at (\ANx,\ANy,0);
  \tdplotsetrotatedcoordsorigin{(Shift)};
  \tdplotsetrotatedcoords{180}{90}{180+\O)};
  %\begin{scope}[tdplot_rotated_coords]
  %\draw [tdplot_rotated_coords,blue] (0,0,0) -- (\r,0,0) node{xx};
  %\draw [tdplot_rotated_coords,blue] (0,0,0) -- (0,\r,0) node{yy};
  %\draw [tdplot_rotated_coords,blue] (0,0,0) -- (0,0,\r) node{zz};
  %\draw [tdplot_rotated_coords,black] (\r,0,0) circle (0.5pt) node[anchor=south west]{(r,0,0)};
  \tdplotdrawarc[tdplot_rotated_coords,thick,stealth-,black]
    {(Shift)}{0.2*\r}{0}{\i}{anchor=west}{$\bm{i}$};
  %\end{scope}

\end{tikzpicture}

  \caption{Geometry of Orbital Elements}
  \label{fig:orbital-elements-3d}
\end{figure}

$\Omega$, $\omega$ and $i$, which define the orientation of the orbit in space 
are sometimes called \emph{Euler angles}. Given these six elements, it is always 
possible to uniquely calculate the state vector. 

\begin{description}
  \item $\bm{\alpha}$: the \emph{semi-major axis} (sometimes $\bm{h}$, the specific 
  \emph{angular momentum} is used instead),
  \item $\bm{i}$: \emph{inclination},
  \item $\bm{\Omega}$: \emph{right ascension of the ascending node},
  \item $\bm{e}$: \emph{eccentricity},
  \item $\bm{\omega}$: \emph{argument of perigee},
  \item $\bm{\theta}$\footnote{True anomaly is often deisgnated with the letter 
  $\bm{\nu}$ of $\bm{f}$, but here we are following the notation $\bm{\theta}$}.: 
  \emph{true anomaly} (sometimes $\bm{M}$, the \emph{mean anomaly} 
  is used instead)
\end{description}

\begin{figure}
  \centering
  %% https://tex.stackexchange.com/questions/386030/how-to-draw-orbital-elements
\def\r{3.5}
\pgfmathsetmacro{\inclination}{35}
\pgfmathsetmacro{\nuSatellite}{55}
\pgfmathsetmacro{\gammaAngle}{290}
\pgfmathsetmacro{\omegaSatellite}{90}

\tdplotsetmaincoords{70}{165}
\begin{tikzpicture}[tdplot_main_coords]

    % Earth
    \fill (0,0) coordinate (O) circle (3pt) node[left=7pt] {$M_\oplus$};
            
    % Draw equatorial plane
    \draw[] (0,-\r,0) -- 
        (\r,-\r,0) node[below]{Equatorial plane} -- 
        (\r,\r,0) -- 
        (-\r,\r,0) -- 
        (-\r,-0.65*\r,0) ;
    \draw[dotted] (-\r,-0.65*\r,0) -- 
        (-\r,-\r,0) -- 
        (0,-\r,0);
            
    % Draw Line of nodes
    \draw[dashed] (0,-1.3*\r,0) -- (0,1.4*\r,0) node[right] {Line of nodes};

    % Draw Equinox line
    \draw[black,->] (0,0,0) -- (1.5,2.0*\r,0) node[right] {Vernal Equinox (\aries)};

    % Draw Omega angle
    \tdplotdrawarc[thick,-stealth,black]
        {(0,0,0)}{0.6*\r}{77}{90}
        {anchor=north}{$\boldsymbol\Omega$};

    % vertical vector to orbital plane 
    %\tdplotsetcoord{Pg}{1.3*\r}{90}{\gammaAngle}
    %\draw[->] (0,0,0) -- (Pg) node[anchor=east] {Reference direction $\boldsymbol{\gamma}$};

    % Create a new rotated system in the center
	\tdplotsetrotatedcoords{0}{\inclination}{90}
				
	% Draw orbital ellipse
	\tdplotdrawarc[tdplot_rotated_coords,thin,blue]{(0,0,0)}{\r}{-125}{180}{label={[xshift=-5.7cm, yshift=-2.2cm]Orbital plane}}{}
	\tdplotdrawarc[tdplot_rotated_coords,dotted,blue]{(0,0,0)}{\r}{180}{235}{}{}

    % Draw periapsis line
	\draw[dashed,tdplot_rotated_coords,blue] (0,0,0) -- (0,\r,0) node[anchor=south west] {Periapsis};

    % Draw omega angle
    \tdplotdrawarc[tdplot_rotated_coords,thick,-stealth,blue]
        {(0,0,0)}{0.4*\r}{0}{\omegaSatellite}
        {anchor=south west}{$\boldsymbol\omega$};

    % Draw inclination angle
    \tdplotresetrotatedcoordsorigin
	\tdplotsetrotatedcoords{0}{0}{180}
    \coordinate (Shift) at (0,\r,0);
    \tdplotsetrotatedcoordsorigin{(Shift)}
    \tdplotsetrotatedthetaplanecoords{0};
    \tdplotdrawarc[tdplot_rotated_coords,thick,-stealth,brown]
        {(Shift)}{0.3*\r}{90}{90-\inclination}{anchor=west}{$\bm{i}$}
\end{tikzpicture}
  \caption{Orbital Elements $\Omega$, $\omega$ and $i$}
  \label{fig:orbital-elements-Ooi}
\end{figure}

To avoid confusion, we are going to adopt the distinction of orbital elements 
sets proposed by \cite{Vallado} and clarify between the following cases:
\begin{description}
  \item \textbf{two-body elements} are the elements derived from or used with 
    the two-body equations of motion,
  \item \textbf{osculating elements} are the \emph{instaneous} elements, under 
    the influence of perturbations, and
  \item \textbf{mean elements} are the elements obtained when averaging the 
    effect of perturbations over a specified time interval
\end{description}

Hence, it is clear that the osculating elements \emph{describe the perturbed 
motion and are defined for a prticular instant in time} (\cite{Vallado}).
