\subsection{Orbital Elements}\label{ssec:orbital-elements}

A total of six independent parameters are needed to unambiguously define an 
arbitrary and unperturbed orbit at some instant $t$ in time. Two parameters, 
eccentricity $e$ and angular momentum $h$ (or alternatively the semi-major axis, 
$\alpha$) define the form of the orbit. To locate a point on the orbit we need a 
third parameter, the true anomaly $\theta$. Describing the orientation of the 
orbit in three dimensions requires three additional parameters, inclination $i$, 
argument of perigee $\omega$ and right ascension of the ascending node, $\Omega$. 
These six parameters are called the \emph{orbital} or 
\emph{Keplerian elements}\footnote{There exist alternate sets of (six) parameters 
that can uniquely define the orbit, but this set is by far the most widely used 
in celestial mechanics.} (see \autoref{fig:orbital-elements-3d} and 
\autoref{fig:orbital-elements-Ooi}).
\begin{description}[labelindent=1cm]
  \item[$\alpha$]: the \emph{semi-major axis} (sometimes $h$, the specific 
    \emph{angular momentum} is used instead),
  \item[$i$]: \emph{inclination},
  \item[$\Omega$]: \emph{right ascension of the ascending node},
  \item[$e$]: \emph{eccentricity},
  \item[$\omega$]: \emph{argument of perigee},
  \item[$\theta$]\footnote{True anomaly is often designated with the letter 
    $\nu$ of $f$, but here we are following the notation $\theta$}.: 
    \emph{true anomaly} (sometimes $\bm{M}$, the \emph{mean anomaly} 
    is used instead)
\end{description}

\begin{figure}
  \centering
  %% see https://tex.stackexchange.com/questions/386030/how-to-draw-orbital-elements
\tdplotsetmaincoords{70}{110}
\begin{tikzpicture}[tdplot_main_coords,scale=5]
  \pgfmathsetmacro{\r}{.8}
  \pgfmathsetmacro{\O}{45} % right ascension of ascending node [deg]
  \pgfmathsetmacro{\i}{30} % inclination [deg]
  \pgfmathsetmacro{\f}{35} % true anomaly [deg]

  \coordinate (O) at (0,0,0);
  
  % rotate and plot orbital plane (so that the ecliptic goes above it)
  \tdplotsetrotatedcoords{-\O}{\i}{0};
  \tdplotdrawarc [tdplot_rotated_coords,fill opacity=0.8,fill=red!20]
    {(O)}{\r}{0}{360}{}{};

  % rotate back and plot the ecliptic RF
  \tdplotsetrotatedcoords{\O}{-\i}{0};
  \node at (0,-\r,0) [left,text width=4em] {Ecliptic Plane}; % name of plane
  \tdplotdrawarc [dashed,fill=gray!10,fill opacity=0.9]
    {(O)}{\r}{0}{360}{}{}; % draw plane and fill it
  \draw [] (O) -- (\r,0,0); % x-axis, solid part
  \draw [-stealth,dashed] (\r,0,0) -- (2*\r + \r /2,0,0) 
    node[anchor=north east] {$\bm{x} (\aries)$}; % x-axis, dashed part
  \draw [-stealth] (O) -- (0,\r,0) node[anchor=north west] {$\bm{y}$};
  \draw [-stealth] (O) -- (0,0,\r) node[anchor=south] {$\bm{z}$};
  
  % rotate and plot the line of nodes
  \tdplotsetrotatedcoords{\O}{0}{0};
  \draw [tdplot_rotated_coords] (-1,0,0) -- (1,0,0) 
    node [below right] {Line of Nodes};
  \tdplotdrawarc[thick,-stealth]{(O)}{.33*\r}
    {0}{\O}{anchor=north}{$\bm{\Omega}$} % Omega angle

  % rotate back and plot the orbital RF
  \tdplotsetrotatedcoords{-\O}{\i}{0};
  % re-plot the part of the orbital plane above the ecliptic
  \tdplotdrawarc [tdplot_rotated_coords,fill opacity=0.8,fill=red!20]
    {(O)}{\r}{90}{270}{}{};
  \begin{scope}[tdplot_rotated_coords]
    \draw[red,-stealth] (O) -- (0,0,\r) node [above] {$\hat{\bm{h}}$};
    \tdplotdrawarc[thick,-stealth,red]{(O)}{.33*\r}{90}{180}
      {anchor=west}{$\bm{\omega}$};
    \coordinate (Sat) at (180+\f:\r);
    \draw [-stealth] (O) -- (Sat);
    \filldraw [black] (Sat) circle (0.5pt) node[anchor=south west]{Sat};
    \tdplotdrawarc[thick,-stealth,red]{(O)}{.33*\r}{180}{180+\f}
      {anchor=south west}{$\bm{\theta}$};
    \draw [thick] (O) -- (-\r,0,0) node[anchor=south west]{Periapsis/Perigee} ;
  \end{scope}
 
  % inclination ....
  \pgfmathsetmacro\ANx{\r * cos(\O)}   
  \pgfmathsetmacro\ANy{\r * sin(\O)}   
  \coordinate (Shift) at (\ANx,\ANy,0);
  \tdplotsetrotatedcoordsorigin{(Shift)};
  \tdplotsetrotatedcoords{180}{90}{180+\O)};
  %\begin{scope}[tdplot_rotated_coords]
  %\draw [tdplot_rotated_coords,blue] (0,0,0) -- (\r,0,0) node{xx};
  %\draw [tdplot_rotated_coords,blue] (0,0,0) -- (0,\r,0) node{yy};
  %\draw [tdplot_rotated_coords,blue] (0,0,0) -- (0,0,\r) node{zz};
  %\draw [tdplot_rotated_coords,black] (\r,0,0) circle (0.5pt) node[anchor=south west]{(r,0,0)};
  \tdplotdrawarc[tdplot_rotated_coords,thick,stealth-,black]
    {(Shift)}{0.2*\r}{0}{\i}{anchor=west}{$\bm{i}$};
  %\end{scope}

\end{tikzpicture}

  \caption{Geometry of orbital elements.}
  \label{fig:orbital-elements-3d}
\end{figure}

$\Omega$, $\omega$ and $i$, which define the orientation of the orbit in space, 
are sometimes called \emph{Euler angles}. Given these six elements, it is always 
possible to uniquely calculate the \emph{state vector}, that is the three spatial 
dimensions defining the position $\begin{pmatrix}x&y&z\end{pmatrix}$ in a Cartesian 
coordinate system, and their corresponding velocities 
$\begin{pmatrix}\dot{x}&\dot{y}&\dot{z}\end{pmatrix}$.

\begin{figure}
  \centering
  %% https://tex.stackexchange.com/questions/386030/how-to-draw-orbital-elements
\def\r{3.5}
\pgfmathsetmacro{\inclination}{35}
\pgfmathsetmacro{\nuSatellite}{55}
\pgfmathsetmacro{\OmegaAngle}{-290}
\pgfmathsetmacro{\omegaSatellite}{90}

\tdplotsetmaincoords{70}{165}
\begin{tikzpicture}[tdplot_main_coords]

    % Earth
    \fill (0,0) coordinate (O) circle (3pt) node[left=7pt] {$M_\oplus$};
            
    % Draw equatorial plane
    \draw[] (0,-\r,0) -- 
        (\r,-\r,0) node[below]{Equatorial plane} -- 
        (\r,\r,0) -- 
        (-\r,\r,0) -- 
        (-\r,-0.65*\r,0) ;
    \draw[dotted] (-\r,-0.65*\r,0) -- 
        (-\r,-\r,0) -- 
        (0,-\r,0);
            
    % Draw Line of nodes
    \draw[dashed] 
      (0,-1.3*\r,0) -- (0,1.4*\r,0) node[right] {Line of nodes};

    % Draw Equinox line
    \tdplotsetcoord{V}{2.0*\r}{90}{\OmegaAngle};
    \draw[->] (0,0,0) -- (V) node[anchor=west] {Vernal Equinox (\aries)};

    % Draw Omega angle
    \tdplotresetrotatedcoordsorigin
    \tdplotsetrotatedcoords{0}{0}{180}
    \tdplotdrawarc[tdplot_rotated_coords,thick,-stealth,black]
      {(0,0,0)}{0.6*\r}{250}{270}
      {anchor=north}{$\boldsymbol\Omega$};

    % Draw orbital ellipse
    \tdplotsetrotatedcoords{0}{\inclination}{90}
    \tdplotdrawarc[tdplot_rotated_coords,thin,blue]
      {(0,0,0)}{\r}{-125}{180}
      {label={[xshift=-5.7cm, yshift=-2.2cm]Orbital plane}}{}
    \tdplotdrawarc[tdplot_rotated_coords,dotted,blue]
    {(0,0,0)}{\r}{180}{235}
    {}{}

    % Draw satellite on orbital plane
    \tdplotsetrotatedcoords{0}{\inclination}{90};
    \pgfmathsetmacro{\omegaSatellite}{90}
    \pgfmathsetmacro{\xmRot}{\r*cos(\omegaSatellite+\nuSatellite)}
    \pgfmathsetmacro{\ymRot}{\r*sin(\omegaSatellite+\nuSatellite)}
    \pgfmathsetmacro{\zmRot}{0}
    \draw[tdplot_rotated_coords,thin,->,blue] 
      (0,0,0) -- (\xmRot,\ymRot,\zmRot);
    \filldraw[tdplot_rotated_coords, blue] 
      (\xmRot,\ymRot,\zmRot) circle (2pt) node[above left] {$sat$};

    % Draw periapsis line
    \draw[dashed,tdplot_rotated_coords,blue] 
      (0,0,0) -- (0,\r,0) node[anchor=south west] {Periapsis};

    % Draw omega angle
    \tdplotdrawarc[tdplot_rotated_coords,thick,-stealth,blue]
        {(0,0,0)}{0.4*\r}{0}{\omegaSatellite}
        {anchor=south west}{$\boldsymbol\omega$};

    % Draw v angle (true anomaly)
    \tdplotdrawarc[tdplot_rotated_coords,thick,-stealth,blue]
      {(0,0,0)}{0.4*\r}{\omegaSatellite}{\omegaSatellite+\nuSatellite}
      {anchor=south west}{v};

    % Draw inclination angle
    \tdplotresetrotatedcoordsorigin
    \tdplotsetrotatedcoords{0}{0}{180}
    \coordinate (Shift) at (0,\r,0);
    \tdplotsetrotatedcoordsorigin{(Shift)}
    \tdplotsetrotatedthetaplanecoords{0};
    \tdplotdrawarc[tdplot_rotated_coords,thick,-stealth,brown]
        {(Shift)}{0.3*\r}{90}{90-\inclination}{anchor=west}{$\bm{i}$}

\end{tikzpicture}

  \caption{Orbital Elements $\Omega$, $\omega$ and $i$}
  \label{fig:orbital-elements-Ooi}
\end{figure}

A real orbit and its elements change over time due to various perturbations (see 
\autoref{ssec:perturbed-motion}). A Kepler orbit is an idealized, mathematical 
approximation of the orbit at a particular time. To avoid confusion, we are going 
to adopt the distinction of orbital elements sets proposed by \cite{Vallado2001} and 
distinguish between the following cases:
\begin{description}
  \item \emph{two-body elements} are the elements derived from or used with 
    the two-body equations of motion,
  \item \emph{osculating elements} are the \emph{instantaneous} elements, under 
    the influence of perturbations, and
  \item \emph{mean elements} are the elements obtained when averaging the 
    effect of perturbations over a specified time interval
\end{description}

Mathematical formulae to transform between state vector and orbital elements can 
be found in relevant literature. In the software developed for this Thesis, the 
methodology described in \cite{Montenbruck2000} is adopted and implemented.
