\section{Introduction}\label{sec:podwdoris-introduction}

With the introduction of the \gls{doris} system and the observation model 
discussed in \autoref{ch:doris}, and using the developments presented earlier 
in the Thesis, a full orbit determination process can now be put together. The 
theoretical background and implementation details of the fundamental building 
blocks of the analysis (discussed above), include:
\begin{description}
  \item[Astrodynamics] and the effective modelling of the \emph{perturbed} motion 
    of an artificial earth orbiting satellite (see \autoref{ssec:orbital-mechanics})
  \item[Spatial and Temporal Reference Systems] where the equations of motion 
    can be described in, as well as the transformation mechanisms between their 
    realizations (see \autoref{ssec:the-celestial-reference-frame} and 
    \autoref{ssec:time-systems-and-scales})
  \item[Earth's Attitude], i.e. modelling the Earth's variable rotation and dynamics 
    (see \autoref{ssec:earth-attitude})
  \item[Orbit integration] for the efficient and precise extrapolation of the 
    satellite's trajectory (see \autoref{ch:orbit-integration}),
  \item[Orbit estimation] or \emph{orbit improvement}, where data are processed in 
    a robust fashion to produce estimates for the satellite's state vector or 
    equivalently orbital elements (see \autoref{ch:pod})
  \item[Data analysis] where measurements obtained using the \gls{doris} system are 
    processed using the observation model described in \autoref{ch:doris}, to obtain 
    precise relative velocity values between the emitter and the receiver
\end{description}

State-of-the-art \gls{pod} software packages utilizing \gls{doris} observations, 
can reach accuracies in the centimeter level (\cite{Rudenko2023}). It should be 
noted that such software are only seldom limited to one satellite technique; most 
often either \gls{slr} and/or \gls{gnss} datasets are included in the analysis, 
either for validation purposes or for integrated processing.

\subsection{Goals of The Current Chapter}\label{ssec:jason3-pod-goals}

In this chapter a software tool to perform \gls{doris}-data analysis is presented, 
using the models, methods and algorithms already discussed in the previous sections.
The program is designed to tackle a ``pure orbit determination'' problem, or more 
specifically orbit improvement; that is, given an initial, reference state at some 
initial epoch, process \gls{doris} observations (in an iterative manner) to compute 
the ``best'' estimate of the satellite state.

The problem is labelled ``pure'' in the sense that it does not consider improvements 
or estimates for other parameters, as e.g. site coordinates. However, a number of 
parameters must be included in the filtering process (e.g. drag and radiation 
coefficients, $C_d$ and $C_r$ respectively, relative frequency offsets for the 
beacons, $\Delta f_e / f_{e_N}$, and wet zenith tropospheric delay $zd_{wet}$). The 
theoretical implications and algorithmic design of the \gls{doris} observation equation 
has already been discussed in \autoref{sec:doris-observation-equation-implementation} 
and will be used here to analyze the \gls{doris} measurements.

To test the validity of the estimates, the results are checked against \gls{cnes}/SSALTO 
computed trajectories, given in sp3 files. The latter have been computed using a 
multi-technique approach, using both \gls{doris} and \gls{gps} measurements. Differences 
in the force model and various processing options also exist between the two analysis 
procedures, that introduce further inconsistencies. However, these sp3 files were 
the only available in the \gls{cddis} archive.
