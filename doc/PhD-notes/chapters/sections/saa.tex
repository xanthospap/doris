\section{The South Atlantic Anomaly (SAA)}\label{sec:saa}

\iffalse
https://www.aviso.altimetry.fr/en/news/image-of-the-month/2007/oct-2007-south-atlantic-anomaly-as-seen-by-doris.html
https://ids-doris.org/images/documents/report/ids_workshop_2010/IDS10_s1_Stepanek_Spot5SAA.pdf
\fi

The \emph{Van Allen} radiation belts 
are toroidal zones around the Earth where high energy particles coming mostly 
from the solar wind are trapped. Because the inner belt is not symmetrically 
centered on the Earth, it comes closer to the Earth surface in a region 
located above South America (\cite{Jalabert2018}).
The \gls{saa} is a region of reduced magnetic intensity, where 
the inner Van Allen radiation belt makes its closest approach 
to the Earth's surface. This region is centered in southeast South America (see 
\autoref{fig:saa-aviso}). Because 
of the weakened magnetic 
field, inner radiation belt particles can mirror at lower altitudes increasing 
the local particle flux. It is thus the region where the inner radiation belt 
makes its closest approach to the Earth's surface (\cite{Anderson2018}). 
Satellites in low-Earth orbit pass though the \gls{saa} periodically, thus being 
exposed to several minutes of strong radiation each time, creating problems for 
scientific instruments on-board satellites.

\begin{figure}
  \centering
  \includegraphics[height=.6\textwidth,keepaspectratio]{saa_aviso}
  \caption{\gls{jason}-1 exposure to \gls{saa} effects, measured on 200-2004 
    period using the \gls{doris} ultra-stable oscillator. (Credits \gls{cnes}/\gls{cls}, 
    source Aviso \url{https://www.aviso.altimetry.fr/en/news/image-of-the-month/2007/oct-2007-south-atlantic-anomaly-as-seen-by-doris.html})}
  \label{fig:saa-aviso}
\end{figure}

\gls{doris} measurement processing assumes a stable oscillator frequency over 
the duration of each pass. It also assumes a smooth (i.e. low degree polynomial) 
evolution of the instrument frequency for both the receiver and the transmitter. 
The oscillator in the on-board \gls{doris} instrument is an \gls{uso} which is 
designed to meet these stability requirements (\cite{Jalabert2018}).
The high radiation level dominating \gls{saa} impacts the behavior of on-board 
oscillators: a rapid change in the frequency can be observed when the satellites 
fly across this area. This effect causes precision degradation in \gls{pod} and 
positioning results.

\gls{saa} effects on satellites carying \gls{doris} payload have been identified 
for \gls{jason}-1 \cite{Willis2003}, \gls{jason}-2 \cite{Willis2016b}, SPOT-5 \cite{Stepanek2013} 
and Sentinel-3A \cite{Jalabert2018}. \cite{Capdeville2016a}, show that the \gls{jason}-3 
satellite is also affected by \gls{saa}, evident both in \gls{pod} and station positioning 
results. Although corrective models exist for some cases (e.g. for \gls{jason}-1 
\cite{Lemoine2006} and \cite{Capdeville2016b} for SPOT-5), no such model exists 
in the case of \gls{jason}-3. Other possibilities for mitigating the impact include the 
improved model of the \gls{uso} frequency to correct the data and the attenuation 
of the side-effects of the changes in the \gls{uso} by downweighting or eliminating 
measurements from beacons placed within the region covered \gls{saa}. In this Thesis, 
the latter method is used.
