\chapter{Tides}
\label{ch:tides}

\section{Solid Earth Tides}
\label{sec:solid-earth-tides}

\subsubsection{Solid Earth Tide Effect on Geopotential}
\label{sssec:solid-earth-tide-geopotential}
The changes induced by the solid Earth tides in the free space potential are 
most conveniently modeled as variations in the standard geopotential 
coefficients $C_{nm}$ and $S_{nm}$, labled $\Delta C_{nm}$ and $\Delta S_{nm}$ 
respectively, expressible in terms of the Love number $\lovek$. The effects of 
ellipticity and of the Coriolis force due to Earth rotation on tidal 
deformations necessitate the use of three $\lovek$ parameters, $\lovek ^{(0)}_{nm}$ 
and $\lovek ^{(\pm)}_{nm}$ (except for $n = 2$) to characterize the
changes produced in the free space potential by tides of spherical harmonic 
degree and order $(nm)$, whereas only two parameters are needed for $n = 2$ 
because $\lovek ^{(-)}_{2m} = 0$ due to mass conservation (\cite{iers2010}).

Solid Earth tides within the diurnal tidal band (for which $(n,m) = (2,1)$) are 
not wholly due to the direct action of the \gls{tgp} on the solid Earth; they 
include the deformations (and associated geopotential changes) arising from 
other effects of the \gls{tgp}, namely, ocean tides and wobbles of the mantle 
and the core regions (\cite{iers2010}).

The computation of the tidal contributions to the geopotential coefficients is 
most efficiently done by a three-step procedure:
 
\subsubsection{Step 1 Corrections}\label{sssec:step1-corr-earth-tides}
In Step 1, the $(2m)$ part of the tidal potential is evaluated in the 
time domain for each $m$ using lunar and solar ephemerides, and the 
corresponding changes $\Delta \bar{C}_{2m}$ and $\Delta \bar{S}_{2m}$ are 
computed using frequency independent nominal values $\lovek _{2m}$ for the 
respective $\lovek ^{(0)}_{2m}$. The contributions of the degree 3 tides to
$\bar{C}_{3m}$ and $\bar{S}_{3m}$ through $\lovek ^{(0)}_{3m}$ and also of 
those of the degree 2 tides to $\bar{C}_{4m}$ and $\bar{S}_{4m}$ though 
$\lovek ^{(+)}_{2m}$ may be computed by a similar procedure.

With frequency-independent values $\lovek _{nm}$, changes induced by the 
$(nm)$ part of the \gls{tgp} in the \emph{normalized} geopotentials 
coefficients of the same degree and order $(nm)$, are given in the time 
domain by (\cite{iers2010}):
\begin{equation}
\Delta \bar{C}_{nm} - \iim \Delta \bar{S}_{nm} = \frac{\lovek _{nm}}{2n+1}
  \sum^{3}_{j=2} \frac{GM_j}{GM_\Earth} \left( \frac{R_e}{r_j} \right) ^{n+1} 
  \bar{P}_{nm} \left( \sin \Phi _j \right) e^{-\iim m \lambda _j}
  \label{eq:iers1066}
\end{equation}
where:
\begin{description}
  \item $\lovek _{nm}$\footnote{Tables of relevant Love numbers are listed 
    in \cite{iers2010}, Table 6.3.} is the nominal Love number for degree 
    $n$ and order $m$, 
  \item $R_e$ and $GM_{\Earth}$ are the equatorial radius and the 
    gravitational parameter of the Earth,
  \item $GM_j$ is the gravitational parameter of the Moon and Sun, for 
    $j=2$ and $j=3$ respectively,
  \item $\Phi _j$ is the body-fixed geocentric latitude of the Moon and 
    Sun ($j$ indexes as above), and
  \item $\lambda _j$ is the body-fixed (east) logitude of the Moon and 
    Sun ($j$ indexes as above)
\end{description}

For $n=4$, formula \ref{eq:iers1066} becomes (\cite{iers2010}):
\begin{equation}
\Delta \bar{C}_{4m} - \iim \Delta \bar{S}_{4m} = \frac{\lovek _{nm}}{5}
  \sum^{3}_{j=2} \frac{GM_j}{GM_\Earth} \left( \frac{R_e}{r_j} \right) ^{3} 
  \bar{P}_{2m} \left( \sin \Phi _j \right) e^{-\iim m \lambda _j} \text{ for } m=0,1,2
  \label{eq:iers1067}
\end{equation}
to account for the changes in the degree 4 coefficients produced by the 
degree 2 tides.

In Step 1, we compute corrections for 
\begin{equation}
  \Delta \bar{C}_{nm}, \Delta \bar{S}_{nm} \text{ for }
    \begin{cases}
      n=2 & m=0,1,2 \\
      n=3 & m=0,1,2,3 \\
      n=4 & m=0,1,2\\
    \end{cases}
\end{equation} 

\subsubsection{Step 2 Corrections}\label{sssec:step2-corr-earth-tides}
in Step 2 we compute corrections for the deviations of the 
$\lovek ^{(0)}_{21}$ from the constant nominal value $\lovek _{21}$ 
assumed (for this band) in the first step. Similar corrections need to be 
applied to a few of the constituents of the other two bands also.

The contribution to $\Delta \bar{C}_{20}$ from the long period tidal 
constituents, each with a frequency $f$, can be computed by (\cite{iers2010}):
\begin{equation}
  \operatorname{Re} \sum _{f(2,0)}(A_0 \delta k_f H_f) e^{\iim \theta _f} = 
    \sum_{f(2,0)} \left[ \left(A_0 H_h \delta k^\Re _f \right) \cos \theta _f 
      - \left(A_0 H_h \delta k^\Im _f \right) \sin \theta _f 
    \right]
  \label{eq:iers1068a}
\end{equation}

We can further compute the contribution for $(nm)=(21)$ from the 
diurnal tidal constituents and to $(22)$ from the semidiurnal, using 
(\cite{iers2010}):
\begin{equation}
  \Delta \bar{C}_{2m} - \iim \Delta \bar{C}_{2m} = 
    \eta _m \cdot \sum _{f(2,m)} \left( A_m \delta k_f H_f\right) e^{\iim \theta _f} 
    \text{ for } m=1,2
  \label{eq:iers1068b}
\end{equation}
where 
\begin{description}
  \item $\delta k_f = \delta k^\Re _f + \iim \delta k^\Im _f$ is the difference 
  between $k_f$ defined as $k^{(0)}_{2m}$ at frequency $f$ and the 
  nominal value ($k_f - k_{2m}$), plus a contribution from ocean 
  loading\footnote{\label{fn:set-coefs}Values of the imaginary and real part, $\delta k^\Re _f$ and 
  $\delta k^\Im _f$ respectively, can be found in \cite{iers2010}, Tables 6.5a 
  through 6.5c. Note however, that in the computation we use the amplitude values 
  for the in-phase and out-of-phase components ($A_{in-phase} = \left(A_m H_f \delta k^\Re _f \right)$ 
  and $A_{out-of-phase} = \left( A_m H_f \delta k^\Im _f \right)$) directly, recorded 
  in the same tables.}
  \item $H_f$ is the amplitude (in meters) of the term at frequency $f$
  \item $\theta _f$ is given by 
  \begin{equation} \theta _f = m \cdot ( \theta _g + \pi ) - \sum ^5_{j=1} N_j F_j \end{equation}
  \footnote{Here we use the expression based on the expansion using the Fundamental 
  Arguments. For alternate formulations, e.g using the \emph{Doodson} fundamental 
  arguments, see \cite{iers2010}, Sec. 6.2.1.}
  where $\theta _g$ is the \gls{gmst} expressed in angle units

  \item The terms $\eta _m$ and $A_m$, are given by: 
  \begin{equation}
  \eta _m = 
    \begin{cases} -\iim , m=1 \\ 1 , m=2\end{cases}
  \end{equation} and
  \begin{equation} 
    A_m = \begin{cases} 
        \frac{1}{R_{\Earth}\sqrt{4 \pi}}, m=0\\
        \frac{(-1)^m}{R_{\Earth}\sqrt{8 \pi}}, m \ne 0
    \end{cases}
  \end{equation}\footnote{As with the $\delta k^\Re _f$ and $\delta k^\Im _f$ terms 
  (see \ref{fn:set-coefs}), explicit computation of $A_m$ is not needed if the 
  amplitude terms $A_{in-phase}$ and $A_{out-of-phase}$ are used from 
  \cite{iers2010} Tables 6.5a through 6.5c.}
\end{description}


Steps 1 and 2 can be used to compute the total tidal contribution, including 
the time independent (permanent) contribution to the geopotential coefficient 
$\bar{C}_{20}$, which is adequate for a ``conventional tide free'' model. 
When using a ``zero tide'' model, this permanent part should not be counted 
twice.

\subsubsection{Solid Earth Tide Effect on Reference Points}
\label{sssec:solid-earth-tide-reference-points}

Site displacements caused by tides of spherical harmonic degree and order $(nm)$ 
are characterized by the Love number $loveh _{nm}$ and the Shida number $\lovel _{nm}$. 
The effective values of these numbers depend on station latitude and tidal 
frequency (\cite{WahrEtAl1981}).

Computation of the variations of station coordinates due to solid Earth tides, 
like that of geopotential variations, is done most efficiently by the use of a 
two-step procedure (\cite{iers2010}). The evaluations in the first step use 
the expression in the time domain for the full degree 2 tidal potential or for 
the parts that pertain to particular bands $(m = 0, 1, \text{ or } 2)$. Nominal 
values common to all the tidal constituents involved in the potential and to 
all stations are used for the \emph{Love} and \emph{Shida} numbers $\loveh _{2m}$ 
and $\lovel _{2m}$ in this step. Along with expressions for the dominant 
contributions from $\loveh ^{(0)}$ and $\lovel ^{(0)}$ to the tidal displacements,
relatively small contributions from some of the other parameters are included 
in Step 1 for reasons of computational efficiency. The displacements caused by 
the degree 3 tides are also computed in the first step, using constant values 
for $\loveh _3$ and $\lovel _3$ (\cite{iers2010}).

Corrections to the results of the first step are needed to take account of the 
frequency-dependent deviations of the \emph{Love} and \emph{Shida} numbers from 
their respective nominal values, and also to compute the out-of-phase 
contributions from the zonal tides. Computations of these corrections constitute 
Step 2. The total displacement due to the tidal potential is the sum of the 
displacements computed in Steps 1 and 2 (\cite{IERS2010}).

A short overview of the relevant computation formulas follows. More information 
can be found in \cite{iers2010} and references therein.

\textbf{\ul{In-phase displacement due to degree 2 tides, with nominal values for 
$\loveh ^{(0)}_{2m}$ and $\lovel ^{(0)}_{2m}$}}.\\
The displacement vector of the station due to the degree 2 tides is given by 
(\cite{iers2010}):
\begin{equation}
  \Delta \bm{r} = \sum ^{3}_{2} \frac{GM_j R^4_{\Earth}}{GM_{\Earth}R^3_j}
    \biggl\{ \loveh _2 \hat{r} \left( \frac{3 (\hat{R}_j \cdot \hat{r})^2 -1}{2} \right) 
      + 3 \lovel _2 (\hat{R}_j \cdot \hat{r}) \left[ \hat{R}_j - (\hat{R}_j \cdot \hat{r}) \hat{r} \right] 
    \biggl\}
    \label{eq:iers105}
\end{equation}
where \begin{description}
  \item $GM_j$ is the gravitational parameter for the Moon ($j=2$) and Sun ($j=3$),
  \item $GM_{\Earth}$ is the gravitational parameter for the Earth,
  \item $\hat{R}_j, R_j$ is unit vector from the geocenter to Moon or Sun and 
    the magnitude of that vector,
  \item $R_{\Earth}$ is the Earth’s equatorial radius,
  \item $\hat{r}, r$ is the unit vector from the geocenter to the station and 
    the magnitude of that vector,
  \item $\loveh _2$ and $\lovel _2$ are the \emph{Lova} and \emph{Shida} numbers 
    of degree 2
\end{description}

\textbf{\ul{In-phase displacement due to degree 3 tides, with nominal values for
$\loveh ^{(0)}_{2m}$ and $\lovel ^{(0)}_{2m}$}}.\\

Only the Moon’s contribution needs to be computed, the term due to the Sun being
negligible. The transverse part of the displacement does not exceed \SI{0.2}{\mm}, 
but the radial displacement can reach \SI{1.7}{\mm} (\cite{iers2010}).

The displacement vector due to these tides is then given by (\cite{iers2010}):
\begin{equation}
  \begin{aligned}
  \Delta \bm{r} = \frac{GM_{Moon} R^5_{\Earth}}{GM_{\Earth}R^4_{Moon}} &
    \biggl\{ \loveh _3 \hat{r} \left( \frac{5}{2} (\hat{R}_{Moon} \cdot \hat{r})^3) -\frac{3}{2} (\hat{R}_{Moon} \cdot \hat{r}) \right) \\
      & + \lovel _2 \left( \frac{15}{2}(\hat{R}_{Moon} \cdot \hat{r})^2 - \frac{3}{2}\right) \left[ \hat{R}_{Moon} - (\hat{R}_{Moon} \cdot \hat{r}) \hat{r} \right] 
    \biggl\}
    \end{aligned}
    \label{eq:iers106}
\end{equation}

/* Add formulas here from Section 7.1.1 Effects of the solid Earth tides */

\gls{iers} publishes a FORTRAN program to perform the Step 1 and Step 2 computations 
called \texttt{DEHANTTIDEINEL.F}\footnote{Available from the \gls{iers} \href{https://iers-conventions.obspm.fr/}{Conventions Centre} at \url{https://iers-conventions.obspm.fr/content/chapter7/software/dehanttideinel}}.

\section{Secular polar motion and the pole tide}
\label{sec:pole-tides}

Changes in the direction of the Earth's rotation axis with respect to locations 
on the Earth's surface cause local deformations (\cite{Desai2002}) that result in 
variations in station coordinates up to a few centimeters.

To perform the corrections of the induced effect, we have to compute the so-called 
``secular pole'' (\cite{iers2010}, Sec. 7.1.4). Its coordinates, designated by 
$(x_s, y_s)$ are computed according to (\cite{iers2010}):

\begin{equation}
  \begin{aligned}
    x_s &= 55.0 + 1.677 \cdot (t-2000) \text{ in } \si{\milli\larcsecond} \\
    y_s &=  320.5 + 3.460 \cdot (t-2000) \text{ in } \si{\milli\larcsecond}
  \end{aligned}
\end{equation}
where $t$ is the date in years of 365.25 days.
