\chapter{Time, Time Scales and Calendars}
\label{ch:time-scales}

\section{Julian Date and Modified Julian Date}
\label{sec:julian-date}
Calendar dates are usually quaite cumbersome and inefficient to use when 
performing date computations; instead, a continuous count of days is prefered. 
To this end, the \emph{Julian day number} was introduced with JD zero located 
about 7000 years ago (for example Julian day number 2449444 began at noon on 
1994 April 1). Note that JD (and MJD) can be used in conjunction to most 
time scales used in astronomy, such as TAI, TT and TDB.

\emph{Julian Date} (JD) is the same system but with a fractional part appended; JD 2449443.5 was the
midnight on which 1994 April 1 commenced (\cite{sofa_18161_tscb}). Removing the leading 
`24' and dropping the fractional `.5' part, yileds the so called \emph{Modified Julian Date}, 
aka:
\begin{equation}
    MJD = JD - 2400000.5
\end{equation}
Thus 1994 April 1 commenced at MJD 49443.0. Within \ref{ssec:dso-datetime}, dates are 
internally stored as MJD (integers).

\subsection{Julian Epoch}
It is often convinient to work with fractional years; in the past, this was done 
using a system called \emph{Besselian epoch} (\cite{sofa_18161_tscb}) but since the 
mid-80's, the \emph{Julian epoch} took over. It uses the Julian year of exactly 365.25 days,
and the TT time scale. Julian epoch 2000.0 is defined to be 2000 January 1.5, which
is JD 2451545.0 or MJD 51544.5 (\cite{sofa_18161_tscb}). Julian epochs are denoted with 
a `J' prefix, hence e.g. `J2000.0' is Julian epoch 2000.0.

\section{Time Scales}
The most common time scales used in astronomical computations, are:
\begin{table}
  \centering
\begin{tabularx}{\textwidth}{>{\raggedright\arraybackslash}X >{\raggedright\arraybackslash}X c}
  %\hline
    \bf{Time-Scale} & \bf{Description} & \bf{Type} \\
  \hline
  \textbf{TAI}\\ \scriptsize{(International Atomic Time)} & The official timekeeping standard & Atomic \\
  \textbf{UTC}\\ \scriptsize{(Coordinated Universal Time)} & The basis of civil time & Atomic/Solar hybrid \\
  \textbf{UT1}\\ \scriptsize{(Universal Time)} & Based on Earth rotation & Solar \\
  \textbf{TT}\\ \scriptsize{(Terrestrial Time)} & Used for solar system ephemeris look-up &  Dynamic \\
  \textbf{TCG}\\ \scriptsize{(Geocentric Coordinate Time)} & Used for calculations centered on the Earth in space & Dynamic \\
  \textbf{TCB}\\ \scriptsize{(Barycentric Coordinate Time)} & Used for calculations beyond Earth orbit; for most common cases, may be approximated by \textbf{TT} & Dynamic \\
  \textbf{TDB}\\ \scriptsize{(Barycentric Dynamical Time)} & A scaled form of TCB that keeps in step with TT
on the average & Dynamic \\
  %\hline
\end{tabularx}
\caption{Common Time Scales used in Astronomical and Celestial Computations.}
\end{table}

Time scales that are obsolete, according to \cite{sofa_18161_tscb}, are:
\begin{description}
  \item[UT0, UT2]: specialist forms of universal time that take into account polar motion and
known seasonal effects; no longer used.
  \item[GMT] (Greenwich mean time): an obsolete time scale that can be taken to mean either
UTC or UT1.
  \item[ET] (ephemeris time): superseded by TT and TDB.
  \item[TDT] (terrestrial dynamical time): the former name of TT.
\end{description}

Note that \emph{Sidereal time} ins not really a time scale but rather an angle. 
The same can be said of UT1; however, the interrelation between UTC and UT1 makes 
it clearer and more convenient to treat the latter as a time scale.

Each of the time scales Each has a distinct role, and there are offsets of tens 
of seconds between some of them. The transformation from one time scale to the next can take a number of forms. In some cases,
for example TAI to TT, it is simply a fixed offset. In others, for example TAI to UT1, it is
an offset that depends on observations and cannot be predicted in advance (or only partially).
Some time scales, for example TT and TCG, are linearly related, with a rate change as well as an
offset. Others, for example TCG and TCB, require a 4-dimensional spacetime transformation (\cite{sofa_18161_tscb}).

\begin{figure}
\centering
\begin{tikzpicture}
\coordinate (TAI) at (-4,5.0);
\coordinate (UTC) at (-4,2.5);
\coordinate (UT1) at (-4,0);
\coordinate (TT) at (0,0);
\coordinate (TCG) at (2,-2.5);
\coordinate (TCB) at (2,-5.0);
\coordinate (TDB) at (2,-7.5);

\draw [thick] (UT1) rectangle node{\textbf{UT1}}  ($(UT1)+(1.4,0.8)$);
\draw [thick] (UTC) rectangle node{\textbf{UTC}}  ($(UTC)+(1.4,0.8)$);
\draw [thick] (TAI) rectangle node{\textbf{TAI}}  ($(TAI)+(1.4,0.8)$);
\draw [thick] (TT)  rectangle node{\textbf{TT}}   ($(TT)+(1.4,0.8)$);
\draw [thick] (TCG) rectangle node{\textbf{TCG}}  ($(TCG)+(1.4,0.8)$);
\draw [thick] (TCB) rectangle node{\textbf{TCB}}  ($(TCB)+(1.4,0.8)$);
\draw [thick] (TDB) rectangle node{\textbf{TDB}}  ($(TDB)+(1.4,0.8)$);

\draw [blue,thick,->] ($(TAI) + (0.7,0.0)$) -- node[anchor=west]{\(-\Delta AT\) (leap seconds)} ($(UTC) + (0.7,+0.8)$);
\draw [red,thick,->] ($(UTC) + (0.7,0.0)$) -- node[anchor=west]{\(+\Delta UT1\) } ($(UT1) + (0.7,+0.8)$);

\end{tikzpicture}
\caption{Geometry of Alcatel DORIS Ground Antenna/Beacon}
\label{fig:alcatel-antenna}
\end{figure}

\subsection{Leap Seconds and $\Delta$AT}
\label{ssec:leap-seconds-dat}
Leap seconds are introduced when necessary to keep the time difference UT1-UTC 
to within \(\SI{\pm 0.9}{\second}\). They are injected at the end of December or 
June. Each time a leap second is introduced, the offset \(\Delta AT = TAI - UTC\) 
changes by exactly \SI{1}{\second} (leap seconds are in practice always positive, hence 
the offset is augmented, but provision for negative seconds exists if needed). In 
practice, are indicators of the accumulated difference between atomic time and time 
measured by Earth rotation. At the time of writing, the average solar day 
is now \SIrange{1}{2}{\milli\second} longer than the nominal \SI{86400}{\second}, accumulating 
to \SI{1}{\second} over a period of 18 months to a few years. As the Earth rotation 
slows, leap seconds will become ever more frequent.

The procedure of leap second injection, can be thought of as stopping the UTC clock 
for a second to let the Earth catch up.

\subsubsection{Leap Seconds Implications}
If a UTC date is not expressed using the hours, minutes, seconds format, but instead 
is expressed as a Julian Date (or MJD), an ambiguity will 
arise at the time of leap second injection. E.g., \cite{sofa_18161_tscb}, the dates 
June 30 1994, 235960.0 and July 1 1994, 000000.0 would both result to the same 
MJD, namely 49534.000; obviously, subtracting the two, identical JDs, would not 
yield the correct interval.

\subsection{Solar Time: UT1, UTC and $\Delta$UT1}
UT1 is the modern equivalent to \emph{mean solar time} and is really an angle rather 
than time (\cite{sofa_18161_tscb}). Historicaly its definition has involved the 
`ficticious mean Sun' and sidereal time, but now is defined through its relationship 
with Earth rotation angle. Because the Earth's rotation rate is slightly irregular
and is gradually decreasing, the UT1 second is not precisely matched to the SI second.
This makes UT1 itself unsuitable for use as a time scale in physics applications.
Nevertheless, it is still in use in a wide variety of applications.

Coordinated Universal Time (UTC) is the standard atomic based time scale in normal 
everyday use throughout the world. UTC is a compromise between the demands of 
precise timekeeping and the desire to maintain the current relationship between 
civil time and daylight (\cite{sofa_18161_tscb}).

To obtain UT1 starting from UTC, it is necessary to look up the value of 
\( \Delta UT1 = UT1 - UTC \) for the date concerned in tables published by the 
\gls{iers}; this is then added to the UTC. The quantity \(\Delta UT1\), which 
typically changes by \SIrange{1}{2}{\milli\second} per day, can be obtained only 
by observation, principally \gls{vlbi} using extragalactic radio sources, though seasonal effects 
are present and the \gls{iers} listings are able to predict some way into the future 
with adequate accuracy for most applications.

\( \Delta UT1 = UT1 - UTC \) is used to determine the \gls{era}, which is the 
angle measured along the intermediate equator of the \gls{cip} between the \gls{tio} and the 
\gls{cio} positively in the retrograde direction (\cite{IersBulABC04}; for details on 
\gls{era} computation, see \cite{iers2010}).

Values for \( \Delta UT1 \) are published by the \gls{iers} within the regularly disseminated 
\href{https://www.iers.org/IERS/EN/Publications/Bulletins/bulletins.html}{Bulletins}; 
\href{https://datacenter.iers.org/productMetadata.php?id=6}{Bulletin A} contains 
rapid determinations for earth orientation parameters, 
\href{https://datacenter.iers.org/productMetadata.php?id=207}{Bulletin B} 
contains monthly earth orientation parameters and 
\href{https://datacenter.iers.org/productMetadata.php?id=17}{Bulletin D} contains 
announcements of the value of \( \Delta UT1 \).
