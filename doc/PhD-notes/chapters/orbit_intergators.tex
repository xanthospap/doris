\chapter{Orbit Integrators}
\label{ch:orbit-integrators}

\section{Runge Kutta Nystr{\"o}m Methods}
Families of \gls{rkn} formulae are used when the problem at hand involves 
a 2\textsuperscript{nd} order \gls{ode}, where :
\begin{equation}
  \label{eq:dorm1}
  y''(t) = f(t, y(t)) \text{ with } y(t_0) = y_0 \text{ , } y'(t_0) = y'_0
\end{equation}
that is \( f \) is independent of \( y'(t) \). While it is possible to reduce 
\ref{eq:dorm1} to a system of first order \glspl{ode} and apply e.g. a Runge-Kutta 
method, we can exploit its special structure directly applying \gls{rkn} technique,
which consists of formulae of orders \(q\) and \(p\) (with \(q>p\)) of the form 
(\cite{dormand87}):
\begin{equation}
  \begin{aligned}
    \hat{y}_{n+1}  & = \hat{y}_n + h_n \hat{\phi}(x_n , \hat{y}_n , \hat{y'}_n , h_n) & 
    \hat{y'}_{n+1} & = \hat{y'}_n + h_n \hat{\phi}' (x_n , \hat{y}_n , \hat{y'}_n , h_n) \\
    y_{n+1}        & = \hat{y}_n + h_n \phi (x_n , \hat{y}_n , \hat{y'}_n , h_n) & 
    y'_{n+1}       & = \hat{y'}_n + h_n {\phi}' (x_n , \hat{y}_n , \hat{y'}_n , h_n) \\
  \end{aligned}
\end{equation}

where
\begin{equation}
  \begin{aligned}
    \begin{aligned}
    \hat{\phi}(x_n , \hat{y}_n , \hat{y'}_n , h_n) & = \hat{y}'_n + h_n \sum_{i=0}^{s-1} \hat{b}_i g_i & 
    \hat{\phi}'(x_n , \hat{y}_n , \hat{y'}_n , h_n) & = \sum_{i=0}^{s-1} \hat{b}'_i g_i \\
         \phi (x_n , \hat{y}_n , \hat{y'}_n , h_n) & = \hat{y}'_n + h_n \sum_{i=0}^{s-1} b_i g_i & 
         {\phi}' (x_n , \hat{y}_n , \hat{y'}_n , h_n) & = \sum_{i=0}^{s-1} b'_i g_i \\
    \end{aligned}
    \\
    g_i = f(x_n + c_i h_n , \hat{y}_n +c_i h_n \hat{y}'_n + h^2 \sum_{j=0}^{i-1} {\alpha}_{ij}g_j) \text{ ,  } i=0,2,\cdots ,s-1\\
  \end{aligned}
\end{equation}

In the above, \( x_{n+1} = x_n + h_n\) where \( h_n = \theta (x_n) h\) with 
\( 0< \theta \leq 1 \), \( \hat{y}_0 = y(x_0) \) and \( \hat{y'}_0 = y'(x_0)\) 
and \( \hat{y}_n \), \(\hat{y'}_n\), \(y_n\) and \(y'_n\) denote approximations to the 
`true' solution \(y(x_n)\) and \(y'(x_n)\). The q\textsuperscript{th} order approximations 
\( \hat{y}_n \) and \( \hat{y'}_n\) are used as initial values for the \( (n+1)^{th} \) 
step.

Estimates \({\delta}_{n+1} = y_{n+1} - \hat{y}_{n+1}\) and \({\delta '}_{n+1} = y'_{n+1} - \hat{y'}_{n+1}\) 
of the local truncation errors, \(t_{n+1}\) and \(t'_{n+1}\) can be computed and used 
to control the step size \(h_n\), via \cite{dormand87}:
\begin{equation}
  h_{n+1} = 0.9 h_n (T/E)^{\frac{1}{p+1}} \text{ where } 
  E = max\{\lVert {\delta}_{n+1} \rVert_{\infty} , \lVert {\delta '}_{n+1} \rVert_{\infty} \}
\end{equation}
\(T\) is the maximum allowable local error.

\subsection{RKN4(3) Family}
For a process with \(q=4\) and \(p=3\), \cite{dormand87} provides coefficients for \(c_i\), \({\alpha}_{ij}\), 
\(\hat{b}_i\), \(\hat{b'}_i\), \(b_i\) and \(b'_i\) which are listed in \ref{table:dormand87-rkn434fm} 
(for a detailed description of their derivation see therin); nomenclature follows the 
original work of \cite{dormand86}, in which the second \(4\) indicates the number of function 
evaluations, the \textbf{F} indicates that \emph{FSAL}\footnote{FSAL refers to the idea that 
the last evaluation at any step is the same as the first of the next evaluation, used when 
deriving the RKN coefficients; see e.g. \cite{dormand78}.} and the \textbf{M} indicates that the fourth-order truncation-error terms have
been reduced.

\renewcommand{\arraystretch}{2}
\begin{table}[h!]
    \centering
    \begin{tabular}{cccccccc}
        \hline
        \(c_i\) & \multicolumn{3}{c}{\({\alpha}_{ij}\)} &  \(\hat{b}_i\) & \(\hat{b'}_i\) & \(b_i\) & \(b'_i\) \\
        \hline
        \(0\) & & & & \(\frac{1}{14}\) & \(\frac{1}{14}\) & \(-\frac{7}{150}\) & \(\frac{13}{21}\) \\
        \(\frac{1}{4}\) & \(\frac{1}{32}\) & & & \(\frac{8}{27}\) & \(\frac{32}{81}\) & \(\frac{67}{150}\) & \(-\frac{20}{27}\) \\
        \(\frac{7}{10}\) & \(\frac{7}{1000}\) & \(\frac{119}{500}\) & & \(\frac{25}{189}\) & \(\frac{250}{567}\) & \(\frac{3}{20}\) & \(\frac{275}{189}\) \\
        \(1\) & \(\frac{1}{14}\) & \(\frac{8}{27}\) & \(\frac{25}{189}\) & \(0\) & \(\frac{25}{189}\) & \(-\frac{1}{20}\) & \(-\frac{1}{3}\) \\
        \hline
    \end{tabular}
    \caption{The RKN4(3)4FM, \cite{dormand87}.}
    \label{table:dormand87-rkn434fm}
\end{table}
\renewcommand{\arraystretch}{1}

\section{Gauss Jackson}
The Runge-Kutta methods (discussed above), may be characterized as \emph{single-step} 
methods. No use is made of function values calculated in earlier steps, which
means that all integration steps are completely independent of one another. Enhanced with  
step-size controll, single-step methods are well suited for differential 
equations with rapid changes in the function to be integrated \cite{Montenbruck2000}.

For differential equations defined by complicated functions, with a lot of arithmetic operations, 
\emph{multi-step} methods are usually more efficient \cite{Montenbruck2000}; these 
methods store and use values from previous steps.

\section{Introduction}
Determination and prediction of orbits requires an orbit propagator that finds the
phase space state of a satellite at one time based on its state at another time.
An \gls{ode} of order \( n \) is an equation of the form:
\begin{equation}
	\frac{d^n y}{dt^n} = f(t,y,y',y'',\ldots,y^{n-1})
\end{equation}
Initial value problems involving higher-order \glspl{ode} require initial conditions 
for each derivative through \( y^{n-1} \). 
The solution to an initial value problem, \( y(t) \) most often must be found numerically;
algorithms that numerically solve initial value problems are known as \emph{numerical integrators}.

Higher-order \glspl{ode} may be transformed to an equivelent set of 1\textsuperscript{st} 
order equations and be solved with a standard numerical integrator; however, some 
numerical integrators are designed to directly solve higher-order \glspl{ode}.

\emph{Multi-step integrators} (also called \emph{predictor-corrector} methods) 
integrate forward from a particular time to the next mesh
point using function values at the current point as well as several previous mesh
points. The set of the previous points used as well as the current point is called the
set of \emph{backpoints} (\cite{berry2004}). The methods develop a Taylor series 
in the time separation between mesh points, and this series must be truncated after 
some number of terms, which thereby gives the \emph{order} of the method. The order is one 
less than the number of backpoints, and is not related to the order of the differential equation.
There a re various multi-step integrators, depending on the implementation.

The function \( f(t,y) \) must be continuous and smooth through the set of 
backpoints. If there are any discontinuities in \( f \), for example going through eclipse when 
solar radiation pressure is considered, the integration must either be restarted, or 
modified to handle the discontinuity (\cite{berry2004}).

\section{Difference Tables}
Predictor-corrector integrators can be defined in terms of difference tables. 
For a fixed-step method, assume that solution values \( y_n \) are known on a 
discrete set of equally-spaced mesh points 
\( \ldots, t_0 - 2h , t_0 - h , t_0 , t_0 + h , t_0 + 2h , \ldots \)
where \( h \) is the \emph{step}. If we define \(  t_n \equiv t_0 + n h \), then the 
set of points is \( t_{-2}, t_{-1}, t_0 , t_{1}, t_{2} \) with the corrsponding 
function values \( f_n = f(t_n , y_n ) \), where \( y_n \) is the numerical solution 
at the mesh point \( t_n \).

There are three kinds of differences \footnote{Note that a more general backward difference formula, 
reads:
\begin{equation}
  \begin{aligned}
    & \text{\textbf{forward differene} } {\Delta}_h [f](x) = f(x+h) - f(x)\\  
    & \text{\textbf{backward difference} } {\nabla}_h [f](x) = f(x) - f(x-h) = {\Delta}_h [f](x-h)\\
    & \text{\textbf{central difference} } {\delta}_h [f](x) = f(x+\frac{h}{2}) - f(x-\frac{h}{2}) = {\Delta}_{h/2} [f](x) + {\nabla}_{h/2} [f](x)\\
  \end{aligned}
\end{equation}
When the step size \( h \) is ommited, it is taken to be \(1\). }, 
represented by different operators (following \cite{berry2004}):
\begin{equation}
  \begin{aligned}
    \text{forward differene     } &   \Delta f_i    = & f_{i+1} - f_i \\
    \text{backward difference    } & \nabla f_i  = & f_i - f_{i-1} \\
    \text{central difference    } &   \delta f_i   = & f_{i+1/2} - f_{i-1/2} \\
  \end{aligned}
\end{equation}

The differences of the
differences, or \emph{second differences}, can also be computed. For instance,
the square of the backward difference operator is the operator applied to the first
difference \footnote{
  On a more genral manner, the n\textsuperscript{th} order forward, backward and central differences read :
\begin{equation}
  \begin{aligned}
  & {\Delta}^n_h [f](x) = \sum_{k=0}^n \binom{n}{k} (-1)^{n-k} f(x+k)\\
  & {\nabla}^n_h [f](x) = \sum_{i=0}^n (-1)^i \binom{n}{i} f(x-ih)\\
  & {\delta}^n_h [f](x) = \sum_{i=0}^n (-1)^i \binom{n}{i} f(x + (\frac{n}{2} -i)h)\\
  \end{aligned}
\end{equation}
respectively.}
\begin{equation}
  {\nabla}^2 f_i = \nabla \nabla f_i = \nabla (f_i - f_{i-1}) = 
  (f_i - f_{i-1}) - (f_{i-1} - f_{i-2}) = f_i - 2 f_{i-1} + f_{i-2}
\end{equation}

In addition to the three difference operators, there is also a 
\emph{displacement operator} \(E\). The displacement operator is defined as:
\begin{equation}
  E f_i = f_{i+1} \text{ and } E^n f_i = f_{i+n}
\end{equation}

The difference and displacement operators  can be written in terms of each other:
\begin{equation}
\label{eq:berry6}
  \nabla = 1 - E^{-1} \text{ and } E = \frac{1}{1 - \nabla}
\end{equation}

\begin{figure}
\centering
\begin{tikzpicture}
\matrix[
  matrix of math nodes,
  inner sep=3pt,
  row sep=2em,
  column sep=2em
] (M)
{
    \bm{i} & \bm{{\nabla}^{-2} f_i} & \bm{{\nabla}^{-1} f_i} & \bm{f_i} & \bm{\nabla f_i} & \bm{{\nabla}^2 f_i} & \bm{{\nabla}^3 f_i} \\
    \cdots \\
    -3 & {\nabla}^{-2} f_{-3} & {\nabla}^{-1} f_{-3} & f_{-3} &  \nabla f_{-3}  &  {\nabla}^2 f_{-3}  & {\nabla}^3 f_{-3} \\
    -2 & {\nabla}^{-2} f_{-2} & {\nabla}^{-1} f_{-2} & f_{-2} &  \nabla f_{-2}  &  {\nabla}^2 f_{-2}  & {\nabla}^3 f_{-2} \\
    -1 & {\nabla}^{-2} f_{-1} & {\nabla}^{-1} f_{-1} & f_{-1} &  \nabla f_{-1}  &  {\nabla}^2 f_{-1}  & {\nabla}^3 f_{-1} \\
     0 & {\nabla}^{-2} f_{0}  & {\nabla}^{-1} f_{0}  & f_{0}  &  \nabla f_{0}   &  {\nabla}^2 f_{0}   & {\nabla}^3 f_{0}  \\
     1 & {\nabla}^{-2} f_{1}  & {\nabla}^{-1} f_{1}  & f_{1}  &  \nabla f_{1}   &  {\nabla}^2 f_{1}   & {\nabla}^3 f_{1}  \\
     2 & {\nabla}^{-2} f_{2}  & {\nabla}^{-1} f_{2}  & f_{2}  &  \nabla f_{2}   &  {\nabla}^2 f_{2}   & {\nabla}^3 f_{2}  \\
     3 & {\nabla}^{-2} f_{3}  & {\nabla}^{-1} f_{3}  & f_{3}  &  \nabla f_{3}   &  {\nabla}^2 f_{3}   & {\nabla}^3 f_{3}  \\
    \cdots \\
}
;
\draw[->] (M-3-4.south east) -- (M-4-5.north west);
\draw[->] (M-4-4.south east) -- (M-5-5.north west);
\draw[->] (M-5-4.south east) -- (M-6-5.north west);
\draw[->] (M-6-4.south east) -- (M-7-5.north west);
\draw[->] (M-7-4.south east) -- (M-8-5.north west);
\draw[->] (M-8-4.south east) -- (M-9-5.north west);
\draw[->] (M-4-4.east) -- (M-4-5.west);
\draw[->] (M-5-4.east) -- (M-5-5.west);
\draw[->] (M-6-4.east) -- (M-6-5.west);
\draw[->] (M-7-4.east) -- (M-7-5.west);
\draw[->] (M-8-4.east) -- (M-8-5.west);
\draw[->] (M-9-4.east) -- (M-9-5.west);
\draw[->] (M-3-5.south east) -- (M-4-6.north west);
\draw[->] (M-4-5.south east) -- (M-5-6.north west);
\draw[->] (M-5-5.south east) -- (M-6-6.north west);
\draw[->] (M-6-5.south east) -- (M-7-6.north west);
\draw[->] (M-7-5.south east) -- (M-8-6.north west);
\draw[->] (M-8-5.south east) -- (M-9-6.north west);
\draw[->] (M-4-5.east) -- (M-4-6.west);
\draw[->] (M-5-5.east) -- (M-5-6.west);
\draw[->] (M-6-5.east) -- (M-6-6.west);
\draw[->] (M-7-5.east) -- (M-7-6.west);
\draw[->] (M-8-5.east) -- (M-8-6.west);
\draw[->] (M-9-5.east) -- (M-9-6.west);
\draw[->] (M-3-6.south east) -- (M-4-7.north west);
\draw[->] (M-4-6.south east) -- (M-5-7.north west);
\draw[->] (M-5-6.south east) -- (M-6-7.north west);
\draw[->] (M-6-6.south east) -- (M-7-7.north west);
\draw[->] (M-7-6.south east) -- (M-8-7.north west);
\draw[->] (M-8-6.south east) -- (M-9-7.north west);
\draw[->] (M-4-6.east) -- (M-4-7.west);
\draw[->] (M-5-6.east) -- (M-5-7.west);
\draw[->] (M-6-6.east) -- (M-6-7.west);
\draw[->] (M-7-6.east) -- (M-7-7.west);
\draw[->] (M-8-6.east) -- (M-8-7.west);
\draw[->] (M-9-6.east) -- (M-9-7.west);
\draw[gray,->] (M-3-3.south east) -- (M-4-4.north west);
\draw[gray,->] (M-4-3.south east) -- (M-5-4.north west);
\draw[gray,->] (M-5-3.south east) -- (M-6-4.north west);
\draw[gray,->] (M-6-3.south east) -- (M-7-4.north west);
\draw[gray,->] (M-7-3.south east) -- (M-8-4.north west);
\draw[gray,->] (M-8-3.south east) -- (M-9-4.north west);
\draw[gray,->] (M-4-3.east) -- (M-4-4.west);
\draw[gray,->] (M-5-3.east) -- (M-5-4.west);
\draw[gray,->] (M-6-3.east) -- (M-6-4.west);
\draw[gray,->] (M-7-3.east) -- (M-7-4.west);
\draw[gray,->] (M-8-3.east) -- (M-8-4.west);
\draw[gray,->] (M-9-3.east) -- (M-9-4.west);
\draw[gray,->] (M-3-2.south east) -- (M-4-3.north west);
\draw[gray,->] (M-4-2.south east) -- (M-5-3.north west);
\draw[gray,->] (M-5-2.south east) -- (M-6-3.north west);
\draw[gray,->] (M-6-2.south east) -- (M-7-3.north west);
\draw[gray,->] (M-7-2.south east) -- (M-8-3.north west);
\draw[gray,->] (M-8-2.south east) -- (M-9-3.north west);
\draw[gray,->] (M-4-2.east) -- (M-4-3.west);
\draw[gray,->] (M-5-2.east) -- (M-5-3.west);
\draw[gray,->] (M-6-2.east) -- (M-6-3.west);
\draw[gray,->] (M-7-2.east) -- (M-7-3.west);
\draw[gray,->] (M-8-2.east) -- (M-8-3.west);
\draw[gray,->] (M-9-2.east) -- (M-9-3.west);
\end{tikzpicture}
\caption{Backward Difference Table. The arrows in the table point towards the difference; 
the upper component is always subtracted from the lower, e.g. \( {\nabla}^2 f_3 = \nabla f_3 - \nabla f_2\)
\cite{berry2004}.}
\label{fig:differences-table-integrator}
\end{figure}

Differentiation (aka the operator \( D = d / dt \)) can be represented approximately 
in terms of  the \( E \) operator, noting that (\cite{berry2004}):
\begin{equation}
  \begin{aligned}
  E^p f(t_0 ) & = f(t_0 + p h) \\
              & = f(t_0) + p h D f(t_0) + \frac{(ph)^2}{2!}D^2f(t_0) + 
      \frac{(ph)^3}{3!}D^3f(t_0) + \ldots \\
              & = e^{phD}f(t_0)\\
  \end{aligned}
\end{equation}

We identify two operators:
\begin{equation}
  E^p = e^{phD}
\end{equation}
or taking the logarithm:
\begin{equation}
  p h D = p \log E
\end{equation}
which gives (using \ref{eq:berry6}):
\begin{equation}
  \label{eq:berry13}
  h D = \log E = - \log(1-\nabla)
\end{equation}

\section{Adams Method}
The integration operator \( D^{-1} \) can be computed as the inverse of the 
differentiation operator; using \ref{eq:berry13}:
\begin{equation}
  D^{-1} = - \frac{h}{\log(1-\nabla)}
\end{equation}

In the following, focus is placed in satellite orbits, hence the general function 
\( f \) will be replaced with \( \ddot{\vec{r}} \).

The definite integral is computed by using the diplacement operator on the indefinite 
integral (\cite{berry2004}):
\begin{equation}
  (E^p -1) D^{-1} \ddot{\vec{r}}_n = (E^p -1) \dot{\vec{r}}_n 
  = \int_{t_n}^{t_n +ph} \ddot{\vec{r}} \,dt
\end{equation}

This integration corresponds to a \emph{predictor} in a multi-step numerical 
integration, because the equation finds a value \( \dot{\vec{r}} \) at a future 
time. If \( p = 1 \), the predictor operator \( J \) can be written in terms of 
the backward difference operator as (\cite{berry2004}):
\begin{equation}
  \begin{aligned}
  J & = \frac{1}{h} (E-1) D^{-1} \\
    & = \frac{1}{h} [ {(1-\nabla)}^{-1} - 1 ] D^{-1} \\
    & = \frac{1}{h} \frac{\nabla}{1-\nabla}D^{-1} \\
    & = - \frac{\nabla}{(1-\nabla) \log(1-\nabla)}
  \end{aligned}
\end{equation}
