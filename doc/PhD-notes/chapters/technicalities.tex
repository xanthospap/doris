\chapter{Technicalities}
\label{ch:technicalities}

\section{Constexpr Math}
Currently (\today), the standard C++ library \texttt{math.h} does not offer trigonometric mathematical 
functions qualified as \texttt{constexpr} (see \href{https://en.cppreference.com/w/cpp/header/cmath}{Standard library header \texttt{<cmath>}}).
\texttt{constexpr} functions can spped-up computations by performing them at compile-time 
(see \href{https://en.cppreference.com/w/cpp/language/constexpr}{\texttt{constexpr} keyword}). 
The \href{https://gcc.gnu.org/onlinedocs/libstdc++/}{GNU C++ Library} implementation \texttt{libstdc++} 
does offer such functions as an extension. It was however decided to not use these 
extensions, because:
\begin{itemize}
    \item they do not conform to the standard,
    \item we want to be able to build with any compiler and standard library implementation
\end{itemize}

Hence, we are using the \href{https://www.kthohr.com/gcem.html}{Generalized Constant Expression Math} 
(\texttt{gsem}) library, when such functionality is needed.

It is expected that in the near future, most of the standard mathematical functions 
will be marked/implemented as \texttt{constexpr} (\cite{rostencpp}). When such functionality is 
offered, the dependency on \texttt{gcem} should be dropped in favor of using the standard 
functions.

Source files affected:
\begin{itemize}
    \item \path{occultation.cpp}
\end{itemize}

The most common time scales used in astronomical computations, are:
\begin{table}
  \centering
\begin{tabularx}{\textwidth}{>{\raggedright\arraybackslash}X >{\raggedright\arraybackslash}X c}
  %\hline
    \bf{Time-Scale} & \bf{Description} & \bf{Type} \\
  \hline
  \textbf{TAI}\\ \scriptsize{(International Atomic Time)} & The official timekeeping standard & Atomic \\
  \textbf{UTC}\\ \scriptsize{(Coordinated Universal Time)} & The basis of civil time & Atomic/Solar hybrid \\
  \textbf{UT1}\\ \scriptsize{(Universal Time)} & Based on Earth rotation & Solar \\
  \textbf{TT}\\ \scriptsize{(Terrestrial Time)} & Used for solar system ephemeris look-up &  Dynamic \\
  \textbf{TCG}\\ \scriptsize{(Geocentric Coordinate Time)} & Used for calculations centered on the Earth in space & Dynamic \\
  \textbf{TCB}\\ \scriptsize{(Barycentric Coordinate Time)} & Used for calculations beyond Earth orbit; for most common cases, may be approximated by \textbf{TT} & Dynamic \\
  \textbf{TDB}\\ \scriptsize{(Barycentric Dynamical Time)} & A scaled form of TCB that keeps in step with TT
on the average & Dynamic \\
  %\hline
\end{tabularx}
\caption{Common Time Scales used in Astronomical and Celestial Computations.}
\end{table}

Time scales that are obsolete, according to \cite{SOFA}, are:
\begin{description}
  \item[UT0, UT2]: specialist forms of universal time that take into account polar motion and
known seasonal effects; no longer used.
  \item[GMT] (Greenwich mean time): an obsolete time scale that can be taken to mean either
UTC or UT1.
  \item[ET] (ephemeris time): superseded by TT and TDB.
  \item[TDT] (terrestrial dynamical time): the former name of TT.
\end{description}

Note that \emph{Sidereal time} ins not really a time scale but rather an angle. 
The same can be said of UT1; however, the interrelation between UTC and UT1 makes 
it clearer and more convenient to treat the latter as a time scale.

Each of the time scales Each has a distinct role, and there are offsets of tens 
of seconds between some of them. The transformation from one time scale to the next can take a number of forms. In some cases,
for example TAI to TT, it is simply a fixed offset. In others, for example TAI to UT1, it is
an offset that depends on observations and cannot be predicted in advance (or only partially).
Some time scales, for example TT and TCG, are linearly related, with a rate change as well as an
offset. Others, for example TCG and TCB, require a 4-dimensional spacetime transformation (\cite{SOFA}).

\begin{figure}
\centering
\begin{tikzpicture}
\coordinate (UT1) at (-2,0);
%\coordinate (DUT1) at (-2,1);
\coordinate (UTC) at (-2,2);
%\coordinate (DAT) at (-2,3);
\coordinate (TAI) at (-2,4);
\coordinate (TT) at (0,0);
\coordinate (TCG) at (2,-2);
\coordinate (TCB) at (2,-4);
\coordinate (TDB) at (2,-6);

\draw[black,thick] (UT1) rectangle ++(1.0,0.5) node[pos=.5] {\textbf{UT1}};
\draw[black,thick] (UTC) rectangle ++(1.0,0.5) node[pos=.5] {\textbf{UTC}};
\draw[black,thick] (TAI) rectangle ++(1.0,0.5) node[pos=.5] {\textbf{TAI}};
\draw[black,thick] (TT)  rectangle ++(1.0,0.5) node[pos=.5] {\textbf{TT}};
\draw[black,thick] (TCG) rectangle ++(1.0,0.5) node[pos=.5] {\textbf{TCG}};
\draw[black,thick] (TCB) rectangle ++(1.0,0.5) node[pos=.5] {\textbf{TCB}};
\draw[black,thick] (TDB) rectangle ++(1.0,0.5) node[pos=.5] {\textbf{TDB}};

\draw[blue,->] (TAI.south) -- (UTC.north);

\end{tikzpicture}
\caption{Geometry of Alcatel DORIS Ground Antenna/Beacon}
\label{fig:alcatel-antenna}
\end{figure}
