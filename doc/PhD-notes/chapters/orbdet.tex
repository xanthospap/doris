\chapter{Orbit Determination}
\label{ch:orbit-determination}

\section{Orbit Determination via DORIS}
\label{sec:pod}
Let's say we have a DORIS RINEX file with the first (block of) observations 
taken at some time $t_i$ (in TAI). We search through a corresponding sp3 file, 
to get the state of the satellite at an epoch as close as possible to $t_i$; 
Let's call this epoch $t_0$ and the satellite state at this epoch (extracted 
from the sp3) as $\bm{y}_0$.

Starting from initial conditions $\bm{y}_0 \text{ at } t_0$, we use the integrator 
(\ref{sec:Integrator}) to get the state and state transition matrix at $t_i$, 
that is we compute 
$\bm{y}(t=t_i) = \begin{pmatrix} \bm{r}^T(t=t_i) & \bm{v}^T(t=t_i) \end{pmatrix}$ 
and $\Phi _i$.

We will use an Extended Kalman Filter to perform the orbit determination. The 
filter's parameter vector will be the satellite state, plus two parameters per 
beacon for the zenith tropospheric wet delay ($L^w_z$) and the relative frequency 
offset ($\Delta f$), that is:
$\bm{X}=
    \begin{pmatrix}
        r_x & r_y & r_z & \dot{r}_x & \dot{r}_y & \dot{r}_z 
        \begin{pmatrix} L^w_z & \frac{\Delta f_e}{f_{eN}} \end{pmatrix}_{beacon_1} & 
        \begin{pmatrix} L^w_z & \frac{\Delta f_e}{f_{eN}} \end{pmatrix}_{beacon_2} &
        \cdots 
    \end{pmatrix}^T$
with size=$6+2 \times \text{ num of beacons}$

Now start filtering for the current epoch (in the RINEX file) $t_i$. For each 
beacon in the current block:
\begin{enumerate}
    \item If this is the first observation for this beacon, store measurements 
        and exit. If we already have previous measurements for the beacon (so 
        that we can compute the Doppler count), formulate the observation 
        equation, using the two quantities:
        \begin{equation}
            \begin{aligned}
              V_{observed} &= \frac{c}{f_{eN}} 
                \left( f_{eN} - f_{rT} - \frac{N_{DOP}}{\Delta _{\tau_r}} \right)
                + \Delta _{V_{IONO}} + \Delta _{V_{REL_C}} \\
              V_{theoretical} &= \frac{\rho(t_i) - \rho(t_{i-1})}{\Delta _{\tau_r}} + \Delta _{V_{TROPO}} 
                - \frac{c}{f_{eN}} \left( \frac{N_{DOP}}{\Delta _{\tau_r}} + f_{rT} \right) 
                \frac{\Delta f_e}{f_{eN}}
            \end{aligned}
          \end{equation}
        so that $z_i = V_{observed}$
    \item Now compute the partials of the observation w.r.t. the the vector $\bm{X}$, that is 
        \begin{equation}
          \bm{H}_i = \frac{\partial z_i}{\partial \bm{X}} = \frac{\partial V_{theoretical}}{\partial \bm{X}} = 
          \begin{pmatrix} 
            \begin{pmatrix}\frac{\partial z_i}{\partial \bm{r}}\end{pmatrix}^T_{(3 \times 1)} \\
            \begin{pmatrix}\frac{\partial z_i}{\partial \bm{v}}\end{pmatrix}^T_{(3 \times 1)} \\
            \begin{pmatrix}\frac{\partial z_i}{\partial L^w_z} & \frac{\partial z_i}{\partial \frac{\Delta f_e}{f_{eN}}}\end{pmatrix}^T_{(2 \times 1)} \\
            \begin{pmatrix}\frac{\partial z_i}{\partial L^w_z} & \frac{\partial z_i}{\partial \frac{\Delta f_e}{f_{eN}}}\end{pmatrix}^T_{(2 \times 1)} \\
            \cdots
          \end{pmatrix} 
          \end{equation}
        , which is given by:
        \begin{equation}
            \bm{H}_i = \frac{\partial z_i}{\partial \bm{X}} = 
            \begin{pmatrix} 
                \begin{pmatrix}
                     \bm{R} \cdot \left[ \bm{s}_{t_i} / \norm{\bm{s}_{t_i}} - 
                     \bm{s}_{t_{i-1}} / \norm{\bm{s}_{t_{i-1}}} \right] / \Delta _{\tau_r} \end{pmatrix}^T_{(3 \times 1)} \\
                 \bm{0}_{(3 \times 1)} \\
                 \begin{pmatrix} -(C / f_{eN}) * (N_{DOP} / \Delta _{\tau_r}  + f_{rT}) & (L^w_z\at{t_i} - L^w_z\at{t_{i-1}}) / \Delta _{\tau_r} \end{pmatrix}^T_{(2 \times 1)} \\
                 \cdots
            \end{pmatrix}
        \end{equation}
        where $\bm{s}_{t_i}$ is the topocentric satellite-beacon vector and 
        $\bm{R}$ is the topocentric-to-ECEF rotation matrix.
\end{enumerate}

\section{Integrator}
\label{sec:Integrator}

The integrator has the following constructor:
\begin{lstlisting}
  SGOde(ODEfun f, int neqn, double rerr, double aerr,                         
        dso::IntegrationParameters *params = nullptr)
\end{lstlisting}

where:
\begin{itemize}
    \item \label{it:odefun} \texttt{f} is the (vector) function $\bm{f}$ that 
        computes the partial derivatives (at some point $t$)
    \item \texttt{neqn} is the number of equations in $\bm{f}$
    \item \texttt{rerr} and \texttt{aerr} are relative and absolute error 
        tolerances, and
    \item \texttt{params} is a set of parameters used within $\bm{f}$ to 
        compute the derivatives (e.g. some reference epoch $t_{ref}$)
\end{itemize}

Via this structure, we can ``integrate'' the state, solving a first order 
initial value problem. The signature to perfom the integrations is:
\begin{lstlisting}
  int de(double &t, double tout, const VectorXd &y0,                     
         VectorXd &yout) noexcept;
\end{lstlisting}
where:
\begin{itemize}
    \item \texttt{t} is the initial time $t_0$
    \item \texttt{tout} is the target time of integration (at success, we 
        should get \texttt{t}=\texttt{tout})
    \item \texttt{y0} is the vector of initial conditions $\bm{y}_0$
    \item \texttt{yout} is the solution vector, $\bm{y}_{tout}$
\end{itemize}

Hence, in a ``real-world scenario'', if we have the satellite state vector 
$\bm{y}$ at some initial epoch $t_0$, we can extrapolate the state to some 
future time $t$. To get the state transition matrix $\Phi (t,t_0)$, will can 
augment the ODE system with the state transition matrix. Initial values for 
the state transition matrix can be the identity matrix, because 
$\Phi (t_0,t_0) = \bm{I}$. A call to the function thus, would be:
\begin{itemize}
    \item \texttt{t} is $t_0$
    \item \texttt{tout} is $t$
    \item \texttt{y0} is 
    $\begin{pmatrix} \bm{y}^T_{t0} & \bm{I}_{(6 \times 6)} \end{pmatrix} = 
    \begin{pmatrix} \bm{r}^T_{t_0} & \bm{v}^T_{t_0} & \bm{I}_{(6 \times 6)} \end{pmatrix}$
\end{itemize}
and the result would be 
$\begin{pmatrix} \bm{y}^T_{t} & \Phi(t,t_0)_{(6 \times 6)} \end{pmatrix}$
The function providing the derivatives would be (some form of) the 
variatiational equation system (\ref{sec:variational-equations}) and the number 
of equations to solve for would be \texttt{neqn} = $6$ for the state + $6 \times 6$ for 
the state transition matrix.


\section{Variational Equations}
\label{sec:variational-equations}
The variational equations has the signature:

\begin{lstlisting}
void VarEquations(double tsec, const VectorXd &yPhi,
                  VectorXd yPhiP,
                  IntegrationParameters &params) noexcept;
\end{lstlisting}

Computes the variational equations, i.e. the derivative of the state vector 
$\bm{y} = \begin{pmatrix}\bm{r}^T & \bm{v}^T \end{pmatrix}$ and the state 
transition matrix $\Phi$.

\texttt{tsec} is seconds since reference epoch $t_0$.

\texttt{yPhi} (6+36)-dim vector comprising the state vector $\bm{y}$ and the
state transition matrix $\Phi$ in column wise storage order, that is:
$yPhi = \begin{pmatrix}
    \bm{y}^T &  \Phi ^T _{col0} & \Phi ^T _{col1} & \cdots & \Phi^T _{col6}
\end{pmatrix}$

\texttt{yPhiP} (6+36)-dim vector comprising the state vector 
$\dot{\bm{y}}$ and the state transition matrix $\dot{\Phi}$ derivatives, in 
column wise storage order, that is:
$yPhiP = \begin{pmatrix}
    \dot{\bm{y}}^T &  \dot{\Phi} ^T _{col0} & \dot{\Phi} ^T _{col1} & \cdots & \dot{\Phi} ^T _{col6}
\end{pmatrix}$

\begin{enumerate}
    \item Add seconds to reference time $t_0$ to get current time $t$
    
    \item Construct terretrial to celestial matrix, aka $R^{GCRF}_{ECEF}$ at 
        $t$ (used in \ref{it:ag} to transform state to ECEF).
    
    \item \label{it:ag} Compute gravity-induced acceleration and partials, 
        $\bm{a}_g$ and derivaties  
        $\frac{\partial \bm{a}_g}{\partial \bm{r}}$. Note that 
        $\frac{\partial \bm{a}_g}{\partial \bm{v}} = \bm{0}$.
    
    \item \label{it:atbp} Compute third body perturbation accelerations and 
        partials for Sun and Moon, aka $\bm{a}_{sun}$, $\bm{a}_{moon}$ and 
        $\frac{\partial \bm{a}_{tbp}}{\partial \bm{r}}$. Note that 
        $\frac{\partial \bm{a}_{tbp}}{\partial \bm{v}} = \bm{0}$.
    
    \item Extract state transition matrix (from \texttt{yPhi}):
        \begin{equation}
            \Phi = 
            \begin{pmatrix}
                yPhi[1:7, 1:7]
            \end{pmatrix}
        \end{equation}
    
    \item Construct the derivative of the state transition matrix:
        \begin{equation}
            \dot{\Phi} = 
            \begin{pmatrix}
                \bm{0}_{(3 \times 3)} & \bm{I}_{(3 \times 3)} \\
                \frac{\partial \bm{a}_g}{\partial \bm{r}} + 
                    \frac{\partial \bm{a}_{tbp}}{\partial \bm{r}} & \bm{0}_{(3 \times 3)}
            \end{pmatrix}
        \end{equation}
    
    \item construct $yPhip = \dot{\Phi} \Phi$
    
    \item augment the above matrix with the state partials 
        \begin{equation}
            yPhiP = 
            \begin{pmatrix} 
                \dot{\bm{y}} & \dot{\Phi} \Phi 
            \end{pmatrix}  = 
            \begin{pmatrix} 
                \begin{pmatrix} \bm{v} & \bm{a}_g + \bm{a}_{sun} + \bm{a}_{moon} \end{pmatrix}^T 
                & \dot{\Phi} \Phi
            \end{pmatrix}
        \end{equation}
\end{enumerate}