\chapter{Orbit Determination}
\label{ch:orbit-determination}

\section{Integrator}
\label{sec:Integrator}

The integrator has the following constructor:
\begin{lstlisting}
  SGOde(ODEfun f, int neqn, double rerr, double aerr,                         
        dso::IntegrationParameters *params = nullptr)
\end{lstlisting}

where:
\begin{itemize}
    \item \label{it:odefun} \texttt{f} is the (vector) function $\bm{f}$ that 
        computes the partial derivatives (at some point $t$)
    \item \texttt{neqn} is the number of equations in $\bm{f}$
    \item \texttt{rerr} and \texttt{aerr} are relative and absolute error 
        tolerances, and
    \item \texttt{params} is a set of parameters used within $\bm{f}$ to 
        compute the derivatives (e.g. some reference epoch $t_{ref}$)
\end{itemize}

Via this structure, we can ``integrate'' the state, solving a first order 
initial value problem. The signature to perfom the integrations is:
\begin{lstlisting}
  int de(double &t, double tout, const Eigen::VectorXd &y0,                     
         Eigen::VectorXd &yout) noexcept;
\end{lstlisting}
where:
\begin{itemize}
    \item \texttt{t} is the initial time $t_0$
    \item \texttt{tout} is the target time of integration (at success, we 
        should get \texttt{t}=\texttt{tout})
    \item \texttt{y0} is the vector of initial conditions $\bm{y}_0$
    \item \texttt{yout} is the solution vector, $\bm{y}_{tout}$
\end{itemize}

Hence, in a ``real-world scenario'', if we have the satellite state vector 
$\bm{y}$ at some initial epoch $t_0$, we can extrapolate the state to some 
future time $t$. To get the state transition matrix $\Phi (t,t_0)$, will can 
augment the ODE system with the state transition matrix. Initial values for 
the state transition matrix can be the identity matrix, because 
$\Phi (t_0,t_0) = \bm{I}$. A call to the function thus, would be:
\begin{itemize}
    \item \texttt{t} is $t_0$
    \item \texttt{tout} is $t$
    \item \texttt{y0} is 
    $\begin{pmatrix} \bm{y}^T_{t0} & \bm{I}_{(6 \times 6)} \end{pmatrix} = 
    \begin{pmatrix} \bm{r}^T_{t_0} & \bm{v}^T_{t_0} & \bm{I}_{(6 \times 6)} \end{pmatrix}$
\end{itemize}
and the result would be 
$\begin{pmatrix} \bm{y}^T_{t} & \Phi(t,t_0)_{(6 \times 6)} \end{pmatrix}$
The function providing the derivatives would be (some form of) the 
variatiational equation system (\ref{sec:variational-equations}) and the number 
of equations to solve for would be \texttt{neqn} = $6$ for the state + $6 \times 6$ for 
the state transition matrix.


\section{Variational Equations}
\label{sec:variational-equations}
The variational equations has the signature:

\begin{lstlisting}
void VarEquations(double tsec, const Eigen::VectorXd &yPhi,
                  Eigen::Ref<Eigen::VectorXd> yPhiP,
                  dso::IntegrationParameters &params) noexcept;
\end{lstlisting}

Computes the variational equations, i.e. the derivative of the state vector 
$\bm{y} = \begin{pmatrix}\bm{r}^T & \bm{v}^T \end{pmatrix}$ and the state 
transition matrix $\Phi$.

\texttt{tsec} is seconds since reference epoch $t_0$.

\texttt{yPhi} (6+36)-dim vector comprising the state vector $\bm{y}$ and the
state transition matrix $\Phi$ in column wise storage order, that is:
$yPhi = \begin{pmatrix}
    \bm{y}^T &  \Phi ^T _{col0} & \Phi ^T _{col1} & \cdots & \Phi^T _{col6}
\end{pmatrix}$

\texttt{yPhiP} (6+36)-dim vector comprising the state vector 
$\dot{\bm{y}}$ and the state transition matrix $\dot{\Phi}$ derivatives, in 
column wise storage order, that is:
$yPhiP = \begin{pmatrix}
    \dot{\bm{y}}^T &  \dot{\Phi} ^T _{col0} & \dot{\Phi} ^T _{col1} & \cdots & \dot{\Phi} ^T _{col6}
\end{pmatrix}$

\begin{enumerate}
    \item Add seconds to reference time $t_0$ to get current time $t$
    
    \item Construct terretrial to celestial matrix, aka $R^{GCRF}_{ECEF}$ at 
        $t$ (used in \ref{it:ag} to transform state to ECEF).
    
    \item \label{it:ag} Compute gravity-induced acceleration and partials, 
        $\bm{a}_g$ and derivaties  
        $\frac{\partial \bm{a}_g}{\partial \bm{r}}$. Note that 
        $\frac{\partial \bm{a}_g}{\partial \bm{v}} = \bm{0}$.
    
    \item \label{it:atbp} Compute third body perturbation accelerations and 
        partials for Sun and Moon, aka $\bm{a}_{sun}$, $\bm{a}_{moon}$ and 
        $\frac{\partial \bm{a}_{tbp}}{\partial \bm{r}}$. Note that 
        $\frac{\partial \bm{a}_{tbp}}{\partial \bm{v}} = \bm{0}$.
    
    \item Extract state transition matrix (from \texttt{yPhi}):
        \begin{equation}
            \Phi = 
            \begin{pmatrix}
                yPhi[1:7, 1:7]
            \end{pmatrix}
        \end{equation}
    
    \item Construct the derivative of the state transition matrix:
        \begin{equation}
            \dot{\Phi} = 
            \begin{pmatrix}
                \bm{0}_{(3 \times 3)} & \bm{I}_{(3 \times 3)} \\
                \frac{\partial \bm{a}_g}{\partial \bm{r}} + 
                    \frac{\partial \bm{a}_{tbp}}{\partial \bm{r}} & \bm{0}_{(3 \times 3)}
            \end{pmatrix}
        \end{equation}
    
    \item construct $yPhip = \dot{\Phi} \Phi$
    
    \item augment the above matrix with the state partials 
        \begin{equation}
            yPhiP = 
            \begin{pmatrix} 
                \dot{\bm{y}} & \dot{\Phi} \Phi 
            \end{pmatrix}  = 
            \begin{pmatrix} 
                \begin{pmatrix} \bm{v} & \bm{a}_g + \bm{a}_{sun} + \bm{a}_{moon} \end{pmatrix}^T 
                & \dot{\Phi} \Phi
            \end{pmatrix}
        \end{equation}
\end{enumerate}