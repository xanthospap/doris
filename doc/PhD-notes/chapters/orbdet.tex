\chapter{Orbit Determination}
\label{ch:orbit-determination}

\section{Integrator}
\label{sec:Integrator}


\section{Variational Equations}
\label{sec:variational-equations}
The variational equations has the signature:

\begin{lstlisting}
void VariationalEquations(double tsec, const Eigen::VectorXd &yPhi,
                          Eigen::Ref<Eigen::VectorXd> yPhiP,
                          dso::IntegrationParameters &params) noexcept;
\end{lstlisting}

Computes the variational equations, i.e. the derivative of the state vector 
$\bm{y} = \begin{pmatrix}\bm{r}^T & \bm{v}^T \end{pmatrix}$ and the state 
transition matrix $\Phi$.

\texttt{tsec} is seconds since reference epoch $t_0$.

\texttt{yPhi} (6+36)-dim vector comprising the state vector $\bm{y}$ and the
state transition matrix $\Phi$ in column wise storage order, that is:
$yPhi = \begin{pmatrix}
    \bm{y}^T &  \Phi ^T _{col0} & \Phi ^T _{col1} & \cdots & \Phi^T _{col6}
\end{pmatrix}$

\texttt{yPhiP} (6+36)-dim vector comprising the state vector 
$\dot{\bm{y}}$ and the state transition matrix $\dot{\Phi}$ derivatives, in 
column wise storage order, that is:
$yPhiP = \begin{pmatrix}
    \dot{\bm{y}}^T &  \dot{\Phi} ^T _{col0} & \dot{\Phi} ^T _{col1} & \cdots & \dot{\Phi} ^T _{col6}
\end{pmatrix}$

\begin{enumerate}
    \item Add seconds to reference time $t_0$ to get current time $t$
    
    \item Construct terretrial to celestial matrix, aka $R^{GCRF}_{ECEF}$ at 
        $t$ (used throughout to transform state vector).
    
    \item Compute gravity-induced acceleration and partials, $\bm{a}_g$ and 
        $\frac{\partial \bm{a}_g}{\partial \bm{r}}$. Note that 
        $\frac{\partial \bm{a}_g}{\partial \bm{v}} = \bm{0}$.
    
    \item Compute third body perturbation accelerations and partials for 
        Sun and Moon, aka $\bm{a}_{sun}$, $\bm{a}_{moon}$ and 
        $\frac{\partial \bm{a}_{tbp}}{\partial \bm{r}}$. Note that 
        $\frac{\partial \bm{a}_{tbp}}{\partial \bm{v}} = \bm{0}$.
    
    \item Extract state transition matrix (from \texttt{yPhi}):
        \begin{equation}
            \Phi = 
            \begin{pmatrix}
                yPhi[1:7, 1:7]
            \end{pmatrix}
        \end{equation}
    
    \item Construct the derivative of the state transition matrix:
        \begin{equation}
            \dot{\Phi} = 
            \begin{pmatrix}
                \bm{0}_{(3 \times 3)} & \bm{I}_{(3 \times 3)} \\
                \frac{\partial \bm{a}_g}{\partial \bm{r}} + 
                    \frac{\partial \bm{a}_{tbp}}{\partial \bm{r}} & \bm{0}_{(3 \times 3)}
            \end{pmatrix}
        \end{equation}
    
    \item construct $yPhip = \dot{\Phi} \Phi$
    
    \item augment the above matrix with the state partials 
        \begin{equation}
            yPhiP = 
            \begin{pmatrix} 
                \dot{\bm{y}} & \dot{\Phi} \Phi 
            \end{pmatrix}  = 
            \begin{pmatrix} 
                \begin{pmatrix} \bm{v} & \bm{a}_g + \bm{a}_{sun} + \bm{a}_{moon} \end{pmatrix}^T 
                & \dot{\Phi} \Phi
            \end{pmatrix}
        \end{equation}
\end{enumerate}