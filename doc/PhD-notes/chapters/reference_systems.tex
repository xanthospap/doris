\chapter{Reference Systems and Frames}
\label{ch:reference-systems-frames}

\section{Basic Definitions}
\label{sec:basic-definitions}

\gls{icrf} is the realization of the \emph{barycentric}, fixed, stable, celestial 
reference system based on observations of extragalactic radio sources. The \gls{bcrf} 
is a result of a transformation of the \gls{bcrf} to the geocenter and thus, a 
realization of the \gls{gcrs} (\cite{Gurfil18}); it is a \emph{geocentric} \gls{icrf}.

The \gls{cip} is the geocentric equatorial pole, determined by the \gls{iau} 
precession-nutation model for the transformation from the \gls{icrf} to the 
\gls{gcrf}. It is is an intermediate pole separating, by convention, the motion 
of the pole of the \gls{itrs} in the \gls{gcrs} into a celestial part and a 
terrestrial part (\cite{iers2010}). To be consistent with \gls{iau} 2000 Resolution 
B1.6 and 2006 Resolution B1, starting on 1 January 2009, the matrix \(Q(t)\) in 
\label{eq:tn3651} should be based on the \gls{iau} 2006 precession and on the 
nutation model \gls{iau} 2000A or \gls{iau} 2000B depending on the required 
precision.

The \gls{cio} is an 
origin of right ascensions on the instantaneous celestial true equator of date 
(\cite{Gurfil18}).

The \gls{cirs} is a geocentric reference system, related to the \gls{gcrs} by a 
time-dependent rotation, taking into account precession-nutation. 

\section{Transformation between \gls{itrs} and \gls{gcrs}}
\label{transformation-itrs-gcrs}
The transformation to be used to relate a vector in the \gls{itrs} (\(\vec{r}_{ITRS}\)) 
to the \gls{gcrs} (\(\vec{r}_{GCRS}\)), is:
\begin{equation}
  \label{eq:tn3651}
  \vec{r}_{GCRS} = Q(t) R(t) W(t) \vec{r}_{ITRS}
\end{equation}
\ref{eq:tn3651} is valid for any choice of celestial pole and origin on the equator 
of that pole (\cite{iers2010}).

According to the \gls{iau} resolution (\cite{iers2010}), time coordinates for the 
\gls{bcrs} should be expressed in \gls{tcb}, whereas for \gls{gcrs}, time coordinates 
should be expressed in either \gls{tcg} or \gls{tt}. \gls{tcg} and \gls{tt} differ 
by a constant rate. The parameter \(t\) to be used in \ref{eq:tn3651}, is defined by:
\begin{equation}
  \label{eq:tn3652}
  t = (TT - \text{ 2000 January 1d 12h } TT) \text{ in days} / 36525.
\end{equation}

Two equivalent procedures can be followed to implement the relation \ref{eq:tn3651}, 
consistent with the \gls{iau} 2000/2006 resolutions; they differ by the origin 
that is adopted on the \gls{cip} equator (i.e. the equinox or the \gls{cio}) and 
are conventionaly called ``equinox based'' and ``CIO based'' respectively. The matrix 
\(W(t)\) is the same for both procedures, while \(Q(t)\) and \(R(t)\) depend on the 
corresponding origin on the \gls{cip} equator. Of the two, only the `CIO based'
procedure can be in agreement with \gls{iau} 2000 Resolution B1.8, which requires 
the use of the ``nonrotating origin'' in both the \gls{gcrs} and the \gls{itrs} as 
well as the position of the \gls{cip} in the \gls{gcrs} and in the \gls{itrs} 
(\cite{iers2010}).

When applying the transformation \(W(t)\) at date \(t\), the system is transformed 
from \gls{itrs} to \gls{tirs}, which uses the \gls{cip} as its \(z\)-axis and 
the \gls{tio} as its \(x\)-axis. The formula for the transformation is
\begin{equation}
  \label{eq:tn3653}
  W(t) = R_3 (-s') R_2 (x_p ) R_1 (y_p )
\end{equation}
where \(x_p\) and \(y_p\) are the polar coordinates of the \gls{cip} in the \gls{itrs} 
and \(s'\) is the ``TIO locator'' which provides the position of the \gls{tio} 
on the equator of the \gls{cip} corresponding to the kinematical definition of 
the ``non-rotating'' origin in the \gls{itrs} when the \gls{cip} is moving with 
respect to the \gls{itrs} due to polar motion (\cite{iers2010}).

The pole coordinate parameters \(x_p \) and \(y_p\) to be used in \ref{eq:tn3653} 
(if not estimated) should be the ones published by the \gls{iers} with corrections 
for the effect of ocean tides \((\Delta x , \Delta y )_{ocean tides} \) and for the 
forced terms \((\Delta x , \Delta y )_{libration} \) with periods less than two 
days in space:
\begin{equation}
  \label{eq:tn36511}
  \begin{pmatrix}
    x_p \\
    y_p
  \end{pmatrix}
  =
  \begin{pmatrix}
    x \\
    y
  \end{pmatrix} _{IERS}
  +
  \begin{pmatrix}
  \Delta x \\
  \Delta y
  \end{pmatrix} _{ocean tides}
  +
  \begin{pmatrix}
  \Delta x \\
  \Delta y
  \end{pmatrix} _{libration}
\end{equation}

In the ``CIO based'' procedure, realizes an intermediate celestial reference system 
at date \(t\) that uses the \gls{cip} as its \(z\)-axis and the \gls{cio} as its 
\(x\)-axis, called the \gls{cirs}. The matrix \(R(t)\) uses the \gls{era}, that 
is the angle between the \gls{cio} and the \gls{tio} at date \(t\) on the equator 
of the \gls{cip} (aka, the sidereal rotation of the Earth) matrix \(R(t)\), relating 
\gls{tirs} and \gls{cirs}, can be expressed as
\begin{equation}
  \label{eq:tn3655}
  R(t) = R_3 (-ERA)
\end{equation}
\(Q(t)\) uses the two coordinates of the \gls{cip} in the \gls{gcrs}
\begin{equation}
  \label{eq:tn3656}
  Q(t) = R_3 (-E) R_2 (-d) R_3 (E) R_3 (s)
\end{equation}
where
\begin{equation}
  \label{eq:tn3657}
  X = \sin d \cos E , \quad
  Y = \sin d \sin E , \quad
  Z = \cos d
\end{equation}
\(\left[ X  Y  Z \right]\) are the coordinates of the 
\gls{cip} in the \gls{gcrs}.
\(s\) is the ``CIO locator'' which provides the position of the \gls{cio} 
on the equator of the \gls{cip} corresponding to the kinematical definition of 
the Non-Rotating-Origin in the \gls{gcrs} when the \gls{cip} is moving with 
respect to the \gls{gcrs}, between the reference epoch and the date \(t\) due to
precession and nutation (\cite{iers2010}). Alternatively, \(Q(t)\) can be written 
as
\begin{equation}
  \label{eq:tn36510}
  Q(t) = \begin{pmatrix}
    1 - \alpha X^2 & -\alpha XY      & X \\
    -\alpha XY     & 1 - \alpha Y^2  & Y \\
    -X             &  -Y             & 1 - \alpha (X^2 + Y^2)\\
  \end{pmatrix} R_3 (s)
\end{equation}
with \(\alpha = 1 / (1 + \cos d ) \).

In contrast, the ``equinox based'' procedure uses an intermediate celestial reference 
system that uses the \gls{cip} as its \(z\)-axis and the equinox as its \(x\)-axis, 
called the ``true equinox and equator of date system''. The matrix \(R(t)\) uses 
the \gls{gst} for Earth rotation, which transforms from the \gls{tirs} to the 
true equinox and equator of date system; \gls{gst} is the angle between the equinox 
and the \gls{tio}
\begin{equation}
  R(t) = R_3 (-GST)
\end{equation}
\(Q(t)\) uses the classical precession and nutation parameters; it can be formed 
in two ways: 
\begin{itemize}
  \item using the classical nutation angles and precession matrix, including a 
  separate rotation matrix for the frame biases, or
  \item referred directly to the \gls{gcrs} pole and origin without requiring the 
  frame bias to be applied separately, and no separate precession and nutation
  steps.
\end{itemize}
