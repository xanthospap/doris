\chapter{Reference Systems and Frames}
\label{ch:reference-systems-frames}

\section{Basic Definitions}
\label{sec:basic-definitions}
In the last two decades, major changes and updates have been developed and adopted 
regarding the theoretical background, modeling and implementation of the 
complicated movements of the Earth and their formulation with respect to a 
\gls{crs}. In this chapter, we are only going to briefly describe the basic 
definitions and algorithms relevant to Satellite Geodesy, adhering to the 
so-called \textbf{IAU 2000/2006 resolutions} that were made by the 2003-2006 
IAU Working Group on \emph{Nomenclature for fundamental astronomy}, \cite{Capitaine2006}.

\gls{icrf} is the realization of the \emph{barycentric}, fixed, stable, celestial 
reference system based on observations of extragalactic radio sources 
(\cite{Gurfil18}). The \gls{gcrf} is a result of a relativistic, coordinate 
transformation of the \gls{bcrf} to the geocenter and thus, a realization of the 
\gls{gcrs}; it is a \emph{geocentric} \gls{icrf}. Accordint to \cite{esaa13}, in 
general, the \gls{gcrs} should be used when the object of interest is within 
several Earth diameters of the geocenter; beyond this distance, the \gls{bcrs} 
should be used.

The \gls{cip} is the geocentric equatorial pole, determined by the \gls{iau} 
precession-nutation model for the transformation from the \gls{icrf} to the 
\gls{gcrf}. It is is an intermediate pole separating, by convention, the motion 
of the pole of the \gls{itrs} in the \gls{gcrs} into a celestial part and a 
terrestrial part (\cite{iers2010}). To be consistent with \gls{iau} 2000 Resolution 
B1.6 and 2006 Resolution B1, starting on 1 January 2009, the matrix \(Q(t)\) in 
\label{eq:tn3651} should be based on the \gls{iau} 2006 precession and on the 
nutation model \gls{iau} 2000A or \gls{iau} 2000B depending on the required 
precision.

The \gls{cio} is an 
origin of right ascensions on the instantaneous celestial true equator of date 
(\cite{Gurfil18}).

The \gls{cirs} is a geocentric reference system, related to the \gls{gcrs} by a 
time-dependent rotation, taking into account precession-nutation. 

\section{Transformation between \gls{itrs} and \gls{gcrs}}
\label{transformation-itrs-gcrs}
The transformation to be used to relate a vector in the \gls{itrs} (\(\vec{r}_{ITRS}\)) 
to the \gls{gcrs} (\(\vec{r}_{GCRS}\)), is:
\begin{equation}
  \label{eq:tn3651}
  \vec{r}_{GCRS} = Q(t) R(t) W(t) \vec{r}_{ITRS}
\end{equation}
\ref{eq:tn3651} is valid for any choice of celestial pole and origin on the equator 
of that pole (\cite{iers2010}).

\begin{figure}
\centering
\begin{tikzpicture}[
  every node/.style = {
    draw=black, 
    rounded corners, 
    fill=gray!30,
    minimum width=2cm,
    minimum height=0.5cm,
    align=center},
  every path/.style = {
    draw,
    -latex}
  ]

  \node (gcrs) {\acrfull{gcrs}};
  
  \node[fill=gray!5] (gcrs2cirs) [below =of gcrs] {
    apply matrix \(Q(t)\), using \gls{cip} coordinates 
    and the \gls{cio} locator:\\
    \(Q(t) = R_3 (E) R_2 (d) R_3 (-E) R_3 (-s)\)\\
    see \ref{ssec:gcrs-to-cirs-via-q}
  };

  \node (cirs) [below =of gcrs2cirs] {\acrfull{cirs}};

  \node[fill=gray!5] (cirs2tirs) [below =of cirs] {
    apply matrix \(R(t)\), using the \gls{era}:\\
    \(R(t) = R_3 (ERA) \)\\
    see \ref{ssec:cirs-to-tirs-via-r}
  };
  
  \node (tirs) [below =of cirs2tirs] {\acrfull{tirs}};

  \node[fill=gray!5] (tirs2itrs) [below =of tirs] {
    apply matrix \(W(t)\), that is:\\
    rotation to the \gls{itrs} origin: \(R_3 (s')\), and \\
    rotation for polar motion: \(R_2(-x_p) R_1(-y_p)\) \\
    see \ref{ssec:tirs-to-itrs-via-w}
  };
  
  \node (itrs) [below =of tirs2itrs] {\acrfull{itrs}};

  \draw (gcrs) -- (gcrs2cirs);
  \draw (gcrs2cirs) -- (cirs);
  \draw (cirs) -- (cirs2tirs);
  \draw (cirs2tirs) -- (tirs);
  \draw (tirs) -- (tirs2itrs);
  \draw (tirs2itrs) -- (itrs);

\end{tikzpicture}

\caption{Transformation from \gls{gcrs} to \gls{itrs} using the so called ``CIO based'' procedure.}
\label{fig:bcrs-to-itrs}
\end{figure}

According to the \gls{iau} resolution (\cite{iers2010}), time coordinates for the 
\gls{bcrs} should be expressed in \gls{tcb}, whereas for \gls{gcrs}, time coordinates 
should be expressed in either \gls{tcg} or \gls{tt}. \gls{tcg} and \gls{tt} differ 
by a constant rate. The parameter \(t\) to be used in \ref{eq:tn3651}, is defined by:
\begin{equation}
  \label{eq:tn3652}
  t = (TT - \text{ 2000 January 1d 12h } TT) \text{ in days} / 36525.
\end{equation}
that is he elapsed time in Julian centuries since J2000 TT.

Two equivalent procedures can be followed to implement the relation \ref{eq:tn3651}, 
consistent with the \gls{iau} 2000/2006 resolutions; they differ by the origin 
that is adopted on the \gls{cip} equator (i.e. the equinox or the \gls{cio}) and 
are conventionaly called ``equinox based'' and ``CIO based'' respectively. The matrix 
\(W(t)\) is the same for both procedures, while \(Q(t)\) and \(R(t)\) depend on the 
corresponding origin on the \gls{cip} equator. Of the two, only the `CIO based'
procedure can be in agreement with \gls{iau} 2000 Resolution B1.8, which requires 
the use of the ``nonrotating origin'' in both the \gls{gcrs} and the \gls{itrs} as 
well as the position of the \gls{cip} in the \gls{gcrs} and in the \gls{itrs} 
(\cite{iers2010}).

\subsection{\gls{tirs} to \gls{itrs} via \(W(t)\)}
\label{ssec:tirs-to-itrs-via-w}
When applying the transformation \(W(t)\) at date \(t\), the system is transformed 
from \gls{itrs} to \gls{tirs}, which uses the \gls{cip} as its \(z\)-axis and 
the \gls{tio} as its \(x\)-axis. The formula for the transformation is
\begin{equation}
  \label{eq:tn3653}
  W(t) = R_3 (-s') R_2 (x_p ) R_1 (y_p )
\end{equation}
where \(x_p\) and \(y_p\) are the polar coordinates of the \gls{cip} in the 
\gls{itrs} (\ref{ssec:motion-of-the-cip-in-the-itrs}) and \(s'\) is the 
``TIO locator'' (\ref{ssec:position-of-the-tio-in-the-itrs}) which provides 
the position of the \gls{tio} on the equator of the \gls{cip} corresponding to 
the kinematical definition of the ``non-rotating'' origin in the \gls{itrs} 
when the \gls{cip} is moving with respect to the \gls{itrs} due to polar motion 
(\cite{iers2010}).

\subsection{\gls{cirs} to \gls{tirs} via \(R(t)\)}
\label{ssec:cirs-to-tirs-via-r}
Tthe ``CIO based'' procedure, realizes an intermediate celestial reference system 
at date \(t\) that uses the \gls{cip} as its \(z\)-axis and the \gls{cio} as its 
\(x\)-axis, called the \gls{cirs}. The matrix \(R(t)\) uses the \gls{era}, that 
is the angle between the \gls{cio} and the \gls{tio} at date \(t\) on the equator 
of the \gls{cip} (aka, the sidereal rotation of the Earth) matrix \(R(t)\), relating 
\gls{tirs} and \gls{cirs}, can be expressed as
\begin{equation}
  \label{eq:tn3655}
  R(t) = R_3 (-ERA)
\end{equation}


\subsection{\gls{gcrs} to \gls{cirs} via \(Q(t)\)}
\label{ssec:gcrs-to-cirs-via-q}
\(Q(t)\) is the matrix expressing the combined effect of nutation, precession and 
frame bias, performing the \emph{intermediate-to-celestial} transformation.
\(Q(t)\) uses the two coordinates of the \gls{cip} in the \gls{gcrs}
\begin{equation}
  \label{eq:tn3656}
  Q(t) = R_3 (-E) R_2 (-d) R_3 (E) R_3 (s)
\end{equation}
where
\begin{equation}
  \label{eq:tn3657}
  X = \sin d \cos E , \quad
  Y = \sin d \sin E , \quad
  Z = \cos d
\end{equation}
\(\left[ X  Y  Z \right]\) are the coordinates of the 
\gls{cip} in the \gls{gcrs}.
\(s\) is the ``CIO locator'' which provides the position of the \gls{cio} 
on the equator of the \gls{cip} corresponding to the kinematical definition of 
the Non-Rotating-Origin in the \gls{gcrs} when the \gls{cip} is moving with 
respect to the \gls{gcrs}, between the reference epoch and the date \(t\) due to
precession and nutation (\cite{iers2010}). Alternatively, \(Q(t)\) can be written 
as
\begin{equation}
  \label{eq:tn36510}
  Q(t) = \begin{pmatrix}
    1 - \alpha X^2 & -\alpha XY      & X \\
    -\alpha XY     & 1 - \alpha Y^2  & Y \\
    -X             &  -Y             & 1 - \alpha (X^2 + Y^2)\\
  \end{pmatrix} R_3 (s)
\end{equation}
with \(\alpha = 1 / (1 + \cos d ) \) and \(X\) and \(Y\) being the coordinates of 
the \gls{cip} in the \gls{gcrs} (\ref{ssec:coordinates-of-the-cip-in-the-gcrs}).

In contrast, the ``equinox based'' procedure \todo{Probably skip the equinox based procedure 
alltogether. Only describe the `new', CIO-based procedure} uses an intermediate celestial reference 
system that uses the \gls{cip} as its \(z\)-axis and the equinox as its \(x\)-axis, 
called the ``true equinox and equator of date system''. The matrix \(R(t)\) uses 
the \gls{gst} for Earth rotation, which transforms from the \gls{tirs} to the 
true equinox and equator of date system; \gls{gst} is the angle between the equinox 
and the \gls{tio}
\begin{equation}
  R(t) = R_3 (-GST)
\end{equation}
\(Q(t)\) uses the classical precession and nutation parameters; it can be formed 
in two ways: 
\begin{itemize}
  \item using the classical nutation angles and precession matrix, including a 
  separate rotation matrix for the frame biases, or
  \item referred directly to the \gls{gcrs} pole and origin without requiring the 
  frame bias to be applied separately, and no separate precession and nutation
  steps.
\end{itemize}

\subsection{Motion of the \gls{cip} in the \gls{itrs}}
\label{ssec:motion-of-the-cip-in-the-itrs}
The rotation of the Earth is represented by the diurnal rotation around a reference 
axis, the \gls{cip}, whose motion with respect to the \gls{irf} is represented 
by the theories of precession and nutation (\cite{esaa13}). The \gls{cip} moves 
slowly, in a \gls{trf}, is a quasi-circular path around the axis of figure (maximum 
proncipal moment of inertia). The motion of the of the \gls{trf} with respect to 
the \gls{cip} is known as \emph{polar motion} (\cite{esaa13}).

Polar motion consists of three major components: a free oscillation called 
\emph{Chandler wobble} with a period of about 435 days, an annual oscillation 
forced by the seasonal displacement of air and water masses, and an irregular 
drift, partly due to motions in the Earth's core and mantle, and partly to the 
redistribution of water mass (\cite{iersPolarmotionWs}). Unpredictable geophysical forces can 
affect the exact period and amplitude of polar motion, thus its determination 
can only be performed via observation.

The \gls{iers} publishes values for coordinates of the \gls{cip}, \(x_p\), \(y_p\).
When applied in \ref{eq:tn3653} though, additional corrections/components have 
to be applied, to account for a number of effects.

The pole coordinate parameters \(x_p \) and \(y_p\) to be used in \ref{eq:tn3653} 
(if not estimated) should be the ones published by the \gls{iers} with corrections 
for the effect of ocean tides \((\Delta x , \Delta y )_{ocean tides} \) and for the 
forced terms \((\Delta x , \Delta y )_{libration} \) with periods less than two 
days in space:
\begin{equation}
  \label{eq:tn36511}
  \begin{pmatrix}
    x_p \\
    y_p
  \end{pmatrix}
  =
  \begin{pmatrix}
    x \\
    y
  \end{pmatrix} _{IERS}
  +
  \begin{pmatrix}
  \Delta x \\
  \Delta y
  \end{pmatrix} _{ocean tides}
  +
  \begin{pmatrix}
  \Delta x \\
  \Delta y
  \end{pmatrix} _{libration}
\end{equation}

where \((x, y)_{IERS}\) are pole coordinates provided by the \gls{iers}, 
\((\Delta x, \Delta y)_{ocean tides}\) are the diurnal and semi-diurnal variations 
in pole coordinates caused by ocean tides, and \((\Delta x, \Delta y)_{libration}\) 
are the variations in pole coordinates corresponding to motions with periods 
less than two days in space that are not part of the IAU 2000 nutation model.
A detailed discussion on the variations caused by ocean tides and so-called ``libration'', 
can be found in \cite{iers2010}.

To apply the corrections, we can use one of the following schemes (after retrieving 
\gls{eop} information, e.g. from \gls{iers} Bulletin files):
\todo[color=red!40]{Is this correct? See also \cite{iers2010}, section 5.5.1. It seems that 
a call to \texttt{interp\_pole} is the same as a call to \texttt{ortho\_eop} followed 
by a call to \texttt{pmsdnut2}. The latter may give a bit better precision, but 
we first need an interpolation for the given MJD. \texttt{interp\_pole} e.g. uses 
a Lagrange interpolation algorithm using 4 data points.}

\begin{enumerate}
  \item use the function \texttt{iers2010::interp\_pole}, or
  \item use the functions \texttt{iers2010::ortho\_eop} followed by \texttt{iers2010::pmsdnut2}
\end{enumerate}

\subsection{Position of \gls{tio} in the \gls{itrs}, the ``\gls{tio} locator''}
\label{ssec:position-of-the-tio-in-the-itrs}
The \gls{tio} locator, provides the position of the \gls{tio} on the equator
of the \gls{cip}, such that there is no component of polar motion about the pole 
of rotation. This constitues a kinematical definition of the \emph{non-rotating origin} 
in the \gls{itrs}, when the \gls{cip} is moving with respect to the \gls{itrs} 
due to polar motion. The use of the \gls{tio} locator, 
\begin{displayquote}
which was neglected in the classical form prior to 1 January 2003, is
necessary to provide an exact realization of the ``instantaneous prime meridian'' 
(designated by ``TIO meridian'').
\end{displayquote} (\cite{iers2010}).

\ref{eq:tn3653} needs the \gls{tio}-locator \(s'\), which can be computed by:
\begin{equation}
  \label{eq:tn3654}
  s'(t) = \frac{1}{2} \int_{t_0}^{t} \left( 
    x_p \dot{y}_p - \dot{x}_p y_p \right) \,dt
\end{equation}
According to \cite{iers2010}, the value of \(s'\) will be less than 
\SI{0.4}{\milli\arcsecond} after one century. Using the current mean amplitudes 
for the Chandlerian and annual wobbles gives:
\begin{equation}
  \label{eq:tn36513}
  s' = -47 \text{ } \mu as \text{ } t
\end{equation}

\subsection{\acrfull{era}}
\label{ssec:earth-rotation-angle}
For the computation of \ref{eq:tn3655}, we need the \gls{era}, which is given by 
(\cite{iers2010}):
\begin{equation}
  \label{eq:tn36515}
  \begin{aligned}
  ERA(T_u ) &= 2 \pi ( UT1 \text{ Julian day fraction } \\
            &+ 0.7790572732640 + 0.00273781191135448 T_u )
  \end{aligned}
\end{equation}

where \(T_u = \text{ Julian UT1 date } - 2451545.0 \) and \( UT1 = UTC + \Delta UT1 \), 
and \(\Delta UT1 \) is provided by the \gls{iers} (if not estimated). Similarly 
to polar motion, additional components should be added to the values
published by the \gls{iers} for UT1 and LOD to account for the effects of ocean 
tides and libration.

The function \texttt{iers2010::interp\_pole} can only interpolate and apply the 
correction for the tidal terms \(\Delta UT1_{ocean\text{ }tides}\), or 
\(\Delta LOD_{ocean\text{ }tides}\) but the \(\Delta UT1_{libration}\) and 
\(\Delta LOD_{libration}\) are missing (and probably could be added, see \cite{iers2010}, 
section 5.5.3.1).

\texttt{iers2010::ortho\_eop} can be used to compute diurnal and
semi-diurnal variations in UT1 or LOD caused by ocean tides, aka 
\(\Delta UT1_{ocean\text{ }tides}\) and \(\Delta LOD_{ocean\text{ }tides}\).

\texttt{iers2010::utlibr} can be used to compute \(\Delta UT1_{libration}\) and 
\(\Delta LOD_{libration}\).

\subsection{Coordinates of the \gls{cip} in the \gls{gcrs}}
\label{ssec:coordinates-of-the-cip-in-the-gcrs}
Following the IAU 2006 precession and IAU 2000A nutation models, the parameters 
\(X\) and \(Y\) in \ref{eq:tn36510} can be computed in the microarcsecond level, 
using the formulas (\cite{iers2010}):
\begin{equation}
  \label{eq:tn36516a}
  \begin{aligned}
  X &= \SI{-0.01661700}{\arcsecond} + \SI{2004.19189800}{\arcsecond} t - \SI{0.429782900}{\arcsecond} t^2 \\
  &- \SI{0.1986183400}{\arcsecond}t^3 + \SI{0.00000757800}{\arcsecond} t^4 + \SI{0.000005928500}{\arcsecond} t^5 \\
  &+ \sum_{i} \left[ (a_{s,0})_i \sin \theta + (a_{c,0})_i \cos \theta \right] \\ 
  &+ \sum_{i} \left[ (a_{s,1})_i t \sin \theta + (a_{c,1})_i t \cos \theta \right] \\ 
  &+ \sum_{i} \left[ (a_{s,2})_i t^2 \sin \theta + (a_{c,2})_i t^2 \cos \theta \right] \\ 
  &+ \cdots \\
  \end{aligned}
\end{equation}
and
\begin{equation}
  \label{eq:tn36516b}
  \begin{aligned}
  Y &= -\SI{0.00695100}{\arcsecond} - \SI{0.02589600}{\arcsecond} t - \SI{22.407274700}{\arcsecond} t^2 \\
  &+ \SI{0.0019005900}{\arcsecond} t^3 + \SI{0.00111252600}{\arcsecond} t^4 + \SI{0.000000135800}{\arcsecond} t^5 \\
  &+ \sum_{i} \left[ (b_{s,0})_i \sin \theta     + (b_{c,0})_i \cos \theta \right] \\ 
  &+ \sum_{i} \left[ (b_{s,1})_i t \sin \theta   + (b_{c,1})_i t \cos \theta \right] \\ 
  &+ \sum_{i} \left[ (b_{s,2})_i t^2 \sin \theta + (b_{c,2})_i t^2 \cos \theta \right] \\ 
  &+ \cdots \\
  \end{aligned}
\end{equation}

where \(t\) is expressed in Julian centuries TT (\ref{eq:tn3652}) and \(\theta\) 
is a function of the fundamental lunisolar and planetary arguments. Complete 
list of coefficients for \ref{eq:tn36516a} and \ref{eq:tn36516b} is provided 
by \gls{iers}. 

Details on the position of the \gls{cip} in the \gls{gcrs} and implementation 
details, can be found in \cite{CapitaineAndWallace2006} and 
\cite{Capitaineetal2003a}.
