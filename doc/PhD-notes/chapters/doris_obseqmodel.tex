\tikzstyle{every picture}+=[remember picture]
\everymath{\displaystyle}

\section{Introduction}
\label{sec:doris-introduction}

For the DORIS observatrion equation model, we use the formula described in 
\cite{lemoine-2016}, Eq. 17:
\begin{subequations} \label{eq:lem13}
    \begin{align}
        v_{measured} & = \frac{c}{f_{e_N}} (f_{e_N} - f_{r_T} -
          \frac{N_{DOP}}{\Delta\tau_r}) + \Delta u_{REL} + 
          \Delta v_{IONO} \label{eq:lem13a} \\
        v_{theo} &= \frac{\rho_2 - \rho_1}{\Delta\tau_r} 
          (1- \frac{U_e}{c^2} - \frac{{V_e}^2}{2 c^2}) + 
          \Delta v_{TROPO} - \frac{c(\frac{N_{DOP}}{\Delta\tau_r} + 
          f_{r_T})}{f_{e_N}} \frac{\Delta f_e}{f_{e_N}} \label{eq:lem13b}
    \end{align}
\end{subequations}

where $v_{measured}$ and $v_{theo}$ represent the relative velocity between  
the emitter and the receiver as measured and computed respectively.

\subsection{Nominal Receiver \& Emitter Frequencies}
\label{ssec:nominal-frequencies}
In the observation equation model \ref{eq:lem13} we distinguish between 
\emph{nominal} and \emph{true} receiver/emitter frequencies, to account for 
the fact that in ``real world'' these two are not actually equal.

\subsubsection{Emitter/Beacon Nominal Frequencies, $f_{e_N}$}
\label{sssec:beacon-nominal-frequencies}

RINEX file headers, contain values of the \emph{station frequency shift 
factor} $k$ for each of the beacons involved (\cite{DORISRNX3}, Sec. 6.16). 
These are used to compute the ``nominal'' frequencies of the beacon/emitter. 
(usually, this shift factor is just $0$, but it can be an integer 
$k \neq 0$). The frequencies are computed as (\cite{DORISRNX3}, Sec. 6.16):
\begin{equation}
  \begin{aligned}
    L_{2GHz}   &= 543 \cdot F_0 \left( \frac{3}{4} + \frac{87\cdot k}{5 \cdot 2^{26}} \right) \\
    L_{400MHz} &= 107 \cdot F_0 \left( \frac{3}{4} + \frac{87\cdot k}{5 \cdot 2^{26}} \right) 
    \label{eq:nominal-freq}
  \end{aligned}
\end{equation}
where $F_0 = 5e6 \text{ Hz}$ the USO frequency. These value, are the ones 
labelled as $f_{e_N}$ in \ref{eq:lem13}.

The \emph{true proper frequency} of the emitter $f_{e_T}$, can be computed 
from:
\begin{equation}
  f_{e_T} = f_{e_N} \cdot \left( 1 + \frac{\Delta f_e}{f_{e_N}} \right)
\end{equation}
but the quantity $\Delta f_e / f_{e_N}$ is not know a-priori and has to be 
estimated during the processing.

We estimate $\Delta f_e / f_{e_N}$ using a \colorbox{blue!20}{linear model using TAI}, 
aka $\frac{\Delta f_e}{f_{e_N}}\at{t=t_i} = \alpha + \beta \cdot \delta t_{TAI}$
with an a-priori value of 0 and no process noise. For the estimation, we need 
the partials of the observation equation \ref{eq-lem13b} w.r.t. the $\alpha$ and 
$\beta$ parameters, which are:
\begin{equation}
  \begin{aligned}
  \frac{\partial v_{theo}}{\partial \alpha} &= 
    \frac{c(\frac{N_{DOP}}{\Delta\tau_r} + f_{r_T})}{f_{e_N}} \\
  \frac{\partial v_{theo}}{\partial \beta} &= 
    \frac{c(\frac{N_{DOP}}{\Delta\tau_r} + f_{r_T})}{f_{e_N}} \cdot \delta t_{TAI}
  \end{aligned}
\end{equation}

\subsection{Receiver True Proper Frequency $f_{r_T}$}
\label{ssec:receiver-true-proper-frequency}
In equation \ref{eq:lem13}, $f_{r_T}$ is the \emph{true proper frequency of 
the receiver} with nominal value 
\begin{equation}
  f_{r_T} = f_{r_N} \cdot \left( 1 + \frac{\Delta f_r}{f_{r_N}} \right)
\end{equation}
where $f_{r_N}$ is the ``nominal'' frequency value.

In practice the value $\Delta f_r / f_{r_N}$, called the \emph{relative 
frequency offset} of the receiver, is given in the RINEX files for each epoch 
(under the observable tagged \texttt{F}). Note that these values are scaled to 
$10^{-11}$ (\cite{DORISRNX3}, Sec. 6.11), so that for a given epoch $t_i$, the 
true frequency is
\begin{equation}
  f_{r_T}\at{t=t_i} = f_{r_N} \cdot \left( 1 + F_{t_i} \cdot 10^{-11} \right)
  \label{eq:tpf-rec}
\end{equation}
where $F_{t_i}$ is the relative frequency offset value recovered from the RINEX 
file.

In our implementation, we use ``smoothed'' values of RINEX-provided 
$\Delta f_r / f_{r_N}$ estimates. In a first step/pass we use a linear model 
to filter all the values in the RINEX file (as suggested by \cite{lemoine-2016}, 
Sec. 2.5.4). We then use this linear model to compute relative frequency offset 
values when needed. 

\subsection{Beacon Coordinates}
\label{ssec:beacon_coordinates}
Coordinates for the DORIS Network sites, are extracted from the \texttt{dpod*} 
SINEX files (\cite{Moreaux2020}). To get the coordinates of the sites at the 
beacon reference point, we procced as follows:
\begin{enumerate}
  \item extrapolate coordinates (in the DPOD, ECEF reference frame) at the 
    requested epoch, $\bm{r}\sp{\prime}$
  \item get site eccentricity $\Delta h$ from the respective IDS-distributed 
    log file,
  \item compute the cartesian-to-topocentric rotation matrix $\bm{R}(\bm{r}\sp{\prime})$,
  \item add the topocentric eccentricity vector to get the antenna reference
    point coordinates for the beacon:
    \begin{equation}
      \bm{r} \equiv \bm{r}^{dpod}_{ARP} = 
        \bm{r}\sp{\prime} + \bm{R}^T \cdot 
          \begin{pmatrix} 0 & 0 & \Delta h \end{pmatrix}^T
      \label{eq:rg-to-rarp}
    \end{equation}
\end{enumerate}

\subsection{Coordinate \& Proper Time}
\label{ssec:coordinate-proper-time}
We use TAI as coordinate time; to transform RINEX observation time (given in 
proper time $\tau$) to coordinate time $t$, we use the \emph{receiver clock 
offset} values, extracted from the RINEX file (one value per observation block).

Hence, if an observation block is taged at proper time $\tau _i$ at the RINEX 
file, and the receiver clock offset for this block is $\Delta \tau _i$ (again from 
RINEX), then the coordinate time of the event in TAI is:
\begin{equation}
  t^{TAI}_i = \tau _i + \Delta \tau _i
\end{equation}

\subsection{Tropospheric Correction}
\label{ssec-tropospheric-correction}

\subsubsection{Mapping Functions}
We use the GPT3/VMF3 (\cite{Landskron2018}) to handle tropospheric refraction. 
Given the elevation angle $el$ and the beacon's ellipsoidal coordinates 
$\bm{r}=\begin{pmatrix} \lambda & \phi & h\end{pmatrix}$, we use the \texttt{gpt3} 
$\ang{5} \times \ang{5}$ grid to interpolate the $ah$ and $aw$ coefficients; 
we then compute the mapping function (with height correction) $mf_{wet}$ and 
$mf_{hydrostatic}$.

\subsubsection{Zenith Delays}
For the hydrostatic zenith delay $zd_{hydrostatic}$ we use the ``refined'' 
\emph{Saastamoinen} model (\cite{Davisetal85} and \cite{Saastamoinen72}). 
The corresponding value for the wet delay, $zd_{wet}$ is estimated during the 
analysis, \emph{per beacon and per pass}, using an initial value provided by 
\cite{Askneetal87}.

\subsubsection{Tropospheric Correction}
Using the mapping function and the zenith delay, we derive the tropospheric 
delay for an observation at $t=t_i$, given by:
\begin{equation}
  ztd [\si{\m}] = zd_{hydrostatic} \cdot mf_{hydrostatic} + zd_{wet}\at{t=t_i} \cdot mf_{wet}
  \label{eq:tropo-delay}
\end{equation}

The tropospheric correction term in \ref{eq:lem13}, is actually the ``time-differenced'' 
tropospheric delay between two measurements (the same ones used to derive the 
Doppler count), given in \si{\m \per \s}. That is:
\begin{equation}
  \begin{aligned}
    \Delta v_{TROPO} [\si{\m \per \s}] 
      &= \left( ztd\at{t=t_{i-1}} - ztd\at{t=t_{i}} \right) / \Delta \tau\\
      &= \left( \left[ zd_{h} \cdot mf_{h} + zd_{w}\at{t=t_{i-1}} \cdot mf_{w} \right]\at[\big]{t=t_i} - 
        \left[ zd_{h} \cdot mf_{h} + zd_{w}\at{t=t_{i-1}} \cdot mf_{w} \right]\at[\big]{t=t_{i-1}} \right) \ \Delta \tau
    \label{eq:dtrop-diff}
  \end{aligned}
\end{equation}

Note that in the above equation we use the same value $zd_{w}\at{t=t_{i-1}}$ 
for the wet part of the zenith delay, that is the best estimate prior to 
incorporating the (new) measurement at $t=t_i$.

The parameter $zd_{w}$ is estimated using no constraints and a simple white 
noise model (no process noise). Since it is an estimated parameter, we need 
to compute the (partial) derivative of the observation equation w.r.t to it, 
that is:
\begin{equation}
  \frac{\partial v_{theo}}{\partial zd_{w}} = \frac{mf_{w}\at[\big]{t=t_i} 
    - mf_{w}\at[\big]{t=t_{i-1}}}{\Delta \tau}
\end{equation}

\subsection{Ionospheric Correction}
\label{ssec:iono-correction}
The basic observation equation \ref{eq:lem13}, is formed for the \SI{2}{\GHz} 
carrier. For each measurement, we have to account for the ionospheric path 
delay, by computing a correction (in cycles) as (\cite{lemoine-2016}, Sec. 2.5.7):
\begin{equation}
  \delta_{ION} [\SI{2}{\GHz}\text{ cycles}] = 
    \frac{L_{\SI{2}{\GHz}} - \sqrt{\gamma} \cdot L_{\SI{400}{\MHz}}}{\gamma - 1}
  \label{eq:iono-delay-cycles}
\end{equation}

which is added to the \SI{2}{\GHz} measurement at time $t=t_i$ (obtained by the 
RINEX file). Thus, the corrected observation is:
\begin{equation}
  L_{\SI{2}{\GHz},IF} [\SI{2}{\GHz}\text{ cycles}] = 
    L_{\SI{2}{\GHz}} + \delta_{ION}
  \label{eq:l2if}
\end{equation}

Note that after applying \ref{eq:l2if}, we should refer to the ``Iono-Free'' 
geometrical endpoints of the observations (and not the \SI{2}{\GHz} endpoints). 
This means that we have to apply respective \emph{phase center} corrections 
both at the satellite and at the beacon.

When applied to \ref{eq:lem13}, we have to ``differenciate'' the ionospheric 
delay computed from \ref{eq:iono-delay-cycles}, affecting two observations 
(the same ones used to derive the Doppler count). Hence, the term 
$\Delta v_{IONO}$ appearing in \ref{eq:lem13} is:
\begin{equation}
  \Delta v_{IONO} [\si{\m \per \s}] = 
    \tikz[baseline]{
      \node[fill=blue!20,anchor=base](en1){$\frac{c}{f_{e_N}}$};
    }
    \cdot 
    \frac{\delta_{ION}\at[\big]{t=t_{i-1}} 
    - \delta_{ION}\at[\big]{t=t_i}}{\Delta \tau}
  \label{eq:dion-diff}
\end{equation}

\begin{itemize}
  \item is this correct? should this be $f_{e_N}$?
    \tikz\node[fill=blue!20,draw,circle] (in1) {};
  \item maybe use the ``true'' emitter frequency, from 
  $f_{e_T} = f_{e_N} \cdot ( 1 + {\frac{\Delta f_e}{f_{e_N}}}\at[\big]{t=t_i})$ 
  but then we will have to re-arrange \ref{eq:lem13} so that the term 
  $\Delta v_{IONO}$ appears in the $v_{theo}$ part
\end{itemize}

\begin{tikzpicture}[overlay]
  \path[->] (in1) edge [bend left] (en1);
  %\path[->] (n2) edge [bend right] (t2);
  %\path[->] (n3) edge [out=0, in=-90] (t3);
\end{tikzpicture}

\subsubsection{2GHz \& Iono-Free Phase Center}
\label{sssec:2ghz-ionofree-pco}
To get to the iono-free phase center from the beacon \gls{arp}, 
we use (\cite{lemoine-2016}, Sec. 2.5.7, Eq. 20):
\begin{equation}
  \bm{r}_{iono-free} = \bm{r}_{\SI{2}{\GHz}} + \frac{\bm{r}_{\SI{2}{\GHz}} 
    - \bm{r}_{\SI{400}{\MHz}}}{\gamma - 1}
  \label{eq:ionf-pco}
\end{equation}

where $\bm{r}_{\SI{2}{\GHz}}$ and $\bm{r}_{\SI{400}{\MHz}}$ are the 
eccentricities of the \SI{2}{\GHz} and the \SI{400}{\MHz} carriers respectively 
from the beacon antena phase center (given eg at \cite{DORISGSM}, Sec. 5.2.1).

Note that in \ref{eq:ionf-pco}, the eccentricity vector $\bm{r}_{iono-free}$ 
is in a topocentric RF. Hence, to compute the ECEF coordinates of the 
iono-free phase center, given that we have the ECEF coordinates of the beacon's 
\gls{arp} $\bm{r}_{arp}$ (see \ref{ssec:beacon_coordinates}), we use:
\begin{equation}
  \bm{r}^{ecef}_{iono-free} = \bm{r}_{arp} + \bm{R}^T \cdot \bm{r}_{iono-free}
  \label{eq:arp-to-if-pc}
\end{equation}
where $\bm{R}$ is the cartesian-to-topocentric rotation matrix, computed 
at $\bm{r}_{arp}$.

In accordance to the beacons, a similar geometric reduction must be applied 
at the satellite's end, to correct for the discrepancy between the \SI{2}{\GHz} 
and the \emph{iono-free} phase center:
\begin{equation}
  \bm{r}^{satf}_{iono-free} = \bm{r}^{satf}_{\SI{2}{\GHz}} + 
    \frac{\bm{r}^{satf}_{\SI{2}{\GHz}} - 
    \bm{r}^{satf}_{\SI{400}{\MHz}}}{\gamma - 1}
\end{equation}

where the superscript $satf$ denotes the \emph{satellite-fixed} body/reference 
frame. On-board satellite antenna phase center offset values can be found in 
\cite{DorisSatModels}.

\subsubsection{Relativistic Correction}
\label{sssec:relativistic-correction}
\ref{eq:lem13} contains a relativistic correction term, $\Delta u_{REL}$. This 
correction is plit into two parts, $\Delta u_{REL_c}$ the part containing the 
clock correction and $\Delta u_{REL_r}$, containing the effect of the travel 
path. Here, we are only considering the $\Delta u_{REL_c}$ part.

The correction is computed (for a given obserbation) from the formula:
\begin{equation}
  \Delta u_{REL_c} = \frac{1}{c} \left( U_r - U_e + \frac{V^2_r - V^2_e}{2} \right) \si{\m}
  \label{eq:lem14}
\end{equation}
For the receiver part, the respective quantities in \ref{eq:lem14} are given by:
\begin{equation}
  \begin{aligned}
    V^2_r &= \norm{\bm{v}_{ecef}}^2 \\
    U_r   &= \frac{\mu}{\norm{\bm{r}_{ecef}}} \cdot \left( 1 - 
      \left(\frac{\alpha}{\norm{\bm{r}_{ecef}}}\right) ^2 \cdot J_2 \cdot
        \frac{3 \cdot \sin ^2{\phi} -1}{2} \right)
  \end{aligned}
  \label{eq:potential-receiver}
\end{equation}
where $\bm{r}_{ecef}$ and $\bm{v}_{ecef}$ are the position and velocity of the 
satellite at the given instant, in the terrestrial RF, aka ITRF.

For the emitter, things are further simplified. We consider $V_e = 0$ and the 
potential is computed as 
\begin{equation}
  U_e = \frac{\mu}{\norm{\bm{r}_{ecef}}}
  \label{eq:potential-emitter}
\end{equation}
where $\bm{r}_{ecef}$ is the position of the beacon in ITRF.

In \ref{eq:lem13}, we need the ``differentiated'' relativistic corrections, thus 
\begin{equation}
  \label{eq:drel-diff}
  \begin{aligned}
    \Delta v_{REL} &= \frac{1}{c} \left( \left[ U_r - U_e + \frac{V^2_r - V^2_e}{2} \right]\at{t=t_i} - \left[ U_r - U_e + \frac{V^2_r - V^2_e}{2} \right]\at{t=t_{i-1}} \right) \\
      &= \frac{1}{c} \cdot \left( U_r\at{t_i} - U_r\at{t_{i-1}} +  \frac{V_r\at{t_i}-V_r\at{t_{i-1}}}{2} \right) \si{\m\per\s}
  \end{aligned}
\end{equation}

\subsubsection{RINEX Observation Flags}
\label{sssec:rnx-flags}
An observation is rendered as ``non-usable'', if any of the below holds true:
\begin{itemize}
  \item flag1 or flag2 of the L\SI{2}{\giga\hertz} carrier is set to \texttt{1}
  \item flag1 or flag2 of the L\SI{400}{\mega\hertz} carrier is set to \texttt{1}
  \item flag1 of the Power (\texttt{W}) observation on either the L\SI{2}{\giga\hertz} 
    or the L\SI{400}{\mega\hertz} carrier is set to \texttt{1} (restart mode)
\end{itemize}

\section{Processing SetUp}

\subsubsection{General SetUp}
\begin{itemize}
  \item Elevation cut-off angle \ang{10}
  \item Setup new satellite arc if previous observation (of the beacon) is more 
    than \SI{5}{\min} away
\end{itemize}

\subsubsection{Filter SetUp}
\begin{itemize}
  \item A-priori information for $\Delta f_e / f_{e_N}$: $0.0 \pm 1.0$
  \item A-priori information for drag coefficient $C_{d}$: $2.0 \pm .75$
\end{itemize}

\subsubsection{EOP Information}
EOP information is extracted/interpolated from IERS \texttt{C04} files.

\subsubsection{Earth Gravity}
Using gravity model \texttt{GOCO02s} (icgem format), up to degre/order 90.

%\subsubsection{Beacon Information}
%Reference coordinates extracted from \texttt{dpod2014\_053.snx}. Antenna 
%Reference Point (eccentricity) extracted from respective IDS log files, and 
%added to the extrapolated cartesian coordinates (see 
%\ref{ssec:beacon_coordinates}).

\subsubsection{Satellite Information}
Centre of Gravity (CoG) coordinates, in the SV-fixed RF, are extracted from the 
file \texttt{ja3mass.txt}. Phase centre coordinates (in the SV-fixed RF) for the 
two frequencies are extracted from the file \texttt{DORISSatelliteModels.pdf} and 
converted to \emph{iono-free} pco as:
\begin{equation}
  \bm{r}_{iono-free-pco} = \bm{r}_{2GHz,PCO} + (\bm{r}_{400MHz,PCO} - \bm{r}_{2GHz,PCO}) / (\gamma - 1)
\end{equation}
with all $\bm{r}$ vectors wrt the the SV-fixed RF.

\subsubsection{Atmospheric Drag}
To compute atmospheric density, we are using the \texttt{NRLMSISE00} model, 
using flux data from CelesTrack.

\subsubsection{Third Body Attraction/Perturbations}
Using JPL's \texttt{de405} ephemeris to compute Sun and Moon positions.

\subsubsection{Algorithm}
\begin{enumerate}
  \item Setup auxiliary infromation, data files, filter, etc ...
  \begin{itemize}
    \item Extrapolate coordinates for every beacon to RINEX reference epoch using \ref{eq:rg-to-rarp}, see \ref{ssec:beacon_coordinates}
  \end{itemize}

  \item Parse the RINEX file header

  \item For every observation block in the RINEX file:
    \begin{enumerate}\label{every-block}
      \item Transform proper time $\tau _i$ to TAI $t_i$ using the receiver-clock-offset 
        value reported in the RINEX
      \item Integrate orbit to $t_i$ (results are state vector $\bm{X}_{t_i}$ and state 
        transition matrix $\bm{\Phi}_{t_i}$ in GCRS)
      \item Perform the Kalman filter time-update, aka compute new $P$ matrix as 
        \begin{equation}
          P^- _{t_i} = \bm{\Phi} \cdot P^+ _{t_{i-1}} \cdot \bm{\Phi}^T
        \end{equation}
        where $\bm{\Phi} = \begin{pmatrix} \bm{\Phi}_{6\times 6} & \bm{0} \\ \bm{0} & \bm{I} \end{pmatrix}$
        \item Compute SV-coordinates of Iono-Free Phse Center, in ITRF aka $\bm{X}^{ecef}_{iono-free,t_i}$
        \item For every observation set in the current block:
          \begin{enumerate}\label{every-observation}
            \item Check observation flags reported in the RINEX file (see \ref{sssec:rnx-flags}). 
              If the observation is un-usable, then mark a discontinuity (for the beacon) and procceed 
              to next observation (goto \ref{every-observation}). If flags are ok, continue.
            \item Compute the Iono-Free Phase Center of the beacon using \ref{eq:arp-to-if-pc} (see \ref{sssec:2ghz-ionofree-pco}) 
              to get $\bm{r}^{ecef}_{iono-free}$
            \item Transform the satellite-to-beacon vector from ITRF to topocentric coordinates, 
              and compute azimouth $Az$, elevation $El$ and range $s$ (use $\bm{X}^{ecef}_{iono-free,t_i}$ and 
              $\bm{r}^{ecef}_{iono-free}$ so that the vector is wrt the Iono-Free 
              Phase center at both ends).
            \item If $El < \text{ minimum Elevation Angle}$ goto \ref{every-observation}; else procceed
            \item Check if this is the first observation of the beacon, or we have to start 
              a new arc, i.e. previous observation of the beacons happened more than 
              \SI{5}{\min} away
            \item Compute beacon's nominal frequency $f_{e_N}$, via \ref{eq:nominal-freq} 
            \item Compute ionospheric correction $\Delta I_{t_i}$ (in cycles) using \ref{eq:iono-delay-cycles}
            \item Compute tropospheric delay for the observation $\Delta T_{t_i}$, 
              using \ref{eq:tropo-delay}. The value $zd_{wet}$ is the last estimated 
              value, aka $zd_{wet}\at{t=t_{i-1}}$
            \item Compute relativistic correction using $\Delta R_{t_i}$ using \ref{eq:lem14}
              (use \ref{eq:potential-receiver} and \ref{eq:potential-emitter}; values 
              $\mu$, $\alpha$ and $J_2$ needed to compute \ref{eq:potential-receiver} 
              are extracted from the gravity model
            \item Check for discontinuity or start of new arc; if true, then store 
              observation and goto \ref{every-observation}. Else retrieve last 
              stored dataset (latest beacon observation)
            \item Compute the true proper frequency of the receiver, $f_{r_T}\at{t=t_i}$ 
              using \ref{eq:tpf-rec} (scaled with $10^6$ to get \si{\Hz})
            \item Compute the Doppler count $NDpod$, by just subtracting the current 
              \SI{2}{\GHz} carrier observation with the previous one, aka
              $NDop = L_{\SI{2}{\GHz}}\at{t=t_i} - L_{\SI{2}{\GHz}}\at{t=t_{i-1}}$
              as recorded in the RINEX file
            \item Differentiate corrections wrt to (\textbf{proper}) time:
              \begin{itemize}
                \item $\Delta v_{ION}$ from \ref{eq:dion-diff}
                \item $\Delta v_{TRO}$ from \ref{eq:dtrop-diff}
                \item $\Delta v_{REC}$ from \ref{eq:drel-diff}
              \end{itemize}
          \end{enumerate}
      \end{enumerate}
\end{enumerate}
