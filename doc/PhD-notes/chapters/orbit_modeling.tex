\chapter{Satellite Orbit modeling}
\label{ch:satellite-orbit-modeling}

\section{Quick Note: The State-Space Notation}
It is often helpfull to represent a linear differential equation in the the
\emph{state-space} form, that is a first-order matrix differential equation of
the form:
\begin{equation}
	\dot{\vec{x}} = \bm{F} \vec{x} + \bm{G} \vec{u} + \vec{w}
\end{equation}

where:
\begin{itemize}
	\item \(\vec{x}\) is the system \emph{state vector},
	\item \(\bm{F}\) is system's \emph{dynamic matrix},
	\item \(\vec{u}\) is a deterministic input, sometimes called a \emph{control vector}, and
	\item \(\vec{w}\) is a random forcing function, which is also known as \emph{process noise}
\end{itemize}

\section{Linearization of the Orbit Determination Process}
In the general orbit determination problem, both the dynamics and the measurements
involve significant nonlinear relationships. For the general case, the governing
relations involve the nonlinear expression:
\begin{subequations}
	\begin{align}
		\dot{\vec{x}} = F( \vec{x}, t ),
		 & \quad \vec{x}(t_k ) \equiv \vec{x}_k
		\label{eq:tapley421} \\
		\vec{y}_i = G( \vec{x}_i , t_i ) + {\epsilon}_i ,
		 & \quad  i=1,2,\ldots ,l
		\label{eq:tapley422}
	\end{align}
\end{subequations}

where \(\vec{x}_k\) is the unknown \(n\)-dimensional state vector at time \(t_k\) and
\(\vec{y}_i\) for \(i=1,2,\ldots ,l\) is a \(p\)-dimensional set of observations. The
\emph{best estimate} of the state vector \(\vec{x}_k\) will be denoted as \(\hat{\vec{x}}_k\).
In general, \(p<n\) and \( m = p \times l \gg n \). The formulation described by the
system \ref{eq:tapley421} and \ref{eq:tapley422}, is characterized by: (\cite{tapley})
\begin{enumerate}
	\item the inability to observe the state (\(\vec{x}_k\)) directly,
	\item nonlinear relations between the observations and the state,
	\item fewer observations at any time epoch \(i\) than there are state vector
	      components (\(p<n\)), and
	\item errors in the observations represented by \({\epsilon}_i\)
\end{enumerate}

If a reasonable reference trajectory \(\vec{x}*\) is available and if
\(\vec{x}\), the true trajectory, and the reference trajectory remain sufficiently
close throughout the time interval of interest, then the trajectory for the actual
motion can be expanded in a Taylor’s series about the reference trajectory at
each point in time. Eliminating higher order terms, the deviation in the state
from the reference trajectory can be described by a set of linear differential
equations. Corresponding linear relations can be derived between the observation
and the state deviations, thus transforming the nonlinear orbit determination problem
to a linear one.

If \(\vec{\delta x}\) is the \( n \times 1 \) state deviation vector and
\(\vec{\delta y}\) is the \(p \times 1\) observation deviation:
\begin{equation}
	\begin{aligned}
		\vec{\delta x} (t) & = \vec{x}(t) - \vec{x}^* (t) \\
		\vec{\delta y} (t) & = \vec{y}(t) - \vec{y}^*(t)
	\end{aligned}
\end{equation}

hence

\begin{equation}
	\dot{\vec{\delta x}} (t) = \dot{\vec{x}} (t) - \dot{\vec{x}}^* (t)
\end{equation}

Expanding \ref{eq:tapley421} and \ref{eq:tapley422} in a Taylor series about the
reference trajectory, leads to:
\begin{equation}
	\label{eq:tapley425}
	\begin{aligned}
		\dot{\vec{x}} (t) & = F (\vec{x}, t) \\
		                  & \approx F (\vec{x}^* , t)
		+ \left.\frac{\partial F(t)}{\partial \vec{x}(t)}\right|_{\vec{x}^*} \left( \vec{x}(t) - \vec{x}^* (t) \right)
		+ O_F \left( \vec{x}(t) - \vec{x}^* (t) \right) \\
		\vec{y}_i         & = G( \vec{x}_i , t_i ) + {\epsilon}_i =        \\ & \approx G( \vec{x}^*_i , t_i )
		+ \left.\frac{\partial G}{\partial \vec{x}}\right|_{\vec{x}^* , i} \left.\left( \vec{x}(t_i) - \vec{x}^* (t_i) \right)\right|_{i}
		+ O_G \left( \vec{x}(t_i) - \vec{x}^* (t_i) \right) + {\epsilon}_i \\
	\end{aligned}
\end{equation}

where \(\left.\frac{\partial}{\partial \vec{x}}\right|_{\vec{x}^*}\) indicates that
the partial derivative matrix is evaluated on the reference solution \(\vec{x}^* (t)\)
which is obtained by integrating \ref{eq:tapley421} with the initial conditions
\(\vec{x}^* (t_0)\). \(O_F\) and \(O_G\) indicate terms higher than the 1\textsuperscript{st}
order which are ignored. Noting that \(\dot{\vec{x}}^* = F(\vec{x}^* ,t)\) and
\(\vec{y}^*_i = G(\vec{x}^*_i , t_i )\), and letting
\begin{subequations}
	\begin{align}
		\delta \vec{x}(t) & = \vec{x}(t) - \vec{x}^*(t) \label{eq:tapley426ua} \\
		\delta \vec{x}_i  & = \vec{x}(t_i) - \vec{x}^*(t_i)\label{eq:tapley426ub} \\
		\delta \vec{y}_i  & = \vec{y}_i - G(\vec{x}^*_i , t_i ) \label{eq:tapley426uc}
	\end{align}
\end{subequations}

\ref{eq:tapley421} can be written as:
\begin{align}
	\label{eq:tapley426a}
	\delta \dot{\vec{x}}(t) & = A(t) \delta \vec{x}(t) \\
	\label{eq:tapley426b}
	\delta \vec{y}_i        & = \tilde{H}_i \delta \vec{x}_i + {\epsilon}_i \quad (i=1,\ldots,l)
\end{align}

where
\begin{equation}
	\label{eq:tapley426ah}
	A(t) = \left.\frac{\partial F(t)}{\partial \vec{x} (t)}\right|_{\vec{x}^*}
	\quad
	\tilde{H}_i = \left.\frac{\partial G}{\partial \vec{x}}\right|_{\vec{x}^* , i}
\end{equation}

\section{State Transition Matrix}
\ref{eq:tapley426a} represents a system of linear differential equations with time-dependent
coefficients; the general solution to the system, can be expressed as:
\begin{equation}
	\label{eq:tapley427}
	\bm{x}(t) = \Phi (t, t_k) \cdot \bm{x}_k \quad \text{with } \bm{x}_k = \bm{x}(t_k)
\end{equation}

The matrix \(\Phi (t_i, t_k) \) is called the \emph{state transition matrix} and has the
following properties:
\begin{itemize}
	\item \(\Phi (t_k, t_k) = I \)
	\item \(\Phi (t_i, t_k) = \Phi (t_i, t_j) \Phi (t_j, t_k) \)
	\item \(\Phi (t_i, t_k) = {\Phi}^{-1} (t_j, t_k) \)
\end{itemize}

Noting that \(\bm{x}_k\) is constant,
\begin{equation}
	\label{eq:tapley429}
	\bm{\dot{x}}(t) = \dot{\Phi} (t, t_k) \cdot \bm{x}_k
\end{equation}

Using \ref{eq:tapley426a} and \ref{eq:tapley427},
\begin{equation}
	\label{eq:tapley4210}
	\begin{aligned}
		\dot{\Phi} (t, t_k) \bm{x}_k & = A(t) \cdot \bm{x} (t) \\
		                             & = A(t) \cdot \Phi (t, t_k) \bm{x}_k \\
		\implies                     & \dot{\Phi} (t, t_k) = A(t) \Phi (t, t_k)
	\end{aligned}
\end{equation}

\ref{eq:tapley4210} represents a linear differential equation. For any practical orbit
determination application, the solution for \(\Phi (t, t_0)\)  will be obtained
via numerical integration, supplying a vector of derivative values for the differential
equation of the nominal state vector and computed values for \(\dot{\Phi} (t, t_0)\).

\section{Observations}
Combining \ref{eq:tapley427} and \ref{eq:tapley426b}, we get:
\begin{equation}
	\label{eq:tapley4237}
	\begin{aligned}
		\bm{y}_0     & = \tilde{H}_0 \Phi (t_0, t_k) \bm{x}_k + {\epsilon}_0 \\
		\bm{y}_1     & = \tilde{H}_1 \Phi (t_1, t_k) \bm{x}_k + {\epsilon}_1  \\
		             & \vdotswithin{=} \\
		\bm{y}_{l-1} & = \tilde{H}_{l-1} \Phi (t_{l-1}, t_k) \bm{x}_k + {\epsilon}_{l-1}
	\end{aligned}
\end{equation}

where the system contains \(m=p\times l\) observations and \(n\) unknown components
in the state vector. For convinience, we define the following notation:
\begin{equation}
	\bm{y} \equiv \begin{bmatrix} y_0 \\ y_1 \\ \ldots \\ y_{l-1} \end{bmatrix},
	\quad
	H \equiv \begin{bmatrix} \tilde{H}_0 \Phi (t_0, t_k) \\ \tilde{H}_1 \Phi (t_1, t_k) \\ \ldots \\ \tilde{H}_{l-1} \Phi (t_{l-1}, t_k) \end{bmatrix},
	\quad
	\bm{\epsilon} \equiv \begin{bmatrix} {\epsilon}_0 \\ {\epsilon}_1 \\ \ldots \\ {\epsilon}_{l-1} \end{bmatrix}
\end{equation}

so that we can now write \ref{eq:tapley4237} as:
\begin{equation}
	\label{eq:tapley4239}
	\bm{y} = H \bm{x} + \bm{\epsilon}
	\quad
	\bm{y} \in \mathbb{R} ^{m \times 1},
	\bm{x} \in \mathbb{R} ^{n \times 1},
	H \in \mathbb{R}^{m \times n},
\end{equation}

where \( m = p \times l \) is the total number of observations.

\section{Sequential Estimation Overview}
In the sequential estimation algorithm, observations are processed as soon as they
are received (in contrast to batch estimation), thus offering the advantage of
inverting a matrix of the same dimension as the observation vector. Hence, if the
observations are processed individually, only scalar divisions will be required to
obtain the estimate of \(\vec{x}_k\). The sequential estimation algorithm discussed
here, is often reffered to as the \emph{Kalman filter}, named after Rudolf E. K\'alm\'an
who was one of the primary developers of its theory.

An estimate \(\hat{\bm{x}}_j \) and a covariance matrix \(P_j\) can be
propagated forward to an epoch \(t_k\) by the relations (known as the
\emph{time update} equations)
\begin{subequations}
	\label{eq:tapley471}
	\begin{align}
		\bar{\bm{x}} _k & = \Phi (t_k, t_j) \hat{\bm{x}}_j
		\label{eq:tapley471a} \\
		\bar{P}_k       & = \Phi (t_k, t_j) P_j \Phi ^T (t_k, t_j)
		\label{eq:tapley471b}
	\end{align}
\end{subequations}


Assume that we have a new, additional obervation at epoch \(t_k\),
\begin{equation}
	\bm{y}_k = \tilde{H}_k \bm{x}_k + \bm{\epsilon} _k ,
	\quad E\left[\bm{\epsilon} _k \right] = 0,
	\quad E\left[\bm{\epsilon} _k \bm{\epsilon} ^T_j \right] = R_k \delta _{kj}
\end{equation}

(where \(\delta _{kj}\) is the \emph{Kronicker delta}). We wish to process
\(\bm{y} _k\) in order to determine \(\hat{\bm{x}} _k\). The best estimate of
\(\bm{x} _k\) is

\begin{equation}
	\label{eq:tapley473}
	\hat{\bm{x}} _k = \left( \tilde{H}^T_k R^{-1}_k \tilde{H}_k + \bar{P}^{-1}_k \right)^{-1} \left( \tilde{H}^T_k R^{-1}_k \bm{y}_k + \bar{P}^{-1}_k \bar{\bm{x}}_k \right)
\end{equation}

\ref{eq:tapley473} implies the inversion of the \(n \times n\) \emph{information matrix} 
(more on the information matrix at \ref{sec:information-matrix-and-information-filter}) \(\Lambda _k\),
\begin{equation}
	\label{eq:tapley474}
	\Lambda ^{-1}_k \equiv P_k =
	\left( \tilde{H}^T_k R^{-1}_k \tilde{H}_k + \bar{P}^{-1}_k \right)^{-1}
\end{equation}

From \ref{eq:tapley474}, it follows that:
\begin{equation}
	\label{eq:tapley475}
	P^{-1}_k
	= \tilde{H}^T_k R^{-1}_k \tilde{H}_k + \bar{P}^{-1}_k
\end{equation}

Premultiplying each side of \ref{eq:tapley475} by \(P_k\) and then postmultiplying
by \(\bar{P}_k\), leads to:
\begin{subequations}
	\begin{align}
		\bar{P}_k & =
		P_k \tilde{H}^T_k R^{-1}_k \tilde{H}_k \bar{P}_k + P_k
		\quad or \label{eq:tapley476} \\
		P_k       & =
		\bar{P}_k - P_k \tilde{H}^T_k R^{-1}_k \tilde{H}_k \bar{P}_k
		\label{eq:tapley477}
	\end{align}
\end{subequations}

Postmultiplying \ref{eq:tapley476} by \(\tilde{H}^T_k R^{-1}_k\)
\begin{equation}
	\label{eq:tapley478}
	\begin{aligned}
		\bar{P}_k \tilde{H}^T_k R^{-1}_k                           & =
		P_k \tilde{H}^T_k R^{-1}_k \tilde{H}_k \bar{P}_k \tilde{H}^T_k R^{-1}_k +
		P_k \tilde{H}^T_k R^{-1}_k                                 &                                     \\
		                                                           & = P_k \tilde{H}^T_k R^{-1}_k \left(
		\tilde{H}_k \bar{P}_k \tilde{H}^T_k R^{-1}_k + I \right)   &                                     \\
		                                                           & = P_k \tilde{H}^T_k R^{-1}_k \left(
		\tilde{H}_k \bar{P}_k \tilde{H}^T_k + R_k \right) R^{-1}_k &
	\end{aligned}
\end{equation}

We can postmultiply \ref{eq:tapley478} by \(R_k\) and then by
\(\left(\tilde{H}_k \bar{P}_k \tilde{H}^T_k + R_k \right) ^{-1} \) to solve for
the quantity \(P_k \tilde{H}^T_k R^{-1}_k\):
\begin{equation}
	\label{eq:tapley479}
	P_k \tilde{H}^T_k R^{-1}_k =
	\bar{P}_k \tilde{H}^T_k \left( \tilde{H}_k \bar{P}_k \tilde{H}^T_k + R_k \right) ^{-1}
\end{equation}

which relates the \emph{a-priori} covariance matrix \(\bar{P}_k\) to the
\emph{a-posteriori} covariance matrix \(P_k\).

Substituting \ref{eq:tapley479} into \ref{eq:tapley477}, yields:
\begin{equation}
	\label{eq:tapley4710}
	P_k =
	\bar{P}_k - \bar{P}_k \tilde{H}^T_k \left( \tilde{H}_k \bar{P}_k \tilde{H}^T_k + R_k \right) ^{-1} \tilde{H}_k \bar{P}_k
\end{equation}

\ref{eq:tapley4710} is an alternate way of computing the inverse in \ref{eq:tapley474},
but here the matrix to be inverted is of dimension \(p \times p\), that is the
same dimensions as the observation error covariance matrix. If the observations are
processed as scalars (i.e. one at a time), only a scalar division is required.

If we define the weighting matrix \(K_k\), usually called \emph{Kalman gain matrix},
as
%\begin{equation}
\begin{tcolorbox}[ams equation]
	\label{eq:tapley4711}
	K_k = \bar{P}_k \tilde{H}^T_k \left( \tilde{H}_k \bar{P}_k \tilde{H}^T_k + R_k \right) ^{-1}
\end{tcolorbox}
%\end{equation}

then \ref{eq:tapley4710} reads
%\begin{equation}
\begin{tcolorbox}[ams equation]
	\label{eq:tapley4712}
	P_k = \left( I - K_k \tilde{H}_k \right) \bar{P}_k
\end{tcolorbox}
%\end{equation}

Note also, that substituting \ref{eq:tapley479} into \ref{eq:tapley4711}, we get
\begin{equation}
	\label{eq:tapley4714}
	K_k = P_k \tilde{H}^T_k R^{-1}_k
\end{equation}

Substituting \ref{eq:tapley474} into \ref{eq:tapley473}, we get:
\begin{equation}
	\begin{aligned}
		\hat{\bm{x}}_k & = P_k
		\left( \tilde{H}^T_k R^{-1}_k \bm{y}_k + \bar{P}^{-1}_k \bar{\bm{x}}_k \right)                                \\
		               & = \underbrace{P_k \tilde{H}^T_k R^{-1}_k}_{K_k} \bm{y}_k + P_k \bar{P}^{-1}_k \bar{\bm{x}}_k \\
		               & = K_k \bm{y}_k + P_k \bar{P}^{-1}_k \bar{\bm{x}}_k                                           \\
		               & = K_k \bm{y}_k + \left( I - K_k \tilde{H}_k \right) \bar{P}_k \bar{P}^{-1}_k \bar{\bm{x}}_k
	\end{aligned}
\end{equation}

and finaly
%\begin{equation}
\begin{tcolorbox}[ams equation]
	\label{eq:tapley4716}
	\hat{\bm{x}}_k = \bar{\bm{x}}_k + K_k \left( \bm{y}_k - \tilde{H}_k \bar{\bm{x}}_k \right)
\end{tcolorbox}
%\end{equation}

\ref{eq:tapley4711}, \ref{eq:tapley4712}, \ref{eq:tapley4716} along with \ref{eq:tapley471} can be used in a recursive fashion to compute the estimate \(\hat{\bm{x}}_k\)
incorporating the observation \(\bm{y}_k\).

\subsection{Sequential Estimation Algorithm}
\label{ssec:sequential-estimation-algorithm}
Given the initial conditions \(\vec{X}^*_{k-1}\), \(\hat{\bm{x}}_{k-1}\) and \(P_{k-1}\),
the observation \(Y_k\) and the corresponding \(R_k\) at \(t=t_k\), the algorithm
for computing the estimate sequentially is summarized as:
\begin{enumerate}
	\item \label{en:kalman-wf-item1} Integrate reference trajectory (\(X^*\)) and
	      state transition matrix, from \(t_{k-1}\) to \(t_k\) (see \ref{eq:tapley421}, \ref{eq:tapley426ah} and \ref{eq:tapley4210})
	      \begin{subequations}
		      \begin{align}
			      \dot{X}^*               & = F( X^* , t )
			      \quad \text{with initial conditions } X^*_{k-1} \label{eq:tapley4717a} \\
			      A(t)                    & =
			      \left.\frac{\partial F(X,t)}{\partial X}\right|_{X=X^*}                \\
			      \dot{\Phi} (t, t_{k-1}) & =
			      A(t) \Phi (t,t_{k-1})
			      \quad \text{with initial conditions } \Phi(t_{k-1}, t_{k-1}) = I\label{eq:tapley4717b}
		      \end{align}
	      \end{subequations}
	      This step results in computation of \(X^*_{t_k}\) and \(\Phi (t_k, t_{k-1})\)

	\item \label{en:kalman-wf-time-update} Compute the time update (see \ref{eq:tapley471}):
	      \begin{subequations}
		      \begin{align}
			      \bar{\bm{x}}_k & = \Phi (t_k , t_{k-1}) \hat{\bm{x}}_{k-1}              \\
			      \bar{P}_k      & = \Phi (t_k , t_{k-1}) P_{k-1} \Phi ^T (t_k , t_{k-1})
		      \end{align}
	      \end{subequations}

	\item Compute observation deviation, observation state matrix and gain matrix (see \ref{eq:tapley426uc}, \ref{eq:tapley426ah} and \ref{eq:tapley4711})
	      \begin{subequations}
		      \begin{align}
			      \bm{y}_k    & = Y_k - G(X^*_k , t_k )                                         \\
			      \tilde{H}_k & = \left.\frac{\partial G(X , t_k )}{\partial X} \right|_{X=X^*} \\
			      K_k         & =
			      \bar{P}_k \tilde{H}^T_k
			      \left( \tilde{H}_k \bar{P}_k \tilde{H}^T_k + R_k \right) ^{-1}
		      \end{align}
	      \end{subequations}

	\item Compute the \emph{measurement update} (see \ref{eq:tapley4712} and \ref{eq:tapley4716})
	      \begin{subequations}
		      \begin{align}
			      \hat{\bm{x}}_k & = \bar{\bm{x}}_k + K_k \left( \bm{y}_k - \tilde{H}_k \bar{\bm{x}}_k \right) \\
			      P_k            & = \left( I - K_k \tilde{H}_k \right) \bar{P}_k
		      \end{align}
	      \end{subequations}

	\item Replace \(k\) with \(k+1\); \(X^*(t_k)\) now becomes \(X^*(t_{k-1})\).
	      Return to \ref{en:kalman-wf-item1}.

\end{enumerate}

The estimate of the state of the nonlinear system at \(t_k\) is given by
\(\hat{X}_k = X^*_k + \hat{\bm{x}}_k\)

Note that if there is an observation at \(t_0\), a time update (\ref{en:kalman-wf-time-update}) is not performed but a measurement update is performed.

Also, note that the differential equations for the state transition matrix
are reinitialized at each observation epoch. Therefore, the state transition matrix
is reinitialized at each observation epoch. If there is more than one observation
at each epoch and we are processing them as scalars, we would set \(\Phi (t_i , t_i ) = I\)
after processing the first observation at each epoch; \(P\) and \(\hat{\bm{x}}\) are not
time updated until we move to the next observation epoch.

\subsection{Shortcomings and Considerations}
One disadvantage of the sequential algorithm lies in the fact that if the true state
and the reference state are not close together then the linearization assumption
leading to \ref{eq:tapley426} may not be valid and the estimation process may diverge.
This problem can be adressed via the \emph{extended sequential filter algorithm}, see
\cite{tapley}.

A second unfavorable characteristic of the sequential estimation algorithm is
that the state estimation error covariance matrix may approach zero as the number
of observations becomes large. The trace of the state estimation error covariance
matrix grows between observations and is reduced by the amount \(trace(K\tilde{H}\bar{P})\)
after each observation. Hence, the magnitude of the covariance matrix elements
will decrease depending on the density, information content, and accuracy of the
observations.

Examination of the estimation algorithm shows that as \(P_k \to 0\), the gain
approaches zero, and the estimation procedure will become insensitive to the
observations. Consequently, the estimate will diverge due to either errors introduced
in the linearization procedure, computational errors, or errors due to an incomplete
mathematical model. To overcome this problem, process noise often is added to
the state propagation equations.

In addition to these two problems, the Kalman filter may diverge because of
numerical difficulties associated with the covariance measurement update, given by
\ref{eq:tapley4712}. The covariance matrix may lose its properties of symmetry and
become nonpositive definite when the computations are carried out with the finite
digit arithmetic of the computer. In particular, this equation can fail to yield a
symmetric positive definite result when a large a priori covariance is reduced by
the incorporation of very accurate observation data (\cite{tapley}). The most common
solution to numerical problems with the covariance update is to use a square root
formulation to update the covariance matrix (\ref{sec:square-root-filtering}).

\subsection{The Extended Sequential Estimation Algorithm}
To minimize the effects of errors due to the neglect of higher order terms in the
linearization procedure leading \ref{eq:tapley426}, the extended form of the sequential
estimation algorithm is sometimes used. This algorithm is often referred to as the
\emph{Extended Kalman Filter} (EKF). The primary difference between the sequential
and the extended sequential algorithm is that the reference trajectory for the ex-
tended sequential algorithm is updated after each observation to reflect the best
estimate of the true trajectory. For example, after processing the \(k^{th}\) observation,
the best estimate of the state vector at \(t_k\) is used to provide new initial
conditions for the reference trajectory,
\begin{equation}
	X^*_{k,new} = \hat{X}_k = X^*_k + \hat{\bm{x}}_k
\end{equation}

The flowchart for the \emph{Extended Sequential Estimation Aalgorithm} is given below,
in contrast to the one presented in \ref{ssec:sequential-estimation-algorithm}.

Given the initial conditions \(\vec{X}^*_{k-1}\), \(\hat{\bm{x}}_{k-1}\) and \(P_{k-1}\),
the observation \(Y_k\) and the corresponding \(R_k\) at \(t=t_k\), the algorithm
for computing the estimate sequentially is summarized as:
\begin{enumerate}
	\item \label{en:kalman-wf-item1} Integrate reference trajectory (\(X^*\)) and
	      state transition matrix, from \(t_{k-1}\) to \(t_k\) (see \ref{eq:tapley421}, \ref{eq:tapley4210})
	      \begin{subequations}
		      \begin{align}
			      \dot{X}^*               & = F( X^* , t )
			      \quad \text{with initial conditions } X^*_{k-1} \label{eq:tapley4717a} \\
			      A(t)                    & =
			      \left.\frac{\partial F(X,t)}{\partial X}\right|_{X=X^*}                \\
			      \dot{\Phi} (t, t_{k-1}) & =
			      A(t) \Phi (t,t_{k-1})
			      \quad \text{with initial conditions } \Phi(t_{k-1}, t_{k-1}) = I\label{eq:tapley4717b}
		      \end{align}
	      \end{subequations}
	      This step results in computation of \(X^*_{t_k}\) and \(\Phi (t_k, t_{k-1})\)

	\item \label{en:kalman-wf-time-update} Compute the time update (see \ref{eq:tapley471b}); \textcolor{red}{in contrast to the sequential estimation filter, the extended version does not use \ref{eq:tapley471a} at this step}
	      \begin{subequations}
		      \begin{gather}
			      \hcancel[red]{\bar{\bm{x}}_k = \Phi (t_k , t_{k-1}) \hat{\bm{x}}_{k-1}} \\
			      \bar{P}_k = \Phi (t_k , t_{k-1}) P_{k-1} \Phi ^T (t_k , t_{k-1})
		      \end{gather}
	      \end{subequations}

	\item Compute observation deviation, observation state matrix and gain matrix (see \ref{eq:tapley426uc}, and \ref{eq:tapley426ah})
	      \begin{subequations}
		      \begin{align}
			      \bm{y}_k    & = Y_k - G(X^*_k , t_k )                       \\
			      \tilde{H}_k & = \frac{\partial G(X^*_k , t_k )}{\partial X} \\
			      K_k         & =
			      \bar{P}_k \tilde{H}^T_k
			      \left( \tilde{H}_k \bar{P}_k \tilde{H}^T_k + R_k \right) ^{-1}
		      \end{align}
	      \end{subequations}

	\item Compute the measurement \textcolor{red}{and reference orbit} update (see \ref{eq:tapley4711}, \ref{eq:tapley4712}, \ref{eq:tapley4716})
	      \begin{subequations}
		      \begin{gather}
			      \hcancel{\hat{\bm{x}}_k = \bar{\bm{x}}_k + K_k \left( \bm{y}_k - \tilde{H}_k \bar{\bm{x}}_k \right)} \\
			      \textcolor{red}{\hat{\bm{x}} = K_k \bm{y}_k} \\
			      \textcolor{red}{X^*_k = X^*_k + \hat{\bm{x}}} \\
			      P_k = \left( I - K_k \tilde{H}_k \right) \bar{P}_k
		      \end{gather}
	      \end{subequations}

	\item Replace \(k\) with \(k+1\); \(X^*(t_k)\) now becomes \(X^*(t_{k-1})\).
	      Return to \ref{en:kalman-wf-item1}.

\end{enumerate}

The estimate of the state of the nonlinear system at \(t_k\) is given by
\(\hat{X}_k = X^*_k + \hat{\bm{x}}_k\)

\subsection{The Prediction Residual}
It is of interest to examine the variance of the predicted residuals, which are
sometimes referred to as the \emph{innovation}, or \emph{new information}, which
comes from each measurement. The predicted residual, or innovation, is the observation
residual based on the a-priori or predicted state, \(\bar{\bm{x}}\), at the observation
time, \(t_k\), and is defined as
\begin{equation}
	\label{eq:tapley4733}
	\beta _k = \bm{y}_k - \tilde{H}_k \bar{\bm{x}}_k
\end{equation}

with
\begin{equation}
	\begin{aligned}
		\bar{\bm{x}}_k & = \bm{x}_k + \eta _k                 \\
		\bm{y}_k       & = \tilde{H}_k \bm{x}_k + \epsilon _k
	\end{aligned}
\end{equation}

where \(\bm{x}_k\) is the true value of the state deviation vector and \(\bm{\eta}_k\)
is the error in \(\bar{\bm{x}}\). Also
\begin{equation}
	E \left[ \bm{\eta}_k \right] = 0, \quad
	E \left[ \bm{\eta}_k , \bm{\eta}^T_k \right] = \bar{P}_k
\end{equation}

and
\begin{equation}
	\begin{aligned}
		E \left[ \bm{\epsilon}_k \right]                     & = 0   \\
		E \left[ \bm{\epsilon}_k , \bm{\epsilon}^T_k \right] & = R_k \\
		E \left[ \bm{\epsilon}_k , \bm{\eta}^T_k \right]     & = 0
	\end{aligned}
\end{equation}

From these conditions it follows that \(\beta _k\) has mean
\begin{equation}
	\begin{aligned}
		E \left[ \beta _k \right] \equiv \bar{\beta}_k & = E \left[
		\tilde{H}_k \bm{x}_k + \epsilon _k - \tilde{H}_k \bar{\bm{x}}_k \right]                                                                \\
		                                               & = E \left[ \tilde{H}_k \left( \bm{x}_k - \bar{\bm{x}}_k \right) + \epsilon _k \right] \\
		                                               & = E \left[ \epsilon _k - \tilde{H}_k \bm{\eta}_k \right]                              \\
		                                               & = 0
	\end{aligned}
\end{equation}

and variance-covariance
\begin{equation}
	\begin{aligned}
		P_{\beta _k} & =
		E \left[ \left( \beta _k - \bar{\beta}_k \right) \left( \beta _k - \bar{\beta}_k \right)^T \right] \\
		             & = E \left[ \beta _k {\beta}^T_k \right]                                             \\
		             & = E \left[ \left( \bm{y}_k - \tilde{H}_k \bar{\bm{x}}_k \right)
		\left( \bm{y}_k - \tilde{H}_k \bar{\bm{x}}_k \right)^T \right]                                     \\
		             & = E \left[ \left( \bm{\epsilon}_k - \tilde{H}_k \bm{\eta}_k \right)
		\left( \bm{\epsilon}_k - \tilde{H}_k \bm{\eta}_k \right)^T \right]                                 \\
		             & = R_k + \tilde{H}_k \bar{P}_k \tilde{H}^T_k
	\end{aligned}
\end{equation}

Using formula \ref{eq:tapley:4711}, we can write the gain matrix \(K_k\) in terms
of the residual variance-covariance
\begin{equation}
	\begin{aligned}
		K_k & =
		\bar{P}_k \tilde{H}^T_k \left( \tilde{H}_k \bar{P}_k \tilde{H}^T_k + R_k \right) ^{-1} \\
		    & =
		\bar{P}_k \tilde{H}^T_k P_{\beta _k}^{-1}
	\end{aligned}
\end{equation}

Hence, for a large prediction residual variance-covariance, the Kalman gain matrix
will be small and the observation will have little influence on the estimate of the
state. Also, large values of the prediction residual relative to the prediction residual
standard deviation, may be an indication of bad tracking data and hence may be used to
edit data from the solution.

\section{State Noise}
In addition to the effects of the nonlinearities, the effects of errors in the dynamical
model can lead to divergence in the estimate. As pointed out previously, for a sufficiently large
number of observations the elements of the covariance matrix \(P_k\) will asymptotically
approach zero and the estimation algorithm will be insensitive to any further
observations. This condition can lead to filter divergence. One approach to preventing
this divergence is to recognize that the linearized equations for propagating the
estimate of the state are in error and to compensate for this by assuming
that the error in the linearized dynamics can be approximated by process noise.

The state dynamics of a linear system under the influence of process noise are
described by:
\begin{equation}
	\label{eq:tapley491}
	\dot{\bm{x}} (t) = A(t) \bm{x} (t) + B (t) \bm{u} (t)
\end{equation}

Where the vector \(\bm{u}\), called the \emph{state} or \emph{process noise} is of
dimension \(m \times 1 \) and the matrix \(B\) is \(n \times m \).
The functional form of \(\bm{u}\) can include a number of processes, including constant,
piecewise constant, correlated, or white noise.

If we assume a white process noise:
\begin{equation}
	\label{eq:tapley492}
	\begin{aligned}
		E \left[ \bm{u} (t) \right]                 & = 0                    \\
		E \left[ \bm{u} (t) \bm{u}^T (\tau) \right] & = Q(t) \delta (t-\tau)
	\end{aligned}
\end{equation}
where \(\delta (t-\tau)\) is the \emph{Dirac Delta} and \(Q\) is called the
\emph{process noise covariance matrix}. The algorithm that results from the assumption
that \( \bm{u} (t) \) is white noise with known covariance is known as \gls{snc}.
The use of more sophisticated models such as the process to compensate for state
and/or measurement model errors generally is referred to as \gls{dmc}.

The solution of \ref{eq:tapley491}, is given by (for a detailed description, see \cite{tapley})
\begin{equation}
	\label{eq:tapley4914}
	\bm{x} (t) = \Phi(t, t_0) \bm{x}_0 +
	\int_{t_0}^{t} \Phi(t, \tau ) B (\tau ) \bm{u} (\tau ) \, d\tau
\end{equation}
which is the general solution for the inhomogeneous \ref{eq:tapley491} and
indicates how the true state propagates under the influence of process noise.

If the mean of the process noise is zero, that is \(E \left[ \bm{u} (t) \right] = 0 \),
then the equation for propagating the state estimate is the same as without process noise
(\cite{tapley})
\begin{equation}
	\label{eq:tapley4919}
	\bar{\bm{x}} (t) = \Phi (t , t_{k-1} ) \hat{\bm{x}}_{k-1}
\end{equation}

One could derive a solution for the case where the mean is nonzero. In the case
where \(E \left[ \bm{u} (t) \right] = \bar{\bm{u}} \),
the solution would be obtained by applying the expectation operator to \ref{eq:tapley492}
to yield
\begin{equation}
	\label{eq:tapley4920}
	\bar{\bm{x}} (t) = \Phi (t , t_{k-1} ) \hat{\bm{x}}_{k-1} + \Gamma (t , t_{k-1} ) \bar{\bm{u}}
\end{equation}
where \(\Gamma (t , t_{k-1} )\) is given by \ref{eq:tapley4947}.

For the propagation of the estimation error covariance matrix, \cite{tapley} derives the
differential equation
\begin{equation}
	\label{eq:tapley4935}
	\dot{\bar{P}} (t) = A(t) \bar{P}(t) + \bar{P}(t) A^T (t) + B(t) q9t) B^T(t)
\end{equation}
which is a \(n \times n \) matrix differential equation whose solution may be obtained
by integrating with the initial conditions \(\bar{P} (t_k) = P_k\); that is, the
measurement update of the estimation error covariance matrix at \(t_k\).

\ref{eq:tapley4935} also can be expressed in integral form by using the method
of variation of parameters. According to \cite{tapley},
\begin{equation}
	\label{eq:tapley4944}
	\begin{split}
		\bar{P}(t) & = \Phi (t , t_{k-1} ) P_{k-1} \Phi ^T (t , t_{k-1} ) \\
		&+  \int_{t_{k-1}}^{t} \Phi(t, \tau ) B (\tau ) Q (\tau ) B ^T (\tau ) \Phi ^T (t, \tau ) \, d\tau
	\end{split}
\end{equation}

\ref{eq:tapley4919} and \ref{eq:tapley4944} are the equations for propagating the estimate
of the state and the covariance for a \emph{continuous system}. Since the orbit determination
problem generally consists of a continuous system (the trajectory) subjected
to discrete observations, it is convenient to use \ref{eq:tapley4919} to propagate the state
estimate and to discretize \ref{eq:tapley4944}. This can be accomplished by replacing
\(t\) with \(t_{k+1}\) and assuming that \(\bm{u}(\tau )\) is a \emph{white random sequence}
rather than a process. Thus, \(\bm{u}(t)\) is considered to be a piecewise constant function
with covariance
\begin{equation}
	\label{eq:tapley4945}
	E \left[ \bm{u}(t_i) \bm{u}^T(t_j) \right] = Q_i \delta _{ij} ,
	\quad \delta _{ij} = \begin{cases}
		1 & \quad i = j    \\
		0 & \quad i \neq j \\
	\end{cases}
\end{equation}

where the Dirac delta function has been replaced by its analog for the discrete
case, the Kroneker delta function. In the discrete case, \ref{eq:tapley4919} becomes
\begin{equation}
	\bm{x} (t_{k+1}) \equiv \bm{x}_{k+1} = \Phi (t_{k+1}, t_k ) \bm{x}_k + \Gamma (t_{k+1}, t_k ) \bm{u}_k
\end{equation}

where
\begin{equation}
	\label{eq:tapley4947}
	\Gamma (t_{k+1}, t_k ) = \int_{t_k}^{t_{k+1}} \Phi (t_{k+1}, \tau ) B (\tau ) \, d\tau
\end{equation}
\(\Gamma (t_{k+1}, t_k )\) is referred to as the \emph{process noise transition matrix}
and \ref{eq:tapley4974} is an \(n \times m\) quadrature since \(\Phi (t_{k+1}, \tau )\)
and \(B (\tau )\) are known functions.

\ref{eq:tapley4944} in the discrete case becomes (\cite{tapley})
\begin{equation}
	\label{eq:tapley4950}
	\bar{P}(t) = \Phi (t_{k+1} , t_k ) P_k \Phi ^T (t_{k+1} , t_k )
	+ \Gamma (t_{k+1}, t_k ) Q_k \Gamma ^T (t_{k+1}, t_k )
\end{equation}

Note that the estimation error covariance matrix \(\bar{P}_{k+1}\) can be obtained
either by integrating the differential equation \ref{eq:tapley4935}, or by using the
state and process noise transition matrices as indicated \ref{eq:tapley4950}. Comparing
the two:
\begin{itemize}
	\item Since \(P(t)\) is symmetric, only \( n (n+1) / 2 \) of the \( n \times n\)
	      system of equations represented in \ref{eq:tapley4935} must be integrated. However,
	      the \( n (n+1) / 2 \) equations are coupled and must be integrated as a single,
	      first order system of dimension \( n (n+1) / 2 \).
	\item The \(n \times n\) system represented by \ref{eq:tapley4950} can be separated
	      into an \( n \times n\) system of differential equations for \(Phi\) and an
	      \( n \times m\) quadrature for \(\Gamma\). Furthermore, the \(n \times n\) system
	      of equations represented by the solution for \(\Phi (t_{k+1} , t_k )\) can be integrated
	      as a sequence of \(n \times 1\) column vectors.
\end{itemize}

The comparison between the two methods indicates that integration of fewer
equations is required to obtain the solution for \(P(t)\) with \ref{eq:tapley4935}.
However, the integration of these equations may be more difficult than the integration
associated with the larger system represented by \ref{eq:tapley4950} since they are 
coupled.

The equations for determining \(\hat{\bm{x}}\) using the sequential processing algorithm are 
unchanged whenever a zero-mean process noise is included. However, as has been 
shown, the equations that propagate the estimation error covariance matrix do change; 
that is \ref{eq:tapley471b} is replaced by \ref{eq:tapley4950}.

The advantage of using the process noise compensated sequential estimation algorithm 
lies in the fact that the asymptotic value of \(\bar{P}(t)\) will approach a nonzero 
value determined by the magnitude of \(Q(t)\). That is, for certain values of \(Q(t)\), 
the increase in the state error covariance matrix \(\bar{P}(t)\) during the interval between 
observations will balance the decrease in the covariance matrix that occurs at the 
observation point. In this situation, the estimation procedure will always be sensitive 
to new observations.

The question of how to choose the process noise covariance matrix, \(Q(t)\), is
complex. In practice, it is often chosen as a simple diagonal matrix and its ele-
ments are determined by trial and error. Also, The Gauss-Markov (see \ref{sec:gauss-markov})
process is used as a 
process noise model; it is computationally well suited for describing unmodeled forces
since it obeys Gaussian probability laws and is exponentially correlated in time.

\section{Information Matrix and Information Filter}
\label{sec:information-matrix-and-information-filter}
Up to now, in the sequential estimation algorithm, we have used to the covariance filter 
\(P\). If we define the \emph{information matrix} as \(\Lambda \equiv P^{-1}\), we can 
derive the \emph{information filter} which offers some numerical properties with better 
characteristics than the covariance filter.

Using the information matrix, we can rewrite \ref{eq:tapley473} as
\begin{equation}
  \label{eq:tapley4101}
  \begin{aligned}
    \hat{\bm{x}} _k &= \left( \tilde{H}^T_k R^{-1}_k \tilde{H}_k 
      + \bar{\Lambda}_k \right)^{-1} \left( \tilde{H}^T_k R^{-1}_k \bm{y}_k + 
        \bar{\Lambda}_k \bar{\bm{x}}_k \right) \\
    \left( \tilde{H}^T_k R^{-1}_k \tilde{H}_k + \bar{\Lambda}_k \right) \hat{\bm{x}} _k &= 
      \tilde{H}^T_k R^{-1}_k \bm{y}_k + \bar{\Lambda}_k \bar{\bm{x}}_k \\
    \Lambda _k \hat{\bm{x}} _k &= 
      \tilde{H}^T_k R^{-1}_k \bm{y}_k + \bar{\Lambda}_k \bar{\bm{x}}_k
  \end{aligned}
\end{equation}

where we have made use of \ref{eq:tapley474}.

According to \cite{tapley} we can derive the formula:
\begin{equation}
  \label{eq:tapley4104}
  \begin{split}
    \bar{\Lambda}_{k+1} = \bar{P}^{-1}_{k+1} &= 
      M(t_{k+1}) - M(t_{k+1}) \Gamma (t_{k+1}, t_k ) \\
    & \cdot \left( \Gamma ^T (t_{k+1}, t_k ) M(t_{k+1}) \Gamma (t_{k+1}, t_k ) + Q^{-1}_k \right) ^{-1} \\
    & \cdot \Gamma ^T (t_{k+1}, t_k ) M(t_{k+1})
  \end{split}
\end{equation}

where
\begin{equation}
  \label{eq:tapley4105}
  M(t_{k+1}) = A^{-1} = \Phi ^T (t_k, t_{k+1} ) P^{-1}_k \Phi (t_k, t_{k+1} )
\end{equation}

In the case where we have no process noise, \ref{eq:tapley4104} reads
\begin{equation}
  \bar{\Lambda}_{k+1} = \bar{P}^{-1}_{k+1} = M(t_{k+1})
\end{equation}

Further defining
\begin{equation}
  \label{eq:tapley4106}
  \begin{split}
  L_{k+1} & \equiv M(t_{k+1}) \Gamma (t_{k+1}, t_k ) \\
          & \cdot \left( \Gamma ^T (t_{k+1}, t_k ) M(t_{k+1}) \Gamma (t_{k+1}, t_k ) + Q^{-1}_k \right) ^{-1}
  \end{split}
\end{equation}

\ref{eq:tapley4104} becomes
\begin{equation}
  \label{eq:tapley4107}
  \bar{\Lambda}_{k+1} =  M(t_{k+1}) - L_{k+1} \Gamma ^T (t_{k+1}, t_k ) M(t_{k+1})
\end{equation}

\section{The Gauss-Markov Process}
\label{sec:gauss-markov}
A first order \emph{Gauss-Markov} process is often used for dynamic model compensation 
in orbit determination problems to account for unmodeled or inaccurately
modeled accelerations acting on a spacecraft. A Gauss-Markov process is one that
obeys a Gaussian probability law and displays the Markov property. The Markov
property means that the probability density function at \(t_n\) given its past history at
\(t_{n-1} , t_{n-2} , \ldots \) is equal to its probability density function at 
\(t_n\) given its value at \(t_{n-1}\).

A Gauss-Markov process obeys a differential equation (often referred to as a 
\emph{Langevin equation}) of the form
\begin{equation}
  \label{eq:tapley4951}
  \dot{\eta} (t) = - \beta \eta (t) + u (t) 
\end{equation}

with \(\beta = \frac{1}{\tau}\), where \(u(t)\) is white Gaussian noise with
\begin{equation}
  \label{eq:tapley4952}
    E \left[ u \right] = 0, \quad E \left[ u(t) u(\tau ) \right] = \sigma ^2 \delta (t - \tau )
\end{equation}
 where \(\tau\) is the time constant or correlation time. 

\section{Square Root Filtering}
\label{sec:square-root-filtering}
Sequential estimation algorithms are subject to the filter divergence phenomenon, during
which the estimate of the state can depart in an unbounded manner from the true value
of the state. There are two fundamental reasons for filter divergence (\cite{tapley}):
\begin{itemize}
	\item due to inaccuracies in the mathematical model used to describe the dynamic
	      process or in the model used to relate the observations to the state, and
	\item the state error covariance matrix during measurement update can become nonpositive
	      definite (a situation that is a theoretical impossibility) due to floating point
	      arithmetic when computing the update of the state error covariance matrix at the
	      point where an observation is incorporated \footnote{When the eigenvalues have a wide spread, the error
		      introduced in the computational process can destroy the symmetry and positive
		      definite character of the covariance matrix and filte r divergence may occur, see \cite{tapley}.}.
\end{itemize}

The latter point is addressed in modifications of the computational algorithm called
\emph{square root covariance filters}, in which the state error covariance matrix is
replaced by its square root. The state error covariance matrix is obtained by
multiplying the square root matrix by its transpose and will always be symmetric
and positive semidefinite.

If we define
\begin{equation}
	\label{eq:tapley571}
	P = W W^T
\end{equation}

where \(W\) is the state error covariance matrix square root, and use \ref{eq:tapley571} to
compute the \(P\) matrix, this can never be nonpositive definite even in the presence
of round-off or truncation errors. Furthermore, since \(P\) is symmetric and positive
definite, there will exist an orthogonal matrix \(M\) such that:
\begin{equation}
	\label{eq:tapley572}
	P^* = M^T P M
\end{equation}

where \(P^*\) is a diagonal matrix whose elements are the eigenvalues of \(P\) and
\(M\) is the corresponding matrix of eigenvectors. Define \(W^*\) as the matrix
whose diagonal elements are equal to the square root of the diagonal elements of \(P^*\)
\begin{equation}
	W^*_{ii} = \sqrt P^*_{ii} \quad i=1,\ldots ,n
\end{equation}
where \(P^*_{ii} > 0\), then
\begin{equation}
	W^* W^{*T} = P^* = M^T P M = M^T W W^T M
\end{equation}

Thus, \(W^* = M^T W \) and since \(M\) is an orthogonal matrix, it follows that
\begin{equation}
	\label{eq:tapley574}
	W = M W^*
\end{equation}

The numerical conditioning of \(W\) is generally much better than that of \(P\) (see
e.g. \cite{lawson1995}, \cite{tapley}).
