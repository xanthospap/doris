\chapter{Satellite Orbit modeling}
\label{ch:satellite-orbit-modeling}

\section{Linearization of the Orbit Determination Process}
In the general orbit determination problem, both the dynamics and the measurements 
involve significant nonlinear relationships. For the general case, the governing 
relations involve the nonlinear expression:
%\begin{equation}
\begin{align}
  \label{eq:tapley421}
  \dot{\vec{x}} = F( \vec{x}, t )                 , & \quad \vec{x}(t_k ) \equiv \vec{x}_k \\
  \label{eq:tapley422}
  \vec{y}_i = G( \vec{x}_i , t_i ) + {\epsilon}_i , & \quad  i=1,2,\ldots ,l
\end{align}
%\end{equation}

where \(\vec{x}_k\) is the unknown \(n\)-dimensional state vector at time \(t_k\) and 
\(\vec{y}_i\) for \(i=1,2,\ldots ,l\) is a \(p\)-dimensional set of observations. The 
\emph{best estimate} of the state vector \(\vec{x}_k\) will be denoted as \(\hat{\vec{x}}_k\). 
In general, \(p<n\) and \( m = p \cdot l \gg n \). The formulation described by the 
system \ref{eq:tapley421} and \ref{eq:tapley422}, is characterized by: (\cite{tapley})
\begin{enumerate}
  \item the inability to observe the state (\(\vec{x}_k\)) directly,
  \item nonlinear relations between the observations and the state, 
  \item fewer observations at any time epoch \(i\) than there are state vector 
  components (\(p<n\)), and 
  \item errors in the observations represented by \({\epsilon}_i\)
\end{enumerate}

If a reasonable reference trajectory \(\vec{x}*\) is available and if 
\(\vec{x}\), the true trajectory, and the reference trajectory remain sufficiently 
close throughout the time interval of interest, then the trajectory for the actual 
motion can be expanded in a Taylor’s series about the reference trajectory at 
each point in time. Eliminating higher order terms, the deviation in the state
from the reference trajectory can be described by a set of linear differential 
equations. Corresponding linear relations can be derived between the observation 
and the state deviations, thus transforming the nonlinear orbit determination problem 
to a linear one.

If \(\vec{\delta x}\) is the \( n \times 1 \) state deviation vector and 
\(\vec{\delta y}\) is the \(p \times 1\) observation deviation:
\begin{equation}
  \begin{aligned}
    \vec{\delta x} (t) &= \vec{x}(t) - \vec{x}^* (t) \\
    \vec{\delta y} (t) &= \vec{y}(t) - \vec{y}^*(t)
  \end{aligned}
\end{equation}

hence

\begin{equation}
  \dot{\vec{\delta x}} (t) = \dot{\vec{x}} (t) - \dot{\vec{x}}^* (t)
\end{equation}

Expanding \ref{eq:tapley421} and \ref{eq:tapley422} in a Taylor series about the 
reference trajectory, leads to:
\begin{equation}
\label{eq:tapley425}
\begin{aligned}
  \dot{\vec{x}} (t) &= F (\vec{x}, t) \\
   & \approx F (\vec{x}^* , t) 
    + \left.\frac{\partial F(t)}{\partial \vec{x}(t)}\right|_{\vec{x}^*} \left( \vec{x}(t) - \vec{x}^* (t) \right)
    + O_F \left( \vec{x}(t) - \vec{x}^* (t) \right) \\
    \vec{y}_i &= G( \vec{x}_i , t_i ) + {\epsilon}_i = \\ & \approx G( \vec{x}^*_i , t_i )
    + \left.\frac{\partial G}{\partial \vec{x}}\right|_{\vec{x}^* , i} \left.\left( \vec{x}(t_i) - \vec{x}^* (t_i) \right)\right|_{i} 
    + O_G \left( \vec{x}(t_i) - \vec{x}^* (t_i) \right) + {\epsilon}_i \\
\end{aligned}
\end{equation}

where \(\left.\frac{\partial}{\partial \vec{x}}\right|_{\vec{x}^*}\) indicates that 
the partial derivative matrix is evaluated on the reference solution \(\vec{x}^* (t)\) 
which is obtained by integrating \ref{eq:tapley421} with the initial conditions 
\(\vec{x}^* (t_0)\). \(O_F\) and \(O_G\) indicate terms higher than the 1\textsuperscript{st} 
order which are ignored. Noting that \(\dot{\vec{x}}^* = F(\vec{x}^* ,t)\) and 
\(\vec{y}^*_i = G(\vec{x}^*_i , t_i )\), and letting
\begin{equation}
\begin{aligned}
\delta \vec{x}(t) & = \vec{x}(t) - \vec{x}^*(t)\\
\delta \vec{x}_i  & = \vec{x}(t_i) - \vec{x}^*(t_i)\\
\delta \vec{y}_i  & = \vec{y}_i - G(\vec{x}^*_i , t_i )
\end{aligned}
\end{equation}

\ref{eq:tapley421} can be written as:
\begin{equation}
\begin{aligned}
  \dot{\delta \vec{x}}(t) &= A(t) \delta \vec{x}(t) \\
  \delta \vec{y}_i &= \tilde{H}_i \delta \vec{x}_i + {\epsilon}_i \quad (i=1,\ldots,l)
\end{aligned}
\end{equation}

where
\begin{equation}
A(t) = \left.\frac{\partial F(t)}{\partial \vec{x} (t)}\right|_{\vec{x}^*} 
\quad
\tilde{H}_i = \left.\frac{\partial G}{\partial \vec{x}}\right|_{\vec{x}^* , i}
\end{equation}
