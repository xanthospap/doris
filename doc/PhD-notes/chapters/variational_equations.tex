\chapter{State Transition Matrix and Variational Equations}
\label{ch:state-transition-matrix-and-variational-equations}

The \emph{State Transition Matrix} $\bm{\Phi} (t, t_0 )$, at some specified epoch $t_0$ 
describes the results of a change of initial values (in the state vector 
$\bm{y}(t_0) = \begin{bmatrix} \bm{r}(t_0) \bm{v}(t_0) \end{bmatrix}$)
in a later epoch $t$. The form of the State Transition Matrix, is:
\begin{equation}
  \begin{pmatrix}
    \frac{\partial \bm{y}(t)}{\partial \bm{y}(t_0)}
  \end{pmatrix}_{(6 \times 6)}
  = \bm{\Phi} (t, t_0)
\end{equation}

Besides initial state, the orbit also depends on a number of parameters, 
$\bm{p}$, with size $n_p$, that determine the forces acting on the given 
satellite at a given instant $t$. Such parameters can be, e.g. the drag 
coefficient $C_D$. The dependence of the orbit on these parameters is 
described by the so called \emph{Sensitivity Matrix} $\bm{S}(t)$, where:
\begin{equation}
  \begin{pmatrix}
    \frac{\partial \bm{y}(t)}{\partial \bm{p}}
  \end{pmatrix}_{(6 \times n_p)}
  = \bm{S} (t)
\end{equation}

Denoting an observation by $z(t)$, we can also derive the partials of the 
measurements with respect to the state vector, which is given by the partials
$ \begin{pmatrix} \frac{\partial \bm{z}}{\partial \bm{y}(t)} \end{pmatrix}_{(1 \times 6)}$

Computation of the observation model also depends on a number of 
parameters $\bm{q}$, of size $n_q$, such as e.g. relative frequency offsets, 
wet part of zenith tropospheric delay, etc. The dependance of the predicted 
observation values on these parameters, is described by the partials:
$ \begin{pmatrix} \frac{\partial \bm{z}}{\partial \bm{q}} \end{pmatrix}_{(1 \times n_q)}$

Combining the above, we can compute the dependence of an individual measurement 
$z$ on the initial state vector $\bm{y}(t_0)$, force model parameters $\bm{p}$ 
and observation model parameters $\bm{q}$ (\cite{Montenbruck2000}):
\begin{equation}
  \begin{pmatrix}
    \frac{\partial \bm{z}}{\partial \bm{y}(t_0)}, 
    \frac{\partial \bm{z}}{\partial \bm{p}}, 
    \frac{\partial \bm{z}}{\partial \bm{q}}
  \end{pmatrix}_{(1 \times (6+n_p+n_q))} = 
  \begin{pmatrix}
    \left( \frac{\partial \bm{z}}{\partial \bm{y}(t)} \right)
    \left( \Phi (t,t_0) \bm{S}(t) \right), 
    \frac{\partial \bm{z}}{\partial \bm{q}}
  \end{pmatrix}
\end{equation}

In applications where accuracy demands are high, the computation of the state 
transition matrix $\bm{\Phi}(t,t_0)$, must take into account at least the major 
perturbations. As is the case for perturbed motion, this augmentation make an 
analytical solution impossible; one has to solve a set of differential equations, 
the so-called ``\emph{Variational Equation}'' by numerical methods.

The state vector $\bm{y}(t) = \begin{bmatrix} \bm{r}(t) \\ \bm{v}(t) \end{bmatrix}$ 
obeys the first order differential equation:
\begin{equation}
  \frac{d \bm{y}(t)}{dt} = \bm{f}(t,\bm{y}) = 
    \begin{bmatrix} \bm{v}(t) \\ \bm{a}(t,\bm{r},\bm{v}) \end{bmatrix}
\end{equation}
so that:
\begin{equation}
  \frac{\partial}{\partial \bm{y}(t_0)}\frac{d\bm{y}(t)}{dt} = 
    \frac{\partial \bm{f}(t,\bm{y}(t))}{\partial \bm{y}(t_0)} = 
    \frac{\partial \bm{f}(t,\bm{y}(t))}{\partial \bm{y}(t)} 
      \frac{\partial \bm{y}(t)}{\partial \bm{y}(t_0)}
\end{equation}

The state transition matrix
\begin{equation}
  \bm{\Phi}(t,t_0) = \frac{\partial \bm{y}(t)}{\partial \bm{y}(t_0)}
\end{equation}

may therefore be obtained from:
\begin{equation}
  \frac{d}{dt}\bm{\Phi}(t,t_0) = 
    \frac{\partial \bm{f}(t,\bm{y}(t))}{\partial \bm{y}(t)}
      \bm{\Phi}(t,t_0)
\end{equation}

or

\begin{equation}
  \frac{d}{dt}\bm{\Phi}(t,t_0) = 
  \begin{bmatrix}
    \bm{0}_{(3 \times 3)} & \bm{I}_{(3 \times 3)} \\
    \frac{\partial \bm{a}(t,\bm{r},\bm{v})}{\partial \bm{r}(t)} &
      \frac{\partial \bm{a}(t,\bm{r},\bm{v})}{\partial \bm{v}(t)} \\
  \end{bmatrix}_{(6 \times 6)}
  \bm{\Phi}(t,t_0)
\end{equation}

with the initial condition:
\begin{equation}
  \bm{\Phi}(t_0,t_0) = \bm{I}_{(6 \times 6)}
\end{equation}

In an analogous way, we can derive the differential equation of the sensitivity 
matrix (partial derivatives of the state vector \emph{w.r.t} the force model 
parameter vector $\bm{p}$):
\begin{equation}
  \begin{pmatrix} \frac{d}{dt} \frac{\partial \bm{y}(t)}{\partial \bm{p}} 
    \end{pmatrix}_{(6 \times 6)} = 
  \frac{\partial \bm{f}(t,\bm{y}(t), \bm{p})}{\partial \bm{y}(t)} 
      \frac{\partial \bm{y}(t)}{\partial \bm{p}} + 
      \frac{\partial \bm{f}(t,\bm{y}(t), \bm{p})}{\partial \bm{p}}
\end{equation}
or:
\begin{equation}
  \begin{pmatrix} \frac{d}{dt}\bm{S}(t) \end{pmatrix}_{6 \times n_p} = 
    \begin{pmatrix}
      \bm{0}_{(3 \times 3)} & \bm{I}_{(3 \times 3)} \\
      \frac{\partial \bm{a}(t,\bm{r},\bm{v},\bm{p})}{\partial \bm{r}(t)} &
      \frac{\partial \bm{a}(t,\bm{r},\bm{v},\bm{p})}{\partial \bm{v}(t)}
   \end{pmatrix}_{(6 \times 6)}
    \bm{S}(t) + 
   \begin{pmatrix}
      \bm{0}_{(3 \times n_p)} \\
      \frac{\partial \bm{a}(t,\bm{r},\bm{v},\bm{p})}{\partial \bm{p}}
   \end{pmatrix}_{(6 \times n_p)}
\end{equation}

with initial conditions $\bm{S}(t_0) = \bm{0}$, since the state vector at $t_0$ 
does not depend on the force model parameters.

The set of variational equations are formulated by augmenting the differential 
equation system of the state transition matrix, with the one derived from the 
sensitivity matrix; hence, we have:
\begin{equation}
  \frac{d}{dt} \begin{pmatrix} \bm{\Phi} & \bm{S} \end{pmatrix} = 
  \begin{pmatrix}
      \bm{0}_{(3 \times 3)} & \bm{I}_{(3 \times 3)} \\
      \frac{\partial \bm{a}}{\partial \bm{r}} & \frac{\partial \bm{a}}{\partial \bm{v}} 
  \end{pmatrix}_{(6 \times 6)}
  \cdot \begin{pmatrix} \bm{\Phi} & \bm{S} \end{pmatrix} + 
  \begin{pmatrix}
      \bm{0}_{(3 \times 6)} & \bm{0}_{(3 \times n_p)} \\
      \bm{0}_{(3 \times 6)} & \frac{\partial \bm{a}}{\partial \bm{p}} 
  \end{pmatrix}_{(6 \times (6+n_p))}
\end{equation}
More suitable representations of the above differential equation systems can be 
formulated, when choosing to directly integrate the set of second order 
differential equations (see e.g. \cite{Montenbruck2000}).

It is important to note, that the variational equations must be integrated 
simultaneously. 
