\documentclass[12pt,a4paper,twoside]{report}

\usepackage[utf8]{inputenc}
\usepackage{graphicx}
\graphicspath{ {./../figures/} }
\usepackage{caption}
\usepackage{subcaption}
\usepackage{xcolor}
\usepackage{fancyvrb} % Verbatim and coloring therein
\usepackage{hyperref}
\usepackage{imakeidx} % for index
\makeindex[columns=3, title=Alphabetical Index]
\usepackage{soul} % allow wrapping of underlined text, via \ul{...}
\usepackage{natbib} % for bibliography
\usepackage[left=2cm,right=2cm]{geometry} % somewhat wider text to allow code
\usepackage{siunitx} %% SI Units
\usepackage{amsmath} %% for math ...
\usepackage{amssymb} %% greek and various toher characters and symbols ...
\usepackage{mathabx}
\usepackage[acronym, toc, nonumberlist]{glossaries} %% for acronyms
\usepackage{tabularx}
\usepackage{multirow} %% for tabular

%% TikZ stuff %%
\usepackage{tikz} % add a few drawings ...
\usepackage{tkz-euclide}
\usetikzlibrary{shapes.geometric, arrows} % for creating tikz flowcharts 
\usetikzlibrary{shapes.misc}
\usetikzlibrary{calc}
\usetikzlibrary{quotes,angles}
\tikzstyle{io} = [trapezium, trapezium left angle=70, trapezium right angle=110, minimum width=3cm, minimum height=1cm, text centered, draw=black, fill=blue!30]
\tikzstyle{process} = [rectangle, minimum width=3cm, minimum height=1cm, text centered, draw=black, fill=orange!30]
\tikzstyle{decision} = [diamond, minimum width=3cm, minimum height=1cm, text centered, draw=black, fill=green!30]
\tikzstyle{arrow} = [thick,->,>=stealth]

\usepackage{listings} % include source code files
% Solarized colour scheme for listings
\definecolor{solarized@base03}{HTML}{002B36}
\definecolor{solarized@base02}{HTML}{073642}
\definecolor{solarized@base01}{HTML}{586e75}
\definecolor{solarized@base00}{HTML}{657b83}
\definecolor{solarized@base0}{HTML}{839496}
\definecolor{solarized@base1}{HTML}{93a1a1}
\definecolor{solarized@base2}{HTML}{EEE8D5}
\definecolor{solarized@base3}{HTML}{FDF6E3}
\definecolor{solarized@yellow}{HTML}{B58900}
\definecolor{solarized@orange}{HTML}{CB4B16}
\definecolor{solarized@red}{HTML}{DC322F}
\definecolor{solarized@magenta}{HTML}{D33682}
\definecolor{solarized@violet}{HTML}{6C71C4}
\definecolor{solarized@blue}{HTML}{268BD2}
\definecolor{solarized@cyan}{HTML}{2AA198}
\definecolor{solarized@green}{HTML}{859900}

% Define C++ syntax highlighting colour scheme
\lstset{language=C++,
        basicstyle=\footnotesize\ttfamily,
        numbers=left,
        numberstyle=\tiny,
        tabsize=2,
        breaklines=true,
        escapeinside={@}{@},
        numberstyle=\tiny\color{solarized@base01},
        keywordstyle=\color{solarized@green},
        stringstyle=\color{solarized@cyan}\ttfamily,
        identifierstyle=\color{solarized@blue},
        commentstyle=\color{solarized@base01},
        emphstyle=\color{solarized@red},
        frame=single,
        rulecolor=\color{solarized@base2},
        rulesepcolor=\color{solarized@base2},
        showstringspaces=false
}

% include the external source file, instead of pasting its contents directly 
% into the LaTeX documen
\newcommand{\codelst}[1]{\lstinputlisting[caption=\texttt{\protect\detokenize{#1}}]{#1}\newpage}

% augment the paragraph skip ... a bit more clear text
\setlength{\parskip}{1em}

\bibliographystyle{plainnat}

\title{Appendum to Runge-Kutta Methods\\Implementation Details}
\author{Xanthos}
\date{\today}

\begin{document}

\begin{titlepage}
\maketitle
\end{titlepage}

%\frontmatter
%\tableofcontents
%\listoffigures
%\listoftables

\section{Solving ODE Systems with Runge-Kutta}
\subsection{Problem Description}
\label{ssec:problem-description}
Start with a simple two-dimensional example of a 2\textsuperscript{nd} degree ODE.
The ODE reads:
\begin{equation}
  \label{eq:orgode}
  \ddot{\vec{r}}(t) = - \vec{r}(t) / {r(t)}^3
\end{equation}
with \( \vec{r}(t) \) the position vector, and
\begin{equation}
  \vec{r} = 
  \begin{bmatrix}
  x_{1}(t) \\
  x_{2}(t) \\
  \end{bmatrix}
\end{equation}

A n\textsuperscript{th} degree ODE can be transformed to a system of \(n\) first-order ODEs.
Transform \ref{eq:orgode} to a system of 1\textsuperscript{st} degree ODEs via parameter transformation 
\textit{(dropping the explicit dependency on \(t\) which is henceforth implied)}:
\begin{equation}
  \vec{q} = 
  \begin{bmatrix}
    \vec{r} \\
    \dot{\vec{r}} \\
  \end{bmatrix}
\Rightarrow
  \dot{\vec{q}} = 
  \begin{bmatrix}
    \dot{\vec{r}} \\
    \ddot{\vec{r}} \\
  \end{bmatrix}
\end{equation}

Note that \( \vec{q} \) is actually the \textbf{state vector}.

Expanding vector components, the above reads:
\begin{equation}
  \vec{q} = 
  \begin{bmatrix}
    x_1 \\
    x_2 \\
    \dot{x}_1 \\
    \dot{x}_2 \\
  \end{bmatrix}
\Rightarrow
  \dot{\vec{q}} = 
  \begin{bmatrix}
    \dot{x}_1 \\
    \dot{x}_2 \\
    \ddot{x}_1 \\
    \ddot{x}_2 \\
  \end{bmatrix}
\end{equation}

Hence the ODE systems becomes:
\begin{equation}
  \dot{\vec{q}} = f( t, \vec{q} )
\end{equation}

where \(f\) is a \emph{vector function}; taking into account the original ODE \ref{eq:orgode}, 
we can write the following system of 1\textsuperscript{st} degree ODEs:

\begin{equation}
  \begin{bmatrix}
    \dot{x}_1 \\
    \dot{x}_2 \\
    \ddot{x}_1 \\
    \ddot{x}_2 \\
  \end{bmatrix}
  =
  \begin{bmatrix}
    \dot{x_1} \\
    \dot{x_2} \\
    - x_1 / {({x_1}^2 + {x_2}^2)}^3 \\
    - x_2 / {({x_1}^2 + {x_2}^2)}^3 \\
  \end{bmatrix}
  =
  \begin{bmatrix}
    q(3) \\
    q(4) \\
    -q(1) / {(q(1)^2 +q(2)^2)}^3 \\
    -q(2) / {(q(1)^2 +q(2)^2)}^3 \\
  \end{bmatrix}
\end{equation}

\subsection{Runge-Kutta 4}
In general, if we have an initial value problem:
\begin{equation}
  \frac{dy}{dt} = f(t,y) \text{ with } y(t_0) = y_0
\end{equation}

where \( y \) is an unknown function (scalar or vector) of time \( t \), which we 
would like to approximate, we procced as follows:\\
Pick a step size \( h > 0 \) and define:
\begin{equation}
  y_{n+1} = y_n + \frac{1}{6} h (k_1 + 2 k_2 + 2 k_3 + k_4)
\end{equation}
and
\begin{equation}
  t_{n+1} = t_n + h
\end{equation}

where

\begin{equation}
  \begin{array}{lcl}
  k_1 & = & f(t_n, y_n) \\
  k_2 & = & f(t_n + \frac{h}{2}, y_n + h\frac{k_1}{2}) \\
  k_3 & = & f(t_n + \frac{h}{2}, y_n + h\frac{k_2}{2}) \\
  k_4 & = & f(t_n + h, y_n + h k_3) \\
  \end{array}
\end{equation}

Here \( y_{n+1} \) is the RK4 approximation of \( y( t_{n+1} ) \) and the next value 
\( y_{n+1} \) is determined by the present value \( y_n \) plus the weighted average 
of four increments, where each increment is the product of the size of the interval 
\(h\) and an estimated slope specified by function \(f\) on the right-hand side 
of the differential equation.

Going back to our example, \( y \) is the state vector and the differential equation 
\( f \) is a vector function. Here is how we would go about implementing a 
Runge-Kutta 4 solution, to time \( t \) (\textit{note that in this contrived example, 
the independant \( t \) variable is not used in the computations; the state vector 
is not dependent on \(t\)}).
\begin{enumerate}
  \item First off, from the initial conditions we should have the value of the state vecotr \( \vec{q} \) for \( t = t_0 \)
  \item Compute the derivative of the state vector, when \( \vec{q} = \vec{q}(t_0) \); 
  store the result in an array (\( k_1 \)).
  \item Compute the derivative of the state vector, when \( \vec{q} = \vec{q}(t_0)+\vec{k_1} \frac{h}{2} \); store the result in an array (\( k_2 \)).
  \item Compute the derivative of the state vector, when \( \vec{q} = \vec{q}(t_0)+\vec{k_2} \frac{h}{2} \); store the result in an array (\( k_3 \)).
  \item Compute the derivative of the state vector, when \( \vec{q} = \vec{q}(t_0)+\vec{k_3} h \); store the result in an array (\( k_4 \)).
  \item Compute the qiantity \( \frac{1}{6} h (k_1 + 2 k_2 + 2 k_3 + k_4) \)
  \item Add it to the last value of the state vector \( q \)
  \item Update \( t \), \( t = t + h \)
  \item Repeat  ....
\end{enumerate}

\subsection{ODE Solution Implementation via Runge-Kutta 4}
Here is a dead simple c++ source code to implement the Runge-Kutta solution for the 
problem described in \ref{ssec:problem-description}.
\lstinputlisting{source/rk4.cpp}

\subsection{Yet Another Example for Runge-Kutta 4}
\emph{The following example was released in the public domain by Gilberto E. Urroz 
and can be found \href{https://smath.com/wiki/GetFile.aspx?File=Examples/RK4-2ndOrderODE.pdf}{here}; 
last accessed 24 March, 2022)}

Consider the 2\textsuperscript{nd} order ODE:
\begin{equation}
  \label{eq:ex2ode}
  y'' + y \cdot y' + 3 \cdot y = \sin x
\end{equation}
subject to the initial conditions:
\begin{equation}
  y(0) = -1 \text{  and  } y'(0) = 1
\end{equation}

Let's transform this to a system of first order ODEs:
\begin{equation}
  \begin{aligned}
    u_1 & \equiv y   \\
    u_2 & \equiv y'  \\
    %u_3 & \equiv y'' \\
  \end{aligned}
\end{equation}

Hence, \ref{eq:ex2ode} can be written as:
\begin{equation}
  \begin{aligned}
    u'_1 & = u_2 \\
    u'_2 & = - u_1 \cdot u_2 - 3 \cdot u_1 + \sin x \\
  \end{aligned}
\end{equation}

or in vector for:
\begin{equation}
  \frac{d}{dx} \vec{u} = 
  \vec{f} ( x, \vec{u} ) , 
  \text{ where } 
  \vec{u} = \begin{bmatrix} 
    u_1 \\
    u_2 \\
  \end{bmatrix}
\end{equation}

with the corresponding initial conditions:
\begin{equation}
  \begin{aligned}
    u_1 (0) & = -1 \\
    u_2 (0) & = 1 \\
  \end{aligned}
\end{equation}

\subsection{No Explicit Transformation to 1\textsuperscript{st} Order ODE}
It may happend that we do not want or cannot transform the second order ODE function 
to a system of 1\textsuperscript{st} order ODEs (\textit{for example when this step 
is only a intialization for a multi-step method}).

In this case, we will not treat the state vector \( \vec{q} \) explicitely, but use three vectors 
for the position \( \vec{r} \) the velocity \( \vec{v} \equiv \dot{\vec{r}} \) and 
the acceleration \( \vec{a} \equiv \dot{\vec{v}} \).

Note that when computing the RK4 factor (aka the weighted sum of \( k_i \text{ for } i=1,\cdots ,4 \))
we needed to compute a new state vector as input for every \( k_i \); 
for example, \( \vec{q} + \frac{h}{2} k_1 \) for the 
\( k_2 \) coefficients. Note also that \( k_{i-1} \) is the 
derivative of the state vector computed in the previous step; hence, for \( k_2 \), 
the computation \( \vec{q} + \frac{h}{2} k_1 \) reads:
\begin{equation}
  \vec{q}_{k_2} = 
  \begin{bmatrix}
    r_x \\
    r_y \\
    \dot{r}_x \\
    \dot{r}_y \\
  \end{bmatrix}_{t_0}
  +
  \frac{h}{2}
  \begin{bmatrix}
    \dot{r}_x \\
    \dot{r}_y \\
    \ddot{r}_x \\
    \ddot{r}_y \\
  \end{bmatrix}_{k_1}
  =
  \begin{bmatrix}
    \vec{r}_{t_0} \\
    \vec{v}_{t_0} \\
  \end{bmatrix}
  +
  \frac{h}{2}
  \begin{bmatrix}
    \dot{\vec{r}}_{k_1} \\
    \dot{\vec{v}}_{k_1} \\
  \end{bmatrix}
\end{equation}

\begin{table}
  \centering
\begin{tabularx}{\textwidth}{c >{\raggedright\arraybackslash}X >{\raggedright\arraybackslash}X}
\hline
Step & \bf{Via State Vector} & \bf{Via Position and Velocity}\\
Start &
\( \vec{state}_{t_0} \equiv \begin{bmatrix} 
  x  \\
  y  \\
  x' \\
  y' \\
  \end{bmatrix}_{t_0} \) &
\( \vec{r}_{t_0} \equiv \begin{bmatrix} 
  x \\
  y \\
  \end{bmatrix}_{t_0} \text{ and } 
  \vec{v}_{t_0} \equiv \begin{bmatrix}
  x' \\
  y' \\
  \end{bmatrix}_{t_0} \) \\
\hline

\( k_1 \) &
input \(t, \vec{state}_{t_0} \) result: \( k_1 = \begin{bmatrix}
  x'_{t_0} \\
  y'_{t_0} \\
  -x_{t_0} / r^3_{t_0} \\
  -y_{t_0} / r^3_{t_0} \\
  \end{bmatrix} \) &
input \( t, \vec{r}_{t_0}, \vec{v}_{t_0} \) result: \( k_1 = \begin{bmatrix}
  -x_{t_0} / r^3_{t_0} \\
  -y_{t_0} / r^3_{t_0} \\
  \end{bmatrix} \)
and
  \( m_1 = \begin{bmatrix}
    x'_{t_0}\\
    y'_{t_0}\\
  \end{bmatrix} \) \\

\( k_2 \) &
input \(t+\frac{h}{2}, \vec{state}_{t_0}+\frac{h}{2} k_1 = \begin{bmatrix} 
  x_{t_0} + \frac{h}{2}x'_{k_1} \equiv x_{k2in}\\ 
  y_{t_0} + \frac{h}{2}y'_{k_1} \equiv y_{k2in}\\
  x'_{t_0} + \frac{h}{2}x''_{k_1} \equiv x'_{k2in} \\ 
  y'_{t_0} + \frac{h}{2}y''_{k_1} \equiv y'_{k2in} \\
  \end{bmatrix} \) result: \( k_2 = \begin{bmatrix}
  x'_{k2in} \\
  y'_{k2in} \\
  -x_{k2in} / r^3_{t_0} \\
  -y_{k2in} / r^3_{t_0} \\
  \end{bmatrix} \) &
input \( t, \vec{r}_{t}, \vec{v}_{t_0} \) result: \( k_1 = \begin{bmatrix}
  -x_{t_0} / r^3_{t_0} \\
  -y_{t_0} / r^3_{t_0} \\
  \end{bmatrix} \)
and
  \( m_1 = \begin{bmatrix}
    x'_{t_0}\\
    y'_{t_0}\\
  \end{bmatrix} \) \\

\end{tabularx}
\caption{Common Time Scales used in Astronomical and Celestial Computations.}
\end{table}

  
\subsection{2\textsuperscript{nd} Order ODE via Runge-Kutta 4: General Formula}
\emph{Retrieved by the internet as `CE563 Computation Methods',
\href{https://www.engr.colostate.edu/~thompson/hPage/CourseMat/Tutorials/CompMethods/Rungekutta.pdf}{here}}

Given the second order ODE
\begin{equation}
  \frac{d^2 y}{dx^2} = f(x,y,\frac{dy}{dx})
\end{equation}
determine \( y(x) \) using the RK4 method:

We begin by writing the equation as a set of two, first order ODEs as follows:
\begin{equation}
  \begin{aligned}
  \frac{dy'}{dx} & = f(x,y,\frac{dy}{dx}) \\
  \frac{dy}{dx} & = F(x,y,\frac{dy}{dx}) \equiv y' \\
  \end{aligned}
\end{equation}

Next, we apply our Runge-Kutta formulas to each of the two equations as follows:
\begin{table}[h!]
\centering
\begin{tabular}{ c | c }
  \hline
  \( \frac{dy'}{dx} = f (x,y,y') \) & \( \frac{dy}{dx} = y' \) \\
  \hline
  \({y'}_{i+1} = y_i + \frac{1}{6}(f_1 + 2f_2 + 2f_3 + f_4)h \) &
  \(y_{i+1} = y_1 + \frac{1}{6}(F_1 + 2F_2 + 2F_3 + F_4)h \) \\

  \( f_1 = f(x_i , y_i , y'_i ) \) & 
  \( F_1 = y'_i \) \\

  \( f_2 = f(x_i + \frac{1}{2}h, y_i + F_1 \frac{1}{2}h, y'_i + f_1 \frac{1}{2}h) \) & 
  \( F_2 = y'_i + f_1 \frac{1}{2}h \) \\

  \( f_3 = f(x_i + \frac{1}{2}h, y_i + F_2 \frac{1}{2}h, y'_i + f_2 \frac{1}{2}h) \) & 
  \( F_3 = y'_i + f_2 \frac{1}{2}h \) \\

  \( f_4 = f(x_i + h, y_i + F_3 h, y'_i + f_3 h) \) & 
  \( F_4 = y'_i + f_3 h \) \\
  
  \hline
  \( y'_{i+1} = y_i + \frac{1}{6}(f_1 + 2f_2 + 2f_3 + f_4)h \) &
  \( y_{i+1} = y_i + y'_i h + \frac{1}{6}(f_1 + f_2 + f_3) h^2 \) \\
  
  \hline
  \multicolumn{2}{c}{ \( f_1 = f(x_i , y_i , y'_i ) \)} \\
  \multicolumn{2}{c}{ \( f_2 = f(x_i + \frac{1}{2}h, y_i + y'_i \frac{1}{2}h, y'_i + f_1 \frac{1}{2}h) \)} \\
  \multicolumn{2}{c}{ \( f_3 = f(x_i + \frac{1}{2}h, y_i + y'_i \frac{1}{2}h + f_1 \frac{1}{4} h^2 , y'_i + f_2 \frac{1}{2}h) \)} \\
  \multicolumn{2}{c}{ \( f_4 = f(x_i + h, y_i + y'_i h + f_2 \frac{1}{2} h , y'_i + f_3 h) \)} \\
\end{tabular}
\end{table}

\section{Solving ODE Systems with Gauss-Jackson}
\subsection{Problem Description}
The initial value problem is again \ref{ssec:problem-description}.

\subsection{4\textsuperscript{th} Order Gauss-Jackson Method}


\bibliography{doris}

\end{document}
