\documentclass[12pt,a4paper,twoside]{report}

\usepackage[utf8]{inputenc}
\usepackage{graphicx}
\graphicspath{ {./../figures/} }
\usepackage{caption}
\usepackage{subcaption}
\usepackage{xcolor}
\usepackage{fancyvrb} % Verbatim and coloring therein
\usepackage{hyperref}
\usepackage{imakeidx} % for index
\makeindex[columns=3, title=Alphabetical Index]
\usepackage{soul} % allow wrapping of underlined text, via \ul{...}
\usepackage{natbib} % for bibliography
\usepackage[left=2cm,right=2cm]{geometry} % somewhat wider text to allow code
\usepackage{siunitx} %% SI Units
\usepackage{amsmath} %% for math ...
\usepackage{amssymb} %% greek and various toher characters and symbols ...
\usepackage{mathabx}
\usepackage[acronym, toc, nonumberlist]{glossaries} %% for acronyms
\usepackage{tabularx}
\usepackage{multirow} %% for tabular

%% TikZ stuff %%
\usepackage{tikz} % add a few drawings ...
\usepackage{tkz-euclide}
\usetikzlibrary{shapes.geometric, arrows} % for creating tikz flowcharts 
\usetikzlibrary{shapes.misc}
\usetikzlibrary{calc}
\usetikzlibrary{quotes,angles}
\tikzstyle{io} = [trapezium, trapezium left angle=70, trapezium right angle=110, minimum width=3cm, minimum height=1cm, text centered, draw=black, fill=blue!30]
\tikzstyle{process} = [rectangle, minimum width=3cm, minimum height=1cm, text centered, draw=black, fill=orange!30]
\tikzstyle{decision} = [diamond, minimum width=3cm, minimum height=1cm, text centered, draw=black, fill=green!30]
\tikzstyle{arrow} = [thick,->,>=stealth]

\usepackage{listings} % include source code files
% Solarized colour scheme for listings
\definecolor{solarized@base03}{HTML}{002B36}
\definecolor{solarized@base02}{HTML}{073642}
\definecolor{solarized@base01}{HTML}{586e75}
\definecolor{solarized@base00}{HTML}{657b83}
\definecolor{solarized@base0}{HTML}{839496}
\definecolor{solarized@base1}{HTML}{93a1a1}
\definecolor{solarized@base2}{HTML}{EEE8D5}
\definecolor{solarized@base3}{HTML}{FDF6E3}
\definecolor{solarized@yellow}{HTML}{B58900}
\definecolor{solarized@orange}{HTML}{CB4B16}
\definecolor{solarized@red}{HTML}{DC322F}
\definecolor{solarized@magenta}{HTML}{D33682}
\definecolor{solarized@violet}{HTML}{6C71C4}
\definecolor{solarized@blue}{HTML}{268BD2}
\definecolor{solarized@cyan}{HTML}{2AA198}
\definecolor{solarized@green}{HTML}{859900}

% Define C++ syntax highlighting colour scheme
\lstset{language=C++,
        basicstyle=\footnotesize\ttfamily,
        numbers=left,
        numberstyle=\tiny,
        tabsize=2,
        breaklines=true,
        escapeinside={@}{@},
        numberstyle=\tiny\color{solarized@base01},
        keywordstyle=\color{solarized@green},
        stringstyle=\color{solarized@cyan}\ttfamily,
        identifierstyle=\color{solarized@blue},
        commentstyle=\color{solarized@base01},
        emphstyle=\color{solarized@red},
        frame=single,
        rulecolor=\color{solarized@base2},
        rulesepcolor=\color{solarized@base2},
        showstringspaces=false
}

% include the external source file, instead of pasting its contents directly 
% into the LaTeX documen
\newcommand{\codelst}[1]{\lstinputlisting[caption=\texttt{\protect\detokenize{#1}}]{#1}\newpage}

% augment the paragraph skip ... a bit more clear text
\setlength{\parskip}{1em}

\bibliographystyle{plainnat}

\title{Runge-Kutta 4\textsuperscript{th}}
\author{Xanthos}
\date{\today}

\begin{document}

\begin{titlepage}
\maketitle
\end{titlepage}

%\frontmatter
%\tableofcontents
%\listoffigures
%\listoftables

\section{Solving ODE Systems with Runge-Kutta}
A simple two-dimensional example of a 2\textsuperscript{nd} degree ODE.
Starting with:
\begin{equation}
  \label{eq:orgode}
  \ddot{\vec{r}}(t) = - \vec{r}(t) / {r(t)}^3
\end{equation}
with
\begin{equation}
  \vec{r} = 
  \begin{bmatrix}
  x_{1}(t) \\
  x_{2}(t) \\
  \end{bmatrix}
\end{equation}

Transform to a system of 1\textsuperscript{st} degree ODEs via (and dropping the 
explicit dependency on \(t\) which is implied):
\begin{equation}
  \vec{q} = 
  \begin{bmatrix}
    \vec{r} \\
    \dot{\vec{r}} \\
  \end{bmatrix}
\Rightarrow
  \dot{\vec{q}} = 
  \begin{bmatrix}
    \dot{\vec{r}} \\
    \ddot{\vec{r}} \\
  \end{bmatrix}
\end{equation}

Note that \( \vec{q} \) is actually the \textbf{state vector}.

Expanding components, the above means:
\begin{equation}
  \vec{q} = 
  \begin{bmatrix}
    x_1 \\
    x_2 \\
    \dot{x}_1 \\
    \dot{x}_2 \\
  \end{bmatrix}
\Rightarrow
  \dot{\vec{q}} = 
  \begin{bmatrix}
    \dot{x}_1 \\
    \dot{x}_2 \\
    \ddot{x}_1 \\
    \ddot{x}_2 \\
  \end{bmatrix}
\end{equation}

Hence the ODE systems becomes:
\begin{equation}
  \dot{\vec{q}} = f( t, \vec{q} )
\end{equation}

or, taking into account the original ODE \ref{eq:orgode}:

\begin{equation}
  \begin{bmatrix}
    \dot{x}_1 \\
    \dot{x}_2 \\
    \ddot{x}_1 \\
    \ddot{x}_2 \\
  \end{bmatrix}
  =
  \begin{bmatrix}
    \dot{x_1} \\
    \dot{x_2} \\
    - x_1 / {({x_1}^2 + {x_2}^2)}^3 \\
    - x_2 / {({x_1}^2 + {x_2}^2)}^3 \\
  \end{bmatrix}
  =
  \begin{bmatrix}
    q(3) \\
    q(4) \\
    -q(1) / {(q(1)^2 +q(2)^2)}^3 \\
    -q(2) / {(q(1)^2 +q(2)^2)}^3 \\
  \end{bmatrix}
\end{equation}

In general, if we have an initial value problem:
\begin{equation}
  \frac{dy}{dt} = f(t,y) \text{ with } y(t_0) = y_0
\end{equation}

where \( y \) is an unknown function (scalar or vector) of time \( t \), which we 
would like to approximate, we procced as follows:\\
Pick a step size \( h > 0 \) and define:
\begin{equation}
  y_{n+1} = y_n + \frac{1}{6} h (k_1 + 2 k_2 + 2 k_3 + k_4)
\end{equation}
and
\begin{equation}
  t_{n+1} = t_n + h
\end{equation}

where

\begin{equation}
  \begin{array}{lcl}
  k_1 & = & f(t_n, y_n) \\
  k_2 & = & f(t_n + \frac{h}{2}, y_n + h\frac{k_1}{2}) \\
  k_3 & = & f(t_n + \frac{h}{2}, y_n + h\frac{k_2}{2}) \\
  k_4 & = & f(t_n + h, y_n + h k_3) \\
  \end{array}
\end{equation}

Here \( y_{n+1} \) is the RK4 approximation of \( y( t_{n+1} ) \) and the next value 
\( y_{n+1} \) is determined by the present value \( y_n \) plus the weighted average 
of four increments, where each increment is the product of the size of the interval 
\(h\) and an estimated slope specified by function \(f\) on the right-hand side 
of the differential equation.

Going back to our example, \( y \) is the state vector and the differential equation 
\( f \) is a vector function. Here is how we would go about implementing a 
Runge-Kutta 4 solution, to time \( t \) (\textit{note that in this contrived example, 
the independant \( t \) variable is not used in the computations; the state vector 
is not dependent on \(t\)}).
\begin{enumerate}
  \item First off, from the initial conditions we should have the value of the state vecotr \( \vec{q} \) for \( t = t_0 \)
  \item Compute the derivative of the state vector, when \( \vec{q} = \vec{q}(t_0) \); 
  store the result in an array (\( k_1 \)).
  \item Compute the derivative of the state vector, when \( \vec{q} = \vec{q}(t_0)+\vec{k_1} \frac{h}{2} \); store the result in an array (\( k_2 \)).
  \item Compute the derivative of the state vector, when \( \vec{q} = \vec{q}(t_0)+\vec{k_2} \frac{h}{2} \); store the result in an array (\( k_3 \)).
  \item Compute the derivative of the state vector, when \( \vec{q} = \vec{q}(t_0)+\vec{k_3} h \); store the result in an array (\( k_4 \)).
  \item Compute the qiantity \( \frac{1}{6} h (k_1 + 2 k_2 + 2 k_3 + k_4) \)
  \item Add it to the last value of the state vector \( q \)
  \item Update \( t \), \( t = t + h \)
  \item Repeat  ....
\end{enumerate}

\lstinputlisting{source/rk4.cpp}

\bibliography{doris}

\end{document}
