\documentclass{beamer}

\mode<presentation>
{
  \usetheme{default}      % or try Darmstadt, Madrid, Warsaw, ...
  \usecolortheme{default} % or try albatross, beaver, crane, ...
  \usefonttheme{default}  % or try serif, structurebold, ...
  \setbeamertemplate{navigation symbols}{}
  \setbeamertemplate{caption}[numbered]
} 

\usepackage[utf8]{inputenc}
\usepackage{epstopdf}
\usepackage{graphicx}
\graphicspath{ {../figures/} }
\usepackage{xcolor}
\usepackage{hyperref}
\usepackage{soul} % for \st{...} aka strikethrough
\usepackage{amsmath} % for math ...
% \usepackage{gensymb} % degree and other symbols ...
\usepackage{tikz} % drawing stuff ..
\usepackage{enumitem}
\usepackage{siunitx} %% SI Units
\usetikzlibrary{tikzmark,shapes}

\tikzset{upper node/.style={fill=red!40},
  lower node/.style={ellipse,fill=red!20},
  ul line/.style={thick,red!20},
  upper/.code={\tikzset{upper node/.append style={#1}}},
  lower/.code={\tikzset{lower node/.append style={#1}}},
  line/.code={\tikzset{ul line/.append style={#1}}},
}

%% highlight with explanation beneath, see 
%% https://tex.stackexchange.com/questions/458694/highlighting-part-of-equation-with-text-underneath
\newcounter{HighLight}
\newcommand{\highlightwu}[3][]{%
  \stepcounter{HighLight}
  \tikzset{#1}
  \underset{\underset{\displaystyle\makebox[0pt]{\text{\tikzmarknode[lower node]{below-\theHighLight}{%
    #3}}}}{\phantom{!}}}{\tikzmarknode[upper node]{above-\theHighLight}{#2}}
    \tikz[overlay,remember picture]{\draw[ul line] (above-\theHighLight) --
      (below-\theHighLight);}
}
%% highlight term
\newcommand{\highlight}[2][]{%
  \stepcounter{HighLight}
  \tikzset{#1}
  {\tikzmarknode[upper node]{above-\theHighLight}{\textbf #2}}
}

%% define a box ...
\setbeamercolor{postit}{fg=black,bg=example text.fg!75!black!10!bg}

%% some itemize styles
\newcommand{\citem}{\item[\checkmark]}
\newcommand{\bitem}{\item[\textbullet]}
\newcommand{\ditem}{\item[\textendash]}
\newcommand{\aitem}{\item[\textasteriskcentered]}

\usepackage{natbib}
\bibliographystyle{plainnat}

\title[]{PhD Status}
\author{Xanthos}
\institute{DSO \& IGN}
\date{\today}

\begin{document}

\begin{frame}
  \titlepage
\end{frame}

\begin{frame}\frametitle{So far ...}\framesubtitle{}
    \begin{itemize}
      \citem Last Meeting, November 2021
    \end{itemize}

    Since then, i have mainly been studying Orbit Determination!

\end{frame}

\begin{frame}\frametitle{Currently ...}\framesubtitle{}
    \begin{itemize}
        \bitem DORIS, technique-dependent part: mostly implemented, need to 
        add details (e.g. time tagging).

        \bitem Filtering: implemented on a basic level. Model \textbf{EFK}, 
        still no square root filtering and process noise.

        \bitem Orbit Determination: implemented a series of models of basic 
        astrodynamics, force models and integrators.
      \end{itemize}

      \medskip

      Hard to validate results apart from examples included in 
      \cite{Montenbruck2000}.

\end{frame}

\begin{frame}\frametitle{Currently ...}\framesubtitle{}
    Plan for the next 2-3(?) months:
    \medskip
    
    Use real data to roughly estimate:
    \begin{itemize}
        \bitem Satellite state
        \bitem Beacon position
        \bitem Tropospheric delay
    \end{itemize}

    \medskip
    Need to have a clue on the efficieny of Orbit models and integrators, aka 
    need to validate!
    
\end{frame}

\begin{frame}\frametitle{Currently ...}\framesubtitle{}
    Idea for validation:
    \medskip

    Extract satellite state from sp3 files.
    \textbf{Use epoch-wise state (as recorded in the sp3) as initial values to 
    propagate state to the next record.}

    \smallskip
    Use a very simple model including some gravity potential model (e.g. JGM 
    degree/order 20/20) and integrate state plus Variational Equations, that 
    is solve the system of 2\textsuperscript{nd} order ODEs.

    \smallskip
    Don't really know what to expect accuracy-wise ...
    
\end{frame}

\begin{frame}\frametitle{Currently ...}\framesubtitle{}
    Idea for validation:
    \medskip

    Once we are happy with the results, incorporate models:
    \begin{itemize}
        \bitem Third-Body attraction (Moon/Sun)
        \bitem Atmospheric drag
        \bitem Solar Radiation Pressure (no macromodel, but including eclipses)
        \bitem Tide effects (not fully implemented yet)
        \bitem Time-varying gravity models
    \end{itemize}

    \medskip
    Then, go back to read data!

\end{frame}

\begin{frame}\frametitle{Conversation ...}\framesubtitle{}
    Need to have a title:\\
    \medskip
    should we look into (near) real-time data?
\end{frame}

\begin{frame}\frametitle{Deadline ...}\framesubtitle{}
    Very heavy pressure!\\
    \textbf{We should be over by December, that is 6 months!!!}\\
    Should we set our own deadlines?
    
    \medskip
    We should probably skip:
    \begin{itemize}
        \bitem SRP macromodels
        \bitem attitude determination
        \bitem ...
    \end{itemize}
\end{frame}

\bibliography{doris}

\end{document}